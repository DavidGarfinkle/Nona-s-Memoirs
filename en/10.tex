% mempages 188 to 199
% pdfpages 202 to 213

-175- 
When we were small, we played hopskotch together.
When we were older, 
we went dancing together, with other friends, and Chaim.
Sometimes 
Eliaou came with us.
"Give me news of the Bassos!"
I said.
"Everyone was taken by the Germans.
Marie, also.
But my brother 
and the baby are alive.
Talking about babies, I want to discuss something very important with you.
Let's sit down in the cafe."
We sat down in the cafe, he ordered something for me, and he started to 
talk.
"I had a cousin who was pregnant.
When the Germans took the 
Jews from her ghetto, she was just starting labour.
Hwr husband took 
her to the hospital in the ghetto zone.
He left his wife and started 
back to the house.
There were two Jewish Police near the entrance of 
the hospital.
They arrested him, and we never saw him again.
Two minutes after my cousin gave birth, the Germans took her.
We never heard 
of her again.
Her baby was a little girl.
A nurse from the hospital 
hid her, and told the Germans that she had died at birth.
The nurse 
smuggled the baby out of the hospital and took her home.
The nurse's 
mother looked after this baby like she was a piece of gold.
The only 
person who knows the address of these people is Mrs.
Riades."
Mrs.
Riades was a volunteer nurse for the Greek Red Cross.
She was 
a very good friend of Georgette Modiano, Daniel's wife.
Her husband was 
the one who checked the milk that I distributed for the Red Cross.
He 
would check the temperature and the quantity.
Mr.
and Mrs.
Riddes were 
both very humanitarian and philanthropic.
"I want this child, and I want you to help me take her from these 
people," Mr.
Rousso said to me.
I went to see Mrs.
Riades.
She was very glad to see me.
I asked 
about her husband.
She told me that he had died.
I spoke to her about 
the case of the little girl, and told her that I wanted to see the 
child.
The next day, Mrs.
Riades took me to see her.
The house was in the district of Tomba.
This district was built 
when the exchange of Greeks and Turks took place.
We went in, and found 
the mother with her two daughters.
Their home was very modest.
The 
little girl was there; she was very sweet, and obviously well looked after.
I think that these people didn't eat enough so that they would be 
able to feed her.
I told them I was a relative of the child.
During the time I was there, the child was in the mother's lap.
She didn't move
from there.
When I had come in, the child had said 'Yassou' ('Health').
-177- 
When I left, she said ‘Yassou’ again.
In the doorway, I asked the nurse 
if I could come again the next day to see the child.
These wonderful 
people didn't say no, even if they didn't want me to come back.
I returned to Salonica and I reported what I had seen to Mr.
Rousso.
I said "I don't think these people will give the child up.
They risked 
their lives to save her.
They don't eat enough so that she can eat well.
And the child is very happy with them.
But I'll go there again tomorrow 
and we'll see what happens."
The next day, before my visit, I went to the Joint Distribution 
Committee.
I asked them for some toys.
In Salonica at that time, toys 
were very expensive, but the Joint had plenty, and no children to give 
them to.
They gave me a teddy bear, and a dress for the child as well.
I brought my gifts to the child, and she became a little friendlier towards me.
I visited the child every day for fifteen days.
One day, I brought 
some lollipops, and I gave them to the nurse's mother to give to the 
child.
The mother refused the lollipops.
The child wasn't there.
She 
said "All of us risked our lives to save this little girl.
If the Germans had known what we did, they would have killed us one by one.
If you 
have some ideas about taking this child away from us, it won't be very 
easy."
I said "I come often because I want the little girl to know me.
I 
want to take her."
"Don't come again unless you have a court order," she said.
"I would like to see her one more time," I said.
Five minutes Later 
the child was back at the house.
She was crying "Don't take me away!"
I went back and explained the situation to Mr.
Rousso.
He wanted to 
know what to do.
"The only thing to do is to go to the Joint Distribu-
tion Committee," I said.
"The director, Mr.
Cohen, is a very well known 
lawyer.
Ask him to give you a court order.
If you get this, it will be 
a pleasure for me to go and get this little girl.” Mr.
Rousso asked
if I would talk to the Joint Committee.
I refused.
I said "You must 
go see Mr.
Cohen yourself."
Mr.
Rousso was very disappointed, and I was 
as well.
I left him and went to have lunch.
In the entrance of the hotel, 
saw Tia Donna's son Samuel, who had been a prisoner of war in Italy.
I 
told him that I had his mother's jewellery, but it was in Palestine.
"Your mother wanted me to give you this jewellery after you were married 
and had your first child."
I didn't give the jewellery to Samuel until 
he came to North America to settle.
He came with his wife and little 
girl; I followed Tia Donna's instructions exactly.
Hasson was to be tried in Salonica.
I attended his trial, but I 
got dizzy and fainted, and therefore could not testify against him.
There 
were too many people, and it was too hot.
Seeing Hasson, I remembered my 
whole past: the milk, the beating, Chaim, the officer, the train, the 
houses that I hid in, too many to remember, in unfamiliar districts.
But 
Daniel Modiano, who never tired of me, and Mr.
Neri, the brother-in-law 
of the Italian consul, would visit me wherever I was, and give me courage.
I was dizzy, but Mr.
Neri was near me.
As I was thinking of all these events, I turned my head and saw the 
two Amario brothers, the Jewish Police of the train, among the spectators.
I was getting dizzier and dizzier.
I tried to go out into the 
hall to get some fresh air, but on the way I fainted.
All I remember 
is cold water being thrown on me by two men.
The grand rabbi Koretz was lucky.
He had been sent to the privileged camp in Germany, in recognition of what he had done for the Nazis.
His wife and children returned to Salonica, but rabbi Koretz had died 
of typhoid, and therefore was not humiliated by a trial.
Hasson had attempted to escape to Albania with other people in his 
entourage.
The Italian consul in Salonica gave him a car to help him escape, in exchange for a group of Jews in Baron de Hirsch who were about to be deported.
The Italian consul said that this group were relatives of Italian people.
You can see that the Italians did their 
utmost to rescue Jews from the claws of the Nazis.
Hasson went to Albania to seek refuge.
When the so-called Italians 
left Baron de Hirsch, the Italians sent them to Athens.
They were on the 
same train that Daniel had put me on.
Hasson's escort was made up of his wife, his mistress, and his son.
When they arrived at Coritza, on the border of Albania, the Italians immediately arrested them and placed them in a concentration camp in Albania.
They were liberated on the eighth of September, 1943.
With their 
jewellery and money, they went to the port of Bari.
From there, they 
went to Egypt on a fishing boat.
They circulated very openly in Alexandria.
One day, a Jewish refugee from Salonica recognized Hasson.
The English police arrested him.
When Greece was liberated, he was sent to Athens, to Haidari.
But no-one pursued him to court, and he was liberated.
He never imagined that there would be Jews in Salonica, liberated 
from the concentration camps.
He returned to Salonica in November of 
1945.
A group of survivors of Auschwitz recognized him.
He was immediately sent to prison at Pavlo Mella by the Greek authorities.
Albala and Tapouz were also arrested.
The denunciation was made by 
the Jewish community.
The two had come to Greek territory from the Yugoslavian frontier.
The trial of the collaborators who had committed treason toward 
their brothers in Salonica opened on July 2, 1946, at ten o'clock in the 
morning.
They were first accused of helping the Germans deport the Jews 
of Salonica to Poland.
They were accused also of acts of violence, complicity, carrying weapons, collaboration with the Germans against the 
Hellenic Jews so as to rob them of their fortunes, bad treatment of many 
non-Jewish Greeks who helped Jews, the brutalization of men and women; 
in short, they were accused of the same barbarism as the Germans.
Hasson had the most serious charges brought against him.
He was like a lion let out of his cage.
His power had been equal to that of the chief of
the camp of Baron de Hirsch, Amster.
Hasson was the iron arm of Gerbin 
the S.S.
director of Baron de Hirsch.
They shared everything that they 
took from the Jews.
Tapouz had Herculean strength.
He just had to tap a person's jaw 
with his hand to break it.
He was always near Hasson.
Hasson would 
give the signal, and Tapouz would do the work.
There were many non-Jew-ish witnesses and all the horrible details were accumulated.
Hasson, Tapouz and company had beautiful limouzines, and would go to all the suburbs of Salonica to find Jews in hiding.
They would beat them to death 
and take all their jewellery and wealth.
These are the atrocities that 
this monster of the century and his accomplices committed.
Peppo Carasso provoked a big sensation when he was on the witness 
stand.
He told of how seven members of his family were in hiding in a 
suburb of Salonica.
Hasson eliminated all of them in a particularly savage manner, and then took all their possessions.
Hasson tried to maintain his innocence by blaming others.
After 
five days of hearings, his sentenced was announced at three o'clock in 
the morning in front of a large audience.
Hasson was condemned to die.
Amster and Boudrian received the same sentence.
Leon Sion, or Tapouz, 
was sentenced to life imprisonment.
Albala was sentenced to 15 years, 
Counio to eight years in prison.
Boudrian was a non-Jew.
He disappeared during the German retreat.
Papanaoum, loaded down with treasures, 
also disappeared at this time.
Amster was killed by the Germans by midtake.
He was escorting hostages to Haidari, and when they arrived, the 
Germans killed all of them, including Amster.
Hasson was transferred to the prison of Corfu.
On Thursday, March 
4, 1948, he had confession with a rabbi of Corfu in the morning.
He 
maintained his innocence.
He had been the chief of Baron de Hirsch and 
he had done only what the Germans had told him to do.
He told his wife 
that he felt guilty, because he had had a mistress, and a child by this 
mistress.
He asked that this child, a little girl, not be persecuted, 
because she was innocent.
He asked to be buried in a Jewish cemetery.
Then the traitor paid his debt to society.
I started to try to trace Daniel.
But no-one knew what had become 
of him.
One day I decided to go to see Alfonse Levy.
He would know 
about Daniel for sure.
I finished my work, went out to have dinner, 
and then started off to Alfonse.
As I was walking, I remembered that 
it was siesta, and in Salonica one did not go visiting at this time.
I 
came to Tsimiski and St.
Sofie Streets.
On the corner there was a pastry 
shop called Diefnes.
I grew up in this pastry shop; it was just oppo-
site my family home.
I sat down and stared at my old balcony.
Mr.
Gar-
finkle passed by.
I asked him where he was going.
"To the Joint Distribution Committee.
And you, what are you doing 
here?"
"Killing time," I answered.
"If you want to kill time, come with me," he said.
I went along.
The doctor, the nurse, and an American soldier were there when we 
arrived.
The American soldier congratulated Mr.
Garfinkle.
"You are 
going to work for the Joint Distribution Committee.
You will reorganize 
the Soupe Populaire."
He turned to me with a dirty look, and he said 
"But not you!"
I answered "I don't know who or what you are, and I don't care.
I 
didn't ask the Joint Distribution Committee for work.
Unlike the doctor 
and the nurse, I don't need the camouflage that the Joint would give me 
to stay in Greece.
I can stay as long as I want, all my life if I want."
I left them and went back to the Diefnes, and sat down to kill time again.
I was sitting opposite my old balcony.
I began to remember how I 
grew up on that balcony, where I passed the best years of my life.
I 
used to sit out on the balcony with my embroidery.
Eliaou would come 
to sing for me: 
En tou balcon kreseran las rosas 
Descoje kouala ese 
La mas ermoza 
In your balcony flowers will grow.
Pick the very best
If Regina was out on the balcony, he would sing for her: 
Regina mou Regina mos 
Mazi sou fa pefano 
Ola tou kosmos ta kala 
Brostamou de ta vazo 
(My Regina, my Regina 
With you I will die 
The best things in the world 
Are nothing compared to you) 
I had just finished thinking about this when the nurse appeared.
(The Joint was just two doors away from the pastry shop.)
"Oh, hello," she said to me.
"How did you like the treatment we 
gave you today?
Give up the coat and everything will be fine.”
"Do you see this balcony?"
I asked.
"It's the biggest balcony on 
St.
Sofie Street.
Just opposite to us is the biggest balcony in Salonica.
There was a family living in the house behind that balcony.
In this
family there was a law of humanitarianism.
This family could sing and 
dance, and they had the most beautiful library that an intellectual person can dream of.
They were philanthropes, zionists, religious.
They 
were respected by all the people who knew them.
This was my family.
I 
lost all of this.
But be sure that I am not going to lose my nice winter coat.
Now go for a walk, because you are polluting my favourite corner in Salonica."
I turned away from her, and she left.
I saw the owner of the pastry shop coming toward my table.
He was 
holding an ice cream cone.
He said "Do you remember the ice cream 
cones?"
"Of course," I said.
"You sold ice cream cones to no-one else but 
Regina and me."
"Of course," he said, "because Eliaou used to bring the cones for me 
to give to you.
This ice cream cone I give to you now was brought to me 
by Eliaou years ago.
I kept it in the hope that you would come back."
-183- 
We both cried a little.
We couldn't say anything to each other.
The siesta was finally over.
The owner of Diefnes asked me where I 
was going.
I explained that I was going to see Alfonse to get news of a 
friend who had helped me greatly in a time of need.
"I'll walk with you," he said.
Alfonse, his wife, and his daughter were very happy to see me.
Alfonse said "I saw your courage when you buried the English soldier in 
this community."
I asked him if he had news of Daniel Modiano.
He said "Don't get 
excited.
We Jews from Salonica must be prepared for everything.
In Sep-
tember, 1943, the Mosseris, the Fernandes and the Torres families, and 
Daniel Modiano were in Italy.
They were assassinated by the Gestapo in 
a hotel on Lake Majeur.
Their bodies were thrown into the lake."
These had been the most distinguished families of Salonica.
I felt 
that I was choking.
I went outside for some fresh air.
I started to 
walk very slowly, and I realized that I was talking to myself.
At the 
corner of the Church of St.
Sofie there was a man selling chestnuts from 
a cart.
I bought some to chew, so that I wouldn't talk to myself.
The man gave me the chestnuts in a wad of paper because they were hot.
The papers came from my mother's hagada of Pesach.
I asked him to sell 
me all his papers.
He refused.
Without papers, he couldn't sell chest-
nuts.
This is what they did with all the precious books that belonged to 
Jewish families.
I started to walk, without knowing where I was going.
I grew up in 
that neighbourhood, but I knew no-one who walked by me.
I saw the building where Alegra Saltiel used to live.
I remembered being at the circumcision of Alegra's son.
For the Sepharadim, the birth of a boy was 
occasion for great happiness.
They were fanatics for boy children.
If 
a girl was born, the father was very angry at his wife.
It took him a 
long time to get used to having a daughter.
Take the case of my father's 
father.
The midwife said 'Mazel toll' to my Nono (grandfather).
They used to say 'mazel toy' for a girl, and 'be siman tov' for a boy.
When my grandfather heard 'mazel tov', he went to work without congra-
tulating his wife.
Five minutes after he arrived at work, someone came to see him and said "Sarfatty, you have two mazel tovs!"
Five minutes 
later, someone else came and said "Hey, Sarfatty, you have three mazel 
tovs!"
He said to the man "My God!
I should go home before I have four 
mazel tovs!"
By the time he got home, the fourth mazel toy had arrived 
Only one of the four babies survived.
Her name was Miriam, and she became the favourite of my Nono.
I remembered when Miriam married.
The 
Nano loved her very much.
She had one daughter, Gilda.
When Gilda was 
taken to Auschwitz, she was called by name to be operated on in a Nazi 
experiment.
But she survived, and is now living in Israel.
I remembered the circumcision of Alegra's son.
The rabbi was also
a surgeon.
Everything was prepared.
The godfather sat down in a chair 
with two cushions, called coultoukes, in his lap.
The godmother same 
with the baby.
This was a great honour.
There was an empty chair in 
the room.
This chair was for Eliaou Anavi (the saint of circumcision).
The rabbi took the baby, and he said "This is the chair of Eliaou Anavi" 
He put the baby on the two cushions.
When the circumcision was finished, the rabbi said 'be siman tov!"
Automatically the music started and
the people began to sing the traditional song of circumcision; 
Tio Ovadia Seror 
Vino de Estambol 
Ande Han Ichoua 
Ayi apozo 
(Uncle Ovadia Seror 
He came from Istanbul 
To the house of Han Ichoua 
And there he stayed).
I remembered what was done if the baby was a girl.
The rabbi came 
after one week.
He was called the fadar, and he would give the girl her 
name, and bless her.
It was a beautiful ceremony.
Even if the father  
was disappointed to have a daughter, the people would dance and sing and 
have a good time just the same.
At the circumcision of Alegra's son, there was a bandleader 
named Tsadik.
He was blind, but he was the best composer that Saloni-
ca knew.
Of course, Leon Botton, one of the best comic entertainers, 
was there.
The band was composed of violins, a hout, a tambourine, and 
a dulcimer.
Tsadik was the best dulcimer player in Salonica.
This com-
bination of instruments was called tsalguin.
Leon Botton sang a humour-
ous song: 

Por non mancar vo explicar 
Lo que el hombre por bevir deve Bouchear 
Por non sofri 
deve offrir 
Siempre devertimientos 
Para non soffric 
Couando oun dia de ventura 
Se presenta 
Cale dar bouelta 
I non penssar 
Couando la pounta tsica 
0 algouna vouerta 
Por devertirsse el tiempo deve passar 
Bayla canta i reyr 
De alegria non soffrir 
Vino likor i reyr 
De alegria non mouerir 
Gozar siempre la mansseves 
I el dia boueno-que vech 
Sin hezitar i profitar 
La saloud bien mirar 
Escarssedad ese bovedad 
El que las vouadra 
Es solo i sin penssar 
Se va mouerir i sin bevir 
Siempre devertimientos para bien bevir 
Dechan los bienes tambien los conaques altos 
Calle yir fartos sin dezear 
El que hereda se bourla i etha salto 
Se imajina que aqui va aquedar 
He then dang another song, 'The Very Modern Salonica': 
Oh Moderno Salonique 
Deantes vistian djoubas 
i se rancavan al peynar 
Los hombres soultoueas con djoubas 
I se vistian sin penar 
vazas con mossandaia 
-1 tSb-
Salonique ya se troco 
D4 couando se tsamoucho 
Ma de lodo el non manco 
Anssi vemos todo trocado 
Oh moderno Salonique 
Vemos los autos corres 
Mos salpican sen quierer 
Polvorina en el barer 
De couando se tsamoueheo 
Ma de lousso 
El non manco 
Anssi vemos todo trocado 
Oh moderno Salonique 
Mos quijemos sivilidar 
En tomando todo areves 
El lousso i la moda bezer 
Es el progresso de la mansseves 
Tenemos cazas ethas con biton 
Mi con gantes fiongos i baston 
Las nignas con sous veloutes 
Exssitan con sous decoltes 
Anssi vemos todo trocado 
Oh moderno Salonique 

I walked on.
Suddenly, I heard someone calling me 'Bouenica!’.
"Oh," I thought, "my mother used to call me that!"
I turned around, 
and I saw one of our good neighbours who lived near us on ST.
Sofie 
Street before the war.
She was the wife of Dr.
Pouliades.
She made the 
sign of the cross and said "Oh, Bouenica, you were walking like a 
ghost!"
She took me up to her house and made me some coffee.
Believe 
me, I needed it.
"We have things for you here," she said.
"I want to give them to
you.
Eliaou gave them to us to keep."
She gave me many more things 
from my trousseau.
She gave me a few pieces of silver and a ring.
I 
recognized the ring right away; Chaim's Nona had given it to me.
Mrs.
Pouliades gave me a ceramic plate with Hebrew letters on it, and I remembered where this plate came from.
My mother used to tell us the legend of this plate.
My maternal 
grandmother's family name was Seror.
One Yom Kippour, one of her 
ancestors, Ovadia Seror, was helping the hazan to sing.
This was a big 
honour, to know Hebrew and to have a good voice, to be able to help the 
hazan.
Everyone was praying and he was singing, when the Turkish police entered the synagogue.
The police took Ovadia Seror away.
Everyone was very frightened.
The police gave no explanation.
There is a 
beautiful bay in Salonica.
Near La Tour Blanche on this bay, there 
was a huge warship.
They took Ovadia Seror onto this ship, and sent 
him to Istanbul, to the King’s palace.
They told him that he must sing 
for the King.
He sang the religious songs of Yom Kippour, and when he 
was finished, the King gave him the plate.
He was brought back to Salonica, to the house of Han Ichoua.
Han Ichoua was the big rabbi of 
the Kiynla (congregation) of the Serors.
All the people of Salonica 
looked to Ovadia like a saint.
They thought that the King had wanted 
to kill him.
Everyone would say "Oh, God!
He is newly born!"
This 
is why they sing the song of Seror at circumcisions.
With the plate in my hand, I thought "I, too, feel newly born, after all I went through.
Now I need courage to start a new life."
I 
thanked them very much for returning my things.
I hadn't even known 
that they had them.
I took a handmade tablecloth from my trousseau and 
offered it to Mrs.
Pouliades, but she refused to accept it.
They saw 
that I was confused and depressed, and the husband took me home.
I started to visit Greek friends of my family.
Wherever I went, people gave me 
things that belonged to me.
One day, I went to see a friend of mine.
Her father used 
to be in the same business as Eliaou.
She told me that she had gone to see Eliaou, Regina, Tia Donna, and Reyna the housekeeper.
Eliaou had wanted to give her something,
but she was afraid of walking out of the ghetto with parcels, with the Jewish Police or 
the corner.
One risked one's life taking Jewish belongings.
"But when I was ready to leave," she said, "I saw the beautiful collection of va-
ses in your house.
I took one small vase.
When Eliaou gave it to me, he said that it 
had come from Germany, from the Kaiser's collection.
'Take this, and if you see Bouena 
give it to her, so that she will have a souvenir of the house.'"
I took the vase and I left.
I walked in the streets, going nowhere.
As I was 
walking, I saw Max Garfinkle.
"Today I start my new job," he said, "of reorganizing the Soupe Populaire.
Come 
with me.
I want you to help me, because you have more experience than I. It will be 
in the Orphelinat Alaatini."
The Orphelinat Alaatini was an achievement of which all the Jews were proud.
It was 
founded in 1910, by the initiative of Mr.
E. M. Salem.
The Alaatini family donated a 
beautiful villa in the best district of Salonica.
In collaboration with the Matanoth 
Laevionim and former students of the Alliance Israelite Universelle, the orphanage was 
founded.
From 1910 to 1930, more than 300 boys were very well educated there.
They 
held competitions, and the winners were sent to Paris to study at the Ecole Normal Israe-
ite Oriental.
They now held the prize jobs at the Ecole Israelite Universel of Morocco, 
and in other parts of the world.
The Rothschilds and Alaatinis helped them while they were in Paris.
Many of them 
were artisans, or held positions in commerce, in Greece and in Palestine.
Moise Jacob 
Abravanel died in Morocco in August, 1944.
He had been the guardian angel of this or-
phanage.
Moise Morpurgo, a big, honest, very wealthy man, was devoted to the orphanage 
until his death in 1939.
Mr.
Isaac Covo was with the orphanage until the very last moiment, when the children were deported.
Isaac Covo was the husband of Ida Simantov, whom I met on the train, fleeing from Salonica to Athens.
The orphanage had one of the largest pedagogical libraries, and the most 
modern sports facilities.
Every boy had a godfather.
Eliaou was the godfather 
of one of these boys, and he treated him like his own son.
He was waiting for him to 
finish his courses in commerce; he was planning to take him on as a junior partner.
After the liberation, many deportees lived at the orphanage.
They didn't want to
return to their homes that they had before the war.
These homes were in a low rental 
district, and this district was now non-Jewish.
The Joint Distribution Committee established a Soupe Populaire at the Orphelinat, and Mr.
Garfinkle was the organizer.
I 
helped also.
The Alaatini family had been the wealthiest Jews in Salonica.
They had the largest 
mill in all of Greece.
It was a very old family, philanthropic and humanitarian.
They 
were very interested in the welfare of these children.
Many years before the war, they 
sold the mill to a wealthy non-Jewish family, called Panoutso, but with a condition.
The mill would never be called anything but Alaatini.
Even today, it bears the name 
of Alaatini.
The Jews contributed greatly to the development of Salonica.
The Alaatinis had 
the biggest mill.
The gas and lighting company was owned by Jews.
The electricity 
company was Jewish.
The Bank of Salonica was founded by Jews.
Saias Threads, and many 
other large factories and workshops were founded by Jews.
One Sunday, Mr.
Garfinkle invited me to go to see the Pinkas Dispensary; Mr.
Silby, 
the director of UNRRA, was going to be honoured by the Jewish community.
He was a man who deserved to be honoured by all the Greek communities; he had helped Greece tremendously after the war.
A large part of the staff of UNRRA was there.
The doctor and the nurse were there, 
of course, as well as Mr.
Cohen of the Joint Distribution Committee.
But I was very 
surprised to see that only one Jew from Salonica was present.
"Oh, God," I thought, "the people who returned from the lager should be here.
For 
example, Dr.
Alalouf has just come back.
He was the chief surgeon of the Hirsch Hospital.
Dr.
Couenca, the otolaryngologist, from the same hospital, should also have 
been invited.
They didn't even invite Rabbi Molko, the author of many books and a 
wonderful man!
They didn't invite the revahs, the founders of Matanoth Laevionim.
I asked the doctor and the nurse why the Jewish community had not been invited.
The nurse pointed at the one Jew who was there from Salonica, and she said "It is
enough for us to see only one Sepharady at a time."
I said "The Jews of Salonica need this dispensary in the middle of nowhere like 
a hole in the head:" I was thinking of a Ladino proverb; 'Quien de afouera vernera 
al de adientro kitara'.
It means 'He who comes from the outside gets rid of the people 
inside.'
Thank God, a lady who worked for UNRRA came to my rescue.
"Oh," she said, "I'm so glad to see you: How do you feel?
The last time I saw , 
you, you were very sick, in the hospital."
Mr.
Sibly was near us, and he asked me when I had been in the hospital.
I said "When I was supposed to go to Athens but missed my plane."
"Oh, no!"
he said, "Why didn't you tell me?"
"Because I wasn't dying, and I was supposed to call to cancel my reservation," I
answered.
"Does your superior know that you were in the hospital?"
he asked.
"You have to ask him," I answered.
The next day, I was called in to the UNRRA office.
UNRRA gave me a large number 
of layettes to distribute to pregnant women.
The officer who gave me the layettes repeated three or four times, "Mr.
Sibly recommended you highly."
There were many women ready to have their babies who had no layettes.
I formed 
a committee and we distributed the layettes in one of the centers of the only zionist organization  
left in Salonica after the war.
I blessed UNRRA, who gave me the pleasure of giving the layettes to those who needed them.
I remembered that I had needed 
diapers to change the baby, and how we both had suffered.
I blessed the people who 
grew the cotton from which the layettes were made.
I blessed the people who sewed the 
layettes.
I blessed the people who had the idea to send the layettes.
I blessed the, 
boat that brought these layettes to Greece.
I blessed Mr.
Sibly for recommending me 
to do the distribution, and I did my work with great pleasure.
A few days later, I was called to the Joint Distribution Committee.
Miss Cohen, 
a representative of the Joint Distribution Committee of London, England, approached me.
-191- 
She said "There are only thirteen or fourteen teenagers left in Salonica after 
the liberation.
This is what is left of the jewels of Salonica, of the next genera-
tion.
These children have been hiding in cellars, or in the back rooms of private 
homes.
They grew up with fear.
You should see them.
Their faces are green.
We 
would like to take them for two weeks of good times and childhood to the mountains 
for a vacation.
We would like you to be a chaperone.
The other chaperone will be a 
man, Mr.
Garfinkle."
"Have you spoken to Mr.
Garfinkle about me?"
I asked.
"It was his idea," she answered.
I married the quartermaster, Max Garfinkle, on the fourteenth of July, 1946, in 
the Monasterlis Synagogue.
I thought my wedding would be the saddest wedding in the 
history of the Jews, but the Union of Deportees organized a choir for my wedding.
Coming to the synagogue, the choir was singing Moche Rabenou and Likra Dodim, Likra Cala: 
Mocha Rabenou 
Ayi en el midbar vide reloumbrar 
Las tavlas de la ley vide abaihar 
2 
I mirar i mirar mis signores 
Nbdhe Rabenou 
Que suvio i abouho 
De los altos cielos 
Ven aqui tou moche 
Detente de mi sia 
21x Yto to tonare 
Que to de ouna tavla 
Que lese agrada 
I mirar i mixer mis signores 
moche Rabenou 
Que souvio i abacho de los aktos cielos 
Rabbi Molko performed the ceremony, and the choir sang Sason be Simka.
It was really 
like a ceremony before the war.
On July 15, 1946, instead of a honeymoon, we took the teenagers on their vacation.
Some Jewish people from Salonica had a house in the country.
The Joint Distribution 
Committee gave us everything that we needed: tents, food, and cooking utensils.
The 
place had a huge lawn, but bandits had destroyed most of the house, as they did to many 
Jewish houses.
One room and the kitchen were still intact.
We put the girls in the room, and the boys in the tents outside.
Mr.
Garfinkle and 
I stayed in a tent.
The children were wonderful; they obeyed us to the letter.
When 
we were settled, we held a meeting to decide on the schedule.
First the children worked out the outside cooking arrangements.
Then they discussed sports.
The Joint Distribution Committee had provided us with sports equipment.
The children also wanted 
to go on excursions, and to have Hebrew lessons.
The wonderful Mr.
Cohen of the Joint Distribution Committee sent a truck every day 
to see what we needed for the next day.
He wanted to make sure that the children gained 
some weight.
Every night we had a meeting, where there was complete freedom of speech.
-193- 
After the meetings we would dance and sing.
We spent fifteen days like this.
We were 
thankful to Mr.
Cohen for making this possible.
The children said "It's been many years since we had steak and french fries:" But they were happiest singing songs loudly in Hebrew.
On the last evening of our vacation, we held a banquet.
The children thanked Max 
and me for giving up our honeymoon to be with them.
I said "We have many people to 
thank, a whole regiment of dedicated men and women, who went to all the Jews to collect 
money.
Freddy Cohen did all that he could to provide us with good food and a good environment."
Most of the children had mothers.
But one girl, who had nobody, said to me, with 
tears in her eyes, "Don't go away!
Stay in Salonica and take me with you."
My heart 
was bleeding, but I was now married, and I had to go with my husband.
I said to her 
"There is a Ladino proverb: deque sivdad sos?
De la de tou marido."
(From what city 
do you come?
From the city of my husband.)
We returned to Salonica.
The Joint Distribution Committee sent me presents of a 
fur cape and a teddy bear.
I didn't want to accept them, but a lady friend, Alegra Cohen, gave me a very good idea.
"We'll hold a lottery with the cape as a prize.
We'll 
give the money we make to KKL (Jewish National Fund)."
We sold many tickets for the 
lottery, to Jews as well as to non-Jews.
One night, I invited everyone who had helped 
me stay alive, and we held the drawing.
Charles Beraha won the cape.
But Charles was in 
Athens.
Later, when I went to Athens, Charles' brother Elico came to get the 
cape, and I gave it to him.
When everyone had left the party, Alegra Cohen stayed to read me a letter that she 
had received from her ten-year-old son in Palestine.
She had sent him there a few months 
before.
"I'm worried," she said.
"Chernovitch assured us when he wanted to take our 
children that they would be sent to school.
Here is a letter, talking only about chickens and eggs."
I said "Your son never saw chickens before, and he's fascinated by them, so he 
writes to you about chickens.
He was hiding from the Germans for many years, and he 
didn't go to school.
He doesn't have a large enough vocabulary to write in more detail.”
She answered "Oh, no.
My son never missed a day of school.
He was three classes 
ahead of the other children his age.
In that dark room where we hid, I gave him lessons in French and Greek and other subjects.
The man who brought us food would also 
bring me the schedule of classes at school, and I followed it."
I said "I will go myself to see your son.
Give me the name of his kibbutz."
Max and I left Salonica by boat.
We went to Pirewus, and then to Athens.
I had a 
good rest on the boat.
Our group from UNRRA was in Athens.
They lived in a beautiful 
villa, and we went to see them.
I was introduced to a Jew from Turkey there.
The other 
nurse had returned from her visit with her relatives in Switzerland, and was now at the 
house.
We said goodbye to everyone, and a few days later, we left Greece from Pirewus 
to go to Palestine, on a Turkish boat.
When we arrived at Pirewus to get our boat, I was surprised to see the Turkish man I 
had met in Athens.
He was leaving the boat.
He tried to tell me something, but he changed his mind.
I asked him if he wanted to send a message to Palestine, but he said no.
The boat departed, and we soon arrived in Alexandria.
The boat was to stay there 
for six hours, loading and unloading cargo.
We had some Egyptian money, and I said to 
Max "The last time I was here, I saw a beautiful vase.
I would like to buy it."
"Let's go.
What are we waiting for," Max said.
As soon as we put our feet on the shore, a lady came up to me.
She was originally 
from Turkey and was now going to Palestine.
She spoke to me in Ladino.
The Turkish man 
must have told her that I spoke that language.
She said “I can talk to you, because you have a magen david on your uniform.
That
is a young man from Turkey on the boat.
He wants to go to Palestine, but he doesn't have 
a certificate to enter the country.
And he doesn't have a return visa for Turkey."
"Look," I said, "he should take a taxi and go to the Turkish consul in Alexandria to 
get a return visa.
If he can't enter Palestine, he will have to spend the rest of his life 
on that boat!"
The lady wasn't very happy with my answer.
Seeing her reaction, I asked "Does he have  
any money?"
She said no.
I said "Look.
Here is some money.
He must go to get a return visa.
It is very important."
She left with the young man to get the visa.
Max and I stayed in the cafe.
We had no more Egyptian money.
I said to him "Since 
-195- 
I have known myself, whenever I want to buy something for myself, something else happens to prevent it."
We started to talk about how this boy could possibly enter Palestine.
"It's impossible," I said, and my husband agreed.
At that time, there were 
many Aliah B. people who entered Palestine illegally, but never on such a big boat 
at Haifa.
A few hours later, a taxi came to the pier, and the young man got out with 
his return visa to Turkey.. 
We arrived at Haifa, and we had to pass through immigration.
The three immigration officers were English, Jewish, and Arab.
The Englishman was blond, so I didn't 
go talk to him.
I went to one of the dark ones, and I asked him if he was a Jew.
"No, I'm an Arab," he answered.
"But this man across from me is Jewish."
I went up to the Jew and told him that I wanted to talk to him.
I explained the 
situation of the young man without a certificate.
"Give me his passport," he said.
He sent someone to Haifa with it.
A half an 
hour later, he returned and said "Cell the young man."
The immigration man gave me 
the passport and I returned it to the boy.
I don't know what was marked on the passport, but the boy came off the boat like someone who had a certificate.
We arrived in the city of Haifa.
A car was waiting for us from Ein Hachofet 
Kibbutz.
We went to the kibbutz, and took the boy with us.
I 
had married Max with a condition, that I would try very hard to learn to live on the 
kibbutz.
A beautiful room was ready for us at the kibbutz.
The people there 
received us very well.
There was barbed wire all around the kibbutz.
This was my first disappointment.
I asked what it was for, and I was 
told it was for security reasons.
In my room, I placed my valises one on top of the other and covered 
them with a nice tablecloth to make it look like a dresser.
I looked at 
the windows, and I said to Max, "I have some nice material.
Tomorrow 
I'll make some curtains."
Max said "Oh, no.
This room is not yours.
We will stay here only for a few days, and then they will give us another one.
What are you worried about?
We are going to go to America for 
six months so that you can meet my family.
By the time we come back, 
a house will be built for us."
In the evening of the first day, phe 
people at the kibbutz had a beautiful party for us.
I stayed at the kibbutz for one month.
During that time, I chang 
ed rooms six times.
This didn't help me get used to kibbutz life.
But 
I don't think I have ever tried harder to do anything than to get used 
to kibbutz life.
The first day that I came to the kibbutz, I didn't work.
I 
went exploring.
I came to the swimming pool, and I saw the children.
My soul was filled with joy to see them.
They were swimming naked, 
and they were splashing water at each other.
Suddenly I saw a little 
boy, about three years old, approaching.
He came up to me and he whispered in my ear: "Tova, Tova.
Wait for me tonight in your room.
I 
will go to see you."
I whispered back "What is your name?"
In Hebrew, every word is either masculine or feminine.
In my surprise at 
what he had said to me, I had asked the question in the feminine.
He 
answered "I'm a boy!
I'm a boy!"
I thought that this must be the most 
beautiful scene in my life.
In the evening, I waited for him.
He came and introduced himself.
"I am Uri, the brother of Efraim."
Efraim was the son of a friend of 
Max.
This friend had died years before in an accident.
The mother remarried, and Uri was born later.
Max had been like a father to Efraim.
Now Uri wanted to compete; he wanted me to be his friend.
Uri and I 
became the best of friends.
Every evening, after supper, Uri waited for 
me outside the dining room.
I used to take him for walks and go to see 
his mother, and I used to put him to bed.
The second day at the kibbutz, I worked in the kitchen, making 
salad.
I cut so many tomatoes that in the evening I couldn't move my 
hand.
I didn't even go to supper; I went straight to my room.
I had 
Uri to thank for cheering me up.
My husband said "Don't worry.
You 
won't cut tomatoes here any more.
I will ask the dressmaker to take you 
in as a seamstress."
The next day, I was working with the dressmaker.
There were two 
women in the workshop.
One of them was a typical kibbutznik.
The other 
was a lady.
The kibbutznik was the dressmaker, and she gave me work 
to do.
I finished my work, but she ripped out everything that I had 
done.
I asked her if she looked at what I had done before she ripped 
it.
She said "Anyone who works as fast as you do can't be doing good 
work."
For fifteen days, she never looked at my work.
I sewed and she 
ripped.
I never said anything, but I was starting to get very nervous 
and angry inside.
I wanted badly to try kibbutz life, but this
woman wasn helping me at all.
One night, I came to my room and I said to Max, "I promised Alegra 
Cohen that I would go to see her son at the children's kibbutz near 
Haifa."
These children were sent to Palestine from Salonica under 
the auspices of the delegates, and through the propaganda of Chernovitch.. 
The mothers and guardians of these children had been terrified of the 
Germans even after the liberation.
The children were sent away with the 
idea that the families would be reunited once the parents could liquidate their belongings and go to Palestine.
Chernovich had assured them 
that the children would go to school, and that they would be very well 
looked after 
I went to Haifa with a big bag of candies for all the children 
in Alegre's son's group.
I took the bus and went to the kibbutz.
Before I arrived there, I saw a man and I asked him where the Greek 
children were staying.
He showed me a warehouse.
I went in.
It was 
very big.
There were rags hung up to form partitions between the children's beds.
It looked like a gypsy camp.
It was dark inside, and the 
walls were of grey cement.
The floors were cement, too.
It was very 
humid, and it smelled like a stable.
At first I thought I was having a 
hallucination.
I thought I was sick.
Suddenly, a little girl pulled 
her partition aside, and came out.
I recognized her right away.
She was 
twelve or thirteen years old.
"Hey, gang!
We have a visitor," she said.
She turned to me and 
said "You've come to see Cohen for sure.
It's the third visit that he's
had, and no-one else has had any visitors."
"No," I said, "I came to see everyone."
She didn't let me finish.
"In what elegant outside cafe did his mother give you his address?
Never mind, since you came to see everyone, we will make an exception 
for you.
We will show you our palace."
She was hysterical.
She started to show me the rags.
"This one was painted by Michael.
angelo.
This one is petit point.
And this one is metrito.
Chernovitch  
brainwashed our mothers and our guardians, and they sent us to this palace.
You know, we were supposed to go to school.
It was to be like a
boarding school.
But instead we have become experts at cleaning the lul 
(chicken coop).
This is the school that we attend."
I started to go near her.
"My little girl," I said.
"Don't baby me!
Do something for us instead!
Take us with you."
I said "I don't even have a room for myself.
I can't take you with.
me."
Everyone started to cry.
The girl came into my arms.
I said "I 
want to see the supervisor."
The girl answered "We are alone here.
There is a supervisor in the lul only, making sure that we don't break one egg 
and that we clean the place well.
These people look after their animals 
-199- 
better than they look after us!"
I left, promising that I would do something to help them.
The only 
thing Alegra's son said to me was “Write to my mother.
Tell her where 
she has sent me.”