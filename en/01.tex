% pages 35 to 54
We came to the tip of the island. The baby's troubles were 
just starting. As long as we were with the owner of the barn, there 
were many women to give him milk. Now there was noone. When we arrived 
at the village, the fisherman found us a family with whom we could pass 
the night. The lady of the house gave me a rag to use as a diaper. I 
changed the baby. She squeezed two tomatoes into a bottle with a nipple 
that I had bought in Athens. The baby drank the tomato juice. It was all 
I had to offer him. I think it was the first time that I cried since I 
had gone to meet Chaim for our wedding. The baby didn't stop crying, and 
neither did I. 
We went to the beach very early in the morning to get the boat. Suddenly, behind every rock on the mountain, we saw a machine gun. A man with 
a big beard approached us. He told us not to be afraid, but we were shaking. He introduced himself. He was one of the partisan leaders. He went to talk with George and Nicolas. He sat down on a rock and wrote a letter, and then he called the other partisans. He gave them the letter 
and he came with us to the boat. The other partisans left. 
He told me that he wanted to go to the Middle East; he had work to do 
there with the Greek soldiers. He wanted to form a new political party 
because he did not want the King to return to Greece. I said to him "You 
don't think there has been too much blood flowing in Greece? You don't 
think it's enough? Let him come who wants, as long as the German flag no 
longer waves in the Acropolis." He told me that I was ignorant. I saw 
that he was getting angry, and he was armed. I shut my mouth. This was 
not the time for discussion with him. 
It was very high tide. The boat came right alongside the shore 
and stopped. Suddenly Miki said "Behind this rock, there is a woman. She 
is praying like you (he meant in Ladino)". I went to see who it was, and 
I found a pregnant woman about to give birth. I took her and put her in 
the boat. All the men were worried about who was going to help her deliver her child. I told them that I would, and that I wasn’t scared at all. "If we leave her here, she and the baby will both die. If we take her, she won't," I said. Suddenly, from behind the rocks, fifty children appeared. They were the survivors from the boat that Miki had been 
on. We took as many as we could onto the boat. How many could we take? 
It wasn't the queen Elizabeth. 
I was very busy with the pregnant woman and with the baby. I had no 
time to pay attention to Miki. The boat sailed. The woman gave birth 
to a baby boy just as the fisherman was changing the flag on the boat from 
Greek to Turkish, but later we told the Turkish government that he was 
born under the Greek flag. We called the baby Moise. In Turkey, soldiers-
received us with great welcome. They didn't ask us for papers, passports, 
visas, anything. They took us to register, and we registered under our 
real names. They took the young men separately. Suddenly I realized that 
Miki was not with us. We had left Miki behind on the island with the other 
Children. They took us to a Turkish bath, which we really needed. But the only 
thing on my mind was how to get Miki from the island. In the bathhouse,  
they took our clothes for disinfection. I gave them all the baby's things 
except the rag the woman had given me that he was wearing as a diaper. 
The Turkish soldiers took the woman with the newborn baby to the hospital., 
I took down the number of the police car that took her. 
Re were supposed to declare all our valuables. I remembered that I 
had three gold lira sewn into my cape, and I went and declared them. Back 
in the bath, I started to remove the baby's diaper. The diaper was glued 
to the baby's skin. We couldn't take the diaper off unless we took a piece of the baby’s flesh.  
I asked for help from God, and Re helped me. I saw a tass in a corner. A tass is a copper bowl used by the women to scoop water from the bath with which to rinse themselves. I was sure that someone had forgotten it. I put some water in the tass and I sat the baby, 
with his diaper still on, in the water. The baby was screaming. I kept 
changing the water in the tass. After about fifteen minutes, the diaper 
came off. The baby's skin was so badly infected that I thought it was 
gangrene. I continued to change the water, trying to clean off all the 
poison. After about half an hour the baby's backside was only very red. 
I used the towels that they had given us as diapers for the baby. They 
gave us each an orange, and some orange juice for the baby. I washed 
his bottle, and I think it was the first time that the baby had had 
a clean bottle. We left the bathhouse and they took us to a large hall. 
There were jute mats on the floor for us to sleep on. But the baby never 
stopped crying. The infection was very painful. 
We went to eat at a restaurant, and on the way I saw a general store. 
From outside I could see that they were selling towels, so I went in and 
bought all the towels that they had. This way I could change the baby 
whenever he needed changing. When I returned to the hall where we 
were staying, I saw a peasant woman from one of the islands next to my 
mat. Her husband had escaped before, and she had arrived in Turkey a few 
days before I had. He had a little boy about two years old. I saw that 
she was still breastfeeding him. I was very surprised and asked her about 
it. She explained that there was a famine on her island, but she was very 
lucky to have a lot of milk with which to nourish her child. I asked her 
if she wanted to earn a gold lira. She opened her eyes wide and said that 
she had never seen one, and had only heard about them. But what would 
she have to do to earn one? I told her that she was to give the baby milk 
until we reached Syria. 
I gave her the money in advance. I went to change the baby. When 
she saw the baby's backside, "Oh!" was all she could say. "You need some 
pork fat!" I told her that I needed many things but didn't have them. 
She opened a rusty old box and took some pork fat with two fingers and put 
it on the baby's skin. This was machia for the baby. 
If the baby cried, she would immediately give him some milk. She 
kept washing his backside and putting more medicine on it. After a few 
times, the baby's skin was only a little red. I asked her if it was just 
pork fat that she used. She said no, that on her island people made their own medicines. This medicine for the baby was made from many herbs. The 
base was pork fat. It was her family's special medicine. They wouldn't 
give the secret away, it was passed on from mother to daughter. 
She assured me that the baby would be fine before we reached Syria. 
I was blessing this peasant woman, and she was blessing me, because it 
first time in her life that she had a gold lira. 

Sitting on the bus with the air conditioning water dripping on my 
backside, I started to scratch and thought about how much that baby had 
suffered, and about how much I had suffered. When the baby cried, I would 
curse my mazel to be alive. The bus stopped at a restaurant before we 
reached Jericho. I had a flight bag with me that El Al had given us be-
fore our departure, and in this bag I had some panties. I was all wet anc' 
uncomfortable, and with this company, all I needed was to scratch my backside in public! I ran to the bathroom. Mrs. X, whose husband was so 
much against me, was waiting at the door of the bathroom. I told her "I 
want to go in, I won't use the toilet and take your place, I will just be 
in the passage. I want to change my underpants; I'm all wet from the air 
conditioning." At first she said no. I said "I can get burned, I don't 
know what kind of chemicals they use in the air condition tank." She tole 
me to go ahead. After a moment I came out. I thanked her and said "You 
see, I didn't take your turn." She answered "I hear so many stories about you, that you take advantage of other people." I answered "There is only one person in this group who has known me for ten years in Montreal. She has told you many stories about me. I would like you to know 
the right one. You'll know it if one day I decide to write my memoirs." 
It was very very hot in Jericho. The only things I could think about 
was how I wished that someone would put me on a tass, like I had done with the baby. The itinerary said that we were to visit the ruins of Jericho, but the heat was so strong that we didn't  go. I was glad. We came to 
-39- 
the bus, and Mr. and Mrs. Bernstein, fellow travellers, as soon as 
they saw me said "Don't go to the back any more! You will sit down in 
front of us, and you won't listen to anyone else any more." I felt as if 
someone had put the soul back into my body.. God bless the Bernsteins! 
It was announced that we would go to Massada. On the way to Massada 
they took us to kibbutz Ste Boker, Ben Gurion's kibbutz. I was very tired, 
exhausted, and disgusted. I was praying that this trip would be over 
very soon, with no more fights. Ben Gurion talked to us, but I was so 
tired that I fell asleep. My husband told me after "You missed Ben Gurion, 
he spoke very nicely. I didn't wake you up because I know you haven't 
been able to sleep lately. Ben Gurion spoke about the Sephardi situ-
ation." I was very sorry that I missed his speech, because there was a 
question period afterwards, and there were things I would have liked to 
ask. 
V
They took us to a very beautiful hotel in the middle of the desert. 
The green of the trees made it very lovely and pleasant. I couldn't be-
lieve that all this was desert only a few years before. I danced and sang, 
and I remembered Chaim who told me not to cry. Mrs. X came up to me and 
asked if my husband was going to go with us to Massada. I said "You can 
tell whoever sent you that my husband doesn't need a wheelchair to go 
to Massada." 
The next morning we were off to Massada. They introduced us to new 
drivers, and told us that they were Temani.- The guide told us that the 
Temanim was the best element in Israel. When we arrived at Massada, my 
husband, as usual, was walking like a soldier. We were both among the 
first to arrive at the top of the mountain of Massada. When everyone else 
arrived, they surrounded me, as if to tell me that they were very sorry 
for what they had done to me. 
In Massada we saw many things. We saw how the people used to take 
baths. The market place was being repaired. We saw how they communicated 
using pigeons. We saw the cages where the pigeons used to roost. There 
were beautiful paintings. At that time they used paints made from the 
juice of fruits. The colours were baked on by the sun. We saw how they 
kept reserves of water. We saw the synagogue with the separate section 
for women. It was really worthwhile to see. 
I went to buy some postcards to send to my son Ely and to friends. 
Suddenly I saw a young man near me, and I heard his voice say "Oh, Maritsa". I was about to pay the salesman. The young man said "Oh, no, don't take her money, it's a present." 
"How do you know my name is Maritsa?" I asked. 
 "You have a nametag on that says Bouena Garfinkle. And you are Maritsa," answered the young man. It was one of the boys that I had taken to Turkey on the boat. We embraced and kissed. I was very pleased to see him all grown up, even if I wasn't sure exactly which one he was. I 
thought of all the people surrounding me on the mountain of Massada, and 
I started once more to remember. 
7 

We left the village of Chesme to go to Ismir. We arrived at Ismir at the station, and we were waiting for the train to go to Syria. I had the baby in my arms, and the other children were all around me, holding onto my clothing. I saw an old couple far away. 
"Oh", I said to the children, "this couple is Jewish". The children asked me how I knew. I 
knew them from Salonica but I didn't remember their name. Of course, everyone turned to look at this old couple. One of the children screamed "Nona”.
The Nona and the Nono turned their heads. It was their grandson. In half an hour, we found Nonas, Nonos, Aunts, Uncles, Fathers for all the children. But we found no relatives for the baby. We didn't even know his name.
The peasant woman was wonderful, washing the baby, changing him, feeding him, everything that a mother would do. When the train came, everyone 
was happy, and we left for Syria. When we arrived, they took us to the 
baths for disinfection and for a medical visit. Thank God, because the 
-41- 
baby had been changed often and been treated with the medicine, he was 
wonderful. They put us in a very big bungalow, clean and with comfortable 
beds. My eyes opened wide when I saw the beds. I had thought that never 
again would I sleep on a bed. We went to bed and fell asleep' right away. 
It was the first night that the baby slept right through. In the morning, when the baby and I woke up, everyone else was outside, except the peasant woman waiting to feed the baby. Even though we were in Syria, the woman never left the baby, changing and washing his diapers. 
I changed the baby. As I was putting in the pins, a woman came to 
stand beside me. I had never seen her before. She was staying in the same bungalow. She said to me "You must love your baby." I looked at her and said of course. She told me that she had a baby, but she had left him with her mother in Athens. She didn't need to escape from the Germans, but she was paid to help other people escape. "I make arrangements with the 
fishermen and I bring people out. If you have people in Athens, I will help them. Naturally, you have to pay." Instead of answering her, I started to cry. I was thinking of Miki who was in the fire with the other children. I was at the end of my strength. The woman told me not to cry, she would help me. I told her that I had no money. 
I told her the story of Miki and the baby, and that the baby wasn't mine. She started to question me. I told her that all I knew about the baby was that the mother had told me not to lose the booties. To me it made no sense. I had declared in Turkey that the baby was mine. I 
didn't know what else to do. I didn't even know the baby's name. The woman asked me to repeat the story of the booties, even if it didn't make sense. I explained that when I took the baby, the mother was still alive, and she begged me to save the baby and to keep the booties. I also knew 
that the baby was not circumcised. But when this baby was born, there were no rabbis to perform the ceremony. 
She asked me where the booties were. I showed them to her. They 
were lined, which was very unusual. She took the lining out. "OH!" she 
yelled, "there's a name in the lining!” I brought a man from Athens with 
the same name. The man was looking for his pregnant wife." I told her 
the story of the pregnant woman. She didn't let me finish. She ran out 
and ten minutes later she returned with a man who had a cast on his leg. 
I asked him if his wife's name was Caroline. "Oh, yes!" he said. "Your 
wife and Moise, your son, are in Turkey," I told him. I gave him the 
number of the police car. He said "You did things that only the Sarfatty 
would do." When he said Sarfatty, I opened my eyes. "I know your brother 
Eliaou very well," he continued, "even though you don't know who I am. I 
1 
used to serve on the same board as your brother Eliaou, the board of Karer 
Hayissod, and in the same zionist organization, too. I used to go to your 
home often. We held the meetings in your library. You had one of the 
most beautiful libraries in Salonica. It was well known for the masterpieces on its shelves. The baby you saved is the son of my brother." Someone came to tell us to make a declaration. They would give us refugee books. The uncle declared the baby as his own son. When we re-
turned to our bungalow, the woman, the man, and myself, the uncle made 
arrangements with the peasant woman . She would continue to give the baby milk. When we came back from registering, the uncle told me he would give me anything that I asked for. 

I told him I needed a boat and a passport to go to Turkey. I wanted to go to the fire to pick up Miki. The lady told us that she had room for two or three people in her boat from Athens, but it 
would take at least two months. To make a long story short, I decided 
to go get Miki and the other children myself. 
Our camp was in the middle of a cemetery. I was going to jump from 
one of the windows since no-one was to know that I was leaving the camp. 
Just as I was ready to jump, someone called me. There was a lady at the 
door who wanted to talk to me. Every second counted for me, but I had 
to see what this woman wanted. It was a middle aged woman. She said "I 
am Rachel Yanait. My husband is Ben Zvi from the Vad Aleoumi." I spoke 
to her for a short time. She was very sweet. "When you get to Jerusalem, come to see me. My house is open to you, and my school also. The name of the school is Havad Halimoud." I thanked her very much, and went back to my bungalow. I jumped from the window. 
An Arab woman was crying at a grave. It was a phony cry; she was my 
guide. She took me by taxi, truck and train, and I walked and walked until finally I got to the beach where a fisherman was waiting for me with his boat. 
When we descended from Massada, my husband and I were the first to 
reach the bottom. Midway on the road down, we found a tourist, not with 
our group, who had fainted. My husband ran to get the drivers. He got 
some water. My husband and the driver helped the man to the bus. 
We came back to the hotel, had a bath, and wept to sleep. When we 
woke up, we got dressed. I was wearing a beautiful print pants suit. 
I went down to the lobby, and someone complimented me. It was the first 
time that someone had said one kind word to me. I went to the dining room. 
Mr. Shofild was sitting at a table with my friend Phyllis Waxman. My hus-
band was right behind me. Mr. Shofild was telling Phyllis that my husband 
walked better than anyone else on the trip, like a soldier. I didn't want 
my husband to hear this conversation. I interrupted them saying "You 
know what?" My friend Phyllis answered me "We don't want to talk to you!” 
Mr. Shofild said "But I want to talk to you." I told him how I had received free postcards in Massada. The young man had recognized me by my nametag. 
Before I left them, I looked Phyllis in the eye, and I thought "Everybody was very pleased that my husband went up to Massada, except you. You are sorry he didn't die in Massada. You are sorry we didn't bring him back on a stretcher. You are a disgrace to humanity, and a shame to Jewish 
women." But Mr. Shofild had been very kind. His wife and her sister were on very good terms with me. 
The next day we went to a kibbutz on the edge of Lake Quinerette. I remembered that my mother was in all senses of the word a Lady. She was a Zionist down to her bones, from the family of Afssakadjis. Afssaka means 'rest' in Hebrew. Beginning on the tenth of the month of Tevet, my 
mother's grandfather would hold one week of tanid (fasting) and tifilatte (prayers). To occupy themselves during this week, the women made linen 
to give to each child when he or she got married. I have a piece of this 
linen, and from it I made a cover for the Hallah for Friday night. My mother 
was called "Golden Hands" because she embroidered so beautifully. Not 
many people could embroider like her. My mother used to say "Grandfather 
doesn't want to die before he bathes himself in Quinerette and sees Sfat." 
What my grandfather was not able to do, I would do. I would bathe myself in Quinerette. 
The next morning, I grabbed my bathing suit, ready to bathe in Quinerette. A few yards from the water, I met Thomas Hecht. Thomas Hecht came to Canada in 1942. He was born in Chekoslovakia. In 1941, during the Nazi occupation, his father, mother, sister and he escaped to France. They crossed the Pyrenees and went to Portugal (in Spain). They were 
there for ten months. They spent six weeks on the Atlantic Ocean on their way to Canada. They came to Montreal to start a new life. Thomas went to Sir George Williams University and to McGill to study Political Science. 
For nine years he was professor of Political Science at Sir George Williams. He speaks many languages. He is on the board of governors of Sokonouth. He is also National President of United Israel Appeal of Canada. I considered Thomas my guardian angel on this trip to Israel. In 
New york, the bus had waited for us thanks to Thomas. He is the most 
humanitarian person that I have ever met. 
Thomas now said to me "Mrs. Garfinkle, the water is very deep. Be 
careful." I told him not to worry. As I started to walk toward the water, 
another woman also came to bathe. Thomas called me; he was still worried 
about me. "Go with this lady." I never found out the name of this woman. I was the only one who had left her nametag on, and we had not been introduced. As soon as the two of us got to the water, I started to remember. 

When I went to get Miki from the fire, the fisherman was very nice 
to me. He put his life in jeopardy for money. He had been born in Jerusalem, of a Turkish father and a Greek mother. At first, I was very afraid of him. I was shaking, but I remembered Chaim. "Sing!" And I started to sing the song of Ani Maamin (I Believe). And the fisherman, since he was born in Jerusalem and knew Hebrew, started to sing with me. We 
came to Greek waters, and he changed the flag. Ten minutes later, we
saw the bodies of two children floating in the sea. The only thing either 
of us could say was "Oh, God, no!" There were a few minutes of silence
between us. The fisherman than said "I am sincerely sorry for these two 
children. I'm sure you are wondering why I'm going with you into this 
fire. All my life I have been very poor. I come from a family of twelve 
children. My older brother went to America the day he turned eighteen. 
I always clothe myself from the packages that he sends.. I am not married 
because I don't want to do what my mother and father did. They brought 
us into the world without being able to take care of us." 
We came to where the children were. Miki yelled "It's Malitsa!” 10 But 
my mother always said "There are no roses without thorns." The tide was 
very low, and the boat could not go to shore. I jumped into the sea. The 
fisherman gave me some rope. I tied one end of it to my wrist, and the 
other end was attached to the boat. I came to where the children were 
waiting. Suddenly I saw two bandits with revolvers in their hands. They 
used to steal goods from the German warehouses and then sell them on the 
black market. Of course the Germans were looking for them. I said to 
them "If you want to eat, we have sandwiches, coffee, milk." "No, we don't want coffee, no food. We want the boat." They pulled the boat to shore with the rope I had brought with me. Since it was low tide, the boat couldn't cast off again. They tied the fisherman to me and held the children 
at gunpoint. They then left to get their families from a neighbouring 
village. 
After they left, I looked around me, but I could not see Miki. I 
asked the children where he was. They answered "When you were approaching, the bandits were busy looking at your boat. Miki and the priest escaped." "Who is this priest?" I asked. "Since you left, we have had the company of the priest. He teaches us how to kill birds, and they are 
good to eat." Four or five minutes after, Miki appeared with the priest. 
They untied us, and there were a few moments of panic. 
"We have to push the boat," said the priest. 
"There's too much mud, it's impossible," said the fisherman. 
I answered "The port of Tel Aviv was opened with songs. Jewish Greeks
from Salonica, they make history. We are Jewish Greeks from Salonica, and 
we are going to make history right here. We're going to push, and I will 
give directions. When I say 'elia mo ya lessa', you will answer 'elia mo 
ya lessa'. When I say 'berden', you'll answer 'yossa'. And when you say 
'yossa' you're going to push with all your might, and I will too." And 
I started. 
 
'17 
"Elia elia mo ya lessa". 
They answered "Elia mo ya lessa." 
I sang "Dali berden". 
They answered "Yossa". 
"Mirande" 
"Yossa" 
"De cachcantica" 
"Yossa" 
I ran out of words. I said anything that came to my head, and they 
answered "yossa". We pushed the boat, singing. I said to Miki "If you 
don't come with us now, I'm not going to come back to pick you up." Miki 
was the first one in the boat. 
As we were pushing the boat, I remembered the German decrees. From the first week that the Germans came, all the Jewish people who had worked in the 
port of Salonica were without jobs. All the Jewish people had to do forced 
labour. They used to send them to regions where malaria was a great danger. Many young people died in forced labour. The Germans asked the Jewish people for a large quantity of gold. If the Jewish community could raise it, the young people would be brought back to Salonica. The community taxed all the Jewish people, rich and middle class. They all paid. 
All the survivors of the forced labour camp were brought back, but they 
were very ill. 
I was pushing the boat and remembering this tragic event. I lost my 
shoes. I was soaking wet. Everybody was wet, but especially me. My stockings were torn and I was freezing. 
The priest came with us. He was merely dressed like a priest; he was 
really a medical doctor, an escapee from Haidary. Just as the last 
child was getting into the boat, the bandits returned with their wives and children. They started shooting at us, and they injured one of the boys. 
They shot him in the arm. The boat started to go full speed, and we were singing, despite the bullets. We sang the slave song: 
Elada doxasmeni 
Patrida aderfomeni 
Den tolmissa 
Pote na scavothis 
When we ended the song, Miki told us that the bandits had thrown two chil-
dren into the sea. The fisherman said "I lost 2000 Palestinian liras." 
At that time, a lira was worth $5.00. "For every child I bring across, I 
am paid 1000 liras. And for bringing Maritsa back in good health (And he 
looked at me as if I was made of gold) I am getting 5000 liras. You," he 
said to the priest "are on the house." 
The priest asked the fisherman if he had any ouzo. He gave some to the wounded boy to get him drunk. There was a first aid kit in the boat. 
The priest removed the bullet from the boy's arm. The boy screamed and 
screamed. I put my hand over his mouth. We didn't want to attract the 
attention of the German patrol. Miki was drunk just from the smell of the 
ouzo, and someone had to grab him so he wouldn't fall overboard. There 
4 
was a thermos of coffee aboard, and we gave it all to Miki. I gave him a 
sandwich, and I told him not to go near the patient. The fisherman and 
the priest drank the rest of the ouzo. We had milk and soup in thermoses. 
I gave some to everybody. We had a blanket for each child. We arrived 
in Turkey singing: 
Ou mi anaknou Israel 
Ou mi coulanou Israel 
Ou mi atem Israel 
Ba mochavoth Israel 
Be kibboutsim Israel 
Be Tel Aviv Israel 
Be Chesme Israel 
The lady who swam with me in Quifierette said to me "Come on, come on!". 
I answered "I can't swim anymore." I remembered something, and I came  out of the water. I apologize now to that wonderful lady for not giving her any explanation. 
We went to the dining room for breakfast, and then we went to Ein Gev 
kibbutz. Every kibbutz in Israel has its own industry. This helps the 
country's economy greatly. The specialty of Ein Gev is the fish industry. People come from all over the world to eat the fish of Ein Gev. 
There is a beautiful gift shop there, and I bought a few ashtrays to bring 
back to Montreal as souvenirs. The ashtrays are in the shapes of fish. 
The designer of the ashtrays must be a very good artist, because the ashtrays 
have all the minute details of the fish. 
As usual, since we had been on the bus for so long, there was a line in 
the ladies' room. A lady near me said to me "You have to suffer very much 
on this trip". I asked her why. "Because we were told that it is very embarrassing for you if people sit at the same table with you." I answered 
"The person who told you this doesn't want anyone to sit with us. My husband shakes, but he eats better than our chairman." 
We went to the restaurant of the kibbutz. As usual, we sat alone. The 
fish we ate was tsipoura. We ate. My husband was shaking, but he ate better than some of the other people. When we finished eating, I showed 
his plate to Phyllis Waxman. I showed her my hands and my husband's hands. It was the first time that I talked to her. "You eat the fish 
with your hands. Look at both of us. We don't eat with our hands." 
Two minutes later, someone called me. "Maritsa:" It was one of the children from the boat. I asked him how he knew it was Maritsa. Again, it was from the name tag that said Bouena Garfinkle. We stayed together for five minutes, but then my bus was leaving. I sat down in the bus, and I started to remember. What courage I had had to talk to Phyllis the way I did! Oh, God, help me to have patience and not to fight!
I remembered that when we arrived in Turkey, Miki said that he wanted 
to eat fish. The fisherman gave us a treat; he took us to a restaurant. 
The children were making a lot of noise because they were very happy, es-
pecially since the fisherman had told them they could order anything they 
wanted. They just had to ask me about the quantity. I was afraid that 
they would eat too much and then have diarrhea. The priest took the responsibility for their health. 
Everybody ordered steak, except Miki. Miki ordered tsipoura, the same 
fish I ate at Ein Gev. He ate the fish like a man. My Nona used to say 
"You can know where people come from when they sit down at the table." 
The bus stopped and we were at the Quinerette Hotel. We went to our 
room to wash up, and I washed a few of my underclothes. We got dressed, 
and I put on my name tag as usual. We went to the salon to sit down. All of 
a sudden a woman said to me "You sat down near us! Can't you see that we 
are talking?" I got up. I really wanted to cry, and I started to go up to 
my room. On the stairs, I saw a lady from our group. She called me. I 
asked her what she wanted, and she said "Why do you always wear your nametag? My name is Shoshone, and everyone knows you are Mrs. Garfinkle. 
Phyllis is right! Take that name tag off your breast!" She grabbed the pin 
and she tore my dress and my brassiere. As she was yanking the pin off, 
the sharp end of it scratched me. She threw the name tag onto the floor. I 
just gave her a dirty look, but I didn't say anything. 
I picked hp the nametag. I said to myself "If it wasn't for my hus-
band, I wouldn't stay with people like you for a minute. But I don't want 
him to worry." I went up to my room. I didn't have another brassiere to 
change into. The others were still wet. I fixed the torn brassiere and 
changed my dress. I fixed my nametag and put it back on. 
4 
When I was dressed again, I started to cry. But right away I remembered 
Chaim. And right away I felt as if someone was choking me. And I started 
to sing the song of the prisoner: 
-51- 
Calselero i piadozo 
Anci el dio to de las vidas 
Que me quites de estas cadenas 
I que me afloches de oun coyare 
He he he he he he he 
He he he he he he he 
I heard my husband calling me to go outside and get some fresh air. 
I didn't sleep all night. I couldn't believe that I didn't spit in the 
face of this Shoshona. Again, I didn't want my husband to lose confidence 
in himself. Suddenly I remembered: "Oh, God, it's not the first time that 
somebody scratched me:" 
Max used to have a children's wear factory. The manager left and Max 
 asked me to help him. I worked very hard. Two days before Christmas, Max 
said to It me "Be a sport, prepare yourself, lets go to Mexico to see my 
sister. She has invited us so often." I said "Two days before Christmas 
we won't be able to get a reservation.' "Try:" he told me. Ely had some 
holidays, and for the three of us the trip would be wonderful. I called 
1 for reservations, and luckily I could get them. The next day we were ready to go. Ely was jumping up and down with joy. 
In the morning we got up and saw that it was snowing. It was one of 
the worst blizzards I had seen. It would be impossible to find a taxi. 
Max said "Don't worry, the factory truck will take us to Dorval airport." 
I prepared our valises, and I opened my box of stockings. It was empty. 
"Oh," I said, "Ely, I have no stockings." I called the drug store and asked 
 if they had stockings. They did, but they wouldn't deliver just stockings. 
So I ordered a few things I didn't really need. The stockings came, three 
pairs in one package. I wore one pair and packed the other two. We sent 
a telegram to my sister-in-law, and we caught our plane and went to Mexico. 
We landed at the Mexico airport, but my sister-in-law wasn't 
there. We took a taxi and went to her house. Nobody was home. We went to 
to our hotel. We called again the next day, but still nobody was home. 
We called some friends and they told us that my sister-in-law and her amily 
had gone to Acapulco. They made reservations for us at the hotel, and we 
went to Acapulco. Oh, Ely was so happy, and we were also. The sea! Ely 
was like a little fish, happy in the water. In the evening we would go to 
see a water-skiing show. It was paradise. My sister-in-law and her hus-
band were very happy to see us. But all good things must end. 
We returned to Mexico City, to my sister-in-law's house. In Acapulco, I hadn't worn stockings, I had worn socks instead. The day after we returned to Mexico City, I wore the same stockings I had worn to go to Dorval, All ad of a sudden, my sister-in-law said to me "Those stockings are mine!". I opened my valise and showed her the other two pairs. 
Without any hesitation, she started to take the stockings off my legs. I 
looked at her as if she were crazy. I said "I think you must be joking!" 
She said "No, I'm not. I was never as serious in my life." As she was 
taking the stockings off, she scratched my leg with her long nails. "Oh, 
Pauline," I said, "you're scratching me!" She answered "As long as you don’t 
agree to an operation for my brother, I'm going to scratch you." Ely asked 
me why I had let her take the stockings off my legs. I answered "We are 
guests here. The only thing we can do after this is to leave." The next 
day we left, and I have never again invited Pauline to my house. 
But you, Madame Chairman, you are only heading the campaign, you don't 
own the community. We're not going to leave. We're going to stay, even 
if it costs me my life! I want to see how far you will go! And I remembered 
how much Ely had suffered in school, and this caused me much pain. 
One day, Ely came in from school, crying. They had called him names. 
Every day, it was the same. One day I spoke to the director, and to 
Ely's teacher, but nothing came of it. I went to see Dr. Stillman, a wellknown urologist in Montreal, and a very good friend. I told him what was 
going on, and he answered "You will go to Mr. Sternthal. He is one of 
the founders of the school." I had wanted to take Ely out of the school, 
but my husband didn't agree. Dr. Stillman told me that my husband was 
right. "This lady can do the same thing anywhere, terrify everyone because your husband's hands are shaking." He called Mr. Sternthal and made an appointment for me. 
I told Mr. Sternthal What was going on in the school. “Since his teacher can't do anything about it, I came to complain to you." After I had explained the situation, Mr. Sternthal said 
"You keep saying 'she, she, she'. Who is 'she'?" I refused to tell him. 
Mr. Sternthal said "I'm not going to permit this in my school. But I want 
you to know one thing. You are protecting a monster, and monsters grow 
bigger." God bless you, Mr. Sternthal, the monster has grown much bigger. 
I started to remember that in 1966, Phyllis Waxman was associate chairman of the Combined Jewish Appeal. I was serving coffee at Sadie Neamtan's 
home. Sadie Neamtan was a friend of mine who served with me on the board 
of Adath Israel Synagogue. We enjoyed talking to each other. She was a 
very determined person. She was born in Montreal and went to school at 
Strathcona Academy. Her father was a doctor of medicine. She is a very 
active member of Technion, involved in cancer research for twenty years, and 
an ex-president of her group. 
The phone rang at Sadie's. It was Pyllis Waxman. She had become
associate chairman of the Combined Jewish Appeal and she wanted 
Sadie to be a district chairman. Sadie accepted the job, and she said "Put Tova (me) 
Down for area chairman, too." But Phyllis refused. The only thing I could hear was Sadie saying very determinedly "Put Tova down! Put Tova down!" I was made area chairman, and I did so well that I was the champion. I finished my district and I took on another one. I used to say 
to myself "If she knows what kind of people we are, maybe she'll leave our boy alone." 
The only incident we had in 1966 was at a meeting at the home of Mrs. 
Kattan. The Kattan's are one of the most respected members of the Sepharady community in Montreal. Mr. Kattan was born in Iraq. He studied in England. He is president of Royal Trust. We ended our meeting and left. 
In the street, Phyllis Waxman asked me "These people have not been in Canada a long time, and they have such a beautiful home. How come? And 
they have such beautiful furniture. Yet Mrs. Kattan speaks English with 
an accent." I said "Is it only the people who speak good English who are 
allowed to have beautiful homes?" "Heh" she answered, "I was born in Mon-
treal and I don't have a house like this." I said "The Kattans didn't make 
their money in Canada. They came with it. But your people came 
without a penny." "How does she know to buy furniture like that?" Phyllis 
asked. I answered "Mrs. Kattan was born in a palace. If all of a sudden 
you have money, you won't know how to buy nice things. But Mrs. Kattan 
has always been surrounded by very good things." She was disappointed 
with my answer.
