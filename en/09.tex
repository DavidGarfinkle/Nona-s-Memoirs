% mempages 166 to 187
% pdfpages 180 to 201

I was born a Greek, and I could stay in Greece without his permission. I said 
to him instead "I would like to be released. I have finished my year." I said good-
bye and I left. 
I returned to Salonica. The driver of the excursion bus for the tuberculosis patients gave me 
the address of the girl who had had my jacket, and the names of the people living there. 
The owner was a rich farmer. Instead of going straight to the farm, I went to the Agricultural Bank; all the farmers had to deal with this organization. I went to see an 
official who had been friends with Eliaou. I asked if he knew the farmer. The official 
answered "Eliaou used to work with him." I told him the story of my jacket. He advised 
me to go and see the farmer, who was a very nice person. "If you have any difficulty, 
I will go with you myself. Give him my regards, and he will give you everything that 
belongs to you." 
The next day, the driver and Max Garfinkle and I went to see this family. They 
lived in a house that had originally been built for rich Turkish farmers. When the Turks 
left Salonica, there was an exchange of Greeks from Turkey and Turks from Greece. Greeks 
returning from Turkey had taken over this house. The gates of the estate were open. 
We entered, onto a very huge lawn. The house itself was built at the end of the lawn, on 
a mountainside. It was a traditional Musulman house. The people in the house could see 
out, but the people outside couldn't see in. Max Garfinkle and the driver stayed outside 
I entered alone. The dogs began to bark and growl at me. 
A lady came out. She said "We don't know any Jewish people, and we don't want to 
know them. If you don't leave immediately, I will untie the dogs." She approached the 
dogs. I was getting scared, but I didn't budge. She said 
"Get out!" again. I started 
to move forward. She became very angry. She said "If the Germans didn't make you into 
soap like they did with the rest of your family, I will do it with these dogs." 
I said "I am Jewish, and very proud of it. I'm going to marry a Jew; I'll have children who will 
will marry Jews, and they will have Jewish children. And I'm not frightened of 
you, or of your dogs, or of your actions, or of your threat to make me into soap. We 
Jewish people have very tough skin; we don't get frightened of people like you. You 
thought Hitler would kill all the Jews. But look! Here I am! And the Jewish soul will 
live. You thought the walls fell down, but we Jewish people know how to build." She 
was untying the dogs as I was talking. As she was about to release them, her husband
arrived. I recognized him immediately; he had been a good friend of Eliaou. He ran over 
to stop his wife from releasing the dogs. 
"Are you crazy?" he asked her. "There's a truck outside full of soldiers." He 
turned to me. "Do me the honour of coming into my house." In the entranceway, near 
the window, there was a cushion on the floor, as one finds in many Greek lam homes. The 
cushion was made in metrito. In Greek, metrito means 'count', or 'measure'. I was ten 
years old when I had made that design. When I finished the cushion, I had to line it. My 
mother looked for material, and she found some left over from the lining of drapes. I 
put the finished pillow on the floor in my room. My mother said to me "Now you can marry 
an intellectual man." I asked "What does my cushion have to do with an intellectual man?" 
She told me the legend of Socrates. When Socrates was young, he wanted to marry, but he 
wanted to marry an intellectual woman. At that time, a woman who knew how to count, and 
who had enough patience to make a cushion of this design, was already considered an in-
tellectual. 
The husband invited me into the living room, and said to his wife "Serve the guest!" 
I said "Whoever made this cushion, health in her hands (a Greek expression)?" The wife 
said "My mother." She went to bring something to drink, and her mother came into the 
room. I said "Health in your hands, you made this cushion." She said "Oh, no. I haven't 
got the patience, or the eyes, and I can't count that well. A Jewish family gave it to 
my daughter as a gift.”
There were two ashtrays, a small table, and a lamp, all beautifully hand-carved wood, 
in the living room. Regina and I had made these works of art. There were two bonbonières filled with candies on the table, one made of silver and the other of early china. 
Sara Trabou had given me the silver one as an engagement present. These was a dragon on 
top of the china bonbonière. Chain's aunt had given me this piece as an engagement present. I got up, I removed the candies and put them on another table, and I took the two 
banbonières. Suddenly the wife came back, and grabbed me by the hair when she saw what 
I was doing. Her husband tried to remove her hands from my hair, but he wound up pulling it some more. He finally bit her hands. I said to him "An official from the Agricultural Bank sent me here. If you don't give me all that belongs to me, he will come here himself." Just then I noticed a sofa with two seats and four chairs, from my own living room. The husband said to me "you 
can tell your friend that we will give you everything that is yours today. Call the 
two men that you have outside to help bring all of your things downstairs. When 
you moved to Mitropoleos St. from the big house without a shelter, your brother put the 
things that wouldn't fit in the smaller house in storage here with me. I will give you 
everything." I called the driver and Mr. Garfinkle. The husband gave me all the carved wood, and 
the cushion, and the four chairs, two armchairs and sofa. He went upstairs, but he re-
fused to let me go. Every Jewish family had a forcel, a large container for the daughter's trousseau. This container was made of special wood, it was lined on the inside, 
and covered on the outside with camelskin. The two forcels that the men brought down 
were filled with my trousseau. They brought down the buffet. They also brought down a 
very large bundle; this was my wool. 
There was enough wool to make four mattresses. These were to be 
part of the bride's trousseau. The wool was already bought by the time 
a girl was fifteen years old. Since Eliaou worked with farmers, he was 
able to buy the best fleece for me. I started to remember the traditional washing of the wool. It was washed in the girl's back yard. 
Friends and relatives arrived at four or five o’clock in the morning. Water was boiled in a large container on 
and boiled on an open fire. They sang traditional songs of the bride.

The Groom’s invitation to the Bride 
Ventanas altas 
Tienes teu con vellas amarias 
Esta nothe rago al dio 
que me souvas ariva 
Tirilaila op Tirilaila opa 
Tirilaila op Tirilaila opa 
En mi sala te combedi 
Non te baganeates 
Tiego sala i camareta 
I ventanas para la vouerta 
Tirilaila op Tirilaila opa 
Tirilaila op Tirilaila opa 
El anio que yevas tou 
El diamante ese mio 
El amigo que te lo dio 
Es primo ermano mio 
Tirilaila op Tirilaila opa 
Tirilaila op Tirilaila opa 
The Bride's Acceptance 
A tanto fouetes i venites 
I en mis tierras me trouchites 
Non me emporta por padre i madre 
I oun castio de donzeas 
Tivila (Mikava) 
Ime alevanti oun lounes 
I oun lounes por la magnana 
Tomi mi arco i mi fletha 
I con la mi mano derecha 
Ande que la fouera a tagnera 
En pouertas de mi namorado 
Coming Out of the Mikva 
Y'a salio de la mar la galana 
Y'a salio de la mar la galana 
Con oun vistido bien blanco 
Y'a salio de la mar 
Antre la mar i el rio 
Mos Krissio oun arvole 
De bembrio 
Y'a salio de la mar 
Antre la mar i larena 
Mos Krissios oun arvole 
De almendra 
Y'a salio de la mar 
By noon, the washing was finished, and the wool was put to dry in 
the sun. The sun of Salonica dried very well, but it still took until 
the end of the summer to dry the wool completely. At noon, big tables 
were set up, and everyone would eat and sing. This beautiful tradition 
was remembered by all who attended for the rest of their lives. 
I brought everything that the farmer had given me back to my house: 
I put kit it all in one room and closed the door, thinking of waiting for 
a few days until I was less depressed to open the forcel. I was very depressed and I said to myself "I have to change. I have no-one to cheer, 
me up, and this room is full of the souvenirs of my life." Wherever I 
turned in the house that day, I remembered the English soldier, Tia Donna, 
Chaim, Eliaou, Regina, Reyna the housekeeper. And there I was, all alone. 
I went out for a walk. I said "I have to cheer myself up. The only 
way is to go to the beauty parlor and get my hair done." I went to the 
beauty parlor that I used to go to before the war. It was full of strangers. I asked to see the boss. A stranger came to see me. I asked him
for the man who usually did my hair. He said "I'm the boss now." I said 
"I want to have my nails done." The manicurist had no customer at the 
moment. He replied "Go across the street. We don't give manicures to 
Jewish people here." 
I left immediately and went across the street. As soon as I entered
I saw one of the workers from the other salon. When he saw me, he opened his arms: "Miss Sarfatty! Come here!" I told him what had happened 
in the other salon. 
"This man killed my boss," he said. "The janitor saw this 
-171- 
man come in with two Germans, and they took my boss away. No one heard 
of him after that. The janitor told some people what he had seen, and 
he was found dead the next day. After the Germans' defeat, my boss' mother went to reclaim the store. But this man showed her legal papers to 
the effect that he had bought the shop." 
Not only did this man do my nails and hair, he even took me out for 
a beer. He saw that I was very depressed, and he cheered me up a bit. 
I went home and I said to myself "I must unpack my things. It's no use 
to leave them any longer." I opened the first forcel and I started to 
cry. I took courage and started to sing the traditional song of our 
land, 'Mouestra Tierra': 
Mouestra tierra non ese venssude 
I demoyevo renassera 
Nouestra nation non es piedrida 
I sou brio renasseta 
Pouerpo i alma ofriremos 
A .este combate 
Libertad mos ganaremos 
Al paez esgate 
Oh venito lavorozo 
Lavorar con fouerssa i roso 
Biva la ora esperada 
De Israel nouestra tierra amada 
But I was still depressed. I started to sing "Sion" 
Sion Tierra amada 
Mouy triste i dessolada 
Couanto tou yoras i esclamas 
Yoras i esclamas 
I este enemigo emplacavle di Israel 
Destrouyo mouestro paez eternel 
Estonsses reloumbrera 
Al monte Maria 
La menora emflamada 
Sembol de libertad 
I found two rolls of material in the first forcel. It was made of 
bilidi. This was for the covers of the mattresses. There were many 
pieces of material of natural silk, for dresses for my trousseau. There 
was an eiderdown. There was material for coats. As soon as I saw the 
material for the lining, I started to cry again. It was a natural silk 
(illegible sentence)
the forcel. Even the buttons, and all different colours of thread, even 
the trimmings for the dresses were there. I remembered how Chaim and I 
had looked through the catalogues to decide on the styles of the dreasse 
I started to figure out what to sell and what to keep. I sold the 
wool, the bilidi, and a few pieces of silk. I sent the money to my sis-
ter who was in France with her two children. Her husband had not come 
back from the concentration camp. 
The other forcel was full of linen, twenty-four pieces of everything.
I sent all this, and everything else I got from the farmer, to my sister 
in Palestine. I asked her to keep everything for me. But my sister is 
very generous. By the time I went back to Palestine, there were very few 
things left for me. She had given everything away to relatives. When 
I asked for the two bombonières to keep as souvenirs, she said that she 
had given the silver one to my sister in France. She said nothing about 
the other one. I said to her "If my mother were alive, she would say 
'lo que dedo del ladron se lo yivo el endevino.'" This means 'What we 
saved from the crooks, we gave to the fortune teller." 
My niece had made herself a housecoat from the lining of my coat, 
the material that Chaim wanted me to make a housecoat of. 
In 1937, my sister Daisy from Palestine and my sister Marie from 
Marseilles and their children came to visit us in Salonica. When they 
were about to leave, Eliaou wanted to buy them a going-away present. 
Marie asked for a carpet. Daisy wanted to show off. She asked for a 
large toy automobile that her daughter, aged four or five, could sit in 
and drive. Eliaou took them to Athens. When he came back, he said "I 
gave the two of them presents, and I have presents for you and Regina." 
He would never give a present to one of us without giving something to 
the others. He said to me "Mama has your present." It was a beautiful 
bracelet. He asked Regina what she wanted. She said "At Mallak (a jeweller) I saw a very nice watch. I would like that watch." 
I started to think now "Where did that bracelet go? To whom did we
-173- 
give it to hide for us?" I couldn't stay with the forcels and their con-
tents any longer. I went out. As I was walking, I saw the director of 
the bank with which Eliaou had dealt. He said to me "Did you know that 
the Germans didn't open the safety deposit boxes? You have one at my 
bank. If you want, I'll arrange to have a key given to you." I said 
"We have the box, but it's empty. We removed everything from it the day 
the war started. Eliaou gave me a bracelet in 1937," I continued, "He 
gave it to someone later for safekeeping, but I can't remember whom." 
The director said "Have you gone to see Mr. So-and-so? Eliaou's 
friend? He's looking for you. Go now, go!" He took me right to this 
man's door. 
When I entered, the secretary was very excited to see me. He rang 
his boss and said "You have a very important visitor." The boss came out 
right away, and he started to kiss me. "Come into my office! Come in, 
come in! We have to have a drink." He opened a bottle of ouzo, and he 
wrote a few words on a piece of paper. He said "There is a bank across 
the street. You know the director." I said "I was speaking to him just 
now." "You will go there and give him this note, and he will give you 
a bracelet. I gave it to him to keep for you." 
As soon as I went in to the director's office, he said "I don't need 
the note. I know you came for your bracelet. Here it is." I was very 
impressed with the behaviour of these two people, especially with the experiences that I had had with my other possessions. 
At about this time, I started to visit orphanages for infants. I 
knew that there were Jewish children there. The mothers had told me 
where they had left their children. But I had no success. I went to 
private homes where I knew there were Jewish adopted children, but the 
people denied it. There was nothing I could do. 
One day I went to an orphanage. As soon as I entered, I saw an old 
classmate of mine, from my dance class. She was very glad to see me. 
She said "I work three times a week here as a volunteer.” 
I said “There are Jewish children here, and I would like to take them." She said that 
she knew only of a four year old boy. I begged her to show me the boy. 
When I saw him, I was frightened. All I could see was a very large head. 
The rest of him was skin and bone. I begged her to help me take him. 
I went to the Joint Distribution Committee. I told them the story 
of this boy. They said "If they give him to you, take him." I went to 
the orphanage every day for a few days to get to know the child. He was 
afraid of people; he had known no one but the nurse for a long time. Finally I took him home to my house. I had him for three weeks. I fed him every two hours, little by little. I toilet trained him. He started to
call me Mama. It was the first time the child had been kissed, and I 
taught him to kiss me back, too. 
After three weeks, The Joint Distribution Committee asked me for th 
child. I said again "I want to take the child to the orphanage in Athens by
myself." Again the nurse said "They have their own people for 
this job." I was told to bring the child to the boat the next morning. 
Two ladies, whom I knew, were going to take him from Salonica to Athens. 
I brought the child to the boat and took him to the ladies' cabin. The 
child was scared. I was the first person that he knew, except for the 
nurse at the orphanage. He started to cry. The two ladies tried to make 
friends with him. They didn't understand that they couldn't become 
friends in five minutes with a child who had never seen people. The 
boat's whistle sounded. The two ladies said "Let him cry, and go." The 
child was screaming and crying "Mama! Mama!" Even today, I can hear the 
screams in my ears. When I left the boat, not only the child was crying 
in the cabin. I was crying in the street as well, and saying to myself 
"Everything in life is politics, even if the child cries so." 
A few days later, I was walking in the street and I met Mr.
Rousso. When I saw new faces, new friends in the street, I felt like I 
was walking on the moon and met my parents. Mr. Rousso's brother 
was married to Marie Basso. Marie and Pepo were old friends of mine. 

-175- 
When we were small, we played hopskotch together. When we were older, 
we went dancing together, with other friends, and Chaim. Sometimes 
Eliaou came with us. 
"Give me news of the Bassos!" I said. 
"Everyone was taken by the Germans. Marie, also. But my brother 
and the baby are alive. Talking about babies, I want to discuss something very important with you. Let's sit down in the cafe." 
We sat down in the cafe, he ordered something for me, and he started to 
talk. "I had a cousin who was pregnant. When the Germans took the 
Jews from her ghetto, she was just starting labour. Hwr husband took 
her to the hospital in the ghetto zone. He left his wife and started 
back to the house. There were two Jewish Police near the entrance of 
the hospital. They arrested him, and we never saw him again. Two minutes after my cousin gave birth, the Germans took her. We never heard 
of her again. Her baby was a little girl. A nurse from the hospital 
hid her, and told the Germans that she had died at birth. The nurse 
smuggled the baby out of the hospital and took her home. The nurse's 
mother looked after this baby like she was a piece of gold. The only 
person who knows the address of these people is Mrs. Riades." 
Mrs. Riades was a volunteer nurse for the Greek Red Cross. She was 
a very good friend of Georgette Modiano, Daniel's wife. Her husband was 
the one who checked the milk that I distributed for the Red Cross. He 
would check the temperature and the quantity. Mr. and Mrs. Riddes were 
both very humanitarian and philanthropic. 
"I want this child, and I want you to help me take her from these 
people," Mr. Rousso said to me. 
I went to see Mrs. Riades. She was very glad to see me. I asked 
about her husband. She told me that he had died. I spoke to her about 
the case of the little girl, and told her that I wanted to see the 
child. The next day, Mrs. Riades took me to see her. 
The house was in the district of Tomba. This district was built 
when the exchange of Greeks and Turks took place. We went in, and found 
the mother with her two daughters. Their home was very modest. The 
little girl was there; she was very sweet, and obviously well looked after. I think that these people didn't eat enough so that they would be 
able to feed her. I told them I was a relative of the child. During the time I was there, the child was in the mother's lap. She didn't move
from there. When I had come in, the child had said 'Yassou' ('Health'). 
-177- 
When I left, she said ‘Yassou’ again. In the doorway, I asked the nurse 
if I could come again the next day to see the child. These wonderful 
people didn't say no, even if they didn't want me to come back. 
I returned to Salonica and I reported what I had seen to Mr. Rousso. 
I said "I don't think these people will give the child up. They risked 
their lives to save her. They don't eat enough so that she can eat well. 
And the child is very happy with them. But I'll go there again tomorrow 
and we'll see what happens." 
The next day, before my visit, I went to the Joint Distribution 
Committee. I asked them for some toys. In Salonica at that time, toys 
were very expensive, but the Joint had plenty, and no children to give 
them to. They gave me a teddy bear, and a dress for the child as well. 
I brought my gifts to the child, and she became a little friendlier towards me. 
I visited the child every day for fifteen days. One day, I brought 
some lollipops, and I gave them to the nurse's mother to give to the 
child. The mother refused the lollipops. The child wasn't there. She 
said "All of us risked our lives to save this little girl. If the Germans had known what we did, they would have killed us one by one. If you 
have some ideas about taking this child away from us, it won't be very 
easy." 
I said "I come often because I want the little girl to know me. I 
want to take her." 
"Don't come again unless you have a court order," she said. 
"I would like to see her one more time," I said. Five minutes Later 
the child was back at the house. She was crying "Don't take me away!" 
I went back and explained the situation to Mr. Rousso. He wanted to 
know what to do. "The only thing to do is to go to the Joint Distribu-
tion Committee," I said. "The director, Mr. Cohen, is a very well known 
lawyer. Ask him to give you a court order. If you get this, it will be 
a pleasure for me to go and get this little girl.” Mr. Rousso asked
if I would talk to the Joint Committee. I refused. I said "You must 
go see Mr. Cohen yourself." Mr. Rousso was very disappointed, and I was 
as well. 
I left him and went to have lunch. In the entrance of the hotel, 
saw Tia Donna's son Samuel, who had been a prisoner of war in Italy. I 
told him that I had his mother's jewellery, but it was in Palestine. 
"Your mother wanted me to give you this jewellery after you were married 
and had your first child." I didn't give the jewellery to Samuel until 
he came to North America to settle. He came with his wife and little 
girl; I followed Tia Donna's instructions exactly. 
Hasson was to be tried in Salonica. I attended his trial, but I 
got dizzy and fainted, and therefore could not testify against him. There 
were too many people, and it was too hot. Seeing Hasson, I remembered my 
whole past: the milk, the beating, Chaim, the officer, the train, the 
houses that I hid in, too many to remember, in unfamiliar districts. But 
Daniel Modiano, who never tired of me, and Mr. Neri, the brother-in-law 
of the Italian consul, would visit me wherever I was, and give me courage. I was dizzy, but Mr. Neri was near me. 
As I was thinking of all these events, I turned my head and saw the 
two Amario brothers, the Jewish Police of the train, among the spectators. I was getting dizzier and dizzier. I tried to go out into the 
hall to get some fresh air, but on the way I fainted. All I remember 
is cold water being thrown on me by two men. 
The grand rabbi Koretz was lucky. He had been sent to the privileged camp in Germany, in recognition of what he had done for the Nazis. 
His wife and children returned to Salonica, but rabbi Koretz had died 
of typhoid, and therefore was not humiliated by a trial. 
Hasson had attempted to escape to Albania with other people in his 
entourage. The Italian consul in Salonica gave him a car to help him escape, in exchange for a group of Jews in Baron de Hirsch who were about to be deported. The Italian consul said that this group were relatives of Italian people. You can see that the Italians did their 
utmost to rescue Jews from the claws of the Nazis. 
Hasson went to Albania to seek refuge. When the so-called Italians 
left Baron de Hirsch, the Italians sent them to Athens. They were on the 
same train that Daniel had put me on. 
Hasson's escort was made up of his wife, his mistress, and his son. 
When they arrived at Coritza, on the border of Albania, the Italians immediately arrested them and placed them in a concentration camp in Albania. They were liberated on the eighth of September, 1943. With their 
jewellery and money, they went to the port of Bari. From there, they 
went to Egypt on a fishing boat. They circulated very openly in Alexandria. One day, a Jewish refugee from Salonica recognized Hasson. The English police arrested him. When Greece was liberated, he was sent to Athens, to Haidari. But no-one pursued him to court, and he was liberated. He never imagined that there would be Jews in Salonica, liberated 
from the concentration camps. He returned to Salonica in November of 
1945. A group of survivors of Auschwitz recognized him. He was immediately sent to prison at Pavlo Mella by the Greek authorities. 
Albala and Tapouz were also arrested. The denunciation was made by 
the Jewish community. The two had come to Greek territory from the Yugoslavian frontier. 
The trial of the collaborators who had committed treason toward 
their brothers in Salonica opened on July 2, 1946, at ten o'clock in the 
morning. They were first accused of helping the Germans deport the Jews 
of Salonica to Poland. They were accused also of acts of violence, complicity, carrying weapons, collaboration with the Germans against the 
Hellenic Jews so as to rob them of their fortunes, bad treatment of many 
non-Jewish Greeks who helped Jews, the brutalization of men and women; 
in short, they were accused of the same barbarism as the Germans. Hasson had the most serious charges brought against him. He was like a lion let out of his cage. His power had been equal to that of the chief of
the camp of Baron de Hirsch, Amster. Hasson was the iron arm of Gerbin 
the S.S. director of Baron de Hirsch. They shared everything that they 
took from the Jews. 
Tapouz had Herculean strength. He just had to tap a person's jaw 
with his hand to break it. He was always near Hasson. Hasson would 
give the signal, and Tapouz would do the work. There were many non-Jew-ish witnesses and all the horrible details were accumulated. Hasson, Tapouz and company had beautiful limouzines, and would go to all the suburbs of Salonica to find Jews in hiding. They would beat them to death 
and take all their jewellery and wealth. These are the atrocities that 
this monster of the century and his accomplices committed. 
Peppo Carasso provoked a big sensation when he was on the witness 
stand. He told of how seven members of his family were in hiding in a 
suburb of Salonica. Hasson eliminated all of them in a particularly savage manner, and then took all their possessions. 
Hasson tried to maintain his innocence by blaming others. After 
five days of hearings, his sentenced was announced at three o'clock in 
the morning in front of a large audience. Hasson was condemned to die.
 Amster and Boudrian received the same sentence. Leon Sion, or Tapouz, 
was sentenced to life imprisonment. Albala was sentenced to 15 years, 
Counio to eight years in prison. Boudrian was a non-Jew. He disappeared during the German retreat. Papanaoum, loaded down with treasures, 
also disappeared at this time. Amster was killed by the Germans by midtake. He was escorting hostages to Haidari, and when they arrived, the 
Germans killed all of them, including Amster. 
Hasson was transferred to the prison of Corfu. On Thursday, March 
4, 1948, he had confession with a rabbi of Corfu in the morning. He 
maintained his innocence. He had been the chief of Baron de Hirsch and 
he had done only what the Germans had told him to do. He told his wife 
that he felt guilty, because he had had a mistress, and a child by this 
mistress. He asked that this child, a little girl, not be persecuted, 
because she was innocent. He asked to be buried in a Jewish cemetery. 
Then the traitor paid his debt to society. 
I started to try to trace Daniel. But no-one knew what had become 
of him. One day I decided to go to see Alfonse Levy. He would know 
about Daniel for sure. I finished my work, went out to have dinner, 
and then started off to Alfonse. As I was walking, I remembered that 
it was siesta, and in Salonica one did not go visiting at this time. I 
came to Tsimiski and St. Sofie Streets. On the corner there was a pastry 
shop called Diefnes. I grew up in this pastry shop; it was just oppo-
site my family home. I sat down and stared at my old balcony. Mr. Gar-
finkle passed by. I asked him where he was going. 
"To the Joint Distribution Committee. And you, what are you doing 
here?" 
"Killing time," I answered. 
"If you want to kill time, come with me," he said. I went along. 
The doctor, the nurse, and an American soldier were there when we 
arrived. The American soldier congratulated Mr. Garfinkle. "You are 
going to work for the Joint Distribution Committee. You will reorganize 
the Soupe Populaire." He turned to me with a dirty look, and he said 
"But not you!" 
I answered "I don't know who or what you are, and I don't care. I 
didn't ask the Joint Distribution Committee for work. Unlike the doctor 
and the nurse, I don't need the camouflage that the Joint would give me 
to stay in Greece. I can stay as long as I want, all my life if I want." 
I left them and went back to the Diefnes, and sat down to kill time again. 
I was sitting opposite my old balcony. I began to remember how I 
grew up on that balcony, where I passed the best years of my life. I 
used to sit out on the balcony with my embroidery. Eliaou would come 
to sing for me: 
En tou balcon kreseran las rosas 
Descoje kouala ese 
La mas ermoza 
In your balcony flowers will grow. 
Pick the very best
If Regina was out on the balcony, he would sing for her: 
Regina mou Regina mos 
Mazi sou fa pefano 
Ola tou kosmos ta kala 
Brostamou de ta vazo 
(My Regina, my Regina 
With you I will die 
The best things in the world 
Are nothing compared to you) 
I had just finished thinking about this when the nurse appeared. 
(The Joint was just two doors away from the pastry shop.) 
"Oh, hello," she said to me. "How did you like the treatment we 
gave you today? Give up the coat and everything will be fine.”
"Do you see this balcony?" I asked. "It's the biggest balcony on 
St. Sofie Street. Just opposite to us is the biggest balcony in Salonica. There was a family living in the house behind that balcony. In this
family there was a law of humanitarianism. This family could sing and 
dance, and they had the most beautiful library that an intellectual person can dream of. They were philanthropes, zionists, religious. They 
were respected by all the people who knew them. This was my family. I 
lost all of this. But be sure that I am not going to lose my nice winter coat. Now go for a walk, because you are polluting my favourite corner in Salonica." I turned away from her, and she left. 
I saw the owner of the pastry shop coming toward my table. He was 
holding an ice cream cone. He said "Do you remember the ice cream 
cones?" 
"Of course," I said. "You sold ice cream cones to no-one else but 
Regina and me." 
"Of course," he said, "because Eliaou used to bring the cones for me 
to give to you. This ice cream cone I give to you now was brought to me 
by Eliaou years ago. I kept it in the hope that you would come back." 
-183- 
We both cried a little. We couldn't say anything to each other. 
The siesta was finally over. The owner of Diefnes asked me where I 
was going. I explained that I was going to see Alfonse to get news of a 
friend who had helped me greatly in a time of need. 
"I'll walk with you," he said. 
Alfonse, his wife, and his daughter were very happy to see me. Alfonse said "I saw your courage when you buried the English soldier in 
this community." 
I asked him if he had news of Daniel Modiano. He said "Don't get 
excited. We Jews from Salonica must be prepared for everything. In Sep-
tember, 1943, the Mosseris, the Fernandes and the Torres families, and 
Daniel Modiano were in Italy. They were assassinated by the Gestapo in 
a hotel on Lake Majeur. Their bodies were thrown into the lake." 
These had been the most distinguished families of Salonica. I felt 
that I was choking. I went outside for some fresh air. I started to 
walk very slowly, and I realized that I was talking to myself. At the 
corner of the Church of St. Sofie there was a man selling chestnuts from 
a cart. I bought some to chew, so that I wouldn't talk to myself. 
The man gave me the chestnuts in a wad of paper because they were hot. 
The papers came from my mother's hagada of Pesach. I asked him to sell 
me all his papers. He refused. Without papers, he couldn't sell chest-
nuts. This is what they did with all the precious books that belonged to 
Jewish families. 
I started to walk, without knowing where I was going. I grew up in 
that neighbourhood, but I knew no-one who walked by me. I saw the building where Alegra Saltiel used to live. I remembered being at the circumcision of Alegra's son. For the Sepharadim, the birth of a boy was 
occasion for great happiness. They were fanatics for boy children. If 
a girl was born, the father was very angry at his wife. It took him a 
long time to get used to having a daughter. Take the case of my father's 
father. The midwife said 'Mazel toll' to my Nono (grandfather). 
They used to say 'mazel toy' for a girl, and 'be siman tov' for a boy. 
When my grandfather heard 'mazel tov', he went to work without congra-
tulating his wife. Five minutes after he arrived at work, someone came to see him and said "Sarfatty, you have two mazel tovs!" Five minutes 
later, someone else came and said "Hey, Sarfatty, you have three mazel 
tovs!" He said to the man "My God! I should go home before I have four 
mazel tovs!" By the time he got home, the fourth mazel toy had arrived 
Only one of the four babies survived. Her name was Miriam, and she became the favourite of my Nono. I remembered when Miriam married. The 
Nano loved her very much. She had one daughter, Gilda. When Gilda was 
taken to Auschwitz, she was called by name to be operated on in a Nazi 
experiment. But she survived, and is now living in Israel. 
I remembered the circumcision of Alegra's son. The rabbi was also
a surgeon. Everything was prepared. The godfather sat down in a chair 
with two cushions, called coultoukes, in his lap. The godmother same 
with the baby. This was a great honour. There was an empty chair in 
the room. This chair was for Eliaou Anavi (the saint of circumcision).
The rabbi took the baby, and he said "This is the chair of Eliaou Anavi" 
He put the baby on the two cushions. When the circumcision was finished, the rabbi said 'be siman tov!" Automatically the music started and
the people began to sing the traditional song of circumcision; 
Tio Ovadia Seror 
Vino de Estambol 
Ande Han Ichoua 
Ayi apozo 
(Uncle Ovadia Seror 
He came from Istanbul 
To the house of Han Ichoua 
And there he stayed). 

I remembered what was done if the baby was a girl. The rabbi came 
after one week. He was called the fadar, and he would give the girl her 
name, and bless her. It was a beautiful ceremony. Even if the father  
was disappointed to have a daughter, the people would dance and sing and 
have a good time just the same. 
At the circumcision of Alegra's son, there was a bandleader 
named Tsadik. He was blind, but he was the best composer that Saloni-
ca knew. Of course, Leon Botton, one of the best comic entertainers, 
was there. The band was composed of violins, a hout, a tambourine, and 
a dulcimer. Tsadik was the best dulcimer player in Salonica. This com-
bination of instruments was called tsalguin. Leon Botton sang a humour-
ous song: 

Por non mancar vo explicar 
Lo que el hombre por bevir deve Bouchear 
Por non sofri 
deve offrir 
Siempre devertimientos 
Para non soffric 
Couando oun dia de ventura 
Se presenta 
Cale dar bouelta 
I non penssar 
Couando la pounta tsica 
0 algouna vouerta 
Por devertirsse el tiempo deve passar 
Bayla canta i reyr 
De alegria non soffrir 
Vino likor i reyr 
De alegria non mouerir 
Gozar siempre la mansseves 
I el dia boueno-que vech 
Sin hezitar i profitar 
La saloud bien mirar 
Escarssedad ese bovedad 
El que las vouadra 
Es solo i sin penssar 
Se va mouerir i sin bevir 
Siempre devertimientos para bien bevir 
Dechan los bienes tambien los conaques altos 
Calle yir fartos sin dezear 
El que hereda se bourla i etha salto 
Se imajina que aqui va aquedar 
He then dang another song, 'The Very Modern Salonica': 
Oh Moderno Salonique 
Deantes vistian djoubas 
i se rancavan al peynar 
Los hombres soultoueas con djoubas 
I se vistian sin penar 
vazas con mossandaia 
-1 tSb-
Salonique ya se troco 
D4 couando se tsamoucho 
Ma de lodo el non manco 
Anssi vemos todo trocado 
Oh moderno Salonique 
Vemos los autos corres 
Mos salpican sen quierer 
Polvorina en el barer 
De couando se tsamoueheo 
Ma de lousso 
El non manco 
Anssi vemos todo trocado 
Oh moderno Salonique 
Mos quijemos sivilidar 
En tomando todo areves 
El lousso i la moda bezer 
Es el progresso de la mansseves 
Tenemos cazas ethas con biton 
Mi con gantes fiongos i baston 
Las nignas con sous veloutes 
Exssitan con sous decoltes 
Anssi vemos todo trocado 
Oh moderno Salonique 

I walked on. Suddenly, I heard someone calling me 'Bouenica!’. 
"Oh," I thought, "my mother used to call me that!" I turned around, 
and I saw one of our good neighbours who lived near us on ST. Sofie 
Street before the war. She was the wife of Dr. Pouliades. She made the 
sign of the cross and said "Oh, Bouenica, you were walking like a 
ghost!" She took me up to her house and made me some coffee. Believe 
me, I needed it. 
"We have things for you here," she said. "I want to give them to
you. Eliaou gave them to us to keep." She gave me many more things 
from my trousseau. She gave me a few pieces of silver and a ring. I 
recognized the ring right away; Chaim's Nona had given it to me. Mrs.
Pouliades gave me a ceramic plate with Hebrew letters on it, and I remembered where this plate came from. 
My mother used to tell us the legend of this plate. My maternal 
grandmother's family name was Seror. One Yom Kippour, one of her 
ancestors, Ovadia Seror, was helping the hazan to sing. This was a big 
honour, to know Hebrew and to have a good voice, to be able to help the 
hazan. Everyone was praying and he was singing, when the Turkish police entered the synagogue. The police took Ovadia Seror away. Everyone was very frightened. The police gave no explanation. There is a 
beautiful bay in Salonica. Near La Tour Blanche on this bay, there 
was a huge warship. They took Ovadia Seror onto this ship, and sent 
him to Istanbul, to the King’s palace. They told him that he must sing 
for the King. He sang the religious songs of Yom Kippour, and when he 
was finished, the King gave him the plate. He was brought back to Salonica, to the house of Han Ichoua. Han Ichoua was the big rabbi of 
the Kiynla (congregation) of the Serors. All the people of Salonica 
looked to Ovadia like a saint. They thought that the King had wanted 
to kill him. Everyone would say "Oh, God! He is newly born!" This 
is why they sing the song of Seror at circumcisions. 
With the plate in my hand, I thought "I, too, feel newly born, after all I went through. Now I need courage to start a new life." I 
thanked them very much for returning my things. I hadn't even known 
that they had them. I took a handmade tablecloth from my trousseau and 
offered it to Mrs. Pouliades, but she refused to accept it. They saw 
that I was confused and depressed, and the husband took me home. 
