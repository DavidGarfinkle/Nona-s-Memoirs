% mempages 255 to end
% pdf pages 282 to end

ALLIED JEWISH COMMUNITY SERVICES ® UNITED ISRAEL APPEAL 
493 SHERBROOKE STREET WEST 
Dear Friends, 

25th July 1969.
In just a few days, you will be leaving 
on a most important and deeply gratifying Mission to 
Israel.
Since you will be representing the entire 
Montreal Jewish Community, we will look forward to 
your return for a full report on your memorable visit.
The effect of your participation in the 
forthcoming campaign will be significantly strengthened 
as a result of the Mission.
Be assured that you leave 
with our sincere good wishes for a most rewarding 
experience and successful journey.
General Chairman.
Sincerely

THOMAS 0.
HECHT 
Associate Chairman.
BORIS G. INE 
President, 
Allied Jewish Community 

to Nicola.
They gave him very bad diarrhea.
I asked the nurse to 
give him something else to eat.
She gave him some bread and butter, 
and I asked her where the owner's of the home were.
She told me that 
they were away on a trip.
I came home.
It was too late to bring Nicola some decent food, i 
so I started to open the mail.
Among the letters., there was one 
from the Combined Jewish Appeal.
They wished us a good trip, and 
said that they expected a full report of our trip.
The letter was 
signed by three wonderful people, Morley Cohen, Boris Levine, and 
Thomas Hecht.
When I finished reading the letter, I said to myself 
"My dear, wonderful people, if I gave you a report of my trip, your 
hair would stand on end.
If you had no hair, it would grow!
But 
Thomas Hecht has a pretty good idea of what was going on, since he 
was with us."
Now, I'm writing something; call it a report, or call it a book.
My dear three fellows, now that I am writing, I can confess to you 
that I have two things to ask of God.
And God always answers my 
prayers.
I want, first, to see Ely married and living in a traditional Jewish home.
And I want you three wonderful people to read my 
report.
And I would like Mr.
Bronfman to read it, too.
The Bronfmans are the wealthiest Jews in Montreal.
They participate seriously in the Jewish community.
They are generous and philanthropic.
They aid McGill, University of Montreal, and other organizations, both Jewish and non-Jewish.
I met Mr.
Bronfman in 1966, 
1967, and 1968, during the campaigns of AJCS.
He played an active 
part in the campaigns, and to this date continues to do so.
I started to bring food again and clean clothing to Nicola again, 
every day.
Without the owners of the home present, Nicola was very 
neglected.
And I was afraid to complain to Baron de Hirsch, to start 
another battle.
One day, as I was leaving the home, a lady called 

Mr.
and Mrs.
Max Garfinkle, 
6612 Wilderton Avenue, 
Montreal, Quebec.
Dear Max and Bouena: 
CorribilLed jeWislicyl'ppeal 
493 SHERBROOKE STREET WEST, MONTREAL, QUE.
LS.
844-8621 
August 6, 1969 
When our people get to Israel, Israel gets to our 
people -- with the magnitude of her accomplishments, with the 
determination of her men and women and with her youth -- her 
handsome, vibrant, dedicated youth.
Our mission to Israel was an unforgetable experience.
Never have I been so moved, never have I been so proud and 
gratified and never was a group more congenial, more spirited 
or more co-operative.
Each of us helped to add a link to the chain that 
ties diaspora Jewry to our brethren in Israel.
Our task now 
is to strengthen that chain by relating our experiences in 
Israel to our friends, our neighbors and our families.
I thank you for participating in our mission.
I 
thank you for your generosity, for your patience and for your 
understanding -- and I thank you for affording me the opportunity 
to share a memorable experience with you.
Most cordiall 
THOMAS 0.
HECHT 
Associate Chairman 

my name.
"Are you Mrs.
Garfinkle?"
"Yes."
"Oh, I have to thank you so much for what you did for my mother!” 
"And who is your mother?"
"Oh, you must remember, you gave me the phone number of Nicola's 
nursing home.
Everything you said was true.
My mother is very well 
looked after."
"I just came from there, but I didn't see any new people."
"Oh," she said, "my mother is in the new nursing home that the 
owners of this home recently opened.
I'm on my way to see her now."
I asked her if she wanted a ride.
"Oh, yes."
I took her to a place in Snowdon.
I asked for the new home's 
phone number, and then I went home.
I first called Nicola's nursing 
home, where they had been telling me that the owners were away on 
a trip.
I told the lady that the nursing home in Cote St.
Luc was 
very neglected.
She promised me that in a few days a couple would 
come to take charge of the home, and everything would be all right.
I didn't have any alternative; I had to wait and see.
Three weeks later, the couple finally arrived, and the nursing 
home returned to its former routine.
But I kept an eye on things 
just the same.
The first visitor I had after my trip to Israel was Annette 
Wolff.
Annette is Rosetta Elkin's sister.
Annette has more protegés than hairs on her head, exactly like her father and mother, may 
they rest in peace.
We talked a bit about my trip.
I told Annette 
that I had a few documents describing the atrocities of Hitler.
I 
wanted to donate them to an institution where people in North Amer--

-256- 
could come and see them.
She advised me to give them to Bronfman House, but I decided to give them to Yad Ve Chem.
Annette 
arranged a meeting with the Israeli Consul, and we gave him the documents.
A few weeks later, I received a letter of thanks from Yad 
Ve Chem.
I have to thank Annette for the encouragement that she gave me 
to write my memoirs.
Even if it was painful for me to write about 
my past, I have her to thank for her support in this effort.
I soon got back into the old routine of sending a taxi for Nicola every Sunday and spending that day with him at my house.
One 
Sunday, I called as usual to ask them to prepare Nicola.
The lady 
in charge told me that he would not come that day because his leg 
was sore.
She assured me it was nothing serious.
I called the owners 
at the other home and asked them to send a doctor.
The next day, I went to see Nicola's leg for myself.
Oh, God!
I saw that his feet were all burnt.
It looked like some-
one had put his feet in boiling water.
I went to the Jewish General 
Hospital and spoke to the doctor.
I told the doctor to call Baron 
de Hirsch, because I had not been able to find out how Nicola's 
feet had been burnt.
Nicola was in the Jewish General for 10 days.
From there, they 
sent him to the Jewish Convalescent Hospital.
From there, he was 
sent to the Jewish Nursing Home.
Here, under the direction of Mrs.
Cabely, Nicola was very well looked after.
I knew Mrs.
Cabely when 
she was a nurse in the operating room at the Jewish General, and I 
knew how devoted she had been to her work.
She had been a nurse in 
Israel, in Bet Haolim Rambam.
Mrs.
Cabely has an M.A.
in adminis-
tration from McGill University.
She was the director of nursing at 
Maimonedes Hospital and Home for the Aged.
She is the administrator of the Jewish Nursing Home.
Nicola died some time later in the Jewish Nursing Home.
Naturally, I had the gravestone made within 30 days of 
his death, in the Sepharady tradition.
When I ordered the stone, 
my first condition with the monument makers was that they would obtain permission for the stone from Baron de Hirsch.
And they did.
My conscience was clear.
A few days after Nicola died, someone from the Jewish Convalescent Hospital called Max.
They were going to start to treat him with 
L-dopa in pill form, to control his tremors.
Max took the pills 
very well, and when I went to visit him, I could see that he was getting better day by day.
One day, as I arrived home from the hospital, my phone was ringing.
It was the monument maker.
He told me that he was going to remove the stone from Nicola's grave.
I wanted to know why.
"The money is in trust, and I gave orders to the trust company 
to pay you," I said.
"Oh, no, it's not that.
Baron de Hirsch gave me orders to remove the stone.
They want you to go there and sign a paper stating 
that Nicola was your cousin."
I said, "Nicola was not my cousin."
"Well," he said, "I must remove the stone."
I said "You go ahead and remove the stone.
But I will go to 
the Vad Hair."
"The cemetery is not mine, and I have to do what Baron de Hirsch 
wants," he said.
"You do what you have to do, and remove the stone, but I will 
not sign a paper saying that Nicola was my cousin.
He was not."
I hung up the phone, and I felt sick.
It was a terrible feeling 
to hear that a stone I had laid in respect for a dead person was 
going to be removed.
When I hung up the phone, I thought, "Oh, God, I've had 
enough.
To be accused of someone I'm not!
Oh, God, I've had enough!
To be accused of something I didn't do!
Removing the 
gravestone is like removing the epaulets of an innocent officer 
accused of treason.
Extxtitaxamiyxpercx Like Dreyfus, who was accusei 
of treason.
But the only person who believed that Dreyfus was innocent was Theodore Herzl."
Herzl followed the Dreyfus affair very 
closely."
After the Dreyfus affair was cleared up, Herzl gave new, 
life to the Zionist organization.
"Poor Dreyfus," I thought.
"I 
understand what you went through, because I am now in your shoes.
I
was born on November 15, 1918, a few days after the signing of the 
Armistice during the first World War.
Oh, God, which dying soldier's 
soul did You see fit to give me?
The soldier who was wounded in the 
mountains of Serbia, or maybe in the valleys of Albania, or in the 
hills of Monestir, You gave me his soul.
Oh, God, I can't be a 
soldier any more.
I saved Nicola from the Germans.
I looked after 
him for many years.
And now, they are blackmailing me.
But I will 
not forge any signatures.
Nicola was not my relative, distant at 
or near.
Oh, God, I need a Theodore Herzl.
Send me one."
And I 
remembered the song that we used to sing in the Zionist organization on the anniversary of Herzl's deaths 
Kon sou fegoura 
Da1ma poura 
Soubito mourio 
El grande ijo 
El noble padre 
Del pouevlo djidio 
Herzel ho nouivo profeta 
Tou venites liberar 
A Israel martiro 
I trouchites libertad 
Theodore Herzl's Song 
With his face 
And pure soul 
He died suddenly 
The gr'eat son 
And the noble father 
Of the Jewish people 
Herzl, o new prophet 
You came to liberate 
The martyrs of Israel 
And you brought liberty 
I didn't know of anyone whose advice I could ask.
I didn't 
know anyone in Vad Ayir.
Ely was in Israel.
Max was on L-Dopa 
treatment in the hospital.
I didn't want to upset him.
I went to 
see him twice a day.
That day, I had a good cry first, and 
thought to myself, "Oh, Chaim, I have to cry or I'll burst.
I can't 
sing any more."
I freshened up, washed my face, and prepared to go to the hospital for the second time that day, to see how my husband was reacting to the ladibillta pills.
As I was getting dressed, I said to 
myself "Oh, God, thank you for reminding me.
Baruch Pollack will be 
my Herzl.
He's the only one who will believe me."
Full of confidence, 
full of courage, I called Baruch Pollack.
I met Baruck Pollack at Camp Massad.
He was a young law stu-
dent at the time.
If there was an award given to camp counsellors, 
Barouch would take first prize.
He took his work very seriously.
He looked after the children who were assigned to him very well.
The 
children adored him and listened to 12 him.
Baruch was like God to 
them.
He worked at the camp for only one season.
The two seasons 
following, he went to work at another camp in Val Morin.
In 1954, 
Baruch passed his Bar examinations.
He was born in Montreal, and he studied at McGill University.
He speaks Hebrew, French, English and Yiddish very well.
He is a 
religious man and chomer shabat.
When Raphael Naphthali, his son, 
had his bar mitzvah, we received a very original invitation, and 
to my knowledge, the first invitation of its kind in Montreal.
On 
it was marked that instead of giving Ralph a present, the guests 
should donate the money to a school in Israel.
Baruch is philanthropic and humanitarian.. I will give you an 
example.
One day, we received a letter from him.
A certain child 
in Israel had been brought to his attention.
The child had been 
born without arms.
He asked us to give a donation to help bring this 
child to North America and provide him with artificial arms.
But 
he didn't just bring the child to North America.
He took the child 
into his own home, to spend a few days with his three children.
Ruth, 
his wife, is an angel.
I asked ruth "Don't you think it will be depressing for your children to see this poor boy?"
She looked me 
in the eye, and she said "It is a mitzvah."
Baruch and Ruth have a 
pen pal in Russia.
Both of them went to see first-hand the conditions in which their pen pal lived.
Baruch is a member of the 
Board of the Hebrew Academy.
This is the man whom Phyllis Waxman 
called a stinker because he had asked me to give a donation to the 
Academy.
He is very active in the Young Israel Synagogue.
He is the 
most honest lawyer that I know, and this is confirmed by other people 
He is a very good friend of my family, and we are proud to call him 
our friend.
I called him up and told him that Baron de Hirsch wanted to 
remove Nicola's gravestone.
"If I don't sign a paper saying that 
Nicola is my cousin, they will remove the stone."
"Is he your cousin?"
he asked.
"No."
"Then don't sign!"
said Barouch.
"And don't worry.
I will do 
all I can to prevent them from removing the stone.
I will call you 
this evening or tomorrow morning.
Don't worry.
Go visit Max.
and 
don't be upset."
He really gave me a boost.
I got dressed and left for the hospital.
I had a new car.
It 
had maybe two or three hundred miles on it.
As I was driving to the 
Jewish Convalescent Home in Chomedey, I had a flat tire, in the 
middle of nowhere.
I got out of the car to try to find someone to 
help me.
But no-one would stop.
I pulled the car over to the side 
of the road, and in my brand new white dress, I started to change the 
tire.
When I finally managed to get the tire off, it started to 
rain cats and dogs.
I couldn't get into my car to wait for the rain 
to stop.
So I changed my tire.
When I finished, I was soaking wet, 
and my white dress was covered with grease.
My face looked like the 
face of a circus clown, because I had wiped it with my dirty 4 hands.
Instead of continuing on to Chomedey, I made a U-torn and drove home.
As soon as I arrived, I called Max and told him that I would not be 
goIng to see him that night, because of the weather.
As soon as I hung up, the phone rang.
It was Baruch Pollack.
"I'm calling you now because I know you are very upset.
Don't 
worry.
The stone will not be touched."
"Oh, Baruch," I said, "God bless you.
I knew that God would 
send me a Theodore Herzl."
"He sent you two," he said.
"Thomas Hecht helped me greatly in 
this matter."
"Oh, God bless Baruch Pollack and Thomas Hecht:" 
Days went by, and Max was reacting very well to the pills that 
he was taking at the Jewish Convalescent Hospital.
Thank God, he was 
not shaking any more, and he does not shake to this day.
There is a 
God in heaven.
I believe it.
The Combined Jewish Appeal campaign started, and I was no excep-
tion; they called me for a donation, too.
I told the lady who 
called me, "I will give them nothing!"
She asked me why.
"I will send the money directly to Israel," I said.
"And here, 
I will give to the organization of my choice."
A few days after this conversation, I received a 
letter from Thomas Hecht.
He asked me for a donation.
This is 
what I had been waiting for.
Instead of sending the money, I made 
an appointment to see Thomas.
I told him a few things about my 
beautiful trip with the Mission to Israel.
And Thomas had a very 
good idea of what had gone on.
I gave him an anonymous donation 
for the Men's Division.
I told him, "As long as the Chairman is 
there, I will send my contributions directly to Israel."
When Thomas became the chairman of the Combined Jewish Appeal, 
sent a donation in his honour, again anonymously.
I had spoken to 
Thomas to arrange to pay my pledge little by little.
But instead I 
decided not to pay it at all.
As long as the Chairman was swelling 
with importance day by day, I would send money directly to Israel.
In 1970, I received another invitation from the community.
This 
time, it was for a seminar in Israel.
I like to attend seminars very 
much, but I refused the invitation.
I had enough during the trip of 
1969.
It took four years before I received another invitation from the 
community.
In 1974, I was invited to attend the community convention 
in Montreal.
I decided to go.
-263- 
The conference was held at Shaar Hoshamayim Synagogue.
At the 
door, they gave us a blue plastic envelope with the words "Ninth 
Annual Meeting and Community Conference of Allied Jewish Community 
Services" written in gold.
It contained the programme for the meet-
ing as well as other kinds of literature.
The theme of the conference was "The Jewish Family Under Attack".
The keynote speaker was 
Dr.
Lionel Tiger.
At the information desk, I asked for the name of 
a person in the USSR who wanted a pen-pal.
They gave me a name.
I 
later wrote to this person, but I never got an answer.
The convention was dominated by the theme of Jewish education.
The Master of Ceremonies began the proceedings.
He told us that 
he called all the ladies on the Board of the Women's Federation by 
their first names.
And he began reeling off names.
When he men-
tionned "Phyllis", I said to myself, "Mr.
Master of Ceremonies, you 
were supposed to be available at the time of the amalgamation of 
Adath Israel and Young Israel.
If you had been present, you would 
realize that there are people in your community who are in fact destroying Hebrew education."
The Master of Ceremonies announced that there were microphones 
set up in the isles, and if people in the audience wished to speak, 
they could do so.
In the meantime, I opened my blue envelope and I 
saw a pamphlet that listed all of the services available in the community.
People started to go up to the microphones.
I went, too, 
and waited for my turn.
I had the pamphlet in my hand.
Everyone 
was talking about Jewish education.
It was my turn to speak.
I held the pamphlet in my hand, and I spoke.
"I see in this 
pamphlet a list of all the wonderful things that the community is 
doing."
After this, I started to change the subject, and the Master 
of Ceremonies interrupted me many times.
But I persisted.
I said 
"Young man, you are going to listen to me whether you like it or not."
-264- 
And I started again.
"Jewish Convalescent Hospital and Maimonedes are the centers 
of research on Parkinson's Disease.
And they have done a wonderful job.
But how is it that people who destroy the children and the wife 
of men who have Parkinson's disease?
For example, take the Chairman 
of the 1969 Campaign.
How do these people have the face to go to the 
public to ask for money in the name of Maimonedes or the Jewish Convalescent Hospital, both of which are so good to people?"
The Master of Ceremonies Shouted orders: "Close the microphone!” 
All I could think of at the time was that in Russia, if you speak 
out against the Communist Party at a convention like this,
they send you to Siberia.
But the Jewish community has an easier way: 
"Close the microphone!"
I sat down in my chair; I felt like I did that 
time I waited for the boat to pick up Tamo, Nicola's and George's 
cousin.
I was paralysed.
If I could have walked, I would have left 
immediately.
A man came over to me.
I asked him who he was.
"I am a person who is worried about you," he said.
"What is your name?"
I asked.
"Mr.
Vineberg," he answered.
He asked me how the Chairman of 
1969 had mistreated me and caused others to fear my family.
"You just have to ask Thomas Hecht," I replied.
"He knows
everything, and I have officially complained to him.
I want the 
community to understand that we, the survivors of Hitler, are not going to permit Jewish Police in our community.
The community must 
understand that we, the survivors of Hitler, have very strong hides, 
because we emerged from the fire."
When the man left me, I thought, "Oh, God, this wonderful man, 
Mr.
Vineberg, was like the owner of the cafe who brought me ouzo when 
I was waiting for the boat."
The next day, my phone didn't stop ringing.
People I have never.
even met were phoning to congratulate me on what I had done, and to 
give me the news.
Phyllis had been replaced by the vice-president.
Everyone said "It's time that the community understood that we do 
not want Jewish Police here."
The case seemed to be closed, but Phyllis' friend, the rabbi, 
had not said the last word yet.
He formed a committee of big donors 
to put pressure on the community to give the post back to Phyllis, 
and to even promote her.
Some weeks passed, and I received the com-
munity bulletin.
The headline said: "Waxman Officer".
That same day, my doorbell rang.
When I opened the door, I saw 
four ladies standing there.
One of them held a piece of paper in her 
hand.
I invited them in and asked what I could do for them.
They 
showed me the bulletin and said that they wanted to draw up a petition.
"We will not allow people like this to be the leaders of our 
community.
We are going to draw up a petition," they said again.
"If you circulate the petition now," I said, "you will ruin the 
campaign."
"You, too, are impressed by the big donors," one of them said.
"I've seen many miseries in my life," I said, "and believe me, 
some of the people involved were millionaires.
As a matter of fact, 
take a look at my desk.
I'm arranging all my notes.
Believe me, I 
have enough notes to fill a book of 500 pages.
But I'm not going to 
use all of them.
I'll do like the composer Rubinstein.
When he 
gave his first concert, everyone congratulated him.
He said: 'I 
could give another concert with the notes I left out!'
Ladies, I can 
write another book with the notes I'm leaving out.
You know, ladies, 
do you remember Esther?
May she rest in peace."
One of the ladies said "Yes!
How did you meet Esther?"
"In the train coming from New York to Montreal.
I went to a 
wedding one weekend.
Daisy, my cousin Dora's daughter, got married.
Ely was still a baby, and we didn't want to disturb him for such 
a short trip.
I went alone, and Max stayed home with Ely.
On Mon-
day morning, I took the train back to Montreal.
In the train, near 
me, sat a lady.
She was knitting.
She introduced herself as Esther.
I told her my name.
I had 
a lot of food with me.
Dora had packed the leftovers from the wedding in a shopping bag.
I could have fed the whole train.
Of 
course, I shared my food with Esther.
She had been ready to 
go to the restaurant car.
After we had lunch, Esther took off her 
jacket.
I saw the concentration camp number tattooed on her arm.
She told me that she was one of the women that the Germans had operated on.
She had adopted a little boy, her brother-in-law's son.
He had
been put in an orphanage for the duration of the German occupation.
After the war, she waited to be repatriated in her homeland, with 
this little boy.
One day, an American lady came to her camp, to 
ask the people if they had relatives in North America, South America, 
or England.
'Even if you only have a name, and no address, we will 
find them for you,' she said.
Esther told the lady that she had a 
relative in Montreal, but she didn't know the address.
It was her 
grandmother's sister.
The American lady said to her, 'There is a 
man in another camp asking about this same person!'
Two days later, 
Esther was reunited with her husband."
"Esther was knitting a baby jacket.
I asked if anyone in 
her family was expecting a baby.
'Oh, no,' she said.
'I'm knitting 
for money.
My husband works, my boy is in school all day, 
and I have nothing to do.
So I knit.
The money I make I put in the 
bank in a special account.
I want to buy myself a mink coat.'
'Oh,' I said, 'It's going to take you ten years, at the rate 
you're going!'
'Even if it takes me twenty years,' she said, 'I'm going to 
buy that coat with my own money!'
And she added 'If I knew how to 
make booty and hat sets, I could make more money.'
I taught her how to make booties and hats on the train.
After we got back to Montreal, Esther and I became very close friends.
Esther was the most truthful friend I ever had.
Every Tuesday, 
after her son went off to school, she came to spend the day with me.
We would have a knitting party.
She would knit, and I would do the 
finishing on what she had knit during the week.
In 1967, Esther ordered her mink coat.
The day she put in the 
order, we went out to Moishe's Steak House to celebrate.
“The first person to wear the coat must be you,” she said.
The day that the 1967 Israeli war broke out, I was short of 
workers, and I asked Esther to come help me with the campaign.
A few days into the campaign, I called her again to help me.
There 
were a few cards of people who had lived in Outremont and then moved 
to Chomedey.
I asked Esther to come with me to Chomedey.
On the 
way, she said to me 'Don't be angry at what I did.
I lost the down-
payment on the coat.'
'How come?'
I asked.
She said 'I've seen many miseries in my life.
I saw many Jewish people, millionaires, looking through the garbage for potato skins.
Israel is in trouble.
It needs money.
How can I wear my 
Women's Federation of the Combined Jewish Appeal, and I don't regret 
it.'
'Esther!
Welcome to the club, as they say in Canada.
If you 
were in my home town, I would sing you a song.
I'll sing it anyway, 
because I'm like you.
I never wind up getting what I have planned
for!
It's called "Come, Come, We Must be United": 

Venir, venir mos adjountemos 
A nombre de mouestros ideal 
El I con coraje mos fravouemos 
I el estado National 
De sou pouevlo rebenido 
Insi azer sou nido 
Mozotros la djouventoud 
Lavoraremos por Eretz Israel 
Come, come, let's reunite 
In the name of our ideals 
Amd with courage, let's build 
The National State 
The people will be reborn 
And we will make a nest 
We, the youth, will work 
For Eretz Israel.'
Esther loved the song.
From that day, she was always proud of 
what she had done, and never regretted that she had no fur coat.
Years passed after this incident.
Esther came to visit me less 
and less often.
I knew she was sick; she had been in the hospital 
twice, but no-one said what was wrong with her.
One morning, after 
I sent Ely to school, I was preparing to spend the day with Esther.
The doorbell rang.
I opened the door and saw Esther and her husband.
Esther looked like a shadow.
She had a copy of The Montreal Star in 
her hand.
I invited them both in to have coffee, but her husband said that he had to go to the bank.
He would pick Esther up on his 
way back.
I walked him to the door, and he whispered in my ear 
'Please, cheer her up.
She hasn't got long to live.'
I sat down at the table and had coffee with Esther.
I said 
-269- 
'Esther, you have The Montreal Star in your hand.
Do you have 
your jewellery wrapped in it?'
Instead of answering me, she said 'Did you know that Phyllis 
Waxman has become the president of the Women's Federation?'
'I don't care if she became the president of the community,' I 
answered.
She said 'I do!
Look:' She wrapped my mink coat in this news-
paper!'
'What are you talking about?'
I asked.
She unfolded the newspaper and showed me an advertisement.
It said 'Adath Israel Sisterhood will hold a luncheon sponsored by 
the Women's Federation.'
'How is it permissible for the Women's Federation to sponsor a 
luncheon at a synagogue?'
Esther asked.
'And we supported them!
But who knows how many people like me pride themselves in the fact 
that they helped Israel?
Look what they do!
They sponsor luncheons!
I don't need a coat.
I'm dying of cancer.
But before I die, I 
will go to all my friends and ask them to help me draw up a petition.
I want the people who encouraged the Women's Federation to give this 
luncheon to be reprimanded.'
'Esther,' I said, 'if we do what you ask, we will ruin the 
campaign.'
She said 'You, too, are impressed by people who find it easy 
to reach into their pockets for their bank books and put many zeros 
after the numbers on their cheques: They can do it because their 
husbands have put enough money into their accounts.'
She was very 
agitated.
She started to vomit.
I was praying for her husband to 
return.
I gave her some towels, and I put a cold wet towel on her 
forehead.
I tried to calm her down.
'We'll call a meeting and talk to our friends about this,' I said.
She started to feel better.
Half an hour later, her husband came.
I was very glad to see him.
The next day, Esther's husband called me.
He had to go out, 
but he couldn't leave Esther alone.
I asked how she was.
He said 
'The doctor came and said it would be all right if I left her with 
you for an hour.'
When I went over, her husband told me that he 
had prepared a deck of cards so that we could play gin rummy.
Esther was sitting in bed, propped up by two pillows.
We played,
Esther won, and she was very happy.
She felt much better, and she 
asked me to get her some ginger ale.
I went to the fridge and got 
the bottle, but it was closed, and it took me some time to get it 
open.
I put some in a glass and headed back to the bedroom.
'Oh,' I called out, 'I'm sorry I took so long, Esther, but I 
couldn't open the bottle.'
Esther didn't answer.
I heard a key turn 
in the front door, and I went to see who it was.
It was Esther's 
husband.
When we went to the bedroom, we realized that Esther was 
dead.
Now, my dear ladies, if I refused to draw up a petition for 
Esther, I will refuse to do it for you as well.
Who is on this 
committee of big donors that the rabbi has formed?"
"The young Bronfman ladies," they answered.
"Oh," I said, "this rabbi, must be a very big shot!
Who is he?"
But the women didn't know.
One of the ladies asked me "What did you do to this rabbi, 
that he is so much against you?"
"First of all, I don't know who he is.
But at the time of the 
amalgamation of Adath Israel and Young Israel schools, I spoke at 
one of the Board meetings.
We had Moroccan children whose parents 
wanted them to learn Hebrew.
But they couldn't afford the tuition.
I urged the meeting to allow these children to study Hebrew.
But 
instead of listening to me, they removed my words from the minutes.
And Phyllis Waxman is right," I continued, "when she told me that 
if I wanted to be somebody in life, I should listen to the rabbi, 
and not give a donation to the Hebrew Academy.
But I gave, and I 
don't regret it."
I appeal to all those who will read my memoirs to give a dona-
tion to the Hebrew Academy.
"The big donors put pressure on the community to take Phyllis 
Waxman back," one of the ladies said.
I said, "First of all, if I knew who this rabbi was, I would 
like to tell him that I believe in God.
There is a Ladino proverb: 
'Ni el rico con sou rikeza, ni el baragan con sou baragania.'
It 
means 'I believe in neither the rich man with all his wealth, nor 
the superman with all his muscles.'
And I would like to tell the 
ladies who find it easy to put a string of zeros on a cheque, that 
they give only money, but I give the gift of life to people.
I 
want to tell them that it is very good to fight, but only in a good 
cause, and not for people who destroy meetings so that they will be 
able to sit down with Mrs.
Bronfman.
Fight to enlarge the Jewish 
Nursing Home.
Who knows how many senior citizens didn't have a meal 
today because the people in charge are perhaps drunk.
Who knows how 
many Nicolas are cold in the middle of the night, and want to go to 
a friend to have some tea.
But when you have no money and no health, 
you have no friends.
Ladies, fight to make Maimonedes bigger.
Ladies; fight to stop them from renting the upper floors of Maimonedes 
only to the wealthy.
There are many hotels in this city for wealthy people, and they can have a nurse around the clock.
And we have 
so many senior citizens who have nowhere to go.
Don't fight for 
people who want to have their photographs in the Board room of the 
Women's Federation.
Ladies, fight for our children's Hebrew education, and not for people who discriminate against children whose parents are members of Young Israel Congregation.
Ladies, fight so that 
the wives and children of sick men are not discriminated against."
One of the ladies said, "Ladies, let's fight to put an end to 
the luncheons sponsored by the Women's Federation in the synagogues.
Who knows how many Esthers have sacrificed themselves to help Israel?"
Another said, "The committee of the Jewish Convalescent Hospital was against Phyllis Waxman's being restored to her position.
But this opposition broke down very quickly when the Women's Federation gave them $25,000.00 from the money they had collected."
"Oh," I said, "I'm very glad.
They do good work.
At least my 
speech was worthwhile, if it resulted in a grant to the Jewish Convalescent Hospital."
One of the ladies said, "I will ask you again.
What did you 
do to this rabbi?"
"Do you remember the time of the German occupation in Europe?"
I asked.
"Every Jew asked what the Jews had done to Hitler to make 
him hate them so fanatically.
Oh, each one had a little story.
But I 
one story was very popular among the Jews in Salonica.
A bank wrote 
to a newspaper in Germany, that it needed employees, both for high 
and low positions.
The employees would be selected on the basis of 
a test.
The test was given.
Among the applicants, there was a friend 
of Hitler, and one Jew.
The Jew came first in the test.
When Hitler's friends met with Hitler, he said 'We are going to make a political party out of this incident, because the Jewish people have 
invaded Germany.
We will terrorize everyone who has Jewish friends.
And we will terrorize anyone who employs Jews.
Our goal will be to 
destroy the Jews.
This rabbi didn't win the battle during the amalgamation of 
Adath Israel and Young Israel.
It was my idea, I gave a donation, 
and naturally he is against me now.
But I would like to know who 
-273- 
he is, so that I could tell him that it is better if he occupies 
himself with religion and not with politics."
The ladies asked if I would allow these people to triumph.
"Yes," I said.
"In Ladino, we have a proverb: ‘moadim viene 
se va.
Keda roch kodes i shabat.'
It means that holidays come 
and they go, but the head of the month and the shabat remain.
These 
people have a holiday now.
They won this battle.
But I will remain, 
along with the head of the month and the shabat."
"We admire you," said the ladies.
They said goodbye and left.
Now, as I write, I appeal to the youth of Montreal to take 
part in the community, so that they can remedy this situation.
The 
youth of today doesn't care if one gives a large donations or buys 
a dozen new dresses.
They want action.
I was very impressed when 
the youth of the Hebrew Academy honoured their principal, Mr.
Alpert.
Mr.
Alpert didn't give a big donation.
And he never pushed anyone around.
And Mr.
Alpert didn't buy a dozen new suits.
He was 
honoured because he was very devoted to his pupils He was a fair 
principal.
Come, youth: Take part in the community: We need you.
Only 
in this way will we have a healthy community.
Come, and let's 
sing together, like the Jewish youth of Salonica used tot.
Mouestra tierra non ese piedrida 
I de x nouevo renasera 
Mouestra nation non ese vensida 
I sou brio renasera 
Pouerpo i alma ofriremos 
A este combate 
Libertad mos ganaremos 
Al paez esgate 
Oh venito lavorozo 
Lavorar con fouersa i rozo 
Biva la ora esperada 
De Israel tierra amada 
Mouetra tierra etc.
-271- 
Our land is not lost 
And it will live again 
Our nation is not conquered 
But is proud and will be reborn 
Body and soul we will offer 
In this battle 
We will k win liberty 
We will liberate this country 
Oh, come, and work 
Work with muscles and with roses 
Long live the hour of hope 
For Israel, the land we love 
Our land is not lost 
etc.
In 1976, during the twenty-first Olympics, people wanted 
to come to Canada to see the games, but the hotels were too expensive 
for some.
Ely was in Vancouver, but he had been back in Montreal 
for Mr.
Alpert's party.
We were talking about how expensive the 
hotels would be for some people during the Games.
Ely encouraged 
me to rent the basement to the Quebec Lodging Bureau.
The next day I called the Bureau.
Two days later, they sent 
an inspector to my house.
And two days after that, I received a 
letter authorizing me to rent the basement.
Three young girls, Sara, Jennifer, and Malinda, stayed with me 
for five days, and I had a ball with them.
I felt like a teenager 
again.
Before they left, they gave me a present.
And just yesterday, I received letters from them.
I must thank the Upper May Line 
YMCA for sending such lovely people to my house.
Afterwards, a young man from France, Jean Guy, stayed with us.
I had the most lovely time, and it was a great pleasure to have this 
person stay at the house.
When the French won a gold medal, Jean 
Guy was ecstatic to hear the Marseillaise.
I was feeling young again.
I remembered my first trip to France when I was a teenager.
My great sports hero was Meunier, a cyclist.
During a race, every-
one in the streets would sing a song about him: 

Meunier 
Meunier, tu dors 
Ton moulin, ton moulin 
Marche trop vite 
Meunier, tu dors 
Ta voiture, to voiture 
Va trop fort 

I told Jean Guy about the song.
"Oh," he said, "Madame Garfinkle, everyone still sings that 
song in France.
It has become a folk song.
When these people who 
won the gold medal arrive back in France, the band will be playing 
Meunier's song."
On Monday, August 2, Jean Guy left for France.
But before he left, he gave me an alarm clock with nice big numbers on the dial, that I can read without my glasses.
I was very 
happy to get this present.
At eleven o'clock of the night that Jean Guy left, he called 
me.
He was still at Mirabel airport, waiting for his plane.
He 
wanted to say 'Hi' to me.
Max and I had taught him how to say it.
I thought it was very sweet of him to call.
I told him that if he 
had to spend the night in Montreal, he should come back and sleep 
at our house.
He said that he really had to wait for the airplane
there.
Four days later, we received a letter from Jean Guy, from 
France.
As you can see, I had a lovely time during the Olympics, 
getting to know such beautiful people.
And now I will turn over another page of my life, and I will 
sing.
And I will change the philosophy of Koele.
I will use his 
melodies, but I will sing my own words to them.
Comer came 
I bever vino 
Comer comerech 
Que amagnana mouerirech 
Eat meat 
And drink wine 
Eat, everyone 
Because tomorrow you may die 

Now, for sure, folks, you will ask yourselves how I am feeling, 
after my life of misery.
My answer is that I feel that I am the richest woman on earth.
With the help of God, I accomplished many 
things that not too many others could accomplish.
And I have my 
roch hodesh and my shabat.