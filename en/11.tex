% mempages 200 to 221 
% pdfpages 214 to 235

Alegra's son took me to the bus station a few yards from the 
kibbutz.
Instead of going to Haifa and then on to my kibbutz, I waited 
for the bus to Tel Aviv.
I sat down on the bench in the station.
I 
saw the man from whom I had asked directions earlier, coming near me.
He asked me if I was a social worker.
I said "Why do you ask?"
He answered "Because you are dressed like one, and you didn't go 
to the office of the kibbutz.
I'm sure everything is all right, then, 
because you didn't go to the office."
I asked him who he was.
He said "I am a chaver of the kibbutz."
"when you have the next meeting of the kibbutz," I said, "you can 
spit in the face of your chaverim, for allowing the children to live 
in this way!"
"Lady, these children don't know any better.
They are Sepharady!
We work very hard for our own children, but not for the Sepharadim.
The
Sohonouth pays us five liras a month per child.
You know, the bread 
isn't rationed here!
We put it on the table, and they can eat as much 
as they want, not like on other kibbutzim."
I said, "Mister, you should tell your chaverim that these children know better.
Oh, never mind!
Why am I discussing this with you?"
I sat down on the bench again to wait for the bus to Tel Aviv.
On the bus, I sat in the long seat at the back.
I saw a middle 
aged woman sitting near me, but I didn't pay much attention to her, I 
sat down and I cowered my face with my hands, and I said to myself, "oh 
God!
El dia que nasi yo que Planeta reynaria?
(What kind of planet 
was reigning on the day that I was born?)
To my surprise, the woman 
sitting near me completed the song: "Estrea de plata la que to areloum-
bra.
(The silver star illuminated you."
I opened my eyes and looked 
at her.
She said "Yes, I am Sepharady.
I was born in Jerusalem.
My 
dear child, when you are new in Palestine, you must learn one thing.
Savlanut (patience).
Everything will be all right."
I took the woman's hands and I cried.
She said to me "You think 
-201- 
you are Sepharady?
The Sepharadim don't cry!
No one told you that?
We have so many songs, and we must preserve them."
She started to 
sing.
Immediately I remembered that My nona used to sing these kinds 
of songs.
She sang "La Reyna Brodava," and "Kien me va Kierer Ami": 
La Reyna Brodava 
La reyna estava lavrando 
Lavrando ho ho ho ho ho 
Cavesas de sirma 
I al beni 
De sou sirma le mancava 
Ho la li he he he he he he 
De sous caveyos 
Le adjoustava 
Kien me va Kierer Ami 
Kien me va Kierer ami 
Kien me va Kierer ami 
Salliendo que yo to amo 
I me mouero de amor por Li 
By the time we arrived in Tel Aviv, the woman had cheered me up 
wonderfully.
The first thing I did was go to see the people oh Hith-
achduth Oley Yavau.
I went to see Mr.
Albert Alchech.
He was the god-
father of the Sepharadim, a wonderful person.
He was an educator who 
had come to Palestine a few years before the war.
I found a tele-
phone and called my husband to tell him that I wasn't coming home right 
away, and then I went to find Mr.
Alchech at his library-bookstore.
There was a note on the door: "Back in an hour".
At that time, Mr.
Alchech was sending reports to all the Sepharady organizations overseas.
I didn't waste my time waiting for Mr.
Alchech.
I went to see Mr.
Beja instead.
Before the war., Mr.
Beja had been a member of the Board 
of the Zionist organizations in Salonica.
Before the war, he sold 
everything that he had and established himself in Palestine.
His store 
wasn't very far from Mr.
Alcbechis library.
Of course, he was a member 
of the Board of the Hithachduth Oley Yavan.
I told him that I had just come from the kibbutz where Alegre's 
son was living.
He said "I went there before you did, and I saw exact-
ly what you have seen today."
"But the children didn’t react to you the way they reacted to me.
They knew me."
"Never mind.
I saw everything," he said.
"What did you do?"
I asked.
"Did you write to Alegra?"
He said "Oh, no!
If I did, there would be a revolution in Salonica 
"What about all the children there?
Especially when Alegra is like, 
a sister to you!
I want you to know one thing," I said.
"Alegra is not
poor.
Other children there aren't poor either.
Their parents can afford to send them to the best schools, but on that kibbutz they are treated like charity cases."
He said "I know Alegra is not poor.
She has land here in Tel Aviv.
But it is not easy to remove the children from there."
I said "These children are our next generation.
You people want 
to have kibbutniks, but you will have nervous wrecks instead!"
He begged me not to write to Alegra.
"We will do everything we can 
for these children."
I couldn't stop crying, even while I was talking to him.
I went 
out of there after shaking hands with Mr.
Beja on the promise that he 
would do something.
Outside, I saw a bunch of boys who had arrived 
in Palestine illegally, again through Chernovich.
They were happy to 
see me.
I asked them where they were staying.
"Nowhere.
We are going 
to ask the Oley Yavan for a place to sleep.
We were sent to the Misery 
Kibbutz.
Everything they told us turned out to be a lie."
One of the boys said "My mother used to say 'Whoever talks to you 
more than your mother, lies to you with words."
Chernovich talked to 
us more than our mothers, and with words he lied to us."
I said "This is the second time in the last few months that I have 
heard this proverb."
I went to see Mr.
Alchech.
He said to me "I know.
I saw them."
"Did you write to Salonica?"
-203- 
He said "No.
We can't do that."
"And what do you intend to do?
You must tell someone who can 
correct the situation!
These children don't have to be public wards, 
maybe a few of them, but not all!"
"We can't do anything about this.
They don't listen to Sepharadim."
1 
"You must bring all those people there and show them how these 
children live.
You must show them the open racism that exists in Palestine.
If they like the place, they should send their own children 
there.
Propaganda brought these children here.
If these people want 
alia (immigration), they must prepare a place first.
Where were all 
these people when the Jews in Europe were being deported to Poland and 
dying like flies?
There were many ways to save them, through Turkey.
Many non-Jewish people were saved, and they are now in the camp at Gaza.
These people had organized outside help.
These children were brought 
here after the liberation, and where were they put?
In a gypsy camp!
I want you to tell these people one thing.
Hitler put the Jews in the 
gas chamber without making distinctions between Ashkenazy and Sepharady."
Mr.
Alchech promised me that he would do everything in his power 
to help.
And he, too, made me promise not to write to Alegra.
I left 
Mr.
Alchech's library, and I went to see Janie.
Janie was a survivor of Hitler, from Salonica.
She had hidden 
somewhere with her parents during the occupation.
She got married a 
week before I did, to a soldier of the Jewish Brigade.
She came to 
live in Palestine with her husband.
I told her that I had just come 
from the children's kibbutz, where Alegra's son was living.
"Oh," she said, "you're too late.
I went there yesterday, and I 
couldn't sleep all night.
This morning, while I was still in my night-
gown, I wrote to Alegra.
I told her exactly what I had seen.
"Come 
here', I wrote, 'and take your son away!'"
She told me that Jamie's letter had shocked hex and paralysed her.
She then went to call her son, who was also living in Tel Aviv.
A few 
minutes later, he arrived.
He was a wonderful boy.
He was now married and he had a wonderful business.
I asked him if he remembered me.
He said "I don't remember you, and I don't want to remember my child-
hood.
My mother forgot what my Nona used to say; 'Whoever talks to 
you more than your mother, lies to you with words.'
Chernovitch talk-
ed to my mother more than Mamma, and with words he lied to her."
I think that I spoke to all the officers of Oley Yavan; I asked 
them all to help these children.
Afterwards, I went to my sister's 
house to spend the night.
My niece told me that she was getting mar-
ried in two months.
She said that she was very sorry that I wouldn't 
be at her wedding, since I would be in America at that time.
But she 
begged me to sew her wedding gown.
The next morning, I called Max again, and told him that I would 
be staying in Tel Aviv for another two days.
I finished the wedding 
gown in one day.
My niece was very happy; she would be the most elegant bride in Tel Aviv.
I prepared to go to the kibbutz.
My sister and her family 
begged me to stay longer, but I refused.
Before I left, I asked where 
I could buy some lollipops.
My niece said "We have plenty in the 
house.
I'll give you some."
They were for Efraim and Uri.
When I arrived at the kibbutz, Max and Mei were waiting for me at 
the bus station.
In the meantime, of course, Max had changed rooms again.
After supper, Uri came, and I gave him a lollipop.
Suddenly, 
a lady who was a member of the kibbutz stopped us.
She said "We don't 
give lollipops to individuals.
Either take your lollipops to the comuna and the counsilor will give them to those who deserve them, or 
make the boy eat his lollipop in your room."
I took the lollipop from Urils hand and apologised to the lady.
-205- 
We went to Uri's mother's room, so that Uri could eat his lollipop.
Uri's mother found it only natural that the mothers who wished to give 
their children lollipops must do so only in their rooms.
I didn't say 
anything, but she started to talk to me.
"I know that a it is very difficult for you to adapt to this kind 
of life."
I said "I would like to look after my child myself.
I would like 
to see when my child laughs and cries, when he is sick, and when he is 
happy."
"You are talking about sacrifice.
Do you think your children will 
appreciate you when they grow up?"
I answered "That is a question of luck."
The next day I went back to work in the dressmaker's workshop.
The dressmaker told me to finish some skirts and to put zippers in them.
When I finished one, I asked her if I could irons the skirt.
She grabbed the skirt from my hand and picked up a pair of scissors, to rip the 
zipper out.
I said "Stop!
Don't rip it out!
I don't know why you are doing 
this.
For your information, I am expert at everything that is done 
with a needle."
The other woman said to me "I don't know how you managed to wait so 
long to speak to her."
I said "I think I’m still terrified of my past.
I don't know why 
I didn't say something right away.
I try my best not to fight, because 
I want to make my home here.
I promised my husband I would try.
Do 
you know why she does this to me?"
She said "Look.
These people used to marry within the kibbutz.
My 
husband went into the army, and married me, and I came here.
Your hus-
band went into the army, and married you, and brought you here.
Others 
have done the same thing.
She doesn't have a chance to marry anyone 
"She's very sweet to you," I said.
"Yes," she answered, "because she is teaching me.
But you came, 
and you know everything.
You don't need her.
You know how to sew bet-
ter than she does.
And you must know that the most decent job in the 
kibbutz is this one.
You just arrived, and they gave you this job.
You worked in the kitchen before, and you were very good there."
I said "You still haven't answered my question.
Why does she not 
do the same to you?
Maybe Max promised to marry her before he met me."
"Oh, no," she said, "it's only that she can't do it to me."
"Why?"
"She is very prejudiced, and I am not a Sepharady."
I left the workshop and I went to my room.
Max was waiting for 
me.
He said "The ticket agent called.
It will take a long time to get' 
reservations to go to America."
I said "There won't be any Americas" 
"Why?"
"I can't go to meet your people because I can't stay on the kibbutz?
Our agreement was that I would try very hard to live here, but if I 
couldn't, we would both leave.
I don't want you to leave.
I can't let 
you leave a place you like just because of me."
He wanted to know what it was I didn't like.
If it was the room, 
we would have a house by the time we returned from America.
"It has nothing to do with the room," I said.
"First, with all 
this barbed wire around us, I feel like I'm in prison.
Second, I have 
to run to the dining room to eat in the first shift, because by the
second shift the tables are so dirty that you can't even sit down.
I 
can't eat with all the noise in the dining room.
I want my um own ta-
ble, my own bread, and my own cooking.
I don't mind working, but in a 
place where the boss can fire me if he isn't satisfied with my work, 
and praise me if he is."
"We will go to America, and you will decide after," said Max, 

I said "No.
If you think I will spend the rest of my life 
on the kibbutz, you are mistaken.
I'm going to leave Palestine and go 
back to Greece."
"What are you talking about?
I love you and you love me.
We be-
long together.
And what are you going to do in Greece?
You gave up your house.”
"There is always a place in the Alaatini Orphanage with a bed and 
a blanket.
I prefer that to living here."
"I've had enough here myself," Max said.
"Don't worry, we'll leave 
the kibbutz.
But after we've gone to America."
Fifteen days after this conversation, we had a phone call from the 
ticket agent.
They told us that some Arabs were about to go to New 
York on a special plane.
There were three empty places on this plane, 
two for us, and one for another Jewish person.
We accepted the tickets 
and went to Tel Aviv.
At 3 o'clock in the morning, we took a plane to 
Paris.
From Paris we were to fly to New York.
The plane that left Tel 
Aviv had no seats.
They gave us blankets to sit on.
The Arabs on the 
plane were students and businessmen.
There were also three ladies on the 
plane.
Two of them spent the whole trip looking at themselves in their 
mirrors, making sure their lipstick was all right.
The third lady was 
very young, about 18 years old, and she had two babies with her.
The next morning, the airplane landed.
In Casablanca.
Everyone 
got off the plane without helping the woman with the babies.
I went 
up to her and I saw that she was crying.
I asked her if she needed help.
I took a baby from her, she took the other, and Max took two shopping 
bags filled with diapers.
The woman couldn't stop crying.
She said 
she didn't know how to change the babies.
"I thought that we were in Paris," she said.
"My family arranged 
for a nanny to meet me in Paris to take care of the children."
I showed her how to change the babies.
She was going to throw the dirty diapers away.
I said “Dont’!
Don’t throw them out.
Tomorrow ???
New York”
"how will I put the children to sleep on a big bed?"
She asked.
"They can fall down."
I said "Don't worry.
We'll put the mattresses on the floor for 
them.
If they roll off, they won't hurt themselves."
We washed the 
diapers and hung them to dry on chairs.
With the climate in Casablanca, they dried very quickly.
The next day we took the airplane again.
There were a pile of blankets on the plane.
I said "We'll put some blankets on the floor for 
the children.
You can't hold them in your lap all the way to New 
York!"
The airplane took off.
The next day we arrived in Rio de Janeiro.
Max walked with the two shopping bags, and the Arab woman and I each 
carried a child.
The pilot took us to a beautiful restaurant.
The 
Arab woman said "I am ashamed.
My countrywomen keep looking at themselves 
in their mirrors, and I'm such a mess."
She asked Max where we were.
He said "Rio de Janeiro."
The lady 
looked at me.
She said "Who are you?
How did you know we would come 
here?
You are sent from Mohamet!
Oh, I will serve him, and pray for 
him!"
I said "Do it, if it makes you feel better.
But I am just a plain 
Jewish woman."
A few hours later, we took the airplane again.
They told us that 
this time we were going straight to New York.
-209- 
After many hours of flying, we found ourselves on a banana plantation on 
a tropical island.
It was very hot.
The two ladies who had been looking 
in their mirrors were so hot that the curls they had so carefully put in 
their hair came out.
It was like a hell of heat.
A man came to talk to 
us.
"We can give you a beautiful room, with a comfortable bed and air-conditioning, 
but we have no food for you.
We can only give you bananas."
I've never again seen bananas as big as the ones that grew on that plantation.
The Arab lady asked me "What am I going to feed the children?"
"Bananas," I said.
"How will I mash them?"
"With your fingers."
After we ate the bananas, we went to our room.
It was beautiful.
We put the mattresses on the floor again, changed the babies, and washed 
the diapers.
I said to the lady "Did you notice your friends with their 
curls hanging down?"
"Yes," she said.
"Well, you are the same: I'll watch the children.
You go and have 
a shower and wash your hair."
"But what will I do with my hair after?
It is very long."
"Don't worry.
I'll give you a ribbon for it."
I picked up a diaper.
At the edges of it there was seam binding.
I 
cut off a piece of the binding, and gave it to her for her hair.
I took 
a piece of ribbon for myself, because I also had to wash my hair.
The 
two other Arab women wanted ribbons, but we didn't give them any.
We spent the night in that beautiful cool room.
In the morning, we 
took the plane again, to go to New York.
The plane took off; sometime 
later, it landed, and we found ourselves in Dakkar, at a beautiful military 
base in the middle of nowhere.
The officers on the base gave us a 
beautiful bungalow to stay in.
We went into our routine, removing the 
mattresses, washing and drying diapers.
I've met many people in my life whom I had never known before, but 
the champions of kindness, sweetness and honesty were the people of 
Dakkar.
I don't remember exactly how many days we spent there.
I was at 
home with my Ladino.
I could talk to everyone.
The people of Dakkar had 
exactly the same accent as those of Salonica.
One day, they told us that 
our airplane was fixed.
Apparently, it was a forced landing that had 
brought us to Dakkar, but now we were going to New York!
The airplane took off.
Suddenly, it landed again.
From the window, we could see ambulances and fire engines hurrying after our plane.
I 
said to Max, "Oh, they've arranged for a big reception for us here in 
New York!
Max said, "Never mind New York: Who knows where we are?
It's another 
forced landing!"
That God, the airplane came to a stop with no 
casualties.
We found ourselves on an American military base, in the middle of nowhere, on an island in the ocean.
They gave us bungalows.
We 
moved mattresses, changed babies, washed diapers.
Good thing the Arab 
woman hadn't thrown out the dirty diapers!
We were invited to eat in the 
officers' mess.
By this time, my nice white blouse had turned black.
There was nothing to do on the base.
The only things we could see were 
palm trees with coconuts.
Every night we were told that a plane would 
come from Miami in the middle of the night to pick us up.
One night, at 
one o'clock, we were woken up.
The airplane from Miami was waiting for 
us.
I woke the Arab lady, and we dressed the children.
Again, Max walked with the shopping bags, I took one child, and the lady took the other 
We went to the airplane.
This airplane had seats!
We sat down like human beings, and the
plane took off.
It landed in Miami: As soon as we got off, Max went to 
telephone his brother, who had been expecting us in New York 15 days before.
I asked the Arab lady, "Do you have your husband's telephone number': 
"Yes."
"Leave me with the children, and go and call your husband."
When she came back, she said, "I spoke to my husband, and we decide 
to give you this ring."
She took a ring from her finger.
The stone was, 
aquamarine, with diamonds all around it.
She said "My father chartered the airplane we took from Tel Aviv for 
me, because he believes that a wife must not be separated from her husband 
for too long.
I was supposed to come with a nanny and a maid, but 
at the last minute they were refused visas.
You take this ring and give, 
it to Mohamet.
My father gave it to me as a going away present."
I answered, "If you believe in Mohamet, you must know that he does, 
not get paid for his services.
If you want me to take this ring, you 
don't know me well enough.
I don't ask to be paid for my services.
I 
told you before that I'm just a plain Jewish woman."
It was very hot in Miami, and it took a long time to pass through it 
migration.
All our papers were examined.
We then took a plane and landed in New York!
Max's brother Jack was waiting for us.
He apologized for 
not bringing his wife.
She was at his store now, sot that he could meet
us at the airport.
-211- 
"As soon as we get to the stone, Zinn will take you home, but Max will stay there 
with MB."
Max said, "I have to stay downtown, because tonight I have a meeting with Yad ve 
Chem."
"O.K.," said Jack, "we'll go together, but Tova and Zinn will go home."
"We came to the store.
It was a tobacco store, and it was Ball of telephone 
booths.
Jack's son was in the store, so Jack, Zinn, Max and I went to eat at a cafe-
teria.
When we came back from the cafeteria, Jack asked Zina to take me to their 
home.
But Zina didn't want to listen to Jack.
There was no place to sit in the store, 
so I sat on one of our valises, Sometimes there was no-one in the store, but every 
once in a while there would be a rush of people buying cigarettes and bubble gum, and 
using the phones.
At these times, Zina took me out of the store.
The first time she did this, I thought that we were going home.
But instead, we 
went around the block.
She did this six or seven times, whenever people name into the 
store.
As we walked around the block, she talked to me in beautiful Hebrew.
After a 
few walks, my feet started to swell, and I couldn't walk anymore.
I could hardly 
put my feet down on the ground.
Again, people came into the store, and Zina said to 
me "Let's go."
She took me out.
At the door, I asked her where we were going.
"Around the 
block," she said.
And I answered "If I have to walk one more yard, you will have big troubles with 
me."
"Why?"
"Because," I said, "I am very tired and ready to collapse, here, in the middle of 
the street.
And you will have to call an ambulance to take me to the hospital."
She understood.
She took me to the subway.
It was rush hour.
I had never seen 
so many people before in my life.
There was no seat on the subway.
In Salonica, some 
man would have given me his seat, but here, no-one offered.
We finally go off the 
subway and took a bus to Zina's house.
Jack and Zinn lived in the penthouse of an apartment building without an elevator.
I met their other two sons, as nice and as good-looking as the one I 
had met in the store.
I hardly had the strength to say hello to them.
I took my 
shoes off in the living room and lay down on the couch.
I didn't wake up until midnight.
I saw Max and Jack, and the living room was full of neighbours coming to
see me.
Everyone asked me "You say you are Jewish, but how come you don't speak Yiddish?"
The table was set.
They were waiting for me to have supper.
I didn't want to 
eat.
I just wanted to go to bed, but I had to sit down at the table with my dirty
blouse.
When we finished eating, Max and I were shown to our room, and we went to be 
Max asked, "How do you like my sister-in-law?"
I said, "This woman never laughs and never cries.
She never dances, never sings.
And she is a very bad hostess.
On top of all this, she thinks she is the Queen of 
Sheba."
"I agree with everything you say," said Max, "but why the Queen of Sheba?"
"When I find out, I'll let you know.
I don't know why yet."
The youngest son was six or seven years old.
He reminded me of Miki.
He used to 
sit in my lap and kiss my neck, just like Miki had.
Of course, he only did this when 
his mother wasn't around.
When she came, the boys would run away.
The other two 
boys were just finishing high school.
They were very intelligent and very nice looking.
It didn't take me too long to find out why Zina thought she was the Queen of Sheba.
One day, when I was talking to her, she said "I went to school in America!"
I said "Do you think there are no schools in Europe?"
I had two cousins in the United States.
I knew one of them, Dora, very well from 
Salonica.
She came to New York a few months before the war.
She was Suzanne's 
sister, and she had married my uncle.
My mother cousin had left for Szionagat America 
before I was born.
I only knew her name, Alegra Gategno.
I didn't know how to find 
either of my cousins.
One day, one of the Sepharady organizations invited me to a meeting.
Max and I 
used to go every day to see Max's cousins Charlotte and Izi, who had a factory that 
manufactured flocking.
Charlotte and Izi were wonderful people.
Max and I wanted to 
learn the business, so that if we wanted to stay in America, we could go to Montreal and start our own business.
It was very nice of Charlotte and Izi to 
-213- 
teach us.
Charlotte looked at the address of the meeting of the Sepharady organization, 
and she said that I needed at least two hours to get there.
The designer said "No.
It 
will take three hours."
I said "Look.
I’ll give myself three hours.
If I’m early, 
I’ll wait, but I don’t want to be late."
They drew me a map of how to get there.
Izi 
took me to the subway and put me on the right train.
Two hours later, I was at the place 
of the meeting.
I went in, and of course the place was empty.
I asked the janitor 
where everyone was.
He looked at his watch and he said "You're an hour early for the meeting.
Sit 
down, I'll give you something to drink."
As I sat down, I saw a wall-to-wall plaque 
with Sepharady names on it that were very familiar to me.
There were stars next to 
some of the names.
The janitor returned with my drink, and I asked him what the stars 
meant.
He said "These are the names of the people who went overseas during the war.
The 
people with stars next to their names died on the battlefield.
The rest came back."
Just then, a man arrived.
He said "Oh, you came to the meeting?
Too early!"
For some reason, I asked him "Do you know Alegra Gategno?"
"Of course: She's my wife:" 
We started to talk and I explained who I was.
He said "Look.
My car is downstairs 
and we live just five minutes from here.
I’ll take you home to my wife, and I'll 
bring you back in time for the meeting."
We went to his house, and I met Alegra.
We both cried.
She wanted to hear about 
her two brothers, to whom I had been very close.
The time of the meeting approached, 
and Alegra's husband drove me back.
Before I got out of the car, Alegra said "My children are married.
You can see what a big house we live in.
Come to live with us with 
your husband for a little while."
She gave me Dora's telephone number and told me to 
call her either at night or on the weekend, because she worked.
I came home and told Max that I had found my cousin and that she had invited us to 
stay with her.
He said "I came to America to see my brother.
He works all day and I work all day, and when I come home, I want to be with him.” And so we refused.
Soon after I called Dora, and I went to see her, too.
Dora lived in a three 
room apartment in New York with her husband, and their teenage daughter.
She was 
very sorry that she didn't have room for Max and me.
But I went to see Dora often.
The day after Max said that he wanted to stay with his brother, my sister-in-law 
refused to speak to me in Hebrew any longer.
She was Yiddishista, and I would have 
to speak to her in Yiddish, if I wanted to speak to her at all.
Soon after this 
day, I realized that I was pregnant, but I didn't tell anyone about it.
-215- 
One morning I got up and I was bleeding.
I didn't go to Charlotte's that day.
After everyone had left the house and I was alone, I called Alegra.
There was no answer.
I called her every hour, but she still didn't answer.
My other cousin Dora was 
working.
After three hours, someone answered the phone at Alegre's house.
It was her 
son.
I asked for his mother, and he told me she was out of town.
"My sister is pregnant and she went to help her."
This was a great disappointment d for me.
In the afternoon, when my sister-in-law came home, I told her "I'm pregnant, and I saw a show of 
blood this morning."
"Tell me what you just said in Yiddish," she said.
"This is no game," I said to her.
"Please.
I need a doctor."
She didn't say yes or no.
I heard her dialing the phone.
An hour later, she 
said "I made an appointment for you at the doctor.
It will be in 8 days."
During the week, I had a show of blood once a day.
The night before my doctor's 
appointment, I heard an argument between Max, his brother, and his sister-in-law.
But 
it was in Yiddish, and I didn't understand.
My sister-in-law came into my room.
She 
spoke to me in perfect Hebrew.
"I'm very sorry, but I can't take you to the doctor tomorrow," she said.
"Is this what all the excitement is about?"
I asked.
"Yes."
"Don't worry,' I said,'"Just give me his name and address and the time of my ap-
pointment, and I can go by myself."
"But you don't understand," she said.
"I never made an appointment for you."
I looked her in the eye, and I said to myself "This woman is a monster."
"Why didn't you?"
I asked her.
"I'm bleeding, and I need a doctor."
At that mo-
ment, an old Ladino proverb came to my mind: 'El viejo non se Kiere mourir por mas ver 
i mas sintir' ('The old man doesn't want to die, because he wants more time to hear 
and more time to see').
Two minutes later, my husband came to my room.
He said "Let's get out of here."
We went to see Charlotte and Izi.
I was shivering with cold by the time we arrived.
Charlotte, Izi, and Max undressed me and put me to bed.
Very early in the morning, 
I started to bleed heavily, and Charlotte called an ambulance to take me to the 
hospital.
Of course, I lost the baby.
I was in the hospital for 10 days, after after which I was to go to a convalescent
hospital.
Charlotte's sister Bella recommended one that she had been to 20 years ago.
She said it a was a very beautiful place.
One morning, my husband came to take me from
the hospital to the convalescent home.
It was indeed a beautiful place.
I was taken to my room.
It had six beds in it 
but the curtains were drawn around each bed, and I could see no-one.
It was rest period, and I lay down and fell asleep.
The sound of my curtain being pulled aside woke me up, and I looked to see who it 
was.
All I could see at first was a little hand on the curtain.
It pulled the cur-
tain back, and I saw a little, sweet old lady.
She spoke to me in Yiddish.
She saw 
that I didn't understand so she pointed to herself and said "Mali".
Then she pointed 
to me.
"Tova", I said.
When she heard my name, she asked "Do you speak Hebrew?"
I said that I did.
She told me that she had been a teacher in Poland, and that she was 85 years old.
She spoke the most beautiful Hebrew that I have ever heard.
She asked me what I was 
doing in the convalescent home.
I was surprised at the question.
"I came to convalesce for 2 or 3 weeks," I said.
"My dear child," she said.
"This is an old people's home."
"It was very highly recommended to me by someone who had been here 20 years ago," 
said.
"My dear girl, when the district changed, the hospital changed also, " she said.
"The youngest person here is 75.
Surely you should call your husband to come pick you 
up."
I said "I don't have anywhere to go.
And I need rest."
The rest period was over.
Mali was the most wonderful person that I have ever met 
She asked me to call her Babi, because she had a grand-daughter my age.
We got up, and 
I went to the radio room with all the old women.
There was a Yiddish programme 
on the air.
The nurse came in.
Thank God, because the radio was getting on my nerves.
"It's time to play bingo," she said.
Mali took me by force to the bingo game.
She sat near me.
The game was conducted 
in Yiddish and I ???
Suddenly she screamed "BINGO".
"You see, you won," she said.
I won some woolen ???
It was suppertime.
Mali tried to force me to eat, but I couldn't.
In the ???
evening
there was a talent show.
10 names were drawn, and these people were supposed to 
sing.
The songs they sang reminded them of their youth.
Mali said to everyone, "We 
have a young guest tonight.
We've heard all your songs a hundred times before.
Tonight, she must sing for us."
At first I refused.
All the old ladies stopped singing and talked about the memories that their 
songs brought back to them.
It was boring, and noisy.
But the old ladies enjoyed themselves.
Mali and the nurse came up to me.
"Please.
Sing for us."
I went up on the stage, and I said "I'm going to sing you a song that we used to 
sing only to very special friends and very special people."
It was a song that Chaim 
used to sing: 
Mé yaman la frivol 
Me plaze reyr 
Bevo moutho alcol 
I lo ago sofrir 
Al mansevo habile 
Lo etho al zembil 
ixiixiemaxmomikaxclexeil 
I 14 tomo moutho de mil 
Lousos i cabare 
Dansing i separes 
I siempre visto yo carre 
Me vo al buffet 
Me yaman kalorifer 
Lo impresiono me empatrono 
Lo se rovar i entero es koulkar 
Fouyir de siempre embrolios 
Para non abouzar 
The women voted me the best singer of all.
They gave me some curtain material as a 
prize.
It was time to go to sleep, and everyone went to bed.
I fell asleep and had a 
dream.
Chaim came to see me, and he brought me a box of Floca chocolates.
When he 
was ready to go, I said "No, you must take me with you!"
He said no.
I insisted.
"I can't stay here.
You must take me!"
I was screaming so loud in my sleep that Mali 
rang for the nurse.
The nurse came and took me to have some hot chocolate.
I was crying so much I couldn’t talk.
She said “Don’t worry.
You were just having a nightmare.
Oh, ???
before I forget, someone came and left you some presents.”
She went to her office and brought out a box of Floca chocolates.
I started to 
get very excited again.
"Who brought these?"
I asked.
I found out that my cousin 
had come to visit me after work, but they had not let her in to see me, so she left 
the box of chocolates.
The next morning, at 11 o'clock, I was waiting by the pay phone for Max's call.
When he phoned, instead of speaking to him, I just cried and cried.
In the afternoon 
Max came to pick me up.
He said "Dora went to buy a folding bed this morning for you.
You will stay with her."
So I went to stay with Dora.
The bed had been put in the middle of the house; 
there was no room anywhere else for it.
Dora was very good to me, and treated me like 
a queen.
She even missed work for a week to take care of me.
Every day we sat down
together, to make me a new dress, or a new suit.
When I was a bit stronger, Max and 
I took the train to Montreal.
Here I found many narrow-minded people who had "gone to 
to school in Canada".
Some said that I couldn't be Jewish, and wondered how I had 
learned so quickly to seem Jewish.
Others said "Oh, she must come from a Negro family 
But I will never forget the kindness of my husband's cousins, Clara and Rose Cohen, two 
wonderful ladies.
Clara gave me a room in her house.
Her two daughters were married.
Clara, her
husband, and their son Perry who had just finished High School, were staying in the 
house.
I became very fond of Perry.
To this day, Max and I are very close to him.
Rose would send her son Morrie to pick me up and take her to their house.
To this 
day, we often telephone Morrie.
Rose would take me downtown and buy me boots and 
expensive presents.
God bless Clara and Rose, wherever they are!
At that time, it was very hard to find an apartment.
We finally found one, 
semi-basement, but to me it was my home, my own Buckingham Palace.
Clara gave me some 
pots and pans, and Rose gave me a crystal plate.
They came to see me very often, 
often invited me to their homes.
In the Hebrew circle, there was a lot of talk about Keren Hatarbout.
This organization was going to open a summer camp where children would speak only Hebrew.
They
brought a pedagogue from Calgary to be the director of the office of Keren Hater-
-219- 
bout and the director of the camp.
His name was Rabbi Horowitz.
Rabbi Horowitz asked Max and I to work at the camp, which was called Camp Massed.
Max was to be quartermaster, and I was to be nurse and camp-mother.
We decided to accept.
They asked me to go and help open the camp, and blindly I accepted.
There was no 
plumbing yet, and it was wintertime and very cold.
The workers were building new bungalows, and I had to supervise the work.
The day finally arrived when the camp was 
ready: The children arrived, and every one of them spoke beautiful Hebrew.
At the end of the first day, everyone was very tired, and we all went to sleep.
The next morning, at 5 o'clock, I was in the kitchen, making sure that the cook was 
preparing the breakfast.
At 8 o'clock everyone came to the dining room to eat.
There 
was an empty chair at one of the tables.
I asked the counselor where the boy was.
"He doesn't feel well," he said.
"Did you report it to the doctor?"
I asked.
"No."
The doctor's name was Dr.
Mendes.
He was a very nice middle-aged person with .a 
lot of experience.
I left everyone in the dining room, and I went first to see the boy.
"Where does it hurt?"
I asked him.
"I don't know!” he answered.
I tried to touch his belly, but he didn't let me.
When he moved, I saw that his 
sheet was wet.
He was 10 years old, and he had wet his bed.
"Look," I said, "you and I are going to have secrets.
You change quickly and go 
have breakfast.
I'll change your bed before anyone else comes to see."
I picked up 
his sheet, and I saw that there was a rubber sheet underneath it.
The boy had brought 
it from home.
I knew now that his wetting the bed wasn't just an accident; it was a 
habit.
The boy confessed that his mother woke him up two or three times at night at 
home.
If she didn't, he wet the bed.
The boy went to the dining room, but not before he 
made me promise not to tell anyone.
I agreed, but I said that I would have to tell 
the doctor.
I spoke with Dr.
Mendes, and he said "This boy needs help, from you and from me.
We have to wake him up three times each night.
We'll take turns."
Three times a night 
we woke the boy.
But on the fifth night Rabbi Horowitz became aware of what we were doing.
He called me to his office and asked if it was true.
I said yes.
He
said that I was supposed to tell him about these things first, and not the doctor.
I
was to take orders from him, and not from Dr.
Mendes.
"I am the nurse here," I said.
"The doctor gave me orders to help this boy, and
I did so."
Rabbi Horowitz called Dr.
Mendes in.
You could hear them screaming from outside 
the building.
"You can't teach me how to be a doctor!” said Dr.
Mendes.
And Rabbi Horowitz replied "You can't teach me how to be a pedagogue!
I will 
call the parents and tell them to take the boy away from here."
There was a big commotion about this.
The boy was very embarassed, but I assured
him that no-one knew whom Dr.
Mendes and Rabbi Horowits were talking about.
After this argument, Dr.
Mendes packed his suitcase and left the camp.
It was up 
to me alone to get up three times during the night to wake the boy.
Kidouchim lavado 
Apio partido 
Deseouvrir a bikar 
Mazon 
Motzi matha 
Lethouga en boultero 
Meza ordenada 
Deseouvrir a bikar 
Mazon 
Motzi matha