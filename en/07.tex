% mempages 125 to 147
% pdfpages 136 to 158

- 6.13ctri;.1‘ 
In this part o§ Greece, the people suffered greatly from the B4olgiern occupation.
The misery of these people was as great as the misery of those I seved when I was with 
the Soupe Populaire in Salonica.
But UNNRA helped a great deal.
I was satisfied with 
my job.
One evening the chemist had nothing to do.
He told me that on account of me 
he had lost the battle with the nurses.
One day, without even saying "Shalom", he 
left the camp.
I never had time to sit down in the dining room at lunchtime, nor very often in 
the evening.
When my food was cold, I didn't even want to look at it.
But I had a 
wonderful took, and he was after me to eat all the time.
I would prepare programmes 
and try to figure out how to make the food tastier.
I would even help the cook so 
that the meals would be on time.
No matter how many refugees arrived, I made sure 
that I served them myself.
For me this whole experience was very good.
I felt I 
was doing something useful.
One day, a large number of refugees arrived.
For the first time, there was a 
large group of Jews among them.
We distributed food all day, and the kitchen was quite' 
busy.
I didn't have time to talk to any of the Jewish people I was serving, although 
I knew many of them.
We could only look at each other.
When everyone had been fed, 
I started to prepare the programme for the next day.
Suddenly I heard the beautiful 
sound of a flute.
One of the refugees was playing, and the others were singing.
The 
melody was a Greek classic, 'Tsobanaki Imouna', but the words had been changed: 
Evreopoula imouna 
To astraki foressa 
Mas placossan ta cafonda 
Kie mas pigan stin Rib Polbnia 
Stin Polonia mas piganb 
Po po po si pafamb 
Mas courepsan ta malia 
Kie mas dissan andrica 
To lox' praj sto Aufstehen 
Viename sto Tsell apel 
Pende pende stin sira 
Ak manoula mou glikia 
Pende pende stin grami 
Ak manoula mou krissi 
-126- 
Sto loutro mas piganb, 
Ya psora mas quitakssanb 
Ki i cardiamas tik tik tak 
Min tikon mas pant sto gaz 
Stin doulia pujenamb 
Mb anenous kie vrokies 
Kian siga doulepsamb 
To bastouni vlepamb 
This means: 
I was a little Jewish girl 
I wore my little star 
But they crushed us 
And sent us to Poland 
They took us to Poland 
Po po po, how much we suffered 
They shaved our hair 
And dressed us like men 
In the morning Aufsten 
We went out for tsell apel 
Five by five in the line 
Op 
my sweet little mother 
Five by five in the line 
0, my little mother of gold 
They took us to the baths 
And examined us for lice 
And our hearts would beat tik tik tak 
We were afraid they would gas us 
We went to work 
In the wind and the rain 
And if we worked slowly 
They whipped us 
When the boys finished singing this, they started to sing another song, again a 
Greek classic called 'Black is the Life that we Live': 
Mavri morb mavri ine i Zoi you canomb 
Mb fovo tromb to psomi 
Mb fovo perpatoumb 
Stin vris morb stin vris 
Na pamb den boro 
Pandou mm ley o scopos 
Isb filaquismbne 
Yernano cratimeni 
Den eclepsa morb den eclepsa 
outb scotossa 
Evreopoula imouna yi afto 
Mb filakissan sto 
Aousvits mm Kieissan 
Kalali more kalali sta 
Englezakia mas 
Afta fa mas glitosoun kie fa 
Mas elefterossoun 
Pio messa fa mas kossoun 
The translation of this song is: 
Black, so black, is the life we live 
We eat our bread in terror, in terror we walk 
I can't go to the fountain, just to the fountain 
Everywhere the guard says: you're a prisoner, a 
prisoner of the Germans 
I didn't steal, my God, I have never killed 
But I was a little Jewish girl, so they imprisonned me 
and shut me up in Auschwitz 
Blessings, my God, blessings on the little English 
They will save us and free us, or, who knows, they'll 
lock us up again 
I felt terribly sad just hearing the words.
My work was done, and I was ready to 
take my smock off.
Col.
Sheppeard came to the kitchen, and he said "Such beautiful 
flute music!"
"They are very depressed," I said.
"How do you know?"
he asked me.
"By the words of the songs.
This was the first time that I heard the new words 
to these old songs."
Just as I was saying this, they started to sing a part of the 
Operetta of Esther, the part where Esther comes to the throne and she is pessimistic 
about her future: 
Al borde del yarden 
O Tierra amada de los sielos 
Santos montes 
Por sien miraclos 
Signalados 
R4 IA el doulssb.
paez 
De nouestros padres 
Siemprb vamos asser 
Rojados 
Near the Yarden border 
0 the land that God loves 
The blessed mountains 
And hundreds of real miracles 
In the sweet country 
Of our own fathers 
Where we will always be rejected 
Col.
Sheppeard asked me if I wanted to go to try to raise the morale of these people.
I said yes.
I went to them right away.
Everyone was.happy to see me and talk to me, 
-128- 
not having been able to at mealtime.
I told them"Let's sing songs of Leon Botton, 
because you people feel sorry for yourselves.
We will sing happy songs instead of 
singing that we will always be rejected, as in the song oil of Esther, who was depressed 
and afraid of the future."
They answered "What will we find in Saldnica?
Should we 
not be depressed?
Where are our children?
Our wives?
Our mothers?
Or our fathers?
Everyone was gassed."
I answered "Sing '0 Tiriazi' (a Greek patriot song).
Go to Palestine, fight for 
the state of Israel."
They sang this song.
Oh Tiriazi 
Ta hopla fonazi 
Paidia sikophitb 
Ya tin elefteria 
The translation of this song is: 
Let's make a match 
The army calls us 
Children get up 
And fight for freedom 
I didn't know the rest of the song.
But the whole camp started to sing with us.
It 
was really beutiful to hear.
When the song was over, I said to the man playing the 
flute "You have such a beautiful flute.
Where did you get it?"
The man said "I used 
to work in Canada."
When he said 'Canada', I thpught he was crazy.
I said "What's 
this about Canada?"
"Oh," he said, "You don't kbow.
There was a huge warehouse in the 
lager.
Everything that the Jewish people brought with them was stored there by the 
Germans.
They called the warehouse Canada because that is the richest country.
I was 
Lucky enough to work there.
We opened all the parcels that the Jews brought, and we 
put all their belongings in separate piles.
One day a lot of Jews arrived.
This flute 
was among their baggage.
As soon as I saw it, I had to play it.
I played the song 
I used to sing to my wife, 'I Never have the Courage".
He played this song for we 
and everyone sang.
Nounka touve el koraje 
De dezir qub tb ami 
Kon sospiros nib mantouve 
De tb ver me afarti 
Lo blanco izo el platbro 
Lo moreno izo el dio 
Biva la djenth morena 
-129- 
This means: 
I never have the courage 
To tell you.
that I love you 
I fed myself with big sighs 
I was full just to see you 
He said, "You know, my wife was a brunette, just like you.
I sang to her: 
The white is made by silversmiths 
The brunette is made by God 
Long live the brunette people 
Because I would die for them.
In the evening the Germans called for me.
I thought they were going to send me to 
the gas chamber.
But they put me in the Gercpan orchestra.
I was very well fed.
We 
would go to play from one camp to the other.
There were only two Jews in the orches-
tra.
I played the flute, and the other one played the violin.
One day they took us 
to a very nice camp.It was the birthday of the camp commandant's wife.
She was very 
musical, and she came to every rehearsal to make sure we were playing well.
I'm 
not a professional musician.
I played some wrong notes.
She knew I was a Jew.
I 
felt the strap on my back whenever I played a wrong note.
Once, when she was raising 
the strap, I caught sight of her ring.
It was my wife's engagement ring.
I had or-
dered this ring from Agope (the best jeweller in Salonica).
You can imagine how I 
felt; I thought I would have a heart attack.
I was told that I would have to sing 
at the birthday celebration.
I adked for some paper and a pen to write the words of 
the song down.
I said I didn't remember them very well.
I went to my room which I shared with the violinist.
I wTote and wrote and kept 
throwing papers away.
In the end, I wrote what I had in mind.
The night of the con-
cert came.
I looked only at the ring worm by the commandant's wife.
Suddenly I no-
ticed the yadran she was wearing, strings of pearls with a little bow in the back.
I 
wondered yokes to whose grandmother it had belpnged.
It was my turn to sing.
I sang 
in Ladino: 
Tantas estreas en los sielos 
Tantas ijos pareras 
Sietb vezes cazes 
Inez bivda kedaras 
(You should have as many children 
as there are stars in the sky 
You should marry seven times 
...A2 91 • 
-130- 
I was very proud that I had cursed her as much as I could.
I was liberated by the Americans.
I waited at the door of the commandant, to 
see him being taken 
* 
prisoner.
The commandant and his wife came out, both wearing 
packsacks.
I told the Americans by sign language that the contents of the packsacks 
were mine.
The violinist, who was also there, said "This is mine" -- the commandant 
and his wife had taken his violin.
When I opened the sack, it wasn't only the flute 
that I found.
My wife's ring was there, first of all, and a lot of other jewellery.
Who knows to whom i* had belonged.
And there were American dollars too."
I said "Good for you!"
I was very tired.
They didn't want to let me go.
0 I 
said "Do you remember the song we sang when the bride left her father's house?"
"Yes, the bride had a special farewell song," they answered, "but we're going 
to sing another song from Leon Botton first, called the Matchmaker: 
A si biva Han Liathi 
Bar mi nam 
A si biva yo con el 
Bar mi nam 
Qub mm bouchkes oun noviezico 
Bar mi nam 
'Club sea a mi plazer 
Qub sea a mi plazer 
Non mb mires qub sb bachica 
Bar mi nam 
Tengo agnos en couti 
Bar mi nam 
El boye de mi estan cazada 
Tienen Oos al charchi 
Tienen yos al tsarihi 
Ouna ya sola tingo 
Bar mi nam 
Mb la yaman tendjere 
Bar mi nam 
Couando salt a la pouerta 
Bar mi nam 
Mb la azen kipazel 
Mb la azen kipazel 
Mi querido ese alto i vano 
Bar mi nam 
I ouna vara de espander 
Bar mi nam 
Mi madrb izo colada 
Bar mi nam 
Lo mitio a dentener 
Lo mitio a dentener 
W^ 44,-Ael-krari cinninn +hic cnnn_ and T cAid "Now we'll sina the brides@ bride's farewell 
-131- 
farewell song."
It was called 'Kedavos Embonora': 
Si ves ke mm vo 
Sola por oun camino 
Rogar en el dio 
Ke non tinga enemigos 
Kedavos en bouena ora 
Ke yo ya mm vo ayir 
Si ves ke mb vo 
Sola por ouna kaleja 
Rogar a el dio 
Ke non mb caiTga teja 
Kedavos en bouena ora 
Ke yo ya mm xp vo ayir 
I went to the dining room) feeling very pleased with myself.
I saw Col.
Sheppear 
again.
He asked me if I had ever gone to military school, I said no.
"But what you 
have done here tonight is exactly the basic step that they teach us in military 
school.
The basis for being a good officer is to keep the soldiers happy and to give 
them good food."
I saw a light on in the kitchen.
My cook was there.
I said "Kosta, go to sleepl 
We have to get up at five o'clock tomorrow morning:" He said, "You know, miss, you 
haven't eaten all day.
I made you some eggs and chips."
I thought it was very nice o 
him.
-132- 
My work was both physical and mental.
I thought only of how to give the best 
food, and of how to raise morale.
One day I was very tired.
I had been going from 
one end of the kitchen to the other all day.
I thinked I walked many kilometers.
The bungalows were full of refugees.I was on my feet from five in the morning to 
nine at night.
I finished my work many hours after my group had had supper.
Again 
the cook fixed something for me to eat.
I ate and went up to my room.
The nurses 
were inside and the doctor was in my bed.
I came in.
The nurses told me that the 
doctor would sleep in my bed that night.
They had opened the doctor's bed in the 
hallway for me.
I wanted to get my nightgown.
The nurses, together, said "We put everything you 
need on the bed outside."
I was so tired that I went to bed.
But I wanted to pull 
their hair out.
The next day, I had my room back, and I didn't say anything.
A few 
nights later, the same thing happened again.
Usually the cook placed warm water in the bathroom so that I could wash up in 
the evening.
Everyone in the group bathed on the road to Koula, near the Bulgarian 
frontier.
There were warm water springs there, and they would go to bathe every day, 
rain or shine.
But I had to feed the refugees three times a day, and I had no time 
tp go with my group.
That's why I used to wash up in the bathroom at night.
A week after the incidents in my room, I had finished my work and changed into 
my bathrobe.
I went down to the bathroom to wash.
Again, the doctor was in my bed, 
talking to the nurses.
Again, they had opened my bed in the hallway.
In the mor-
ning I got up very early.
The doctor was still asleep.
I couldn't go to my room to 
get my uniforii.
I went to the kitchen in my housecoat.
The cook asked me if I wasn't 
feeling well.
I gave the orders in the kitchen.
Some refugees were supposed to leave 
the camp that morning.
I went out, without even combing my hair, in my housecoat.
I 
was mad.
I watched for the doctor to gut up.
As soon as he did, I went to talk to 
I 
him.
I said "Look here, Mister.
Even though yam don't belong to any miflaga, I have 
the right to my own corner.
I want you to know something.
When the Sokonouth sent me 
here, they didn't ask me if I belonged to any miflaga.
Do you expect me to go to your 
-133- 
"Oh," he answered, "the nurses are driving me crazy, trying to convince me to 
take your room."
"If they want you to have your own room, they should give you theirs.
Why didn' 
they give you Esther's room after she left?"
The doctor never took my room again.
We 
remained friends after this incident, but I kept my distance from him.
I always had hope that some member of my family would turn up.
But that day ne-
ver came.
I was very happy to see Jewish people arrive.
Every night after my work 
was done, I went to the bungalow where the Jewish people were staying to talk, and to 
raise their morale.
But this wasn't enough for them.
They wanted me to give them 
passes to leave the camp at night, to see Siderokastro.
I explained to them that it 
wasn't up to me, that I wasn't the camp commandant.
"Well, if you don't want to do 
that, go to Siderokastro and buy ws some ouzo," they said.
They proposed that .I go 
Col.
Sheppeard or the doctor to ask permission, or just buy them some ouzo when I 
went.
I explained that many of the refugees, non-Jews, asked me to do the same thing.
"You didn't do it for them, but you can do it for us," they said.
I said "If I do it 
for you, I'll only bring antisemitism to this camp.
If you want, go to my superior.
If he says yes, he'll give me a car, and then I'll go."
I did everything I could for 
them, but they were very disappointed just the same.
The Greek government started to reorganize the armed forces.
It was natural fore' 
them to ask us to gime up a few bungalows for the army.
We would keep the refugees i 
the camp for a shorter time.
One day we received notice that many refugees were to 
arrive, The menu was salmon.
I told the cook "They811 be very disappointed to have 
salmon the first day; they're Greek people, and they'll have no oil to eat.Maybe we 
can make cutlets."
The cook said "Do you know what a job it would be to make cut-
lets for so many people?
We need at least three cutlets per person!"
I said "I'll hE 
you.
We'll do the cutlets; I'll prepare them, and You will fry them.
Let's open a 
box to see how many there are, and how long it will take to cook.
When the refugees 
arrive, they go first to the disinfection room.
By the time they are out, we will 
have the cutlets ready."
The cook said "We need'breadcrumbs."
I said "We'll use 
flour instead."
A box was enough for four people.
We figured out how many boxes we'o 
-134- 
nedd.
I was timing the cook, to make sure the cutlets would be ready when the refugees 
came out of the disinfection room.
We tasted one; it was delicious.
And we started 
to open the boxes.
We heard the trucks coming, and we started to hurry.
Suddenly 
the disinfection officer appeared in the kitchen.
he said "Come and help me.
Thete's 
too much worfl, and there isn't enough water in the camp.
You call the colonel of the 
military camp and tell him to tell his soldiers nat to use the water because we need 
it here."
"Only Colonel Sheppeard can do this, and not me," I said.
"Col.
Sheppeard has 
his own interpreters, and he can do it."
"No, he doesn't want to do it.'
"If Col.
Sheppeard refuses to do this, I'm sorry, but I will not do it."
He saw the cutlets, and he took one.
"Oh," he said, "these are delicious.
Well, 
come and help me put DDT on the refugees."
"I'll go and help because I want the refugees to eat the food when it's ready," I 
answered.
"I'm not obliged to go.
My job is not disinfection.
Whenever I go, you 
are not at your post.
And you have a staff to help you.
Look.
P1 the drivers are 
in the dining room, as well as the extra man with our group.
Ask them to help you."
He answered "Instead of making delicatessen for these Sepharadim bastards, come 
and help me."
"Get out of my kitchen before I kick you out:" I said.
"You are menouvelette (irresponsible)", he said.
When he had said 'Sepharadim', 
I thought "Oh, there are Jews among these refugees."
IXNXXXXXXXXXNXIXIN4 I was eager 
to serve dinner, to see who these people were.
I was still hoping that some member 
of my family would one day turn up, but this never happened.
Col.
Sheppeard was born in Australia.
He was a very nice person, and very sweet.
He had been a career officer in the Australian army.
He knew how to give orders with-
out anyone realising that that's what they were.
He was a good comrade, honest and 
sympathetic.
The officers of the Greek camp came to Col.
Sheppeard's office often.
ne aay 1 went o his o ice.
en en ere 
-135- 
there.
Col.
Sheppeard was about to introduce me.
Suddenly, the officer yelled 
"Maritsa!
", and we were htgging and kissing each other.
It was the brother of the 
owner of the barn in Evia.
Col.
Sheppeard didn't understand what was happening; he 
knew my name was Bouena, and yet this man was calling me Maritsa.
The man said "We 
didn't know if you were still alive."
I said "When I came to Athens from the middle 
east) I went to Evia to see your brother.
But I didn't find him.
The farm was shut 
down."
"This farm was not ours," he said.
My brother only rented it during the war, 
Everyone called us aphendiko.
We didn't tell anyone our name."
I asked for his bro-
ther's address.
He said "One day in the mountains, my brother fell off a horse.
He 
died instantly."
I was very sorry.
There was a moment od silence.
Then I said to 
Col.
Sheppeard "It is a very long story.
I haven't the time now to stay and tell it 
to you.But this officer will explain everything.
The refugees are leavibg today, and 
I'm not ready to go with them.
If there's another car going to Salonica tomorrow, 
I would like to go for the day.
I have some business to attend to."
"I know," Col.
Sheppeard said, "that it's not very easy for you to go to Saloni-
ca."
The officer said that two days from then he was going to Salonica, and he 
would take me.
Two days later, there were no refugees in the camp.
Col.
Sheppeard 
gave us his own driber and a jeep, and I went to Salonica with the officer.
Before 
we left, Col.
Sheppeard said to him "Make sure she has a good time.
She needs it."
In Salonica, I went to the hotel, and the officer went to see some relatives.
Tim 
hours later,,he came back to take me to an open air cafe to dancee We had a really 
good time.
The next day the officer came to the hotel again.
I was ready to leave.
I said "Look.
I didn't come to Salonica just to have a good time with you.
I came 
to go to the convent of Ville Marie.
They have a little Jewish girl there, named' 
Sarica Leah.
I came to, get her."
"Where will you put her if the nuns let you take her?"
he asked.
"The Joint Distribution Committe has an orphanage in Athens for children like 
her," I answered.
The Colonel's jeep was at my disposition, and the officer said 
he wanted to spend his holiday with me.
We went to the convent, and we saw the 
-136- 
Mother Superior.
I told her "You have a little girl hefe$ four or five years old, 
called Saruca Leah.
She is my cousin (this was a lie)."
She said "She is in the 
infirmary.
She was a little sick yesterda."
She took us to the infirmary to see 
Sarica.
She was very sweet, and she had been WRIai well looked after.
She kissed 
me.
I had brought candies which I gave to the Mother Superior *o give to all the chil-
dren.
She said "If you come tomorrow, I will let you take Sarica for a walk.
But 
you have to buy her some shoes.
She needs them."
When we left the convent, the driver told us that he had called Col.
Sheppeard.
There were io no refugees in the camp, and I could stay in Salonica for a few more days.
The next day, Sarica was waiting for us, looking through the bars of the gate.
She 
said to all the children "This is my cousins This is my cousin( You see, she has a 
star, she is an officer:" I took her into town and I bought her a pair of shoes.
The next day, Sarica was again waiting for me, with the Mother Superior.
The 
Mother Superior said "I know you have no other relatives.
If you want to take Sa-
rica for good, you can."
I took Sarica and I went to the office of the Joint Dis-
tribution Committee.
I told them that I wanted to take her to Athens myself, to stay 
with her for a day to help her get used to the atmosphere of the orphanage.
One of 
the nurses from our camp was there.
She answered me instead of the Joint Committee's 
representative.
"The Committee has its own people to do this kind of job."
I left 
0 
Sarica, who was screaming and crying, and I went to Siderokastro, back to the camp.
My thoughts were only with Sarica.
Sarica was sent to Athens, and from Athens to 
Palestine.
Today she is a very happy wife and mother, living on a kibbutz.
A week after I had gone to Salonica, the brother of the owner of the barn told 
me that the Greek army was going to hold a ball.
He asked if I would accompany him.
Col.
Sheppeard asked me to be on the committee organizing the ball.
I was told 
that everyone in my group would be invited.
But the people in my group refused the 
invitation.. As I was getting dressed to go to the ball, one of the nurses came in 
and ordered me not to go.
I said "I accepted the invitation, and I am going."
We arrived at the ball, and they asked Cdtl.
Sheppeard and me to start the danc-
ing.
The band started to play a Greek national dance.
They gave Col.
Sheppeard a 
handkerchief, traditional to this dance.
Col.
Sheppeard didn't know what to do.
I - 
took the handkerchief and started to wave it.
I gave the signal to the orchestra.
I started to dance, and Col.
Sheppeard ran away, but all the Greek officers came to 
join me.
The brother of the owner of the barn shouted "Yasso vre Maritsa!"
("Long 
Live Maritsa").
The atmosphere of gaiety lasted until four in the morning.
They 
held a lottery, and each officer who won gave me his prize.
By the end of the ball, 
I was weighted down with prizes.
We had coffee, and said goodbye, and it was six 
o'clock by the time we got back to our camp.
My cook was already in the kitchen.
I went upstairs to my room, changed into my uniform, and went to work.
A few days later, the doctor went to Athens to attend a conference with the 
chief of the Palestinians.
Our headquarters were in Athens.
When he came back, he 
brought three coats for the girls, one for each of the nurses, and one fpm me.
UNNRA had given them to him for us, because Siderokastro was cold.
The coats we had _ 
been given in Athens weren't warm enough.
The new coats were windbreakers, very 
light and beautiful.
X The lining was imitation fur.
It was the warmest coat I 
ever had.
The doctor gave mez my coat and made me sign a note that I had received it.
I 
took the coat and marked my name on the label.
A few hours after we received the 
coats, one of the nurses spilled a bottle of castor oil on hers.
I was on my way 
up to my room when I saw a shadow coming out of it.
As soon as I entered, I saw my 1 
coat on the bed; I knew that I had hung it up before.
I picked it up, and I immedi-
ately saw the stain and smelled the castor oil.
I went across to the nurses' room 
-138/ 
right away.
The nurse was in such a rush to go in that she hadn't shut the door 
properly.
I pushed it open.
I had the coat in my hand.
I said "I'm sorry, but 
this coat is yours.
The coat you are holding is mine."
I took it from her, and 
showed her my name on the label.
I left.
Needless to say, from that time on, we 
were colder towards each other than before.
A few days later, the extra man in my group asked me for my coat.
I said "If 
you want a coat like this, go ask the nurses.
They will give one to you for sure; 
you are good friends with them."
I said to myself "I have been intimidated long 
enough.
These are just nurses, not Germans," I was very pleased with myself: I 
had started to take courage in life.
I went to my room, and I started to remember that Regina had an exhibition of 
handmade goods in the Italian school in Salonica.
Many people came.
One of the 
ladies asked to buy her work, but we refused to sell.
The first day that the Ger-
mans arrived, this lady came to our house with a German officer, with a list of the 
goods that had been in the exhibition.
They also took all the valuable paintings 
that we had.
Two days after the incident with the coats, the refugees in the camp were ready 
to go home.
Since there were no other refugees in the camp, it was decided that I 
would escort them to Salonica.
Since there weren't too many of us, we would go by 
train, and not by truck.
I asked the doctor how many people there would be, so that 
I could prepare rations for the trip.
He said "You go rest, because you will be 
travelling at night.
I'll tell the cook what to prepare."
Sc I went to rest.
I went tci the kitchen when I was ready to go.
The cook had one sack of con-
serves, and he put it on the truckx that we would take to the station.
I asked the 
cook why he had done this, since usually we distributed the food at the camp before 
people left.
The doctor answered "This time i you will distribute them at the sta-
tion."
We arrived at the station and I gave each person his ration.
I gave some to 
everyone, but the sack was still three-quarters full.
The truck hadn't waited a.
41390 
With, me on' the.train.
i 
fter used to 'send the refugees to ,Saionicas and, from there they Were -sent ie .
all 
parts`' of Greece i- to their honet• There were some German women among'them to had 
marred' Greeks' ini GerManys' and had children* These women end Children had no visas 
• te: enter' Greiecei!
VieF sent' these people to Salonica Where' they were put in -a special 
camp.
The toile vzolierr' we're • placed in , the dining, room and in the forger rooms of Ger•- 
man .Officere- in Pavlo .11ellas • the - prison ittere I• was held after4heprana the •Germans 
• killed Chain.'
• 
• 
.t: 
.was ' already 
fall when I took 'the train with• the refirgeess bat on that day it 
:was Very .very Wars* I 'put my uniform jacket ons but no coats It Was so walls 014 
I -took my Jacket off.
The train -came into the station* , The,conductor got off the 
train and , gave' us -Instructions* The Jewish people get -in.
the.
open.wagen*,• The ..noir* 
'Jews get.
in 'the regular •train•" • i •protesied•: • He replied "The train will leeye in* 
one 'Minutes' and you- will 'stay' here*" hart no truck to take us' back to -the camp.
* I .
was'Obligedd to tell•the 'Jewish 'people to get into the open Wage% :and.
I got in 'with 
their • r'• There were ..no' seats, so .we , sat on the dirty floor*.
When the train.
started tee 
may's' %viao,like being in a ,refrigerator*1 Each refugee .
had a blankets •and theiy..cov.01 
ered'thesiselvet*:, But I only.
had my:jacket* I •pulled my beret -over my ears.
'A few kilometers 'from Salonicas the train stepped iri.
a small station•- I got off 1.
• the-train with my' eacipsi and I went to talk.
to 'the 'conductor's .z "We are in a, refrigerw' 
atoll!
All these peeple• are going •to' be sick; myself inclildede.
• The train started 
to Nove .-I was going to be left behind with' my' sack* There were 'no stair's oh the 
open wagons.
you- had to - jump up• • I went to the regular wagon with my sack instead.
• When- I arrived in, Salonicas, there.
was an- officer- from UNNRA there to greet .
see.
4 
He' took.
my 'sack and put it ,in his.
jeeps and he said "We prepared a' room for you;: in 
• the camp .mere the German Women are staying."
I didn't pay attention.
All I wanted 
watt° be ins a warm place.- • When I came to the camps I realised where I wile.
* PtiVlo 
Hellas: the prison where I was -detained after thaini was killed* ..With my sackfull 
ittiensa 1•1felt :more likez •refUgee*.
I -went 'te my = item and .put my seek down on 
'chain ,' This roots:had been• the.
interrogation groom of the Germans* 
-140- 
The bed in the room was very clean, but it was impossible for me to sleep.
I thought of the experiences I had had in this room.
Suddenly, I said to myself 
"I have to have courage: I think being in this room w.t in a Jewish im i uniform 
with a Magen David on my shoOlder is like sitting spitting in the face of the Ger-
mans.
I'm going to sleep well tonight: I don't have the agony I had here before.
Every night the Germans killed five prisoners.
Every night each prisoner thought it 
was his turn.
Every night eackxyaxismmex we could hear the machine guns and the people 
screaming "Elefteria:" ("Freedom:").
I won't be among the five tonight: All I can 
hear tonight is a baby waking up and crying."
And I slept.
Mothers and children slept in the dining room.
In the middle of the night, the 
sack of rations fell from the chair.
With all the conserves inside, it made a very 
loud noise in the silence.
The children all woke up and started to cry.
I thought 
the Atom Bomb had been dropped.
The night for me was never-endings and I remem-
bered my whole life.
I thought "Hitler told the truth.
I will liv4tettserable 
as I am."
The next morning, the officer came and took me to have lunch at the hotel.
He 
took me to Siderokastro with my sack.
The doctor was in the kitchen when I brought 
the sack in to the cook.
I asked the doctor why he had given me more rations than 
there were refugees.
He said "Oh, we only wanted to play a practical joke on you."
I said "If you knew what life was about, you wouldn't make a joke out of extortion."
protest to 
I told him to prokamil the UNNRA because the Jewish people had been placed in an open 
wagon.
I never found out if he did or not.
Later, when I would ask him if he had 
protested, he would say "O.K., O.K."
I left the sack with the cook and I went to my room.
The nurse was standing 
in her doorway across the hall, and she asked me how I had passed my trip.
I said 
"Very nicely, thank you."
She said "I told you that you would regret taking the 
coat.
This was nothing compared to what we could do to you."
I answered "In this 
world, I lost my brother, I lost my sister.
i I lost my love, I lost my relatives.
I lost my friends, I lost my neighbours.
I lost my house, I lost my belongings.
But you can be sure I will not lose myx this coat."
I went to my room.
I had a 
-14.- 
terrible earache.
The next day, the other nurse asked me to give my coat to the doctor.
I ans-
wered "If you want the doctor to have a coat like this, give him yours."
I went to 
doctor and I asked him to look at my ear.
He said that he was going to Athens, and 
he would look at my ear when he returned.
Instead I asked Col.
Sheppeard's driver 
to take me to Drama to any doctor there.
I found a doctor in Drama, and he gave me 
drops for my ear and some pain killers.
-14L 
Fewer and fewer refugees were arriving at the camp.
My work became lighter, and 
I started to eat in the dining room with the others.
One of the drivers liked mm to 
cook, and he cooked for us.
The first day that I went to the dining room, one of the 
nurses said "Now that you have time you will come to work with us in the infirmary."
-There were four patients in the hospital, attended by two doctors, two nurses, and a clean 
ing woman.
The cleaning woman Is fought with the nurses and they had given her notice.
I told the nurse "It's true I don't have much work now.
But who will do the little 
that there is?"
She said "You'll work for part of the night."
I said "You work one 
part of the day;" And I didn't go to work at the infirmary.
Every day everyone but me would go to the warm springs at the resort to bathe.
had to distribute food to the refugees three times a day.
But this day I made up my 
mind to go, and I did.
It was like a swimming pool inside, with beautiful warm water.
The 
back of the building was in ruins; the cook told me it was from an earthquake.
This was 
the first time in six months that I bathed like a human being.
A month later, there were even fewer refugees, and the camp would soon be closed. 
)
My comrades went either to Salonica or to Athens.
I stayed in the camp with Mr.
Garfinkle 
to close it and to hand emu it over to the Greek authorities.
My ears were still hurt-
ing me very much.
When I finished my work, I was ready to go to R Salonica.
But before I left, I went 
to see the wonderful work that UNNRA had done to rehabilitate the tobacco f growers in the 
region near Cavalla.
These people had been ruined by the Bulgarian occupation.
UNNRA 
provided the people with houses, and with storage places suitable for tobacco.
Mr.
Sibly, 
director of UNNRA and a wonderful humanitarian man, used to say "It's very good to give 
food to the hungry.
But we have to help them to help themselves."
Mr.
Edward was in 
charge of this project.
He was an American Negro, very sympathetic and well educated.
Everyone respected him.
In this project they had also built latrines in the fields, 
but I saw only pictures of these.
1 
The first day I came to Salonica, UNNRA gave me a room in the first class hotel.
few of the pills that the doctor in Drama had given me 
T awnke.
I saw that I was 
-140- 
sharing my room with one of the nurses.
The other had gone to Switzerland to visit rel.. 
atives.
I got dressed, and without Ein even having breakfast, I went to see the guardian of 
my estate.
The first thing I said to him was "I want a house."
He asked me which one.
said "I want the house we used to live in near the Monasterlis synagogue.
I want it 
because there are a few Jewish people living there.
And also, I think that I have my 
last memories in this house."
He went to the lawyer, and applied to the court.
In the meantime, I would stay in 
the hotel.
When I left my representative, I saw a familiar face in the street.
"Oh, Miss Sarfatty$ how are you?"
he asked.
"You don't remember me?"
He told me 
his name, but it didn't mean anything to me.
He was a Jewish depottee who had passed 
through the camp at Siderokastro.
He invited me to an outdoor cafe.
We sat down and he, 
said "I would like very much for you to meet my wife."
"Oh, your wife came back?"
I asked.
He said no.
"Oh, then you remarried?"
No.
I didn't know what to say, and the man started to explain.
"I met a girl who also came from the lager."
"If you love her, marry her!"
I said.
"It's a little complicated," he said.
"The girl is pregnant."
"Pregnant women also get married:" I said.
"The child is not mine.
A Russian soldier raped her."
To give myself time to thin 
of what to say, I ordered another lemonade.
He continued.
"I don't mind that this baby' 
isn't mine.
But it won't be Jewish!"
This gave me an idea 
"Look," I said.
"If you love her, marry her.
According to Jewish law, if the 
ther is Jewish, so is the child.
It is considered one hundred percent Jewish.
If you 
don't believe me, go to Rabbi Molko.
He is a very nice person, and he will help you."
He answered "Many couples who came from the lager now live together.
Fifteen of 
the women are pregnant.
Rabbi Molko talked to us, and told us we should marry.
He told 
us to have just one ceremony; all fifteen couples should marry at once.
But I'm the 
only one who has some complications."
I repeated what I said.
"If you love her, marry her.
There are no complications.
.4414 
The baby is Jewish."
Later that day, I went to court.
My ear was hurting me terribly.
I was given two 
rooms in the house I wanted.
The first night I came to the house, the people who lived 
upstairs started to scream at me, to make me move.
But I went to my rooms and shut the 
door.
All I found in the house was a picture of my friend Sarica Florentin.
The rest 
of the walls were empty.
imdsextamedxsxkleek*xikekRia I also found a copy of the plays 
of Molibre.
Half an hour after I arrived at the house, all the pregnant girls came to see me 
with their boyfriends.
They told me that everyone had agreed to marry.
I was glad.
Theyx told me they had made arrangements with Rabbi Molko to be wed in the hall of Mata-
noth Laevionim.
All the girls wanted to be wed in white gowns.
I said "That's up to 
you:" They had spoken to the Joint Distribution Committee, and the Committee had agreed 
kp to pay for the rental of the dresses and the veils.
We needed refreshments for the weak 
wedding.I asked who would pay for the food for so many people.
They answered that the 
Joint Distribution Committee had agreed to supply the ingrediehts for cakes and so on.
fkiwww The girls would prepare everything themselves.
A few days later, the Joint Distribution Committee approached me to ask me to 
supervise this wedding.
I accepted immediately.
A few days before the wedding, the girls 
wanted to decorate the hall of Matanoth Laevionim, because it had not been painted for 
a long time.
The Joint Distribution Committee agreed to give us as much crepe paper 
as we wanted.
When we finished, the hall of the Matanoth looked like a dirty circus.
But I let them do whatever they wanted.
All the girls were very happy.
They had never 
dreamed of getting married in the Matanoth, because they were not of that milieu.
The day before the wedding, the Joint Distribution Committe gave the girls everything 
they asked for; flour, eggs, vegetables.
I came to the Matanoth very early in the mor-
3 
ning to supervise the work.
One of the girls was going to brek the eggs, another was 
going to beat them, and so on.
They had asked some other girls to help them.
I told 
the others to clean spinach for spinach cake.
Some had to prepare the eggplants.
Others 
:1 had to cut up the cheese, and so on.
Suddenly, three deportees appeared and said o the yi 
II C" 
r : 
-141p-
"Look.
These girls have been dreaming of marrying in Matanoth, and having a party 
They invited many people.
If we don't work, we won't be able to serve them anything.
The eggs are beaten.
The eggplant is baked.
If we don't finish, we'll have to throw 
everything out."
"This wedding is only propaganda," they said.
"The people of the Magen David 
will be able to send a picture to Palestine and say 'Look what we are doing!'
The 
Joint Distribution Committee will be able to send a report to England and to America 
to show what they are doing."
"The Joint Distribution Committee was very nice to us," I said, "and the Magen 
David has had nothing to do with this."
They said "the Joint Distribution Committee gave us food.
The Magen David sent 
us illegally to Palestine."
"What's wrong with this?"
I asked.
"We still sleep on the floor with just a blanket, at the Orphelinat Alatini."
One 
of them said to me "I sent three children to the gas chamber.
My wife and my father 
and mother also went, because I was honest.
I refused to become one of the Jewish Po-
lice and co-operate with Koretz.
But Koretz went to the privileged camp of the Germans 
with his wife and children.
He died of typhoid in the camp, of natural causes.
His 
wife and children came back to Salonica.
The Joint Distribution Committee didn't place 1 
them with us in the Orphelinat, to sleep on the floor.
They are in a better place.
Is this justice?
Don't work, girls!
The Magen David and the Joint Committee want to 
repair the Pinkas Dispensary.
They will spend a lot of money on it, and for what?
There are no Jewish people there, and it's so far away!
The people in need go to Pales-
tine.
The others have never gone to the dispensary, and never will.
It is too far.
This dispensary used to serve three districts.
It filled hundreds of prescriptions ever 
day.
For two or three prescriptions, they want to open the dispensary again.
They gave us food.
We don't want charity.
We want rehabilitation!
People from 
the Joint Distribution Committe go to the cemetery.
They take pictures of the graves 
that were levelled with bulldozers by the Germans.
They send the pictures with their 
reports.
But nobody tells them that it wasn't the Germans who did this.
Koretz, long 
1 
before the war- OAVA 
Greek citizenship.
Long before the war, Koretz removed the remains of the biggest 
rabbis from their graves.
All the Jewish women went to Ziara to pray at the empty graves.
When someone was sick, they prayed there for him to get better.
All the brides' mo-
thers went there to pray for mazel for their daughters.
My father was a rabbi, and when 
he saw this, he said 'The Jews of Salonica are finished'.. This was the traitor, 
Koretz."
The man was crying.
I said "Look.
We have to be proud of the director of the Joint Distribution 
Committee, Freddy Cohen.
He is the most wonderful man I ever met.
Freddy is the son 
of Hamaki Cohen.
His father was a lwayer and a member of Parliament; he was a special-
ist in Jewish affairs.
This family was good friends with the royal family.
We have to 
be proud of this family.
Freddy's father died before the occupation.
One wing of the 
royal palace in Athens is for the princesses.
There were few princesses there during 
the occupation.
The Geriians respected this wing of the palace.
The Cohen family was 
hiding there.
Tilde, Freddy's wonderful sister, took .a large valise filled with wool 
with her.
She knit day and night so that she, would have presents to give to the work-
ers in the palace.
I knew Tilde very well, a most distinguished, educated, humanitar-
ian lady.
Now Freddy Cohen is the director of the Joint Committee.
When we were liberated, 
he was responsible for bringing us mobile synagogues from England.
He does everything 
he can to please the deportees, because his roots are humanitarian.
Mr.
Cohen is do-
ing us a favour by remaining the director of the Joint.
He is a succesful lawyer, and 
he doesn't need this job.
We should be thankful to him."
They said "The booses are the doctor and the nurse."
I said "Not to my knowledge.
Now let's prepare for the wedding and let these girls 
have a happy day.
Let these children be born with names."
"Oh, no," they said.
"The girls are going to leave.
The Joint will see all this 
thrown out.
You will tell them why, and they will do something for sure."
"You're mistaken," I said.
"I will say nothing to the Joint.
Anybody who has com-
plaints should go to Mr.
Cohen, who is a wonderful person.
He will lidten.
If you 
~.
Mr.
Cohen, and don't tell your complaints to me."
Everyone left.
But before they went, the boy said to.
me "Throw everything out!"
pleaded with them not to create a scandal at the wedding the next day.
"If you create 
scandal at the wedding, do you think you will gain anything?"
They left, and I was alone with the beaten eggs and the baked eggplant.
I couldn' 
believe that they wanted me to throw all this out.
These people were about to die from 
stark starvation before they were liberated„ and now they were telling me to throw out 
all this food.
I put my hands up in the air, and I said "Oh, God, send me the strength and I will 
prepare everything myself, and throw nothing out.
I know what hunger means."
And I 
started to make the cakes.
At seven o'clock in the evening, all the cakes were prepared 
and I started putting them in the oven.
The kitchen was very warm.
ihmxkitekommaKI:x 
ximy I left the cakes in the oven, and I went into the salon.
I looked at all the 
decorations, and I started to remember how elegant this salon had once been.
I remembered my coming out party.
I was 18 years old.
Eliaou took me to the ball 
of the KKL (Jewish National Fund).
I was wearing a blue evening gown, with flowers on 
it.
The flowers were handmade from the same material as the dress.
The dress was made 
of natural silk.
When we presented our tickets at the door, two ladies handed me a 
'carnet de bal'.
One of the ladies tied it to my wrist with a blue ribbon.
When we 
entered, we saw all the debutantes with their fathers.
There were many young men near 
the door.
All the men were wearing tuxedos.
As soon as I entered, one of the young men approached Eliaou.
Eliaou knew him; 
his name was Elio.
He wrote his name in my carnet.
I was very curious about which 
dance he wanted.
As soon as we sat down at a table, I looked in my book.
It was the 
first waltz.
I was so disappointed: Elio was very fat!
How could a fat boy like that 
dance the waltz?
Eliaou introduced me to another young man, Chaim.
Chaim took the carnet de bal 
from my wrist.
A few minutes later, the ball was opened with a Strauss waltz.
Elio 
came over, and we started to dance.
Surprisingly, he was the best waltzer in Salonical 
He was flying, and so was I. I had never had such a pleasant partner at a ball before.
iI 
When Elio took me back to my table, Chaim returned my carnet de bal.
He had put