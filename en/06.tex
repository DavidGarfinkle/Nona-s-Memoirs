% mempages 111 to 124
% pdfpages 121 to 135

ISerial 
No.
663 
Military Permit for Exit, Middle East, and Entry into 
Country > 
I Ct F.F:c E 1 Country 
Provided the normal travel documents ( passports, visas, etc.)
are in order the undermentioned is .hereby 
authorised to proceed to the territory shown above.
Rank and Name .
v S R I. 
Business, Employment or Status 3. fit.
S.
".37.1.N 4.19 N .
71.• V‘I Pi • k F 
Duty on which proceeding .
.
W at-P R ItE NnfottK 
Duration of stay Pk- twt 19 N ks.i4T 
Destination on completion of duty 
• 
Particulars of authority for entrance : Signal No.
ea 614.
Cite From Ct REEKtilt,... To.
11 t.pRAST Dated ttm" 146.
Stamp of the Adjutant General's 
Branch, 
•eiti 
G.H.Q., M.E.F.
0 ti les • 
--t 
-- • Ar.V.
Tr- 
•-• 
6091miGHQP/3.000/5-45 • 
CERTIFICATE OF IDENTITY 
This will serve to identify Miss Buena Sarfati, member of 
the PJCRA voluntary society, under the administration of 
UNRRA at the UNRRA Asseldbly Camp, GHQ(S),,M.E.F., pending 
receipt of formal identification d 
• 
June 18, 1945 
ow.
4..ORM No.
EI-5 
• MAY 0, 
UNITED NATIONS • • 
,RELIEF AND REHABILITATION •.- 
.ApiviINISTRA!kTION• •-: • 
AVEL: AI THORIZ:ATI 
DA1 17th June1 1945 
1RAVELER 
Miss.
B.Sarfati 
• TITLE Interpieeter 
.
• 
• OFFICIAL STATION Pales e!
,UNIT ,• for *Relief: AbroaiM'attache d• 
to' ,UNRRA GreeCe5, Sin 
PERSON 
limitio!oREIN IS ACCDFIDANCt41k 
REGIIE06-00.t:*;; 
1HE' UNITED 
44D-1;8E1413.I.91.
.
:ADMINISTRATION, _S ;BJECT TOIT:NE 
coribitiONApt.E0-BECOV.
:;'1g DATE.
OF.
iit AO-fiiiim4- 60107(dr4t4010..adittli 
.
• 
.
.
• 
'.
PEI< DIEM IN LIE..1.;.0F.WBgISTENCE 
• s uppor e receipts 
ESTIMATED COST 
TRANSPORTATION 
.
PER DIEM 
INCIDENTAL 
TOTAL.'
„ 
Air Passage.
authorised.
NOTATIONS • AMOUNT .
TRAVEL E 
hisolg 
chieept,.
v 9 
g1:1:e,- 1 Office."
.
* :.‘ 
4611v441.0N 6,tik‘ 
.
.. 
... 
- .
!
- ' • ""t* tow" r 
-.'
.
: ;-• 1-4,,fie'-;.
: ; 
-- " • - 4 • '' .. ---,*:.•.er 6<" '' 
• 
• .
.- 
- 
. '
'''''':''44' 4 ".
42,...'..-.
tc.
,...L.:4,.... '' r , ... .
.
k .
r ' ' 
COPY 
Mb 0 
Ref: MERCAD 
::•*" 
1213?
.
AIR MINISTRY AIR MOVEMENT.
WARRANT :(civil) 
To: Combined Air Booking Centre 
British Overseas Airways Corporation 
B.A.F.
Transport Command 
Copy to: Passenger for retention and action as overleaf.
British Overseas Airways Corporation.
R.A.F.
Transport Command.
Air passage is authorised and priority recommended on behalf of the Middle East Air Priorities 
Board for the person(s) below to make the journey indicated and competent authorities are requested to 
arrange departure at the earliest appropriate opportunity by military or commercial civil aircraft.
1ority : 
• L.
• • 
Date: 
13th.
June.
1945 
Signature 
fonadd East Representative, 
•Air Ministry Civil Aviation Department, 
Office of the Minister Resident in the Middle 'Fast.
A. 
B. 
D. 
Passenger's name and., 
personal weight (lbs) 
Departure point 
Final Destination 
Original sponsor 
Fare chargeable to 
F. Ready to leave on 
G. Local Telephone No.
.,1 H. Reniarks: return priority 
: 'baggage weight, etc.
N 
— Miss B. Sarfati 
Cairo 
Athens' 
U.N.R.R.A.
** 
:--.1.6.:tila...0trae-a-94Y-- 23rd' June ,194‘5 
ext 7 Dar el Shifa, 
6Vx:rtu Zip to 120• lbs in.c4.uding camp kit.
• -- ...r"-4••••••:.
IN THEIR OWN INTEREST PASSENGERS SHOULD READ THE INSTRUCTIONS 
PRINTED OVERLEAF • 
rtEOYITS1.01.1 IIPONOIAZ 
KAI 
Tr/CETI: /s1TX/k trine 
r itarirron.. =At: axrialt 
.
tvi TrP01411i.
.r(rv"4 rtti .
; ..tut 
• 
O ; .
.
.
9 9 
ri n .
rl 0 
• I 
” 
• it: • ••• 
• 11 il• — - 
•r a s t • .
.
.
lyyCypcp,p11 oi • 6n' atiZ .
&pi 0.
• .......• • • • 'a .
:•2)• .
., — .
.
It.
; • • 
3) 4) 
.
1 
47• ;.. .
• • a • e • ••• • • • 'kat ' en al:Ivo uat v fl tyri s f 5 are 
.
.
'., 
• 
, .
• I 
• , • • ..... / 14 111 :4 ,44 f 0 • I • 1 l •C •1; 4 1 - 11 14 .1 -i ‘ j kA• .
.. .
.
A s ..0 .
..• -V V ; 
.
.
•,.
.
.
.
...... - , -.a- 
.. • ?
.
0 - !
• •_ / .
.• 
;- - P P T 1 s r et.
ITI i A 7 t 
a pot 0 ec pep arcocric xcer67ct v.... 
,...,.
Ch e fie h l 
9 
6,6$4 : Or 0 • • 1 i• I• 6 / i t •C i r•%: - N ?
.
O E* t t: 2 :: C -"C. t e'r t : 1 1 7C/ 2• 
.1... • • 
:').q .
• 
i .
• 
• 
H: 
19 '•• • 91 
•• •• •• 
• 
ts • 
.. • .. .
••• 
.. .
.
.
• t!
•.
:M"• • e•• • • .
mntpctotie 114 ot 
• 
va-11 xplivipcdatift:ki 
O 
TMEJATA PXH , 
r 4,47 
.1-#7 
JO/5 
65) 
• • .0 
• • • 1 11) °Ct ql/YO)V1 .
.44.4ati • • 0711141794 
FA V X af pcp • i.
.V.11./7k.toi 194 r 
• 
••• 
.. • • 
O 
• 
-111- 
The truck stopped, the door was opened, and I came down.
Just as my foot was 
touching the ground, an Arab dressed in a white robe came up to me.
He said "Selam".
I answered politely "Selam".
Suddenly I saw a pair of scissors.
He cut the strap of 
my purse and took it, and disappeared among all the other Arabs in white robes.
I was 
about to tell my driver "He was wearing a white robe:" But there were dozens of people 
there, wearing the same thing.
My passport and my refugee book were in my purse.
I 
was now without them, and without a penny, too.
I didn't go to see Mr.
Soula.
Instead I returned to our camp.
I didn't sleep 
all night.
I kept thinking about what I was to do, without a passport and without any 
identificationi The next morning, half my group took a plane to go to Greece.
The 
rest of the people were to go by boat with the trucks and the baggage.
I went to the 
Greek passport office.
As soon as I entered I saw a friend of Nicola.
As soon as he mmxxx 
saw me, both of us were so happy that I forgot all about my paaspoet.
"What are you doing here?"
I asked him.
"Oh," he said, "I prepare passports for officials returning to Greece."
"Michael, " isa I said, "you are machiak:" 
"Why?"
he asked.
I explained all that had happened to me with my passport.
He 
gave me a little piece of paper right away.
"I'm sure you have no money," he said.
"Oh," I said, "no."
"Take this paper into that room.
The photographer there will take your picture 
for free."
By the time I came back with my picture, the passport was ready.
Michael 
gave me axpasspx the passport and a few liras, so that I would have some money in my 
pockets.
He saw that I had nowhere to put the money; I didn't have a purse.
He came 
outside with me, and he bought me a purse.
I went to the place where the trucks stopped to go to Meady.
My driver was in the 
cafe facing this place.
I showed him my purse from far away.
He couldn't believe that 
I had a passport, sal I showed it to him.
"You see," I told him, "There is a God in 
heaven.
He helped me."
He said, "You know Miss Sarfatty, from this day on, I will 
start to believe in God."
-1 I 
-112- 
The next day I took the plane and went to Greece without having seen Mr.
Soula.
I had good memories of Cairo just the same.
We had gone on many excursions, we had 
seen all the ancient ruins, and the pyramids by moonlight.
We had visited all the Zi-
onist organizations, and I had met many people.
When I arrived in Athens, there was a station wagon on the airfield.
I asked the 
driver where the people from P.J.C.R.A.
were staying.
He said "I don't know P.J.C.R.A.
But I know Magen David."
The driver was Jewish.
He said "They wear uniforms just like 
you, khakis, with a Magen David on the apaulet, and the word 'Palestine'.
They also 
have berets like yours, with a medal with a Magen David carved into it."
The driver 
took me to a hotel.. I found that my k whole group was staying there.
A bed was ready 
for me, to be shared with one of the nurses.
The first day in Athens, I saw two cousind of mine who had reamained alive.
I 
asked tham about Sam.
He was the son of my first cousin, Rikuetta.
I knew that he 
was supposed to be studying in Athens at the Polytehcnic.
"Oh," they said, "he was 
saved from the Germans, but he was killed later in the revolution."
Of course, we 
went to the cemetery right away.
The next day my group went to put flowers on the grave of the Unknown Soldier.
There were twenty people in our group, and we were divided into two.
Half the group 
was to work in the region of Athens.
The other half was to work in the region of 
Salonica.
Three days later, we were in the trucks.
There was a big red Magen David on the 
door of every truck.
We passed through Larisa, but we found very few Jewish people.
We passed through Yanina, and this time there were a few more Jews.
In every place 
that we wnt through, we would meet with the Jewish people in the synagogue.
In every 
synagogue, there would be a speaker.
Everybody would swear never again to let the 
Jewish Police into their communities.
They wanted to prevent what had happened in 
Salonica, where the Jewish Police made sure that not many Jews were saved.
Everyone 
would swear "Not another Rabbi Koretz," 
We passed through Veria.
The trucks with the Magen Davids stopped in the Platea 
-113- 
toglou, was upstairs on her balcony.
She was looking at the trucks, but especially 
at the Magen Davids, with big eyes.
I changed into my cleanest uniform, and went 
to talk to her.
I was walking very tall, very proud to be a Jew.
riO I said hello to 
her.
When she saw me, I think she had a heart attack.
She said "Hah.'
You are a Jew?
:" I said yes.
She made the sign of the cross.
When she finished, she said "If I had known that you were a Jew, you wouldn't be alive 1 
today.
My husband was killed by the Partisans after the Liberation."
Her husband had 
belonged to the Fifth Column.
I looked at her, and I said "My condolences.
I am sorry that your husband is 
dead."
And I was truly sorry; I remembered what my mother used to say: "Kaye tou ene-
migo i non to alegres".
If your enemy falls, don't be happy.
I asked Mrs.
Moratoglou 
about Mrs.
Soula.
She said "Mrs.
Soule is in Cairo.
She went to join her husband."
I was very disappointed.
I had looked so much for her husband.
If I had found him, I 
would have seen her, too.
I went back to the truck and sat down.
I made a mental estimate of the Greeks.
I think that this woman was the only one who was not happy that I was alive.
We finally arrived in Salonica.
When we entered the city, I thought we were enter, 
ing a big morgue.
The trucks stopped in the district where the Soupe Populaire of 
Matenoth Laevionim used to be.
I can't describe the way I felt, sitting in that truck.
; 
ibex We arrived at our Hotel.
Everyone went out onto the balcony to see the beau-
tiful view.
I was the only exception.
I stayed inside.
The few Jews who had sur-
vived came to see me.
I was crying on the inside, but I didn't show it.
The second day in Salonica) Mr.
Chernovich, the Palestinian delegate of the 
whom I knew very well because he had been to Salonica often, was to talk to all the 
Jewish people at the Cinema Palace.
Of course, all our group was to sit on the stage.
The speech was to be held at noon.
In the morning, I got up very early, and I went to 
see the man who had sold the material to my guard.
I went into the store.
In the middle of the store there was a table with two or 
three pieces of material on it.
When I saw this empty store, I thought that I had made 
a mistake about the place.
Suddenly, a voice from the back of the store asked me if I 
-114- 
needed anything.
As soon as the man saw me, he started to yell.
We embraced each 
other, we cried together.
I told him that I thought .I had mistaken the store.
He ex-
plained that the Germans had taken all his merchandise.
I told him that I wanted very badly to see the officer who had saved me from the 
prison.
He replied "The Germans took all of my chairs, too.
But I have this stepladder.
Sit down."
Hex startedz to tell me the story of the officer.
"He took too many chan-
ces, because he spoke such excellent German.
One day your guard saw him in the street, 
and he was immediately arrested.
They took him to the same prison that you were in.
The treatment that he received is very tragic.
They cut open his back, and then they 
put salt on the wounds.
This was how they tried to make him talk.
He died in great 
pain.
But he didn't reveal that I Was the one who had told him where you were.
He 
didn't tell them the name of his wife and child.
And he didn't give them Irhe address 
where you were staying."
We made a rendez-vous for the next day, to go to see the offi-
zaztia cer's wife.
I went back to my hotel, and cried as I was walking.
This time I didn't hold my 
tears back.
At the Hotel, my group was ready to go to hear the speech.
Two nurses came up 
up to me, very angry.
Together they asked "Where were you?"
I answered "I went to see 
a friend."
When they heard this, the two old maids got even madder.
They told me "You 
have to stay with us, at our disposition.
We need an interpreter sometimes."
I ans-
wered, in the same tone, "I didn't come here to be your interpreter.
I'm doing it be-
cause I want to help."
We went to the conference.
Chernovich spoke very well, as usual; he was a good 
orator.
But there weren't many Jews there to hear him.
In the et:ening, I went to 
see a gentleman who was supposed to send Chaim and me to Athens, after our wedding.
He received me very well.
Chaim had given him 1000 gold liras.
He gave me ten.
He 
took me back to my hotel.
On the way, I said "I think Chaim have you a little more 
money."
"Oh, no," he said, "this is what he gave me."
I had helped Chaim carry the va-
lises full of money to him.
I said to myself "I have to take what they give me, and 
not think of what we gave to them."
-115- 
The next day at the time of my rendez vous, I went to the material store, and I 
went with my friend to see the officer's wife.
We came to a beautiful district of 
Salonica.
We entered a lovely home.
The house was empty, without any furniture.
The 
door of the bedroom was opera.
We could see two matreeses, a big one and a little one.
This was her furniture; a matress for herself, and one for her child.
She had sold 
her furniture during the German Occupation so that she and her child could survibe.
I said to my friend "Lets take her to a restaurant".
She was skin and bone, and 
her child was very thin also.
We went to a restaurant, and they both ate very well.
I gave her a few drachmas.
I didn't have any clothes to give her; I only had uniforms.
I told-her "I can't give you clothing.
But I gave many things to people here, to hide 
for me.
If I find them, I'll give them to you.
I have to go now, because I have a 
x conference with my group."
I paid the bill and left them in the restaurant.
When the conference ended, everyone went to the movies.
They asked me to go with 
them6 I said "No, I want to visit some friends."
Again, they didn't like what they 
called my "behaviour" very much.
I went to see a priest.
His daughter had been in my class in the Italian school.
I had given this friend two large valises filled with clothing to keep for me.
The 
priest was very happy and excited to see me.
He told me that his daughter was in 
Athens, but that the two valises were there with him.
He saw that I was very happy to 
have them.
I told him what I was going to do with them, and the priest was happier 
than I. He called his son to help me carry the valises.
The next day, the priest sent 
a valise full of children's clothing to the officer's wife.
After I delivered the valises, I went to see my friend at his store.
I told him 
"I can't bring her husband back.
But I want to help her.
I gave her clothes and some 
drachmas which will last for a few weeks.
But we have to look for a more permanent 
solution.
What can we do?
", he said.
"I can't help her.
The Ger9ans took all the material 
that I had.
I don't have any money with which to buy new material.
I have to close 
.11 
this beautiful store in this lovely district."
"How much capital do you need?
", I asked.
`1.
-116- 
"I would need at least ten gold liras right away," he answered.
I said "I'll 
give you the ten liras, on one condition, You will take this woman in as a partner.
You will be the outside man, and she will be the inside woman."
He was very happy.
We drew up legal papers, and I gave him the money.
They started to work to-
gether, and the arrangement was a very good one.
She didn't need my help any more, 
and I was glad.
They understood each other so well, and they were doing so well in 
the business, that six months later they got married.
-117- 
irnev 5 rbof 'V- I 
The day after I gave the ten,1liras to my friend, um went to the Monasterlis 
synagogue.
All the surviving Jews were there; there weren'ty too many.
Soldiers 
of the Jewish Brigade, Jewish soldiers from the British Army, and our group, were 
also present.
Like this, the synagogue was full.
One of our group, the disinfect-
or,was given the prayer to read.
He was chosen because he was the tallest and the 
fattest.
But he didn't know how to read.
A few people from the Deportee Union 
(survivors of the lager) were there to help him.
The only thing I could hear being 
whispered around me in Ladino was "He doesn't read Hebrew:" 
The Deportee Union gave some speeches.
They said we would never again permit thd 
Jewish Police to exist in our community.
If they came back, they would be prosecuted' 
according to the laws if the land.
We would not tolerate traitors like Koretz again., 
I remembeted that Koretz had come to talk to us in the same synagogue, a few days 
before the deportations began.
Koretz had stood in the same place where the speaker 
was standing now.
If it wasn't for the Jewish Police, Koretz would not have left 
this synagogue alive.
The Jewish Police saw that the people were ready to kill him, 
and they took him out by the back door.
In the synagogue, I met a friend of my family, a man in his late fifties.
He 
was very glad to see me, and he asked if I had done something about my estate.
I 
said I hadn't.
"But," he said, "you will be leaving here, and you must have someone 
represent you in Salonica."
I answered "If you want to take the job, I'll give it 
to you."
The next day we drew up legal papers to this effect.
Two days later, we were downstairs in the hotel, early in the morning.
The 
buses came to take us to a camp in Siderokastro.
It was a small town near the Bul-
garian border.
Originally it had bedn an army camp.
It had been converted into a resi 
camp for the deportees.
The rest camp was for Prisoners of War, Jews and non-Jews.
If was for nonJewish people whom the Germans were had taken to labour camps in Ger-
many.
There were political prisoners from Greece whom the Germans hadn't killed, and 
also Jewish survivors of Hitler.
Our group was composed of a medical doctor, a chemist, two nurses, four chauffeur: 
a disinfector, a quartermaster (Max 
Garfinkle), a dietitian (myself), and another 
-118- 
person who had no official function.
We came to the camp.
They introduced us to 
the camp commandant.
She was a lady from England.
When we were dismissed, we went 
by the bungalow that was to be the dining room.
In the doorway, I saw Esther.
I 
met Esther in Cairo, in the camp at Meady.
Esther was the envoy of the joint Dis-
tribution Committee.
She was not very popular with our group.
The reason that was 
given was that Esther spoke with non-Jews.
Esther had been born in England.
She 
was very well educated, and a very sweet person.
She called me over and whispered 
in my ear "I prepared your room near mine."
I didn't know how to thank her for hav-
ing prepared a room for me alone.
Everyone took their belongings to their rooms.
The nurses had a very big room 
for the two of them.
The men had a very big bungalow, where everyone stayed together.
I went to my room and opened the folding bed that I had been given in Cairo.
Then 
I went to see the cook.
I introduced myself and told him he would be working with 
me.
The cook was a man in his fifties, very nice and polite.
I had never seen a kitche 
as big as the one in the camp.
I asked him where the stove was.
He took me into a 
small room with a small stove in it.
I asked him how we were going to cook hundreds 
of meals on this small stove.
He said "This is for the staff.
We have a very big 
wood burning stove for the refugees."
He showed me two large rooms in the kitchen that we could use for storage.
I 
asked him if he could find a desk or a little table for me.
I could make my 
office in one corner of the kitchen.
I asked him how many refugees there were.
He 
said there were ten, and that the food was ready for them.
I went to see the food.
The soup tasted like boiled water, but I didn't say 
anything.
I told the cook to keep a small fire going all the time, so that we would 
always have warm water.
I remembered that when I arrived in Syria in the middle of 
the night, tired from the train and hungry, they didn't even give us a cup of tea.
We had to wait until the next day.
I wouldn't make the same mistake here.
If ref-
u§ees arrived in the middle of the night, they would have something warm to drink.
The chemist came in as I was talking to the cook.
He told me "Please, the 
came commandant ma 
was_ 
cmpll r.ity naar qiriar01/.2.strn 
-119- 
The camp commandant had a bad infenction of the hands.
She needed a reliable per-
son to open letters for her in the offices she was to visit in Drama.
We arrived in Drama, a small, clean, pretty city.
We went from one office to 
another until the commandant had finished all her business.
She said that she was 
going to see a doctor who would change the dressings on her hands.
There was a gen-_ 
eral store facing the doctor's office.
I said "You go to the doctor.
I'll stay 
here to window shop."
I was looking in the window when I noticed there was a mazuza 
on the door.
I went in to see if the owner was there.
As soon as I saw him, I real.
ized that he was not Jewish.
The people in Drama had not seen a Magen David since the German deportations.
I. picked up the first thing that I saw, and I asked the owner how much it cost, in 
Greek.
I hadn't finished my sentence when a young policeman came in.
He asked me 
to follow him.
I went, and I found myself in the office of the Director of Police.
He apologized for having brought me there.
"But," he said, "the owner of this store is very patriotic.
He called me to say 
that there was a lady in his store XiikIXB wearing a khaki uniform with a Magen David 
the Germans forced the Jews to wear during the occupation.
He wanted to know where 
you learned Greek."
"This uniform is worn by Jews who have come from Palestine," I said.
"We came 
voluntarily to help the Greek people.
You say the owner of the store.
is very patriot 
and he thought I was some kind of intruder.
But he made a mistake.
He just saw a ghn 
The Director asked me if I was from Drama.
"No," I said, "but this man doesn't 
like to see Jewish people very much.
He thought I came to take his store) because 
this store once belonged to Jewish people."
"But you used to come to Drama before the War?"
the Director asked.
I said no.
"Eh," he said, "how do you know this store was owned by Jews?"
"This man changed the name of the store.
But he didn't remove the most obvious 
evidence that this store had been owned by Jews."
"What is that evidence?"
asked the Director.
"Go there and find out for yourself," Z answered.
T 
-120- 
I went back to the general store to wait for the commandant.
then we returned to the camp, the nurses were very angry that I had gone to 
Drama with the commandant.
"You have to understand that you must ask our permission 
to go, because you aren't from any miflaga (political party)."
"First of all," I answered, "I didn't come here to make politics.
When the So-
konouth asked me to come, they didn't ask me if I was a member of any political par-
ty.
And I came here to heal wounds, to feed people, and to try my best to raise 
their morale.
Don't expect me to ask your permission for every move I make.
I'll 
ask the doctor or the chemist if I need advice."
Late in the evening, I went to the kitchen.
There was boiling water on the 
stove.
The table was placed, just as I asked, in a corner of the kitchen.
The tele-
phone rang: refugees would arrive in half an hour.
We made some soup right away.. 
We tasted it.
We put thex soup in the trucks, and we went to Coula, near the Bul-
garian border.
As soon as the truck stopped, we saw the refugees crossing a small wooden bridge, 
far away.
The first person I saw was a very well dressed man, carrying a brand new 
coat and two valises.
I said "Welcome to Greek territory."
"I am the commandant of the refugees," he told me.
I asked him if he was also a 
refugee.
"Yes," he said, "the Russian army gave me the food, and I distributed it to 
the other refugees."
"There is no commandant here," I said.
"We give the food with our own hands."
"Some people need to eat better than others," he said.
"You can see, some can 
hardly walk.
Others need less food."
"We have food for everyone here," I answered.
"No-one will be left hungry."
As I was talking to him, I noticed a lady in her fifties; it was Regina's art 
teacher.
She was a non-Jew, but the Germans took her the first night they entered 
Salonica.
They accused her of making anti-German propaganda in the school, but she 
was far from being a politician.
They put her in a camp for political prisoners.
She was walking with two men between 25 and 30 years old.
The men were each carry-
ing two valises.
She introduced me to them, and told me that they were both deaf 
to go to Palestine from there, and these two boys are Jewish."
"You don't need money to go to Athens," I said.
"Yes," she said, "I gave my address in Salonica when we arrived."
-121- 
mutes.
We gave everyone very good soup, and we took them to the camp to the dis-
infection room.
I went along to help so that they could get to bed more quickly.
I disinfected the teacher with DDT.
She said that she had something very important 
to tell me.
I asked if it could wait until morning, and she said it could.
I talked to her the next day.
She told me that the two boys with her weren't 
really deaf mutes.
But they didn't speak Greek.
They were officers of the Russian 
Army.
Iasked why they didn't go home.
"If they go home," she said, "they will be killed."
I wanted to know why.
"All 
thoseofficers," she answered, "who surrendered to the Germans will be killed if they 
return to Russia, They were not supposed to surrender.
They were supposed to die 
instead."
"How do you know that?"
I asked.
"you were in a political prison camp and they 
were in the Russian prison camp."
"My camp was liberated by the Russians," she said.
"But before the Russians 
came to our camp, the Germans fled and they left us alone.
We saw we were alone and 
we ran to the bushes.. I couldn't run very fast and I fell down.
These two boys 
foynd me.
I spoke to them in Greek, in German, but they didn't understand.
I spoke 
all the languages I knew, but it didn't help.
Finally I spoke to them in French.
They both speak French.
They were scared, more scared than I was.
They told me the 
rules of the Russian Army.
I advised them to pretend to be deaf mutes.
We found 
other people in the bushes, Greeks running away from the camps.
From that time on, 
the two boys have not spoken a word.
One thing," she said, "no matter what happens 
you won't turn them over to the authorities.
I have confidence in you."
I was scared.
I didn't know what to do.
I said "Let me talk to the superior.
He is a doctor.
I'm sure he will help you."
She refused; only I was supposed to 
know.
I said "How can I help You?"
"You will give us money," she answered.
will take them to Athens.
It is easy 
A 
-122- 
"I can change the address and you will go to Athens," I said.
"I want to take them to my house," she said.
"They need new clothes, and they 
need to talk and to yell, because they can't take being silent any longer."
"I have no money here," I said, " but I'll give you a letter.
You will go to 
my representative in Salonioa, and he will give you money.
By the way," I said, 
"when you are in Athens, go to see one of your pupils.
He is alive, and he can help 
you."
ti 
I need jackets for the boys," she said.
"They don't have any."
"I saw them carrying two heavy valises when they came from Bulgaria," I said.
"Those valises," she said, "are full of the best things in the world.
Look at 
me, I don't even have my wedding ring.
The Russians gave enough food for everyone 
in the camp.
But this Greek commandant gave us very little, and he sold the rest.
These two boys, and others, had nothing to give him.
They were carrying the 
commandant's valises when you saw them.
Take me," she said.
"I gave him my watch 
and other things I had until there was nothing left.
In the end I was doing his 
washing in exchange for food.
You didn't see, but there were people who couldn't even 
walk because they hadn't had enough to eat."
I took the two boys to Mr.
Garfinkle, the quartermaster, to get them jackets.
I 
got clothing for others in the camp, too.
I was proud of myself.
From the first 
moment that I saw the commandant, I had told him that xxxx there were no commandants 
in our camp, that the food was given with my own hands.
The next day, the camp commandant had to leave because of the infection of her 
hands, The doctor was a very good diagnostician, but he was not a good leader.
The nurses were the ones really in charge.
They did what they wanted, and they told 
the doctor what to do.
The chemist wanted to make a name for himself, and he was will-
ing to work very hard to do it.
But there was nothing for him to do in the camp.
The 
disinfector would lose his head every time he saw a woman.
I would get up very early in the morning to give the menu to the x cook.
I su-
pervised the cooking to make sure it was tasty and ready on time.
I made a programme 
so that everything ran by the clock.
If the food was ready and if I had time, I would 
go to the disinfection room when new refugees arrived, to help and to give them warm 
food.
The day came when the refugees were read* to go to Salonica.
At eight in the 
morning, the trucks were ready.
I gave them people breakfast, and rations for the roE 
The doctor had gone ahead to Salonica.
The nurses felt that if the doctor was gone, 
it was up to them to replace the camp commandant.
But the chemist wanted to be 
commandant, too.
When the nurses saw that the chemist had given orders to some of 
the refugees to board the first truck, they made all the people get out.
They filled 
the trucks with other refugees, their favourites, instead.
When the chemist saw 
what the nurses had done, he made the refugees get out, and he put his favourites in 
their place.
At noon, I gave evryone lunch.
This process continued.
I felt very sorry for the refugees.
At 4 P.M.
I saw 
that the excercise of getting in and out of the truck had made them very hungry.
They were eating the rations I had given them for the road.
I couldn't bear to watch 
this cruelty any longer.
There were two patients in the infirmary who had been alone 
all day, because the nthrses were too busy with the trucks.
I couldn't bear to watch 
these people who had suffered so much at the hands of the Germans suffering again, 
for no reason.
They were anxious to go home, and they were being prevented.
I spoke to the chemist.
"I like you, because you are humanitarian.
Don't tou 
think that these people have suffered enough?
They go in and out of these trucks, 
but they're still in the same place.
I don't want to discuss whether you are right, 
or the nurses are right.
I appea4 to you in the name of humanity, because these ref-
ugees can't take this suffering any more.
They will remember what happened here for 
all their lives.
Let them go.
it is ridiculous.
These people still remember the 
German discipline.
But don't push your luck, because they will rebel, and they can 
kill you."
He listened to me; the nurses won.
A few days later, UNNRA sent a camp comman-
dant, Col.
Sheppeard, and a lady doctor.
The infirmary took on its original cbaracter:m 
sick people were looked after, without politics.
The new doctor was in charge mg of 11-- 
the infirmary whenever one of the nthrses and our doctor would go to different villages 
-124- 
--, to provide medicine and to look after the sick.
If there were people with tubercul-
osis in the villages, they were brought to our infirmary.
We did whatever necessary 
to send them to the sanatorium in 271k1 V Cavalla.
On weekends when I had time, I 
went to see them, because these miserable people never had visitors.