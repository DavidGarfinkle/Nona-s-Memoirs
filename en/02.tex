% pages 60 to 70
% IN PROGRESS
The next year, in 1967, I was district chairman, and I went to 
the office to ready the allocations for the workers.
Suddenly I realized 
that it was very late, and that I had to go home to prepare supper for my 
husband.
Phyllis Wexler, my associate chairman for 1967, and Phyllis Waxman were in the office.
I went over to Phyllis Wexler and said: "There are only two people in this office, you and Phyllis Waxman.
I'm going to leave everything on my table and go home to give my husband his supper.
I’ll be back in an hour.
Please make sure no-one disturbs anything on my table."
When I returned an hour later, the cards on the table had been shuffled 
around.
I asked Phyllis Wexler "Who did this?"
She answered "Don't ask me.
I don't want to talk."
I said "How am I going to arrange all these cards now?"
As I was saying this, a lady entered.
I had never seen her before.
Her name was Phyllis Jackson.
"Don't worry," she said to me.
"Even if we have to work all night, my sister, Sheila Lipson, and I will arrange all these cards.
She took the cards, and the next day she brought them to my house.
I later found out that she and her sister had worked all night to put the allocations in order.
People like Phyllis Jackson don't get recognition in our community.
My workers were supposed to come to get their envelopes on Sunday morning.
No-one showed up.
They had been told to get their envelopes at 
Adath Israel.
I complained to Phyllis Waxman.
The only answer she had for me was "If I didn't do that, I would be washed out of the Women's Federation, because I refused to work for Mrs.
Fisher, the chairman of the 1967 campaign.” Despite the sabotage, I was the champion once more.
Mrs.
Fisher sent me flowers and put an announcement in our community newspaper.
It 
wasn't easy to finish the campaign so well.
In 1968, Rosetta Elkin was the chairman of the campaign.
Rosetta comes 
from a very old family on her mother's side, the Josephs of Quebec.
They 
came to Montreal in 1760, with the British Army.
The Josephs were one of 
the founders of the Spanish and Portuguese Synagogue.
Her father's family 
-57- 
were the Wolffs.
Her father was a civil engineer, a treasurer
(gabay) of the Spanish and Portuguese synagogue for eighteen years.
The indefatigable Mr.
Wolff also had other duties.
Mr.
Wolff formed the committee that greeted the survivors of Hitler in Montreal, particularly the Sepharadim.
He held an Open House once a week at his home with the help of his four daughters.
His wife had died on Argyle Avenue in Westmount 
in 1940.
He held services in his own home for the Sepharady survivors on 
Yom Kippur and Rosh Hashanah in the Sepharady tradition, since at that time 
there were no pure Sepharady services in Montreal.
When I came to Montreal, 
I was one of the people welcomed at an Open House, and like this, I got to 
know the Wolff family very well.
Rosetta is married to Victor Elkin.
The Elkins do not need an introduction in the Jewish Community.of Montreal.
Rosetta has three sons and 
one daughter, and she has eleven grandchildren.
She is a lady in all senses of the word, and she is a very sweet person.
Educated as a teacher at 
McDonald College, she also has her B.A.
from Queen's University.
In 1968, my district was doing very well, until the day Phyllis Waxman called me.
She said "Don't run any more.
Leave it for next year."
I answered "Israel can't wait for next year.
Why, are 
you scared you won't be successful next year?
First," I continued, "Mr.
King is the best advertising man in Montreal, and he's going to make you 
big.
Second, you have the rabbi, who is going to teach you how to speak."
I didn't know mho the rabbi was, I had just heard about him.
"Third, 
everything will be done by telephone.
All the organizations will send people to do the telephoning.
I'm going to finish my district and not worry about what will happen next year."
One hour later, my phone rang.
It was my associate chairman, Ella Cohen.
She was coming to get my cards; I had done enough.
I called Rosetta and told her what was going on.
Rosetta said "I'm going to call her."
As I was putting the phone down, my associate chairman was ringing my doorbell.
When she asked me for my cards, I answered "I want to finish my district."
She said "I have to be on good terms with the 1969 chairman, because she is going to give me executive work next year."
I felt like spitting in her face, but I didn't do it.
I remembered what my father used to 
say: "If you walk in the street and a donkey kicks you, are you going to 
kick it back?"
She took the cards and she left.
I called Rosetta and said "It's too late to call Ella, she already 
took the cards."
A week and a half later, the closing luncheon of the campaign was held at Beth El Synagogue, and they presented me with the silver 
plate.
Rosetta introduced the 1969 chairman, Phyllis Waxman.
She started to talk.
At the same moment, everyone got up to leave.
Phyllis with 
tears in her eyes, started to say "Wait!
Wait!"
But everyone was leaving; 
they showed her they didn't want her.
At that moment, I felt sorry for her.
I went to console her.
"Don't worry!
Mr.
King will make you a star!"
A few weeks later, she called me.
"I'm a big shot!
I'm a big shot!"
she said.
"I gave a talk today at the golf course, and everyone there will 
help.
I want you to know one thing.
It's not what you were, it's what you 
are, and I'm a bigshot."
A few weeks after this conversation, she called me again.
She asked 
me what I wanted to do for the campaign of that year.
I answered "Nothing!
I'm not used to people destroying my work.
Find yourself somebody else.
I can serve my community without going to find money.
I can do other things.'
Sadie Neamtan, too, called.
I refused her also.
I think she took my refusal a little personally, but now I think she understands.
The next day, we went to Sfat, a very ancient synagogue.
The synagogue was built into the rocks.
As we left the synagogue, I saw an old 
man running.
Our guide was chasing him.
The lady photographer was yelling.
"No, Albert, no!"
The man tripped over a rock and fell.
Two minutes 
later, there was a sea of blood.
There was a doctor in our 
group.
I called him, and he said to call an ambulance right away.
I gave 
-59- 
our guide a nasty look, but I said nothing.
The indefatigable Thomas Hecht 
came to see what was going on.
I started to walk away, thinking that in 
Israel you needed protection against everything, and this man didn't have 
It to sell quipot to the tourists.
By the time this man gets to the 
hospital, he will surely die, just like the mother of the baby.
I remembered that we had gone to see an apartment building in Israel that had been 
built with Canadian money.
They took us to see one of the apartments.
It 
was very nice.
A family of Moroccans was living there.
I spoke with the 
housewife.
She said "I came to Israel after my daughter came.
I have two 
daughters and a son.
One girl works for the government.
My son goes to 
two different schools, one in the morning, and one in the afternoon.
I don't 
pay rent.
I have two pensions."
I asked "How come you have two pensions?"
"Oh," she said, "I have protection.
My daughter was a Zionist in Morocco, 
and the leaders of the town are friends of my daughter."
"Where is your 
daughter now/0 I asked.
She answered "My husband died, and my daughter 
married and went to Paris with her husband."
We left the house, and we 
heard screaming.
A blind man was yelling "I am blind, I have children, and 
I don't have a pension.
These people have two pensions.
Is this justice?"
I spoke to our group about this.
I think that rather than helping this 
blind man, I made more enemies.
Now in Sfat, after seeing all this blood, 
I couldn't walk.
But life goes on, Eliaou used to say.
We went to see an exhibit of paintings.
I found a chair, and instead 
of looking at the exhibit, I sat down.
When the others had finished looking, we went to the bus.
Our guide went to the police station to make a 
report of the accident.
The guide from the chairman's bus, Jacov, came to 
our bus.
The radio had just announced the news.
Mankind had landed on the moon.
At this very moment, the lady photographer came to sell me some photographs.
During the trip, everyone bought photos but me.
She had refused to sell me 
any.
She had orders not to sell me photographs.
But in the end, no-one 
wanted any more pictures, and she wanted to sell them to me.
Because of 
her, I didn't hear what the guide said about man landing on the moon.
I asked Jacov "Could you please repeat what you said for me?
I didn't 
hear you."
He put the loudspeaker up to his mouth, and he answered me: 
"For you!
Ou bou bou!"
Everyone on the bus heard Jacov say this to me.
My husband said "This one is very fresh!"
I said to myself "The chairman 
bought a dozen dresses.
She's pushing everyone around.
Of course, the 
rabbi taught her how to talk.
She spends hours in front of the mirror, imitating Mrs.
Gertsman, like an actress.
But she has to learn that to become a bigshot', as she called herself, one doesn't gossip with a chauffeur."
We came to the hotel, and the bus stopped.
I took a little while to 
get up from my seat.
She was waiting at the door of the bus, the picture 
of innocence, and she asked me what happened.
I looked at her and I said 
to myself, "I don't know if I should spit in your face or pity you.
But I 
pity the Women's Federation who chose someone like you to represent the 
Jewish women of Montreal."
That same night, there was a meeting, but they never called me for a meeting.
We had supper, and I went to sit down with Mrs.
Shofild and her sister and the photographer.
The photographer said she had no place to sleep that night.
They hadn't given her a room because she had sold me 
some pictures.
I looked at her and I left.
Suddenly I saw a taxi cab stop 
Three soldiers and a civilian got out.
The civilian was dressed in white, 
like Eliaou and Chaim used to dress in the summertime.
I was admiring this 
youth.
"Oh," I said to myself, "look what kind of youth is going to the 
battle to die!
Jewish or Arab, his mother is crying.
Oh, God, peace in 
Israel!"
Suddenly the civilian started to kiss me!
“Who are you?!"
I 
asked.
He answered "I'm the baby."
It was a very emotional reunion.
He 
told me that he was studying in England.
He was supposed to spend the summer 
holidays with his uncle in Europe, but his uncle had told him that there was
to be a reunion in Tel Aviv for all the people who knew Maritsa.
The reuni 
would be held in the cemetery, because one of the boys had died.
They would 
hold the unveiling at the same time.
—b1— 
The uncle was a sick man and couldn't come.
The boy told me "You don't know how 
nappy we are that the Jewish Community invited you on this trip."
I answered "I am treat-
-ed here like the poor cousin who came to visit the rich one."
I said, "This chairman is 
making my life miserable.
When my grandmother was young, there was a lady entertainer in 
her town.
Her name was Bona La Tagnadera (Bona the Entertainer), amd she used to say: 
Estach coumbidadas 
I poco encomendadas 
La eaza se corre 
Langer non aye ande acentar 
Myor ese We non vengach 
This means: You are invited and very little recommended.
The house is leaking.
There 
is no place to sit down.
The best thing you can do is not to come.
This was my invitation."
The boy answered "This chairman deserves to be pelted win tomatoes.
My uncle, as I 
said, can't be here, but he gave me a present for you."
He took a little box from his pocket 
"This is a brooch from my mother.
After the war, my uncle went to the ditch where you had 
told him you had found me.
The skeleton of my mother was still there.
Nearby there was 
a shopping bag with things to babies inside, and all of her jewellery.
My mother 
didn't die from a German bullet.
She fell into the ditch and hit her head, and she died from a loss of blood.
The brooch comes from the jewellery that my uncle found in the ditch.
My uncle and I wanted you to have it."
"You will give this brooch to your fiancee," I said.
He answered "You are my fiancee."
I said "When this boy whose unveiling we are going to died, I sent money for the gravestone, but it wasn't enough.
Just last week I learned that his widow has many debts for the rest of the stone.
"When I go to Tel Aviv, I will give her the money to pay off her debts.
First, if I take this brooch, I will never wear it, because I don't want to remember my 
past any more.
I remember enough every moment on this trip.
Second, this brooch is very 
valuable.
Either keep it, or sell it, and with the money buy chicoun (a condominium) for 
the widow, who now lives in a slum.
I'm not going to take it."
He told me he would call his uncle to ask him what to do.
He begged me to let him talk 
to.the chairman.
I said "To talk with people like this only gives them honotr, and she 
ever to anyone.
We were together for an hour, and then I saw him to his taxi.
He introduced me to 
his friends.
"This lady is my mother."
He kissed me and got into the cab, and said "I'l] 
see yiu in Tel Aviv".
When he was in the taxi, he gave me a bottle of Coca Cola that he 
took from a case, and he said "l'chaim".
They left.
I drank the Coca Cola with gusto, as 
if it was the first time that I tasted something very delicious.
Mr.
Wamsan, a very nice 
man, came up to ne.
He said "The bar has been closed for two hours.
You see, they are 
talking about you.
Montt do that!
Where did you get that Coke?"
I answered "You won't 
understand, but my baby gave it to me."
Everyone was leaving for the meeting.
I told W4 Shofild "It is time that I was in-
vited to one of the meetings.
I, too, have an opinion."
I went to bed, but I couldn't 
sleep.
Despite all this, I never stopped dancing and singing.
I remembered Cheame.
Suddenly, Mild appeared with a pair of slippers.
I was walking barefoot because I 
had lost my shoes when I was pushing the bpat.
I got scared.
I asked him where he got 
those brand new slippers.
All the children said in unison "We bought them!"
I asked 
"where did you het the money?"
They answered "The Turkish government gave us one ration of 
% 
corned beef each.
We sold it to get you a present."
I kissed them all.
They invited me to the next meeting.
I spoke about the protection in Israel.
I spoke 
about the two pensions that lady was getting.
But Phyllis interrupted me, and the meeting 10 
adjourned.
A 
The next day, we went to see an airplane factory.
When we got off the bus, we saw sol-
diers training to be parachutists.
I joked with my husband "You know what?
We're going 
to jump".
My guide heard me, and he insulted me.
This was the last straw.
I took him 
into a corner, and I said to him "First of all, your job is to show Israel to the tourists, " 
and not to insult them.
Israel needs tourists.
And don't think I came for free.
I was in-
vited, and I paid, like everyone else.
And for your information, the Jewish people need 
women like me.
The Jewish people don't have missionaries, and I believe that every Jew muslin 
be a missionary.
And I am one of them."
He apologized.
I was very excited about seeing the airplanes factory.
I was eager to see the first 
airplane made in Israel.
I was admiring it when a lady came up to me with her bus-
-63— 
Pol?
band.
Himax Her husband was wearing a quippa on his head, like the religious Jews.
She asked me if I was Tova (my Hebrew name).
I said "Yes.
And who are you?"
She an-
swered "I am the daughter of Mr.
Loewy.
Rath Pollak is my sister."
"Oh!"
Isaid, "Nellie!"
Our chairman was just passing in front of us.
I called her over.
"Phyllis, I'M going to 
introduce you to Mr.
Loevy's daughter."
Mr.
Loewy does not need an introduction in the 
Jewish Commnnity.
I introduced Phyllis to Nellie and her husband.
When they left, I 
spoke with the chairman.
I said "You see, her husband is very religious, he wears a 
quippa.
You see, Phyllis' He is not a gentile, he is a Jew.
It is not true what you 
used to say all over Outremont, that Loewy's daughter had married a gentile.
For you, it 
was exceptiommy good gossip.
I told you that it was a lie.
Now you have seen it with 
your own eyes."
She answered "Oh, God, don't say anything to anyone."
I said "Oh, no, 
Phyllis, I don't gossip.
But when I write my memoirs, I will not forget it."
I almost forgot to write about yad ve chem.
The guide told us when we were there 
that a group of Jewish people, no-one knows who, started the yad ve chem.
I said "The 
chaveria of the kibbutz MiahmAr.
a Emek started the yad ve ahem, and Mr.
Garfinkle was one 
of them".
For me, it was like going to a very big cemetery of all the young people that 
I had seen dying.
They would die of a simple wound.
There were no medicines, no doctors 
to remove the bullets.
They died of loss of blood.
And do you know who vent down for 
the ceremony (kaddish)?
Phyllis Waxman, who enjoys to hear people cry.
I remembered a 
10 little girl in.
Montreal who had told her "How can you laugh at a time like this?"
She 
reprrsented 6 minion Jews, the Jewish women of America.
This made me cry.
The trip was coming to an end.
We were told that we would visit Gaza.
But then a 
bomb had exploded the same day, and our trip was canceelled.
I started to remember Gaza.
When we left the retaurant in Chesme, a lady that I didn't know came to talk to me.
She said "You can't travel with the abildren.
It is going to take too long.
All the ref-
ugees from the camp in Halepo in Syria are going to Palestine.
They have prepared 
accomodations in Gaza for Greek refugees.
You have to be back at the camp tonight, or 
they malfind out you are missing."
I looked at-her with such surprise that she asked 
"You are Maritsa?"
I said "Yes".
I told Miki to stay with the other nhildren and the priest.
I went to Syria with 
a commercial truck.
At the border between Turkey and Syria, a taxi was waiting for me.
It took me to the cemetery where the camp was situated.
The window was still open.
I 
climbed in.
Everyone in the bungalow asked me where I had been.
I answered "I went to 
work.
I need money."
One of the ladies said "After what you did for our children, if yr 
need money, we will give it to you."
I said "I prefer to work".
As I said this, I was looking out the window.
All I could see were rows of tombs.
And I said to myself "If one day I return to Salonica, I don't even have a cemetery that 
I can go and visit.
The Germans destroyed the tombs of my father and of the previous ger 
erations of my family.
My mother and the English soldier are buried in another cemetery.
had 
But bandits kr= gone there to destro* the tombs because they thought that Jewish people-, 
buried their fortunes with the dead".
Everyone was busy preparing and packing.
I asked "Where are you going?"
"Oh, we z 
are going to Palestine.
In a few minutes, all the men will be here, and we will travel , 
together."
Two minutes later, everyone arrived, inclnaing, of course, George and Nicola.
The children and Miki hadn't arrived yet.
We left.
We arrived in Haifa, and I and other Jewish people received certificates to enter 
Palestine.
M But not George and Nicola5.
I procured certificates for them to be sent 
to Gaza to the camp for Greek refugees, and that is where they went.
The next day, they took us to see the Houlpam, a beautiful house, the most I luxar-
ions apartment that I saw in Israel.
There were washing machines, dryers, beautiful 
dressers, a library, a beauty parlour for women and a barber shop for men.
We needed a 
guide to see a big building like this.
A lady officer knew a bit of French, and Mr.
La-
sous was translating what she said to English.
The lady officer mixed Hebrew with French' 
and Mr.
Lazous didn't understand exactly what she wanted to say.
I explained what she war-<
q 
saying to Mr.
Lazous.
This was the biggest mistake of my life.
Phyllis Waxman came over 
and she said "Shut up, you!"
I looked her in the eye, but I didn't say anything.
I started to remember that when I was in jail after they 'killed Chaim, they used to 
take me for interrogation four or five times a day.
I didn't have anything to tell 
them.
"Where are the partisans?
Who is the chief?
Where is the headquarters in 
-65- 
Salonica?
Who is the boss?"
My answer was always the same: "I don't know anything, 
and I don't know them."
When I was in very bad shape, they would put me back in Toy cell.
My guard would come and talk to me.
"The more you shut up, id; the worse it is for you.
Talk.
I'll give you soup."
She used to bring a very good soup.
If I talked, she would 
give me some.
I would say "I have nothing to tell you."
The next day we went to Haifa.
We went to Bat Galim, a beautiful restaurant in the 
go 
mountains.
I remembered that when we arrived in Haifa, after we received our certifi-
cates, they took us to Bat Gahm, to the Bet Hao13m.
The Bet Realm was a very dirty bungalow.
The beds were full of bedbugs.
It was very 
bard to sleep without scratching.
We would open the lights in the middle of the night to 
kill bedbugs.
Marty of us had psoriasis.
They gave us herring to eat.
You can imagine the 
with the bedbugs, the psoriasis, and the herring, our lives were spent scratching.
The day we arrived in Bet Haolim, the lady in charge came to see me, and she said 
"There is a lady here who wants to talk to you." As soon as she walked in the door, I 
recognized her. It was the lady who had had her baby in the boat. She had ziac a parcel 
in her hand, which she gave to me. "This is yours," she said. M "Mine?" I asked. "Yes," 
she answered. "After you left Chesme, I called the fisherman and I asked him to sell me 
all the jewellery that you had given him to pay for the boat. He sold it to me, and 
here it is." It was a very emotional meeting. She said to me "I'm alive only because 
of you. I have my baby only because of you. There isn't enough money in the world with 
which to repay you. My baby and the baby you saved are with me here. My husband still 
has his cast, and he can't come to see you. We will meet you later, but if you need any-
thing in the meantime, let us know. My husband has plenty of money in Palestine." I 
answered "I don't need anything, but I want you. to promise me one thing. You must raise 
x this baby that I saved as if he were your own nhile. She promised. 
Not all the jewellery that was in the corset thpt she gave me was mine. First of all, 
there was some jewellery that had belonged to Tia Donna. The day Samuel returned, I 
would have to give it to him. There was also a bracelet that belonged to Suzane, my first 
cousin. I had been so anxious to save Niki and the baby that I hadn't thought of 
ithet.oc. 4...hirkere 
When Suzane was a baby, her father died of cancer. A few months later, her mo
ther died of pneumonia. The only thing I can remember abciut &mane's parents is that 
they lived in a very big apartment, and that every zudgimddz shabbat they would come to' 
visit us.
One day, when Suzane was 21 years old, she came to my house with an envelope 
in her hand.
It was two days before the German occupation.
With tears in her eyes, she 
said "I want to speak with Eliaou".
I answered "He's not here now, wait for him".
But she couldn't wait.
"Don't tell anyone that I came here.
I have only 6000 drakmes 
left of the fortune that my father left to me.
The money is in this envelope.
Give it 
to Eliaou to keep for me."
And she left.
Mien Elia= same home, I told him what had happened.
Eliaou took the money and 
with it he bought a beautiful bracelet.
In the evening, he showed us the bracelet, and 
he said "The drakmes are turning to water.
I bought a bracelet for Suzane.
You will 
give it to her after she gets married.
I bought it for 7000 drakmes.
Before we give 
it to her, we will not know what it's worth.
If it is more than 6000 drakmes, you will 
givd it to her.
If it is less, you will give her 6000 drakmes."
"You can imagine," I said to the mother of the baby , "what would happen if I 
didn't have this bracelet when I meet Suzane."
When the lady left, I raised my eyes 
to the sky.
"There is a God in heaven!"
