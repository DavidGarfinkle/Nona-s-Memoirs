% mempages 148 to 165
% pdfpages 159 to 177

-150- 
When they left, I said to the man "This bag was full of jewellery when we gave it 
to them.
You see, each one of us has a story."
He said "Tell me a little bit of yours."
I answered "I've had enough for today.
But one day I'll write it down.
I want my 
children and  grandchildren to know what we went through."
Before, this man had been 
caressing his candlesticks.
Now I was caressing that handbag.
My heart was full of 
tears.
The wedding took place the next day.
I sang the 'Barouk Aba' to the brides.
The 
wedding was a great success.
Rabbi Molko performed the ceremony.
Of course, the Joint 
Distribution Committee sent photographers.
The doctor from my group was there.
I asked him if I had received any letters.
He 
said "Yes, I have one for you in my pocket."
He gave it to me.
I opened it and I saw 
the date.
I asked "Does it take three months for a letter to get here from Palestine?"
He answered "Oh, I've had it in my pocket for a long time."
The letter was from the two "deaf-mute" Russian officers.
They had arrived safely 
in Palestine.
The driver who came to get them at the beach turned out to be the uncle: 
of one of the officers.
"In Russia, one would say that this was a coincidence.
But 
I'm going to start to believe in God, like you."
-159- 
The next day I had a rendezvous with the deportee at the bus that went to the suburb by the.sea.
We took the bus and we stopped at a very beautiful beach that had been 
used for swimming before the war.
The man said "I spent all the summers of my youth 
here, with my aunt and uncle and cousin.
It was very nice of them to take me away from 
Salonica for the summers.
My mother couldn't take me.
My father was sick and she had to 
work."
I said "I'm sure you know the landlord who used to rent the bungalow to your aunt."
He said "Yes.
And this is the bungalow."
I said "Let's go there and ask where this girl lives."
We went to the house.
The wife opened the door, and then yelled for her husband.
"Look!
Look who came!"
"Her husband used to sell ice cream, and he would give me some every day," said the 
man.
As soon as he saw him, the owner of the bungalow fell into his arms.
He 
asked after the man's mother.
"You must mean my aunt," the man said.
The landlord said "No.
I'm asking about your mother."
"You don't know my mother.
She came to see me very seldom."
"Your mother is the dressmaker.
She was the one who rented the bungalow.
She paid 
for everything, even for the ice cream I gave you every day.
Your mother paid your aunt 
to bring you here.
But nothing was enough for your aunt.
She used to play poker with 
the neighbourhood big shots."
The man looked at me, and he said "Oh, God, and I was so bad to my mother."
He was 
crying.
The owner said to him "You have things here.
I went to see your mother in the ghet-
to.
She gave me a box of cutlery to keep for you, service for 24, in silver.
She also 
gave me a silver tea set.
Your mother was an angel.
She told me not to sell any of 
this, that I was to keep it as a souvenir of her love for you."
We asked him if he knew the man's old girlfriend.
The owner said "She lives just 
two doors down."
I said to the landlord "Come with us, please."
I was afraid that the man would kill her.
We went to the house.
A lady in her fifties opened the door.
When we asked her 
about the girl, she said "She's my sister, I'll call her."
A lady in her fourties 
came in.
The man said "You have the same name, but you're not the person that we're 
looking for."
The woman said "you're right.
But I have nothing to hide.
I'll tell you the whole 
truth.
Before the war, we used to rent rooms and bungalows here in the summer.
My 
husband had left me a pension, and I made a little money on rents, enough for my sister 
and me to live on.
During the occupation, no-one came to rent cottages, and we were 
both starving.
I began to look for a job as a housekeeper, so we could eat.
I went to 
Salonica one day to see the people who had rented bungalows from me.
At the first house 
I went to they hired me for three days a week.
They paid me very well and they gave me 
food.
They were so well off that they gave me enough to take home to my sister.
One day, this girl came to the house to visit.
I was amazed, because she had the 
same name as I. She was about the same age as I was.
But she looked much younger, with her fine dress and makeup.
I was born in Turkey.
So was she.
This girl was 
my employer's mistress."
"Who was your employer?"
I asked the woman.
"Papanaoum," she said.
"Oh, no!"
The man and I shouted.
For the Jewish people, Papanaoum was the Hitler, 
of Salonica.
"Yes," the woman replied.
"I worked in this house until the time of the deportations.
Then my employer told me that I was to go to work for the girl.
She had a beautiful home near the sea.
It was a Jewish house that had been taken over by the Germans.
It was the most beautiful home that I had ever seen.
They told me to bring my sister; 
there was too much work in this house for just one person.
They gave us the gardener's cottage.
They gave me an identity card, but not with my real name.. That's when I realized that these people had found me - I hadn't found them 
The girl must have changed her name way before the war, because she was a member of the 
Fifth Column."
The man said "I knew her with this name long before the war."
The woman said "Everything was arranged a long time aqo, because they were from the Fifth Column."
The man said "Oh, God.
She must have sent Keety to jail."
I said "I knew this from the time you started your story."
The man asked the woman "Where is this girl now?"
She answered "Let me finish my story.
Then you will know.
We lived in the cottage, 
went to the house every day to cook and clean, and then went back to the cottage.
One day 
we saw cases being removed from the house.
The girl told us not to clean that day; we 
were only to pack.
We packed the best jewellery, the finest treasures, like those of 
Monte Christo.
We put everything in cases.
I don't know where the cases were sent.
Four weeks before the German retreat, a man came to the house, with a huge valise.
They put two beds in a room near the kitchen, and we were told to stay between the kitch-
en and that room.
We were not to go anywhere else.
There must have been many people 
upstairs.
This man, in my opinion, was a doctor.
He would come wearing a white smock and a 
stethoscope in his pocket.
One day, we saw two people with two valises leaving the house.
Their faces were bandaged.
We heard the footsteps of many people coming down the stairs 
and leaving the house.
As the last person was leaving, we heard some shooting, and some-
one falling down.
When we were sure that everyone had left, my sister and I went up-
stairs.
We saw the doctor lying in a pool of blood, dead.
We ran out into the street.
We couldn't take a streetcar because we had no money.
We had left everything in the house.
We walked to the bus depot.
The driver knew us and I 
let us on for free.
And here I am.
All the years we were in the house, those people knew the Germans couldn't win the 
war.
They were taking Spanish lessons.
Spanish was familiar to them and easy to learn; 
they had many Jewish friends before the war, and they spoke Ladino."
We soon left the two women and returned to Salonica.
The man talked only of his 
mother, and of how much he had made her suffer.
"Oh, mother, I desire you now."
And I 
was taking pills to kill the pain of my earache.
The man begged me to go with him to 
meet Keety.
We arrived; the husband and the children were sitting at the table, waiting to have supper.
I stayed only five minutes, but I sensed the sweetness of this family.
Keety asked us if we had found the girl.
"She's somewhere in South America, or in Spain," I said.
I arrived at the hotel, and saw the doctor from my group.
The doctor asked me 
where I had been; he was looking for me.
I said "I went to stop a nice Jewish man from 
becoming a criminal.
He wanted to kill a girl.
But everything is all right now."
The 
doctor said he had a job for me, and we made an appointment to meet the next day in his 
hotel.
I went to the dining room, but I couldn't eat.
I got up and went to my house.
day, I went to see the doctor.
He was staying in a second-class hote.
My work was to b the taking of statistics.
I hated desk work.
But I had to do it.
In the doctor's room there was no desk.
I worked sitting on his bed.
It was winter and 
there was no heat.
The room was very cold.
I worked with my coat on.
Thank God for 
that coat!
I worked there a few days a week.
During the rest of the week, the doctor, the 
nurse, the driver and I would go to different villages to see patients.
My ear was still 
hurting me very much.
One day, there was no work to do.
The doctor 
said "Tomorrow you will go on an excursion with the Jews who have tuberculosis.
The 
Joint Distribution Committee has an excursion for them every day."
The next day, the truck with the tuberculosis patients came to pick me up.
There 
was a British official from the Joint Distribution Committee, the driver, the patients 
and myself.
We arrived at a stream in the mountains and got out to have lunch.
The 
tuberculosis patients asked me why the Joint Committee didn't do for them what Mr.
Arouk did for one of the girls who had been in the lager and contracted tuberculosis.
I knew the girl they were talking about.
This girl had been operated on by the 
Germans.
One day, Greek Jews came to work at this lager.
Among the workers was Mr.
Arouk.
This girl was in one of the windows, and she screamed "Mr.
Arouk!"
She threw a 
loaf of bread into his arms.
Mr.
Arouk, before the war, had been a member of the board 
of Matanoth Laevionim, and this girl knew him.
When Mr.
Arouk left the lager, he passed 
through Siderokastro.
The first thing he asked me was if this girl was alive.
I told 
him that she was, but she was sick.
As soon as Mr.
Arouk returned to Salonica, and recovered his property, he refuted a bungalow for this girl in Asvestokori, in the mountains,
A very good place for victims of tuberculosis.
He hired a very good housekeeper, and got the best doctor for this girl.
She is very healthy now, and lives in Athens.
Mr.
Arouk used to 
say "If I'm alive today, it's because this girl risked her life to throw me some bread 
from the window."
One of the tuberculosis patients said that we should sing the song of Shabbat, as we used to 
do with our families on every Shabbat, before the war.
The song went like this: 
La Cantigua del Shebat 
Dia de Alkad 
Asta dia de viernes 
Lazdro mi alma 
Vino dia de Ghabat 
Non aye komo el 
So benduho el dio 
I lo santefico 
Por el Ghabat 
Non komo el 
Lo benduho el dio 
lo santefico 
When we finished this song, someone else said, "Now let's sing a song of Leon Botton.
Let's sing the song called 'Sex Appeal".
It goes like this: 
Yo quiero moujer 
Que sea de famia 
Que tinga sex apil 
I que sea bien'vistida 
Que sea de salon 
I bouena education 
I djouga violon 
El piano que sea bien sonar 
Como flame a la cama 
I que sepa bien djousgar 
I arangear 
I bien peynar 
Tou idea pepo non la 
Topo fine 
Quieres tener ouna coucla 
De vetrina 
Para tomar anci a la legere 
Es mifor de se saver meter 
The echo of our singing drew the attention of some people who were selling eggs.
A girl and a man came up to us to sell us some eggs.
I saw that the girl was wearing 
a very well made jacket.
I asked "Who made this jacket for you?"
The girl wrapped the 
jacket around her body protectively and said "My aunt gave it to me."
In Ladino, I said to a tuberculosis patient standing near me "This jacket is mine.
Look inside.
There is a label of Katina PAyimada (one of the best dressmakers in Salonica).
I never had a chance to wear this jacket."
The TB patient said to the girl, "If you want us to buy your eggs, you must show 
us the lining of your jacket."
She showed us the lining, and we saw the label.
"Who gave you this jacket?
What is your aunt's name?"
I asked.
The girl started 
to run away.
One of the tuberculosis patients told the driver "Go and see where she goes.
Take 
the address and bring it to us."
We returned to Salonica with the understanding that I would go with the TB patients
again the next day.
They were very pleased with the way they had spent this day.
I 
was just in time for an appointment with Mr.
Silby, the director of UNRRA.
I told him 
that my contract would soon finish.
I wanted to go to Athens to have a conference with 
the head of the Palestinian Jews.
Mr.
Silby immediately arranged for my transportation 
to Athens.
I was to leave three days later.
I went home, but I couldn't sleep all night.
My ear hurt more than ever.
I had a 
very high fever, and I told myself how stupid I had been not to have it treated.
I had 
become sick in the line of duty.
The next morning, the truck with the tuberculosis patients came 
for me.
I told them that I was sick and couldn't go.
I went to the Hirsch Hospital, a Jewish hospital, the best in Salonica.
It had 
been founded by Dr.
Misrachi.
As all other institutions, it had been run by the Jewish 
community before the war.
But in 1940 it had been taken over by the Greek army for the
soldiers.
In 1941, the Germans took it over.
In 1945, the English army used it.
But 
because I was working with with UNNRA I had the right to go to this beautiful hospital.
I called to cancel my reservation to Athens.
An interpreter answered the phone.
When I said I was Miss Sarfatty, he said, "Oh, my God, my mother wants to see you!"
"Who is your mother?"
I asked.
"You stayed at my house for one night," he answered.
"A phony German officer 
brought you."
I said "I'm in the hospital now.
When I get well, I will call you, and we'll go 
to see your mother.
I'd like very much to see her."
In his excitement, the interpreter 
forgot to cancel my trip.
I learned that Mr.
Sibly was very mad that I didn't go to 
Athens.
But he never said anything to me.
And I never said anything, because the interpreter would lose his job.
-165- 
At the hospital, the doctor examined me.
He asked me how long I had had the pain.
I said three months.
"How long??
Three days?"
"No.
Three months," I said.
"And you waited three months with this pain in your ear?"
he asked.
"I was waiting to finish my contract," I said.
He said "You can have a contract for life with this ear.
You could become deaf."
I spent two weeks in the hospital.
By some miracle, I am not deaf.
There were two 
beds in my hospital room.
The other bed was occupied by a lady working for UNRRA, whom 
I knew.
But I was too sick to make conversation.
In two weeks, however, I had recovered.
The first person I saw after I left the hospital was the nurse from my group.
She 
said "I've seen many things in my life, but you take the cake.
I know the Sepharadim 
don't like to work.
But I never thought they would go to the hospital to avoid it."
A few days later, I called the interpreter to tell him I was ready to go see his 
mother.
I went to see her, but I didn't remember her or her house.
Oh, God, I had gone 
from house to house, always at night, so that I didn't remember where I was.
But at 
every house, Daniel and Mr.
Neri, the Italian consul's brother-in-law, would come to see 
me.
Naturally I brought a present for the interpreter's mother.
I brought her some 
nylon stockings that I had bought at the PX in Cairo.
A few days later, I went to see my superior, Mr.
Schaynay, in Athens.
Of course, 
he was angry because I had missed my appointment with him when I went to the hospital.
But he said nothing.
He was a very sweet man.
He didn't even ask why I had missed my 
appointment twenty days before.
The aim of my meeting him was to ask if I could stay 
longer in Greece, because while I was working I had not looked after my personal interests.
I wanted to stay awhile and arrange my estate, and then return to Palestine.
When I arrived at Mr.
Schahnay's office, he paced back and forth for two hours 
between his desk and his secretary's desk, as if I wasn't there.
Suddenly he said to me 
"you refused to work in the hospital.
And you refused to work in the disinfection room."
I said “I couldn't work at two jobs at the same time.
I had to feed the refugees."
(illegible sentence).
I wasn’t going to ask for any favours to be excused by Mr.
Schahnay.