% mempages to 94
% pages 83 to 96

-79- 
While I was thinking about this, she was talking to me. She said "I gave $500 for 
the campaign. But you gave to your sister. I saw thw way she came here last night, so 
elegant." I answered "My sister didn't take anything from the commmaity, even if she 
is so elegant. And I want you to know one thing, Phyllis. As long as you are in the 
Women's Federation, I'm not going to give one penny. Every Jew mnst help Israel. 
But that doesn't mean that I have to give to the Women's Federation. I can send money 
directly to Israel. Or I can give to the Men's Division. 
"I refused to work with you in the 1969 campaign. In 1966, it wasn't so bad to 
work with you, because you were district chairman and you wanted to collect every penny 6 
to show that you were somebody. But in 1967, because you refused to work miik under the 
direction of Hrs. Fisher, you did everything to destroy my district. Friday, before 
the Sunday we were to mike the collection, I went to prepare the allocations. I was al-
most finished. I left for an hour to give my husband his supper, and you destroyed the 
FOlocations. I really wanted to cry. When I asked Phyllis Wexler who had done this, 
she said 'I don't want to talk. Ask Phyllis Waxman.' So I asked you. The only thing 
you had to say was 'So what? 2/ Start your campaign next week!' In 1967, Phyllis, there 
was a war in Israel, and soldiers were dying. 
"A lady came to the office just then, Phyllis Jackson. She told MB not to worry. 
Even ie she and her sister, Sheila Lipson, had to work all night, the cards would be 
finished. They would make the allocations. You wanted to kill her when she offered to 
do this. These two ladies worked all night fisting the cards and putting them in order. 
But people like Phyllis Jackson are of no use to you. 
"It was only on Sunday that I found out why you had done this. You called all my 
workers; it had taken me weeks to convince them to become canvassers. You told them 
to go to Adath Israel for the distribution of the cards. You wanted to form a caval-
cade of canvassers, but Phyllis, if you want to do this, you have to find your own wor-
kers and not take mine. I complained to you. Your answer was 'If I didn't do this, I 
would be washed out of the Women's Federation.' I replied 'You do things at the ex-
pense of otherset Even after all this sabotage, I got the campaign into very good shape. 
In 196 you told me that 1 had done enough, that I shouldn't run around any more. 
I should leave it for the next year. I answered, I don't work for Mrs. *lkin, I 
work for the organization.' You weren't satisfied. You told the associate °halms 
take my cards before the campaign was over. I called Mrs. *akin and I told her what 
was going on. Mrs. *Thin told me that she would speak with the associate chairman. 
When I put the phone down, my doorbell was ringing. It was the associate chairman, 
coming to take my cards. I'm sure she was coming from your house. I asked her not to 
take them, because traditionally the cards are taken after the closing luncheon. She 
answered it 'If I take the cards now, Phyllis will give me a job of honour in 1969.' 
I don't work for honnour. I repeat it again for you. I work for the organization. 
My honour is in my conscience, for the acts I am doing. 
"You ruined my son's life in school with your gossip. You ruined my trip with 
gossip. These wonderful people don't know who you are. You would burn the wholv 
world down to get IT what you want. When you want to ask where my husband is, you 
make your hsnas shake. But I have never answered you. I want you to know that the 
tremor my husband has is a disease, like an ulcer, except no—one can see an ulcer. Et 
in my house we hays love between us, and you don't have this, Madame Chairman. In my 
house, we make sacrifices for each other, and this you haven't done and never will do, 
Madame Chairman. Only when I decide to write my memoirs will these wonderful people 
know who you really are, Madame Chairman. 
"When we went to Yad ve chen, I wanted to vomit seeing you represent the Jewish 
women of Canada. This was a disgrace for all those people who died in the crematoriums 
or by German bullets. There were names of people there who had had salt poured into 
the open wounds on their backs so that they would reveal where the partisans and Jewish' 
families were hiding. And you represented American Jewish women at Yad ya chen. You 
love it when someone else cries. I remember, and you remember, when a little girl ask' 
you howx you could laugh at a time like this. 
You see, Phyllis, it is not the $500 donation, and not the dozen new dresses, ths: 
change you inside. The only thing that has changed is that you no longer wear your car 
camelhair coat. My father was a prophet. He was rikht. But I don't know if I shauldm 
spit in your face or feel sorry for you. But I am sorry for the ladies in the Women' 
-81- 
Federation who voted for you. 
"People like you shouldn't collect money for sick people. How will you have the 
face to ask the public for money for Maimonades and Convalescent Hospitals? Those 
two hospitals are the centers of research for Parkinson's disease. If you want to 
bury the sick people and their families alive, you have to find an organization that 
collects money for cemeteries, and then bury the sick people and their families." 
She replied "Me? Me?" But I didn't let her continue. 
"Yes, you, you! When we went to the hxipmx houlpan, you told me to shut up. But 
now, it is my turn to tell you to shut up. In the haggada of Pesach, we say 'be 
°haw aba be yerounhnlaim.' But you forgot. In the middle of the haggada, we say 
Idaenoul. But with the insane superiority complex that you have developed since Sadie 
taught you how to dress, the daenou is scratched out of your haggada. 
"You told people that my husband and I didn't want other people to sit mi at our 
table, and no-one did. It was a miracle if someone sat down with us. I showed you 
how myhusband and I eat, not like you. You ate the fish with your hands, Madame 
Chairman. In my city, they would throw tomatoes at a person like you. 
"You go to the meeting and you laugh at people. You say they are wearing slippers. 
You make fun of them. They aren't slippers. They are just shoes with low heels. 
People who were in the interrogation room with the Germans can't wear high heeled shoes, 
Madame Chairman. You told everyone in the Women's Federation, 'Hah, Mrs. Garfinkle 
was in jail!' My only crime was being a Jew, Madame Chairman. 
"In 1968, you promised my associate chairman a reward if she took my cards before 
they were due. I refuse to work with people like you. But Mr. King is going to make 
you a star. He is the best advertising person I know. It's a pity that a talented 
man like he doesn't go to Hollywood. 
"Get off my back, Madame Chairman. Many people took advantage of us since my 
husbands hands started to shake. We used to have blind faith, but they swindled us. 
Friends of my husband! Bat it was only money. But you, in these fifteen days, you 
took our blood, Madame Chairman.And one day you will see, Madame Chairman, in Ladino we x 
say 'Las del avoua al avoua, las del vino Al yinn_ r 
God. What belongs in the water will go to the water, what belongs in the wine will go 
to the wine. You practice in the mirror, immitating the way Mrs. Gerstman talks. I 
hope you will be able to look at yourself in the mirror after what you did here these 
past fifteen days." 
I got up and left her. On the stairs, I met Mr. Gelber, a very nice person. 
told him "I want everyone in this group to know one thing. I'm not the poor cousin 
who came to visit the rich one.If you knew where I came from, the religious and hones' 
man that you are, you would understand. If there's one person here who deserves to bA 
invited by the community, it's me. And don't think that I came for free. I paid like 
everyone else. But this Madame Chairman made my life miserable for these fifteen day 
I have remembered all the miseries of my life. I spent the fifteen days remembering 
the miseries of the 1940's, and it is now 1969. Niseries that I never wanted to reme 
ber, and never wanted to talk about then to anyone. Bat $500 light donation is strong. 
er than being a hrmnnitarian. I don't ask that the community give me a gold medo, for 
what I did. But I will not let myself be insulted any more, especially when it is 
owing from some ignorant person because she gave SAR a $500 donation. She thinks 
that because of this, she is permitted to insult someone like me!" 
I left Mr. Gelber and I went to my room. I closed the door. My husband asked 
me what I had bought. This made me feel a little better, and I started to remember 
again. 
I started to remember the things I didn't tell Phyllis. When we were in Quinerette, 
Max wanted to visit Ein Hachofet on our day of leisure. Max had spent thirteen years 
on this kibbutz, and he had laid the first stone. We called a taxi, and we were ready to 
go. The taxi driver asked if there were only the two of us going. We said yes. He askec 
us if he could take his wife along. We went to Sfat, and picked up his wife. 
On the road, I started to talk to his wife. I asked her where she had been born.. 
"I am from YfIninP, in Greece." Oh, we had so much in common. We spoke Greek, and we 
spoke Lading, and we soon arrived at Ein Hachofet. 
The cbaverim were very happy to see us. They showed us the beautiful dining room, 
the two swimming tin: pools, one for little children, and one for adults. They showed 
us their industry; they manufactured screws. We had one day to feel like human beings. 
We returned later in the day to Quinerette. 
Everyone was ready to visit the kibbutz at X Quinerette. We also went to see it. 
Mrs. X was near us. When we started to vac walk, we saw a beautiful tall China vase in 
one of the doorways. 1 
"Oh," I sad to Mrs. X, "this is early Chinn." 
She jumped. "How do you know?!" 
I answered "Is it because I speak English with an accent that you ask me this? Or 
is it the portrait that the chairman has painted of me?" 
They shoimd us where the children played, and the barn. Suddenly Mrs. X said to me 
"you know, Mrs. Garfinkle, my mother wouldn't do what you are doing!" I replied, "of 
course not! I'm special" 
We went back to the dining room of the kibbutz. She said to me "You have money 
for taxis. But you don't have any for the Women's Federation." I answered "Yougre 
mistaken. I have plenty of money. But I won't give one penny to the Women's Federation 
as long as you are there. My grandmother used; to know a proverb: Quien al nouevo rico 
servio sou aiempo piedrio. A la fin del agno lo yanaron ladron o poutagnbro. It means 
'People who serve the nouveau riche are only wasting their time. At the end of the 
r 
year they will call them crooks and whores.'" 
-1114-remitidu1 mu 
Phyllis Waxman asked me if I wanted to speak to n 
group of French-speaking ladies. I accepted right away. Phyllis came to pick me up. 
The subject that I talked about was the situation of the Jews in Arab countries. I 
said that we needed a strong Israel. The next day, Phyllis called me. Instead of tell.' 
ing me if I spoke nicely or not, she said "I'll kill you if you tell Rossetta that you 
spoke last night." I asked her why. She answered "Because I told Rosetta that I spoke 
last night." I never told Rossetta that I had been the speaker. 
I remedbered another episode in 1965. I was district chairman, and I had to call 
a meeting. I told the Women's Federation the date on which I planned to hold ig the 
meeting, so that they would arrange for some speakers. The speaker was Mrs. Fisher. 
Mrs. Fisher always speaks very well. I called all the workers and area chairmen. The 
day of the meeting, most people showed up at my house. Whx: As soon as we were ready to\r 
start the meeting, Phyllis Waxman arrived. I was surprised, because she hadn't been 
invited. As soon as she arrived, she said to Mrs. Fisher "Let's gol2 Let's got" I sail' 
"Phyllis, she's the speaker!" "I want to go to the Y. There's a very important meet-
ing. All the ladies are going to be there." I said "Phyllis, you weren't invited here 
You can go to the Y by yourself. I'm going to wait another five minutes for Ella Cohen,,. 
the associate c►.hAirma_n, and then I'm going to start the meeting." Phyllis answered 
"Ella Cohen is not coming. She's sick." I said that since Ella was sick, I would 
start the meeting right away. 
As soon as I had said this, the doorbell rang. I opened the door. It was a taxi 
driver with a box in his hand. He said "I took a lady from Cote St. Luc to the I, and 
she gave me this box full of papers to give to you." I took the papers. Fla Cohen 
had sent them for the meeting. The taxi driver asked me for # $3.001 and I paid him. 
I said tp Phyllis "You told me Ella Cohen was sick, but she went to the I, like 
you want to do. I just paid 0.00 so that she could go to the Y by taxi." 
Mrs. Fisher started the meeting. In her speech she asked who were the spokesmen 
for the Jewish community of Montreal. One of the ladies answered "The Congress." 
Phyllis Waxman said "Let's go, let's go!" to Mrs. Fisher. At the door. I said to PhArlie 
"I paid $3.00 and the meeting is ruined. You people take others for suckers." They 
left. 
-85- 
The ladies at my house were very insulted and angry. They asked me "Is it only 
the ladies at the Y who are important? If we're not important, why did you call us 
to a meeting?" I said "Let's go, I'll serve some coffee." But everyone got up and 
left without having any coffee. I felt terrible. 
I put my coat on, took the car, and went to the Y myself, just to see what kind 
of meeting they were having. As soon as I entered; I'saw Ella Cohen in the lobby. 
I told her "Between you and Phyllis, you ruined my meeting." She answered "You don't 
understand. Phyllis and I want to climb to the top in km life. And to go up, to 
have honour, you must be seen talking to Mrs. Bronfman. And that's what we did today." 
21 I said "But don't you care about the campaign? Even with all the boycotting that 
you and Phyllis are doing, I'm going to get the campaign in good shape." She replied 
"Of course, because you are friends with Rossetta." "I don't work for Rossetta. I 
work for the campaign. And Mrs. Cohen, please pay me the $3.00 for your taxi to the 
1.." But she never paid me. 
The second person I saw was Phyllis Waxman. "Oh, Tova, are you invited here?" 
I said "No, I came to see what meeting you were having here, important enough to ruin 
mine." She said "You are a coo-coo. Do you think I would sit down with a bunch of 
women from Outremont if I have the chance to be with Mrs, Bronfman?" "But you weren't 
invited to my. meeting," I said. "I went to rescue Mrs. Fisher." "The best thing for 
MB to do now is leave," I said, "because I feel like pulling your hair out in publim," 
You can imagine how hard it was to find workers after this. I did most of the work 
myself. 
A few weeks later, it was decided to have another meeting in my district. Phyllis 
Waxman informed me that the meeting would not be held at my house. I was very glad 
about this, but just the same, I asked why. I also wanted to know where it would be 
held instead. She said "Your house is not up to the standards of the community." 
i said "I'm very proud of my house. I have a beautiful home." They looked for a 
house, and they decided to have the meeting at the home of Mrs. Rosenfeld. Mrs. Ros-
enfeld is always very generous towards all the organizations in Montreal. She lives 
in a beautiful cottage. Not more than five ladies showed up at the meetit Thank 
God that the board of the Women's Peapratinn wpm +.11=ePA fin mmkp fihP mpAting, a -1-0Athrz 
little warmer. 
When the meeting ended, I thanked Mrs. Rosenfeld. "It was very nice of you to offer 
your house, and I'm so.t-Ly that they disturbed you. You have a beautiful home, but mine 
is nothing to be ashamed of." 
Lying in bed in the hotel in Tel Aviv, I wanted to forget all of this unpleasantness, 
so I started to think about all the children and about the widow. She was so happy to 
have the chicoun, to get out of the slum she was living in. And I started to remember 
happy things. 
I had been liberated for three months in Israel, and I had heard nothing about the 
children or the baby. One day I received a letter from the baby's uncle. He told me he 
had just started to walk without crutches. On the third day of Pesach he would be in Tel 1 
Aviv. He wanted to make a third seder with me. At six P.M. someone would come to pick me' 
up and take me to the hotel where he would be staying. And I would see the baby, too. 
At six P.M. on the third night of Pesach, the doorbell rang. From the balcony where 
I was standing, I could see a taxi. The taxi driver was all dressed up in blue, and his 
shoes were newly shined. It was Miki's uncle. He took me to the hdttel where the family wl 
staying, with the babies. 
I took the baby in my arms. My cheek against the baby's cheek felt like velvet. My 
I) 
tears were falling like rain, and I said to the baby's uncle "Don't think I gave you this ri 
baby because I wanted to be rid of him. I'm too attached to him. I suffered so much to 
save him from the Germans. I gave you the baby because I have nothing to offer him." 
Suddenly we heard people singing a Greek song. I asked if we were in a Greek hotel. ;1 
"Oh," said the uncle, "let's go to the dining room to hear the singing!" "No," I said, 
"I want to be with the baby." g "The baby has to go to sleep." He called the nurse and 
she took the baby away. 
We went to the dining room, and the first person I saw was the priest. Of course, he 
was now a doctor, with a very fancy army uniform. The baby's uncle had arranged a reunion 
for me. Miki's uncle was there, too. The priest introduced me to two Jewish officers 
from the Brittish Army who had come with him. Suddenly, the fisherman with a new suit 
and shined shoes appeared. Suddenly, all the children came in as well. 
There was a lovely buffet, with all the best things to eat. The baby's uncle said 
"attention!" And he started to tell us the story of Pesach, how the Jewish people had 
left Egypt. 
"Pharaoh had the ten dinims. I'm telling you this so that you will know that ic his-
tory repeats itself. The Germans, on the Russian front, have exactly the ten dinims. 
Moses helped the Jews get out of Egypt. You see how history repeats itself; Maritsa helped 
us get out of the claws of the Germans. The Jewish people wandered in the desrt for 40 
years, and they received manna so that they could survive. You people, you received manna 
also. The priest taught you how to kill birds for food. If the Turkish government hadn't 
given us trucks and trains to go to Syria, we would have spent 40 years in the desert, too. 
If the Brittish government hadn't taken us by train to the Promised Land, we would have 
spent 40 years in the desert, too. And now, we are going to start the programme." 
The baby's uncle said to me "The children are well prepared." They took a big table, 
and two children tied up all the others around the table. It was exactly the same scene 
as in Evia, when the bandits tied us up. They called me to sing the 'Elia elia mo yalessa." 
When this was finished, we sang the traditional Pesach songs. We ate from the buffet, 
and we sang the traditional song, 'Eloenou the ba chamaym. 
Eloenou chb ba chamaym 
Eloenou chi ba chAwAym 
El dio club mos mand 
A yerouchalaim 
Con la caratana grand 
Coualo son los dos 
Dos maestros padrbs son 
Primero eze el creador 
Barekou barouk chemon 
Eloenou ebb ba chwinnym 
El dio qub mos mand 
A yerordir 1 aim 
Con la caravana grand 
Coualo son los trbs 
Trbs mouestras padrbs son 
Dos Moehb I Aaron 
Primero eze el creador 
Barekou be.rouk chemon 
Eloenou chi: ba cihannym 
El dio qub nos mand 
A yerouchasim 
Con la caravane grand 
Coualo son los quatro 
Quatro madres de Israel 
Tres mouestros padrbs son 
Dos Mochb i Aaron 
Primero ese el creador 
Barekou barouk ahem= 
Eloenou chi ba piipTnnym 
dio qua mos mand 
A yerougthn 1 n im, 
Con la caravane grand 
Coualo son los cinco 
Cinco livros de la ley 
Quatro madres de Israel 
Tres mouestros padrbs son 
Dos Nbchb i Aaron 
Primero ese el creador 
Barekaa barouk chemon 
Eloenou chb ba ehAmaym 
El dio qub mos mand 
A yerouelin 1 slim 
Con la caravana grand 
Coualo son los seih 
Seih dias de la semana 
Sinco livros de la ley 
Quatro madras de Israel 
Tres mouestros padrbs son 
Dos Moab i Aaron 
Primero ese el creador 
Barekou barouk chemon 
Eloenou chl ba elapripym 
El dio qub mos maned 
A yerounhAlAtim 
Con la caravana grandb 
Coualo son los sietb 
Sietb dias con chabat 
Sech dias de la semana 
Sinco livros de la ley 
Quatro nadres de Israel 
Tres mouestros padrbs son 
Dos Mochb i Aaron 
Primero ese el creador 
Barekou barouk chemon 
Eloenou chb ba chamaym. 
El dio qub mos mande 
A yerouchnl aim 
Con la carotene. grandb 
Coualo son los otho 
Otho dias de 4textfala ke9pck 
Sietb dias con ehabat 
Seth dias de la semana 
Sinco livros de la ley 
Quatro =Ares de Israel 
Tree rnouestros padrbs son 
Dos Mochb i Aaron 
Primero ese el criador 
Barekou barouk chemon 
Eloenou dab ba elinTnn3rm 
El dio qub mos mandb 
A yerouvhalaim 
Con la caravan grandb 
Coualo son los noueve 
Noueve mezes de la Pregnada 
Otho dies de las t oopck. 
Siete dias con chabat 
Sech dias de la semAns 
Sinco livros de la ley 
Quatro madras de Israel 
Tres mouestros padrbs son 
Dos Mochb i Aaron 
Primero ese el criador 
Barekou barouk chemon 
Eloenou chb ba chamaym 
El dio qub mos mama 
A yerouelaninim 
Con la caravana grandb 
Coualo son los dies 
Dies comandamientos de la ley 
Noueve mezes de la pregnada 
Otho dias de twine. 1-0 %.) r-
Sietb dias con chabat 
Seth. diets de la semana 
Sinco livros de la ley 
Quatro madras de Israel 
Tres mouestros padrbs son 
Dos Mochb i Aaron 
Primero ese el criador 
Barekou barouk chemon 
Eloenou chb ba clanTvriym 
El dio qub nos mandb 
A yerouelininim 
Con la caravan grand 
Coualo son los pnzb 
Onzb yos sin Joseph 
Dies comandamientos de la ley 
Noueve mezes de la pregnada 
Otho dies de # UsuNka INovrok. 
Sietb dies con ehabat 
Sech dies de la semana 
Sinco livros de la ley 
Quatro madres de Israel 
Tres mouestros padrbs son 
Dos Mochb i Aaron 
Primero ese el criador 
Barekou barouk chemon 
Eloenou chi ba ellgymAym 
El dio qub mos mand 
A yeraunbAlAim 
Con la caravana grand 
Coualo son los dose 
Doje ijos con Joseph 
Onzb izos sin Joseph 
Dies comandpmientos de la ley 
Noueve mezes de la Rregnada 
Otho dies de kedispa 
Siete dies con chabat 
Sech dies de la semmin 
Sinco livros de la ley ' 
Quatro madres de Israel 
Tres mouestros padrbs son 
Dos Mochki Aaron 
Primer() ese el eriador 
Barekou barouk chemon 
Eloenou chb ba chmmaym 
El dio qub mos mand 
A yerounlisa 1 n 
Con la caravana grand 
Coualo son lo treje 
Treje ijos con Dina 
Doje ijos con Joseph 
Once izos sin Joseph 
Dies comandamientod de la ley 
Noueve mezes de la pregnada 
Otho digs de koalas. I..ovpu\ 
Sietb ding% con chabat 
Sech dies de la seminn 
Sinco Iivros de la ley 
"Quatro madres de Israel 
Tres mouestros padrbs son 
Dos Hoch i Aaron 
Primero ese el criador 
Barekou barouk chemon 
This song means, roughly: 
God send us to Jerusalem 
With a big caravan 
First is the Creator 
Second, we have two fathers, 
One is Moshe, and the other, Aaron 
With the Creator there are three 
Four, we have four mothers of Israel 
Five, the five books of the law 
Six, the xi six days of the week 
Seven, seven days with shabbat 
Eight, the eight days of houpa 
Nine, the nine months of pregnancy, 
Ten, the ten commandments 
Eleven, the eleven sons without Joseph, 
Twelve, twelve sons with Joseph 
Thirteen, thirteen children with Ding 
But first is the Creator 
Barekou barouk chemon 
Again, I started to mRke a resume in my mind of this trip I was on to Israel. 
It is a shame that this chairman and others like her think that with a $500 donation, 
they are militant zionists and they have bought Israel. I started to remember the 
boat Sarah Primo, that came to Salonica. 
It was a school on a boat, belonging to the sukonouth. The sukonouth, for the 
Jews of Salonica, was synonymous with the salvation of Israel. We had a reception for 
the crew of the Sarah Primo in one of the clubs in Salonica, called the Zentin. All 
the zionist organizations were gathered there. Eliaou introduced the captain. Eliaou 
represented the Keren Hayssod. Mr. Bessanthi introduced the captain's speewch. Mr. 
Bessanthi was one of the foremost intellectuals in Salonica. He knew ten languages, 
perfectly. He was a journalist. He had a wife and two daughters, as intellectual as 
he was. Mr. Bessanthi was later taken by the Germans, and we never found out what 
happened to him. 
This meeting with the crew of the Sarah Primo was like a holiday. The Jews were 
very happy to see the first boat—school from the sukonouth. It was in the gulf of 
Salonica, at Niki Street. The atmosphere was like that on March 25, the Greek Nat— 
ional Holiday. There were so many people going to see the boat. All the flags on 
the boat has Marron T vAiA nn -Fham Theme wrefirnilma of -Eho nrpm were white with Maven 
1 
David. All the crew was Jewish. Officers, Jewish. We thought at that time that Pales-
tine was ours. We saw the Jewish flags with Magen David on the boat with our own eyes. 
The name of Sarah Primo was written on the boat with Hebrew letters. You had to 
see it all to understand the reaction. The captain gave a present to Eliaou, a. miniature 
of the Sarah Primo. He gave Jewish flags with 'Strap Primo' marked on them to Regina as 
to me. Eliaou ordered a vitrine to be made, in which to put the boat and the flags, and 
this was put in the best place in our library. Everybody would tell Regine and me "How 
lucky- you are to have a brother like Eliaou! Because of him, you have the Sarah Primo, 
too." 
And just look at the way I am considered here, on this trip. All I could think of 
was that people like this made something of themselves while Hitler made us, the ones 
Europe, lose. How many Jews died with the hope that the American Jews would save us? 
I couldn't swallow. She told me she gave $500. Jews of America, it is time to 
show people like this that it is not the $500 that we need. What we need are clean com-
munities, and hrimnnitarianism. 

I started to remember 1968 again. I hated that year, but I couldn't help myself. 
My associate chairman would come every morning to pick up the money I had collected the 
day before. Everything was going very smoothly because I hate to have money that isn't 
mine in the house. One day Phyllis Waxman called me. "Your associate chairman is sick 
and can't come for the money today.0 "Oh, God, I have $6000 that I collected yesterday 
in the house." She said that she would come to collect the money if I wished. I was very 
glad that someone would take the money. Phyllis said that she would be at my hpuse in an 
hour. I said "I won't be home in an hour. But I can take the money to your house right 
away." She agreed. 
Five xxst minutes later I was ringing her doorbell. Her two children opened the door. 
The two together =INN called their mother: "Hey, Mh, Mrs. Contaminated is here!" I 
said to the children in Ladino :"De la cavessa fieda el peche." This is an old proverb 
whicif. means "The fish starts to smell from the head." I had just finished saying this 
when I heard Phyllis' voice. "Tova, I'm in the bathroom, wait a little." She knew to 
whom her children referred by that name. This was the niCknAme she had given us because 
my husband's hands were shaking. But she never dared to call me tic this to my face. 
,I started to remember the question of the amalgamation of Adath Israel School with 
Young Israel School. It was no secret that all the Jewish schools were having money prob-
lems, and Adath Israel was no exception. Mr. Sternthal, a member of the Board of Education 
and one of the founders of the school, called the parents together to try to find a solu4 
tion. Ten or fifteen parents attended the emergency meeting. Mr. Sternthal explained the 
school's situation to us. He told us that the teachers there were very good, and to have 
this high quality of teaching, one had to pay for it. Many pupils who came to the school 
in the afternoons didn't pay. There were others in regular school all day whose parents 
couldn't pay. A solution had to be found, or the school would have to close. 
I spoke and I said "The district has changed. It has become smaller. People are 
moving away. There is no place for two Hebrew schools. If we join forces with Young Is-
rael, we can keep the school." There were many members of Young Israel at the meeting,' 
parents of the pupils in the High School. They found the idea very good. We decided that 
Mr. Sternthal would speak with the Board of Education, of Adath Israel, and they would 
speak with the Board of Education of Young Israel. 
A few weeks later we found out that Mr. Loewy, member of the Board of Education of YovnE 
Israel, and Mr. Sternthal had taken the proposal very seriously. After many discission 
the amalgamation took place. There was a meeting of the Adath Israel Sisterhood, and 
the question of the schools came up. The Sisterhood was split. Some were for the amal 
gamation, and some were against it. Some of the members went so far as to say that if 
the amalgamation took place, the Sisterhood would no longer give assistance to the schoo: 
I explained to the women that we had committments: "We took children without pay. The: 
are many Moroccan children in the school whose parents cannot pay. What is the different 
if the school is Adath Israel or Young Israel? All the children are Jewish pupils lean__ 
ing Hebrew. The pupils of Young Israel don't need us; the families there can pay very 
well. And also, new blood on the Board of Education is very good for the school." All 
of this was marked down in the minutes of the meeting. The meeting was then adjourned. 
At the next meeting, Phyllis sat down near me. Another friend came to sit with U.S. 
They both started to talk to me. "We are going to remove what you said at the last meet., 
ing about the schools from the minutes." I looked at both of them. I couldn't believe 
my ears. I asked them why. They both replied "Yes, people are against this amalgamatiot 
I said "Everyone wan have her own opinion. If you, or you, want to contradict me, you 
can speak today, and it will be marked down in the minutes." "Oh, no; This has to be 
taken out of the minutes." They looked for the Secretary, Mrs. Glazer, and they told 
S. 
her "Remove what Mrs. Garfinkle said at the last meeting from the minutes." 
At first I wanted to walk out of the meeting, but I decided to stay to see how far x: 
would go. A recommendation was made, not to give donations to the Hebi,ew Academy, the n/ 
of the school after eiTolgamation. When the meeting ended, Phyllis gave me a ride home. 
I said to her "I gave a donation to the Hebrew Academy yesterday. Barouch Pollak called 
me, and I'm very happy to be able to give." She answered "You are a fool! The rabbi is 
against this amalgamation. If you want to be somebody in life, you have to agree with th) 
1 
rabbi." I said "Which rabbi? Tell me the name." She didn't want to tell me, and I nevc 
did find out who this famous rabbi was. I told her "you can tell your rabbi that he is 
not the boss of my pocket." She replied "Sternthal formed the school for his own child-
ren. Only Sternthal should give money to the school, and not you. The Sternthal girls 
should come and work for the Sisterhood. But they don't want to come. They are bigshoter 