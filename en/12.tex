% mempages 222 to 253
% pdfpages 249 to 280
 
On visiting day, all the parents came, including the parents of 
this boy.
His parents asked R. Rabbi Horowitz how things were going with their son.
"No problem.
Everything is going wonderfully," answered Rabbi Horowitz.
I opened my eyes wide in surprise.
But before I could say anything, Mrs.
Horowitz took me aside, and she said just one word: "politics".
Despite this, I got a great deal of satisfaction from working in 
the camp, and especially from helping this boy.
I enjoyed that summer 
very much.
Rabbi Horowitz was a very well-educated man.
He knew how to impress 
others.
He was like an actor who thrives on the applause that he re-
ceives.
This man lived on compliments.
We were all supposed to compli-
ment him.
There was a very nice group of conselors from Calgary at the camp.
I was surprised to hear these intelligent people paying Rabbi Hor-
owitz one compliment after another.
But soon everyone learned that if 
they wanted to stay on at the camp and work in peace, they would have to 
praise Rabbi Horowitz.
If not, they could pack their bags and leave like 
Dr.
Mendes.
And finding other jobs was difficult in the middle of the 
summer.
Mr.
Gold a member of the Board of Karen Hatarbout, came twice a 
Week to make, sure that everything was in order at the camp.
I dreaded 
the fights between Mr.
Gold and Rabbi Horowitz.
This wealthy man.
Mr.
Gold gave his time and money to the camp, and of course, he never 
paid Rabbi Horowitz any compliments.
And because of this, the two men 
would always be fighting.
Every Friday night we had a beautiful ceremony to greet the shabbat.
I made a special dress for the Queen of the shabbat, from material that 
Mr.
Gold brought, on a sewing machine that Mr.
Gold brought.
The children were gaining weight and getting stronger, and they enjoyed every minute of the day.
The only thing they didn't like was rest period.
A few weeks later, on a Sunday, a meeting was held in the camp's 
dining room.
The members of the Board came, and each one decided to 
buy shares in the camp.
Each Board member would give $500.
Everyone 
was wealthy except Mr.
Pougath, but he was the first to give $500.
I 
was a witness to this act of generosity.
I worked at Camp Massad for two seasons.
I enjoyed every moment 
of it.
They asked me back for a third season, but by that time I had 
a son, Ely, who was born on April 13, 1949.
Ely was born during Pesach, and my dream was to hold Pesach in the 
traditional Sepharady way.
On Ely's first birthday, I made my first 
seder.
I invited many people, students from the University of Montreal and McGill University, and other girts and boys who were away from home and had nowhere to go for Pesach.
The Haggada was sung in Hebrew, 
in Ladino, and in French.
We sang all the traditional pesach songs.
The next day I gave a birthday party for Ely, and I invited many 
children.
In Salonica there was a tradition: every friend and relative, 
would come to our house in the morning after the synagogue, to visit 
my grandmother.
My grandmother would give a boiled egg to each man that 
came.
The recipe went like this: the eggs were cooked in their shells 
on a very low flame, with a little bit of olive oil and the skins of 
an onion added to the water.
The onion skin gave the eggs a beautiful 
colour.
After the eggs were boiled, they were polished with a rag.
On 
Ely's first birthday, I made these eggs, and I gave one to each child, to 
take to his father.
On Ely's second birthday, I hired a magician, a student from McGill.
As the magician was taking eggs out of the children's ears, the doorbell'.
rang.
I opened the door, and there was a man standing these with a pareel.
I asked "Do I have to sign anything?"
"No," he said, "you have to kiss me."
I looked at his face for the 
first time.
It was Nicola.
-224- 
After that day, Nicola came to eat with us twice a week.
He 
first took a job with JIAS, but he didn't make a good living there, 
so he found another job.
On Ely's third birthday, I was preparing for pesach and for his 
party.
I didn't have much time to be with Ely.
Ely was keeping himself 
busy walking all over the house in his father's slippers.
On the day of 
the party, I dressed Ely in his new clothes and shoes.
The children were 
starting to arrive when I noticed that Ely had put his father's slippers 
on again.
"Look," I said, "the children and their mothers are going to see you 
with your father's slippers on, and they'll think you are misken (miser-
able).. They will say 'What kind of a mother do you have?
She didn't 
buy you new shoes for your birthday?'"
Ely took his father's slippers 
off and never wore them again.
I had just finished talking to Ely when the phone rang.
It was a 
lady who introduced herself.
"I am Mrs.
Strean, the wife of Dr.
Strean, 
the gynecologist.
My husband and I received a letter from someone who 
is also a friend of yours.
I would like to meet you."
I invited her to 
my house, but she said that she would call me after pesach, and we would 
meet somewhere for lunch.
We made a date to meet in the lobby of the 
Jewish General Hospital.
"How will I recognize you?"
I asked.
*There is a switchboard in the lobby," she said.
"You will go up 
to the receptionist and tell her that you have an appointment with Mrs.
Strean."
The day of our appointment came.
Ely and I, both dressed very nice-
ly, went to the Jewish General, and went up to the switchboard in the 
lobby.
As soon as I got to the desk, a very elegant looking lady came 
up to me and asked if I was Mrs.
Garfinkle.
"I invited you to have lunch with me at a restaurant” she said.
“I know” I said.
"But with a child?"
she asked.
I explained to her: "I take this child to the restaurant once a 
month, and he eats like an adult.
This is our social afternoon.
Don't 
worry.
He will behave like a man.
In every restaurant we go to, they 
give us a high chair."
As I was talking, a distinguished looking gentleman came up to us.
He was wearing a white smock, green pants, and slippers.
There was a 
stethoscope in the pocket of his smock, and I realized that he was a 
doctor.
I saw that Ely had become very sad.
I asked him what was wrong in 
Hebrew, the only language that he knew.
He said "The doctor is misken.
His mother didn't buy him any shoeg 
And I said "Don't worry.
I'll tell his mother to buy him some 
shoes."
"Maybe his mother doesn't have any money," said Ely.
"Don't worry, " I said.
"If his mother can't afford them, I will 
buy him a pair."
Ely was happy again.
Mrs Strean asked me what had been 
the matter.
"It's a long story!"
I said.
We went to the restaurant on the ninth floor of Eaton's.
Mrs.
Strean couldn't get over the fact that Ely behaved like a little man.
She asked me how I knew the distinguished gentleman who had written to 
her.
"How did he come to be your friend?"
she asked.
For sure, you are curious because I didn't go to school in Canada:" I said.
"Correct, she answered.
So I told her the story.
When my brother finished high school, Mr.
Nahama was the speaker at the graduation ceremony.
He was the foremost 
intellectual that the Jewish community of Salonica had produced.
Mr.
Nahama had the most famous library in Salonica, with parchments dating 
from the fifteenth century.
When the Germans came, they dismantled the 
library very carefully, and sent it in trucks to Germany.
The graduation was dedicated to the memory of the Russian pogroms.
The choir sang songs about the events in Russia at that time: 
Aye 2 Mille Agnos 
Aye.
2 mille agnos 
Ke estamos penando 
De la sangre que oourio 
De los djidio matados 
Esto ese Russia 
I tambien malisia 
I en todo en lougar 
Ande bive Israel 
I en todo el lougar 
Ande soufre Israel 
Los campos de engneve 
Estan corelados 
De la sangre que courio 
De los djidios matados 
Esto ese Russia 
etc... 
For two thousand years 
We suffered 
The blood flowed from our.
veins 
During the.
Jewish massacres 
This is Russia 
With malice in every place 
Where Israel lives 
In every place 
Israel'suffers 
The fields matt of snow 
They are red 
With the blood from our bodies 
From the assassinated Jews 
This is Russia 
etc... 
The graduation ceremony was closed with a hora.
It was danced by the gra-
duating class, and everyone was: singing.. This is the song, called "Long 
Live Liberty"; 
Biva la libertad 
I de noueva la natura 
Triomphante glorioza 
Aklareska nouestra 
Vieja bandiera 
Despoues que los corasones 
-227- 
Alegria esperada 
Biva la nation 
Biva sou libertad 
And new nature 
Triumphant and glorious 
Will light up our old flag 
And after, our hearts.
We hope for happiness 
Long live our nation 
And long live liberty.
Mr.
Nahama said to the graduating class: "As Jews, we have to show the 
gentiles what we are.
Many gentiles have never seen a Jew.
For them, 
all Jews are like the Merchant of Venice of Shakespeare, because 
they have never had any contact with Jews.
It is you, our youth, who 
must show them who you are, and what you are.
We hope, and I am sure, 
that one day we will have a state of Israel.
And we will need friends, 
and these friends must be convinced that Shakespeare made a mistake.
I 
brought with me a list of pen-pals, from French speaking countries.
Each
of you is to take a name."
My brother, Eliaou, picked the name of a French Canadian.
They 
wrote to each other for many many years.
They became like family.
During the German occupation, a letter came from this man through the.
International Red Cross.
It was like a questionnaire.
We answered all the 
questions and signed it.
This was the last time that we heard from him.
After the war, I worked first UNRRA, distributing food to deportees.
A 
driver from my group told me that an officer wished to speak to me.
I 
told him* to send him to the kitchen.
He came and introduced himself, 
but I was so busy that I didn't pay attention to his name.
I said "If 
you want to talk to me, help me, and I'll finish quicker."
He helped.
When we finished, I asked him if he wanted to eat.
"I'm starving," he said.
The cook prepared something for us, and we ate at my desk in the 
kitchen.
I asked the officer what I could do for him.
"I came to ask you if you know Eliaou Sarfatty," he said.
-228- 
I looked at his face closely for the first time, and I said "Oh, 
God: We kept your picture at our house: You are Eliaou's pen-pal!"
mutual 
"And in this way," I said to Mrs.
Strean, "I met our Imam friend."
Mrs.
Strean was fascinated with my story.
For many years, she would call 
me from time to time, and we would go to a restaurant.
She never invited me to her home.
She used to call our meetings "escapes"; I used to 
call them "Social afternoons".
Soon after Ely's third birthday, we moved to a new apartment.
One 
day, as I was coming back from the park with Ely, I met Mr.
Pougath on 
the stairs.
"What are you doing here?"
I asked.
"oh," he said, "I've lived here with my wife for 25 years."
He saw 
Ely, and said "You have a prospect here for Camp Massad."
He was very 
proud of the camp.
Ely and Mr.
Pougath became very good friends.
Mr, Pougath would practice his Hebrew with Ely.
One day, Mrs.
Pougath invited Ely and me to 
her house.
Mr.
Pougath showed us the certificate of shares that he received for his donation to Camp Massad.
He was very proud of having helped children speak Hebrew.
He said "When Ely will be the right age for 
camp, I will have someone to visit at Camp Massad 
A while after Ely's third birthday, I got a call from a social worker at Red Feather.
She asked me to help them out.
I asked what kind of 
help they wanted.
They wanted me to be a volunteer interpreter for pregnant girls.
"You have been very highly recommended for your discretion," said 
the social worker.
I said that I would like the work very much, but I had a child.
"But 
if you want, you can bring the girls to my home.
My son and I are the 
only ones there during the day."
The social worker thought that this was a very good idea.
We made an appointment and the next day she arrived with a pregnant girl.
I 
would invite the girls for lunch, and serve them their native foods.
One
day I asked the social worker "What will you do with these girls after 
they give birth to rehabilitate them?"
"They will give their children up for adoption, and they will go to 
work," she said.
"But these girls don't have a trade, and they don't want to be cleaning women in private homes," I said.
"I used to have a directory when I 
worked at UNRRA, whose name was Mr.
Sibly.
He said "It is very good to 
give feed to hungry people.
But we have to teach them how to help themselves.'"
"But what could we teach these girls?"
asked the social worker.
"We can teach them to become sewing machine operators.
They can then get jobs that require experience."
"And who will teach them?"
asked the social worker.
And I said "I will."
My husband had a children's wear factory at the time.
I borrowed an 
industrial sewing machine from him.
My living room became a factory, and 
the girls learned their trade better than their teacher.. After they gave 
birth, they found wonderful jobs.
To this day, they call me, and they 
come to see me.
When Ely was old enough to go to school, I received a phone call 
from the Jewish General Hospital.
It was one of the doctors, asking if 
I wanted to be an interpreter for the outpatients.
I accepted.
"First you have to go see the director of volunteers," he said.
I went for an interview with the director.
She said "You will be a 
interpreter, but we would also need you to work as a saleslady in the gift 
shop a few days a week."
I agreed.
"Oh," she said, "in half an hour, you have your first session as an 
interpreter.
You will need a smock."
She opened a parcel and handed me 
a smock.
I tried it on.
'Ten dollars," she said.
I paid her.
-230- 
"Oh, by the way," she said, "to be an interpreter here, you have to be a member of the auxiliary.
It's $4.00" I paid her.
I put my smock on.
At that time, I had long hair.
She pushed my hair away 
From my face to look at my ears.
I said to myself "I know that when someone wants 
to buy a horse, he looks at its teeth.. But why look at the ears of a potential 
interpreter?"
The director saw that I was astonished, and she explained.
"No volunteer is 
allowed to wear earrings, except the doctors' wives and the wives of Board members."
I thought this was the stupidest rule that I had ever heard.
I started to work as interpreter, and in the gift shop, and at lunchtime I 
would also feed patients who could not feed themselves.
I was at the Jewish General Hospital every day from 9 to 3.
I did everything I could to leave at 3 
o'clock, because Ely came home from school at 3:30.
The gift shop was closed for renovations for a while, so I worked in the canteen.
One day, as I was going to sign out at 3 o'clock, the director of volunteers 
was waiting for me in her office.
"Oh," she said, "you're leaving?"
"Yes."
I went to the parking lot to get my car.
To my surprise, the director was in 
the parking lot.
She ,said "I couldn't believe my ears when I heard, but now, I don't 
believe my eyes:" 
"What is this about?"
I asked.
her.
"You just came to Canada.
You don't even know how to speak English well.
And 
yet you have a beautiful car."
I said "I have a secret for you.
Canada is bilingual.
When I talk to my car, 
I speak in French, which is my 'best language."
"Come to my 
office," she said.
"I have a very important job for you."
'I'm sorry," I said.
"I have to go home.
It's Thursday, and my maid has the 
afternoon off."
"Oh, my God, you have a maid?"
“Yes, but I have another secret for you.
I speak Greek to my maid, not English."
"Oh, I'm so sorry you can't come now.
I have such a responsible job for you!
Aren't you at all curious?"
she asked.
I said "With my credentials, a car and a maid, I'm not curious.
I'm sure you 
will give me a high position."
She never even said good morning to me before, but on that day, how wonderfully she treated me!
I decided then and there in the parking lot that no matter 
what this responsible job was, I would refuse it.
But the next day, she started the conversation in a very nice way.
"There's 
a little girl who goes to a school downtown near Mount Royal.
Her mother works, 
and her father died a while ago.
She needs treatments, and she has to be picked up 
from her school.
The teacher will be notified, and after her treatment, she must 
be taken back to school.
It's only a matter of a few days."
I said yes right away, despite my intention to refuse.
I wanted to help this 
family.
But I soon became a taxi driver, picking up many patients every day.
It 
got to a point where I hated the car and hated to drive.
But on account of the little
girl, I continued.
After 2 or 3 months, her doctor told me that she needed 
only one more treatment.
I went to talk to the director of volunteers, and I said 
"I'm very sorry.
I will take the little girl for her last treatment, and then I 
will stop.
I am getting very nervous, and I'm afraid that I will have an accident."
She said "You are going to going regret this:" She was very angry.
"I regret not being able to help all the people I have been bringing to hospital every day," I said.
"You are going to be even sorrier!"
I explained to her that I was neglecting my group.
A group of us brought 
food to old and sick people every day, and my turn came up twice a week.
At that 
time, there was no convalescent hospital, or meals on wheels.
There were only groups 
of dedicated women who performed this service.
And Voula, my maid, used to prepare 
the meals in my kitchen.
The director of volunteers asked me again “You are sure that you won't regret 
refusing to drive these patients to the hospital?
You will have to deal with me!"
-232— 
"Well," I said, "I regret it, but I'm not going to do it any more."
Two days later, I was serving coffee in the canteen, when Mrs.
Kershman came 
in.
.She invited me to have lunch with her at Miss Montreal.
Mrs.
Kershman was
born in Montreal.
She graduated from MacDonald Teacher's College, and the McGill 
University School of Social Work.
She got her B.A.
at Sir George Williams, and her 
M.A.
at University of Montreal.
She was a past president of the Jewish General Hospital auxiliary.
Her late husband was the very well known neurologist,
John Kershman.
Mrs.
Kershman was very active on the Board of Teknion, and at 
the Y. 
We went to Miss Montreal.
As we waited for our meal, we talked.
I bad the 
feeling that she was cross-examining me.
I said "Are you cross-examining me?.
If 
yes, just ask me what you want to know.
I'll answer you."
She asked me why I brought food to all those people.
"You are asking me why I am humanitarian.
There are 15 ladies who do this.
And Mrs.
Strean gives us medicine for these people.
These people can't afford to 
buy their medicine, but Mrs.
Strean can get samples from her husband and the other 
doctors.
Am I under investigation?"
"Yes," she answered.
"The director of volunteers told me that if I didn't chauffeur the patients any 
more, I would regret it.
I beg you to tell her that I do not regret bringing food.
to old and sick people.
I'm going to continue to do it, whether she likes it or 
not.
These people wait for us like machia, especially when we go to cut their nails 
or wash their hair."
After this talk, Mrs.
Kershman and I became very good friends.
To this day, we 
call each other and see each other.
The day after this incident, I went to my job at 
the hospital as interpreter for the outpatients.
I wore a pair of earrings, because I 
was going to a bridal luncheon after work.
As soon as she saw that I was wearing 
earrings, the director said "Take those off!"
I put the earrings in the pocket of 
my smock.
I finished my work and left, but I forgot the earrings in my pocket.
In the middle of the luncheon, I remembered them.
“Oh, God," I said, "my earrings!
Chaim’s mother gave them to me as an engage,.
ment present!"
I ran to the hospital and looked in the pocket of my smock.
But the 
earrings were missing.
The director seemed very happy that I had lost my earrings.
-234-
I didn't go to work the next day.
The director of volunteers 
called me.
She gave me an ultimatum: “If you don't come to work this 
very minute, you will not be able to work at the Jewish General Hospital any more.
I will make sure of that."
I'm not going to complain to the Board about what you are
Doing.
You make the lives of the a volunteers miserable.
But I hope you make someone
who speaks fancy English miserable, and I hope 
this person will complain to the Board.
And I refused to go to work.
I don’t know what she did with one of the members of the Junior League
Fifteen days later.
The girl cried, and the Junior League complained to 'the Board.
The director of volunteers was fired immediately.
I was very glad that I hadn't complained.
I didn't want her to lose her 
job on account of me.
I was just sorry for her stupidity.
Years later, I went to work at the hospital at the time of the 
strike.
I worked in the labour rooms; I have never enjoyed any work more 
in life than my work in the labour room.
Every half an hour, there 
was another baby.
My work consisted of cleaning the labour room and preparing the instruments for sterilisation.
The head nurse came to congratulate me an my work.
And I enjoyed every moment of it, and remembered the circumstances of the lady who had given birth in the beat from Evia.
Here, women gave birth with dignity.
I was very happy.
When the strike 
was over, I received a very lovely thank-you card, signed.
by Mr.
Steinberg, the president of the hospital.
One day, I received a telephone call from the Royal Victoria Nos.
pital.
They wanted me to go to see a patient.
I went, and discovered
that the patient was Nicola!
Maurice Saltiel, a friend of his, had found 
him unconscious in his room.
Nicolas had had a stroke.
After Nicola was discharged from the hospital, Maurice and Jeanette.
his wife, took him to stay in their home for ten days.
Afterwards, we sent 
Nicola to Bermuda for fifteen days.
I tried to find a solution for Nicola, 
but he couldn't work, he had no money, and no-one to look after him.
I tried to rent him a room near my house, but after a few months the 
Landlord asked me to take him away.
I had to find him another place to live.
I would pick him up every day and bring him to my house, or take 
him for a drive.
I bought him a television set to help pass a few hours a day.
He couldn't read, and he couldn't write.
Many times I felt desperate.
Nicolas wanted to live in myhouse, but this was out of the question.
I kept trying to find a solution, but I couldn't.
On top of all this 
Nicola tore a tendon in his ankle.
He had to have an operation, and afterwards he had to wear a cast.
I placed him 
Whardson Convalescent Home for 3 months.
I had to take him to the 
Jewish General Hospital once a week.
At the end of these 3 months.
I took a room for him in Cote des 
Neiges.
I went every day to change his bed and to make him comfortable, to bring him food, to take him for a drive, or to bring him to my house.
Suddenly, Nicola acquired a new habit: he would come to my
house by himself, at three or four in the morning.
My life became in-
tolerable, and I didn't know what to do.
By now, Ely was old enough to go to camp, and I sent him to Camp 
Massed.
On visiting day, I was thinking of Mr.
Pougath.
who had not 
lived long enough to visit Ely at the camp.
Mrs.
Pougath called me 
and said “Today they will honour the people who bought shares in the 
camp.
They have erected a plaque with all of their names.
But no-one 
invited me to the Ceremony.
Please, take my camera, and take a picture 
of my husband's name.” I begged her to come with me to visit Ely.
She 
said “Just take the photo with Mr, Pougath's camera."
When we arrived at the camp, they gave us a programme.
There were 
two columns of names in the programme.
One column was for those still living, and the other, those who had died.
But Mr.
Pougath’s name was not in the programme.
The Chairman of Karen Hatarbouth was Mr.
Gordon.
He was the 
sweetest person I had ever met.
By chance, he was sitting next to me.
I 
sati "Mr.
Gordon.
Why does Mr.
Pougath's name not appear in the programme?” Immediately, he went over to Rabbi Horowits.
When he came 
Back, he said that Rabbi Horowits had assured him that Mr.
Pougath 
never donated $500.
I said “He had better look in the books.
He will
find that Mr.
Pougath did donate the money."
Mr.
Gordon called Rabbi Horowits over, and in front of me, he said 
"Mrs.
Garfinkle says that Mr.
Pougath did give $500."
"No" said Rabbi Horowits.
I said to him "You are making a mistake.
Mr.
Pougath gave 
$500."
"Rabbi Horowits never makes a mistake," he answered.
I came home.
Mrs.
Pougath came to pick up her camera.
I told her 
that I hadn't taken any pictures.
She wanted to know why.
I gave her 
the programme, and I said *Look.
Your husband is not listed among 
those who gave $500."
"And you call yourself a friend of Mr.
Pougath!” You didn't say 
anything?"
I explained to her that Rabbi Horowits claimed that her husband 
had never given $500.
She couldn't believe it.
She called Rabbi Horowits, and he repeated exactly what he had said to me.
"I am a bookkeeper," Mrs.
Pougath said to him, "and I will obtain a court order to look at your books."
After she hung up the phone, I tried to calm her.
I said "Look.
You have the certificate.
Go to Mr.
Gordon and show him."
The next day, Mrs.
Pougath went to Mr.
Gordon's office.
When he 
saw the certificate, he was very embarassed, because he was a very fair, decent man.
Two weeks later, the artist was called and taken to camp Massad, and the name of Mr.
Pougath was was added to the plaque.
The artist, Mr.
Abraham Goldberg, was an expert in Gothic lettering.
After the plaque was finished, there was a special party for all the 
people who had opened the camp, in honour of Mr.
Pougath.
Mrs.
Pougath was invited, as were Max and I. 
One day, a friend of mine came with me to take food to Nicola.
That night, he hadn't come to my house.
I told my friend I was scared to go in.
She asked me why.
“Because he has a habit of visiting me at 3 or 4 in the morning, and last night he didn’t come.
I’m very tired, because my nights are not peaceful, and my days aren’t either.”

We went in.
Nicola was watching television.
I changed his clothes
and left him the food, and my friend and I came back to my house.
My 
friend said “You've looked after Nicola for many years.
It is time 
that you found another solution.
You have to report him to Baron de 
Hirsch.
They will find a place for him.” 
When I heard her say Baron de Hirsch, became very angry, and I 
said “Oh no!
I have had much experience with Baron de Hirsch.
It is 
better to deal with City Hall than with Baron de Hirsch."
“What kind of experience have you had with Baron de Hirsch?
* she asked.
I proceeded to tell her the story.
I was pregnant, very heavy, and very swollen.
One evening, while my husband was out of town and Ely was already asleep, Voula, my housekeeper and I were playing gin rummy.
She loved to play.
It was the only way I could keep her from going to sleep, and I wanted company.
The phone rang.
Voula answered it, and she said to me “It’s in French”.
I took the phone, and the women on the other end introduced herself.
She was a nun from one of the French hospitals in Montreal.
She said that there was a sick man in the hospital who wanted to see me.
He had a Sepharady name, but with the nun’s accent, I couldn’t figure out who he was.
“It’s one of the people who was at your house for the seder,” she said.
I assured her that I would go to see the man the next day.
“Tomorrow, it will be too late,” she said.
One hour later, I was in the hospital.
When I came in, it was very dark.
There was one light bulb over a desk in a corner.
Two nurses were at the desk.
I asked for the mother superior.
She came right away.
“Oh, madame!” she said.
“I didn’t know that you were in this condition!” I was wearing my husband’s slippers, because they were the only shoes that fit me.
The nun said “If I had known, I wouldn’t have obliged you to come.” She took me by the hand and led me to the man’s bed.
She said “Don’t worry.
I won’t leave your side for a minute.”
The man talked to me.
“I’m separated from my wife, who is not Jewish.
I want to be buried in a Jewish cemetery, and in thirty days I want to have the stone, in the traditional Sepharady way.” I shivered as he talked.
I was very scared.
“I don’t have any money,” he said, “but promises me that you will do it anyway.” The mother superior that I was shaking as I promised the man to do what he wanted.
SHe took me downstairs, gave me a glass of orange juice, and called a taxi.
I get home, and Voula came down to help me up the stairs.
The next day, the nun called me to tell me that the man had died.
I got in contact with Baron de Hirsch, and with this organisation’s help, I organized the funeral.
It took place in a Jewish cemetery.
People came, and each one gave me a few dollars for the stone.
The next day, I went to a monument maker.
I wanted to have the stone done before I gave birth.
The man said “You are $50 short.”
“You make the stone in less than 30 days, and I will make up the difference,” I said.
He asked me the name of the cemetery, and I told him it was Baron de Hirsch.
“Oh,” he said, “you have to get permission from Baron de Hirsch before I can make the stone.”
The offices of Baron de Hirsch were on Sherbrooke Street, near the monument maker.
I could hardly walk.
I was too fat and too swollen.
It was only a block away, so I couldn’t even take a taxi.
I hobbled over.
At the office, an employee of Baron de Hirsch asked me how I knew the man who had died.
“Oh,” I said, “I usually make a big seder for those who have no place else to go, and one person brings another.
He was at the seder this year.”
He said “this man died only yesterday, and you have already gone to order the stone.”
“I’m ready to give birth at any moment,” I said, “and I want to make sure that the stone will be ready in time.”
“Wait a year, like everyone else,” he said.
I told him that the Sepharady place the gravestone 30 days after the funeral, and that I had promised this man that I would do it.
He started to laugh.
He said “Do you expect us to believe that you only know this man from your Salvation Army supper?
Why are you doing this in your condition?”
I was angry.
I said “My seder is not a Salvation Army supper.”
“You had better leave,” he said, “because I won’t give you our permission for the stone.”
I left and got a taxi in the street.
I was supposed to go home, but I told the driver to take me to JIAS.
“What am I going to do at JIAS?” I thought.
But I went there anyway.
There was a Sepharady working at JIAS and I told him what had happened at Baron de Hirsch.
He said “You know, Mrs.
Garfinkle, my director is a very nice man.
I’ll go talk to him.
Maybe he can help you.” He went to talk to Mr.
Kage, the director.
Ten minutes later, eh came back and said “We have permission from Baron de Hirsch.” God bless you, Mr.
Kage.
The stone was finished before the thirty days had passed, and was laid according to Sepharady tradition.
And this was my experience with Baron de Hirsch.
A few days after I had told this story to my friend, I took food to Nicola again.
This time I went with another friend.
I was desperate.
I didn’t know what to do about Nicola’s nightly visits.
And if he didn’t come at night, I was afraid that I would open his door and find him dead.
This friend had the same advice for me.
“This is no life for you any more.
You’ve looked after him for many years.
If you are not going to call Baron de Hirsch, then I will!” And she did.
I went home, and half an hour later, the assistant director from Baron de Hirsch called me, and asked me to talk to a social worker.
The social worker wanted me to bring Nicola to Baron de Hirsch for an appointment.
I was very tired, and I said “I’ll give you his address.
You can go there yourself, or you can send someone.”
They went to see Nicola.
Weeks went by without any news of a nursing home for him.
My social worker was a very sweet lady from France, Mrs.
Bier.
I phoned her one day, and she said “Mrs.
Garfinkle, I can’t do anything about Nicola.
I received orders from the assistant director and I have nothing for you.”
I called the assistant director.
He said “Mrs.
Bier will give you a list of nursing homes for Nicola tomorrow.” The next day, Mrs.
Bier called me and gave me one name.
I went to see the place.
It was not even fit for animals.
When I came home, I called Mrs.
Bier and told her what I had seen.
She gave me another name.
And then another.
They were all the same.
As a matter of fact, a few weeks later all the nursing homes that I had seen were closed by the government.
I spoke again with the assistant director.
He said only “I want to know why you are so interested in Nicola.”
“He’s my friend,” I said.
“You know, Mrs.
Garfinkle, we are doing an investigation of you.
When it is complete, we will give Nicola a place.”
“Make as many investigations as you like.
As long as you take care of Nicola.
You know how to discourage people, but this time you picked on the wrong person.
I don’t discourage easily.”
One day, when I went to see Nicola, he wasn’t in his room.
The day before, there had been a big snowstorm, and he hadn’t come to see me during the night.
I was scared.
Maybe he had disappeared in the storm.
I started walking, but I didn’t know where to look for him.
Suddenly I saw him lying on the ground near a snowbank.
A snowblower was coming towards him.
I ran up and started to pull him away.
Two nuns were passing by, and they screamed at the snowblower to stop.
It stopped.
The nuns helped me take Nicola home.
I changed his soaked clothing.
I was very tired, and i cursed my mazel to have sen what I did.
-242-
I came home and wrote a letter to George.
I told him what had happened that day.
To weeks later, I get a reply: the only thing he he could do for Nicola was to send the $500 that Nicola had given to him for safekeeping.
I wrote back that $500 would not be a solution.
A nursing home cost at least $250 a month.
George answered that he could do nothing else.
I got in touch with all the members of the Board of Baron de Hirsch, either by phone or in person.
But I could get no help from any of them.
They all told me that Nicola was too young for Maimonedes.
I spoke to the director of Baron de Hirsch, and he told me that he had finally found a solution to the problem of Nicola.
“You, Mrs.
Garfinkle, will take Nicola into your home, and Baron de Hirsch will pay you.”
I answered “You take him into your home, and I will pay you.”
“But Nicola is not my friend,” he said.
“If it weren’t for people like Nicola who needed Baron de Hirsch, you would be out of work.
You are here to accomodate people like him.” The conversation ended on a very sour note.
I called Mrs.
Fisher.
I told her the whole story, and she was very sympathetic.
She said “Go to see Mr.
Winer.
He is a very nice person, and he will pay attention to you.
I’m sure that he will do something for Nicola.
I know, as everyone else does, of the bureaucracy that exists at Baron de Hirsch.”
Mr.
Winder was born in New Jersey, in the United States.
He attended the City college of New York, nd the school of social work at Columbia University.
His first job was at Madison House.
After he was released from the army, he went to Columbia University for his Masters degree.
He was in Philadelphia for ten years.
He then came to Montreal, and he worked at a Neigbourhood House and at Wood Acres Camp for seven years.
He now works for the AJCS as campaign director and director of community planning.
He is associate executive director of AJCS.
Mr.
Winer is very satisfied with his job.
In my opinion, he is a born social worker.
Mr.
Winer is also an excellent speaker.
I was very impressed with him when I was taking the Board Orientation Course of the Education Department of the Allied Jewish Community Services of Montreal.
I got an appointment with Mr.
Winer to talk to him about Nicola.
The office was on Sherbrooke Street.
In the elevator, I thought “Another sophisticated bureaucrat that I have to go and see.” To my great surprise, he was neither terribly sophisticated, nor a bureaucrat.
He greeted me very nicely, and this gave me some confidence.
I told him what was going on, and he listened attentively.
As I talked to him, I thought “I’m speaking to Eliaou.” The only difference between the two men was that Eliaou had been a redhead, and Mr.
Winer had dark hair.
After he had listened to me carefully, he said “You will go and find Nicola a nursing home by yourself.
Don’t worry about the bill.
If Baron de Hirsch doesn’t want to pay, come see me again.”
I went home, and as soon as I looked in the newspaper, I found a nursing home.
I called the woman in charge, and I made an appointment to see her the next day.
When I arrived, I saw that it was a very clean place.
Some old people were watching TV, others were sitting in the sun and looking out the window.
I asked if I could stay for lunch.
I wanted to see what kind of food thse people were given.
“Come into the kitchen,” she said.
“I’m in the midle of making up the lunch trays.” The food looked good, and there were generous portions.
I helped her carry the trays.
When we finished serving, I said to her, “I like your place.
But I have to call Baron de Hirsch to make arrangements.”
She said “I know Baron de Hirsch, and they know me.
Look, Mr.
Shapire over there is from Baron de Hirsch.
He came here three weeks ago, and Baron de Hirsch knew then I had vacancies.
I will have more vacancies by May 1, because I’m moving to a bigger place.”
I went out and got into my car.
I thought “Why did Baron de Hirsch send me to all those cheap and dirty houses that were just closed by the government?” It didn’t take me long to find out.
My friend who had first reported Nicola to Baron de Hirsch was Mr.
Jacob Loewy, a wonderful philanthropic person and a leader in the community.
At the time, Mr.
Loewy’s committee had suggested an investigation into the management of Baron de Hirsch.
The anger of the Board at this suggestion was taken out on Nicola and myself.
I came home from the nursing home and called Baron de Hirsch immediately.
The social worker said “Tomorrow morning we will go to see the palce together.” I didn’t say anything about the way Baron de Hirsch had treated me before.
We went to the home, and the social worker said “I’ll wait outside.
You go and look.” Again, I didn’t say anything.
I went in and I told the lady that I had come to see the new place.
Five minutes later, we were shown a beautiful duplex.
I said “This is the social worker from Baron de Hirsch.
She is going to help me with Nicola.” The lady in charged acted like she had never seen the social worker before.
She showed us a large room, with two nice comfortable beds, two dressers, and two very comfortable chairs.
And there was a bathroom in the room.
This was very important to Nicola, since he had to go to the bathroom quite often.
On the first of May, the social worker and I took NIcola to the nursing home.
Mr.
Shapiro was already there.
I brought a TV so that they could watch it in their room.
By the time I was ready to leave, Nicola and I were both crying.
I drove away, but I stopped the car soon after.
The social worker wanted to know why I was stopping, and I said “I want to give you an idea of who Nicola is.
HE was born into one of the most distinguished Jewish families in Greece.He was brought up in Switzerland at one of the best boarding schools in the world.”
“Oh,” she said, “you can see that even now, when he is sick, just from the way that he speaks.”
I said “And despite this, you tried to send him to dirty houses not even fit for animals.
Especially the last house you sent me to.
You called me at six o’clock in the evening to go see the house, and if I liked it, we could have it by the next morning.
I took a taxi to the place.
I didn’t like to drive the car in the evening in the snow.
As soon as I arrived, I saw a woman on the floor, dead drunk, an empty bottle of whiskey in her hand.
I was terrified.
I thought that she was dead.
I heard people crying in a room off the hallway.
I went into the room and I saw three old men crying.
I asked them why they were crying.
They answered me, half in Yiddish, half in English.
They hadn’t eaten all day, “cum aer, cum aer,” they called to me.
I went outside to look for a restaurant.
I found a hot dog stand and ordered three hamburgers with french fries.
While the man was preparing the order, I thought of the meetings of the Women’s Federation.
Everyone comes to these meetings very well dressed.
When the meeting is over, the Board Room doors are opened and there is a big table with a snow-white tablecloth over it, laid with coffee, sandwiches, and delicious pastry.
The ladies com up to the table, select a sandwich, take one bit, and put the rest back.
They don’t want to have their mouths full when they compliment each other on how well they look.
And to think of what is going on with our senior citizens, in our own backyards.” 
The social worker answered “But you found a clean house, and we put Nicola there.”
“Are you sure that I found this house?” I asked.
“Don’t you remember?
You called me to tell me about it,” she said.
“Sure, I remember,” I said.
“But I know that this house was well known to Baron de Hirsch before I found it.
Mr.
Shapiro is from Baron de Hirsch.
I would like to know why you didn’t tell me about this place.”
She said, “I want you to know that I’m only a messenger there.
And thank God, in a few months I’m leaving this job and going back to France.” She said again “I want you to know it is not my fault.
I’m only a messenger at Baron de Hirsch.”
Nicola was well looked after in the nursing home, and he was very happy there.
Every day, they changed his shirt, and he always had a clean suit.
I used to pay for cleaning and washing the shirts.
Dave, the owner of the nursing home, brought Nicola to my house once a week.
On another day, I sent a taxi to get him.
And I took care of all his needs - toothbrushes, haircuts, and so on.
Nicola become like a young boy after his first day of school, who can’t wait to tell his mother about everything that happened.
He wanted to tell me about everything that went on in the home.
One day, Nicola came to my place and announced that he wanted to go to Expo 67, to see the US pavillion.
“Who told you about the US pavillion?” I asked.
“Dave told me all about it, and I want to go see it for myself.”
I called Dave, and I said “What kind of problems are you creating for me?
Take him, and I’ll pay the expenses.”
Dave said that he didn’t have the time, but if I wanted, his friend could take Nicola.
The US pavillion wound up costing $25.
When Nicola came back, he was very excited, and told me in detail about what he had seen.
I thought it was a miracle.
He was getting better every day.
Mrs.
Bier left Baron de Hirsch and went back to France.
I was assigned another social worker.
The new social worker told me that she wanted to take Nicola to his regular doctor’s appointment by herself.
I said “That’s all right.
Dave takes him.
If you want, you can go and pretend that you came at the same time by accident, and you can speak to the doctor.”
She didn’t listen to me.
She told Dave that he didn’t have to take Nicola to the doctor anymore, taht she would take him all the time.
She took Nicola to the doctor by bus, and Nicolas had to go to the bathroom every two minutes.
In the evening, she called me.
She wasted he whole day, and she had trouble getting him on the bus.
The next time I went to the thome, I told Dave that in the future, he should take Nicolas to the doctor.
He refused.
“Yous said that you would take him every time.
So you take him,” he said.
The social worker called me.
I said “What do you want from me?
You wanted to take him to the doctor, so why complain to me?”
“Because I want you to take him from now on,” she answered.
This I refused to do.
“You brought this situation onto yourself,” I said.
“Now you take care of it.” Nicolas missed three doctor’s appointments.
I felt very badly, but Baron de Hirsch didn’t care.
Then I had an idea.
I brought Dave a beautiful present.
After that, everything settled down.
Dave resumed taking Nicolas to his doctor’s appointments.
One night, Nicola had a very bad toothache.
The nursing home 
called me the next morning.
They had phoned Baron de Hirsch, and had 
been told that Baron de Hirsch did not provide for dental care.
I
 called Baron de Hirsch myself, but the reply was the same.
No dentist.
I called the nursing home and I told them to take Nicola to Dr.
Laurence Cohen on Decelles St.
I would pay the bill.
I called the 
dentist, and asked him to look after Nicola and to send the bill to me.
A month later, Nicola received $474.24 from George.
I asked Nicola what he wanted to do with the money.
He said "You've looked after 
me better than a sister.
But I don't want to die poor.
Keep this money for me.
When I die, you will make my stone with this money.
I 
don't want either you or the community to have to pay for my stone.
Buy it with this money."
Two or three days later, I saw the social worker, and I told her 
about the money and Nicola's request.
She wanted me to give the money 
to her.
I refused.
Ten minutes later, the assistant director of Ba-
ron de Hirsch called me and asked me for the money.
I told him about 
Nicola's request.
"I also paid the dentist $175 from this money," I said.
He was 
very angry when he heard about the dentist's bill.
"Why didn't you take him to a clinic?"
he asked.
"Why didn't you tell me or the nursing home when we called that 
there was a clinic?"
I asked.
"You must be very stupid not to know that Herzl clinic provides 
dental services," he said.
"You can say that if you like," I answered, "but your job is to 
inform stupid ladies like me about the services available."
The assistant director went too far with the language that he used.
He told me 
he was going to put me in jail.
I was angry.
I said to myself "Let 
them take the money.
It's not worth having so much aggravation over a few dollars.
I looked after Nicola for so many years, I can do it now 
too.
When Nicolas will die, my husband will pay for the stone.
And 
anyways, who knows who will die first?"
As I was thinking about this, my doorbell rang.
It was Dave, 
holding a big parcel.
"What is this, Dave?"
I asked.
"Nicolas has diarrhea, and no cleaner will take his pants.
I 
brought them here for you to wash.
If you won't do it, I will bring 
Nicola here tonight," he said.
"You didn't take him from here," I answered, "you took him from 
Baron de Hirsch.
If you have complaints about his pants, go to them."
He wound up leaving me the parcel.
When he left, we were on 
quite good terms.
I was going to throw out the dirty pants and buy 
Nicola some new wash and wear pants.
They could then wash them in the 
machine instead of bringing them to me.
The next day, I would go to 
Eaton's and buy Nicola two pairs of pants.
As I was throwing the old pants out, the phone rang.
It was 
Phyllis Waxman.
"You know," she said, "I am the Chairman of the campaign for 1969.
And now I can talk."
"What do you want to say?"
I asked.
"I want to know all about Nicola's past."
"I'm not going to sit down and discuss Nicola's private life 
with you."
"I want to know everything you said to Baron de Hirsch."
"First of all, if you're such a big shot, why don't you go to 
Baron de Hirsch and look up his file?"
"You're going to answer all my questions.
I work for AJCS now.
And I'm a member of the Board of the Women's Federation."
"Phyllis," I said, "if the Women's Federation is not satisfied 
with what I told Baron de Hirsch, they must ask me to a meeting and 
ask me more questions there.
And I will answer to the best of my ability.”
-250- 
"I am the Chairman, and you must tell me everything," she in-
sisted.
"Oh, excuse me, Madame Chairman.
Today, Nicolas has diarrhea.
The 
cleaners won't take his soiled pants.
The nursing home sent them to 
my house to be washed.
Come to my house and help me wash the pants.
And please bring the ladies with whom is you gossip about Nicola and me."
"You talk to me like that now, but one day you will see my picture 
on the wall of the Board Room of the Woman's Federation."
"I don't doubt it for a moment.
Maybe your statue will be in the 
Board Room, too.
I know the way you push people around to make a place 
for yourself."
After I hung up the phone, I said to myself, "Oh, no.
I'm not going to give that money to Baron de Hirsch."
Instead, I put 
it in trust for Nicola.
Sitting in the airplane, I covered my eyes with my hands, and I 
said "Oh, God, please put an end to this nightmare of a trip!"
The 
captain told us to fasten our seat belts.
We were landing at Orly 
airport, in Paris.
He told us that all our hand baggage was to be 
placed outside the airplane after we landed, for security reasons.
A 
bus came to take us to the terminal.
As Max was going off to the bathroom, he said "Buy some perfume while we are here.
Arpege de Lanvin."
I was in no mood to buy perfume; I just wanted to be home.
I sat down on a bench.
I saw a man standing near me.
He wasn't 
old, but he had snow white hair.
At first I thought it was a wig.
The man came up to me and looked at my nametag.
I looked at his hair.
Suddenly, he said "Oh!
Bouena Sarfatty!"
"And who are you?"
I asked.
"Oh," he said, "I'm so glad to see you!
I'm the man with the 
candlesticks, from the Matanoth Laevionim.
You must remember the day before the wedding of that one group of deportees.” And he called out
"Cherie!
Cherie!"
He was calling his wife.
He introduced me to her.
She said "We use the candlesticks now.
And my husband really 
acts like Jean Valjean, as you suggested a long time ago.
He performs 
many good deeds.
Instead of two candlesticks, we have three.
We 
bought another one.
And we light a candle for you every Friday night."
We all started to cry in each other's arms.
For me, it was like 
medicine.
I had been wanting badly to cry in public for two 
weeks.
The man told me that he now lived in France and had three children.
He told me the story of how he came to France.
"After you left Salonica, The Jewish community received a letter 
from my mother's uncle in France.
He was looking for his surviving 
relatives.
The Jewish community forwarded the letter to me, and I 
answered it.
A month later, I had a visa and a ticket to go to France.
The day that I was supposed to leave, Ketty's husband, the officer, 
died.
I felt I couldn't leave Ketty alone with the three children.
But Ketty persuaded me to go; maybe I would be able to help her better 
financially from France.
I didn't forget Ketty.
I went to France, and went to work for 
my great-uncle.
Three months later, I was a junior partner in his 
firm.
I married my uncle's second wife's daughter.
My uncle died 
soon after, and left everything to my wife and me.
First, I brought 
Ketty's oldest son to France to study commerce.
He is now my business 
manager.
A year later, I brought the second son, who is now interning 
in a hospital here.
After that, I brought the youngest one, who is 
still in school.
And finally, I brought Ketty to France.
Ketty is 
everyone's mother.
She's the boss.
And we all speak Ladino in the 
house, even the children."
"And where is Ketty now," I asked.
"She's with our three children.
We have a bungalow near the sea.
My wife and I are going on a trip."
-252-
His wife got up and walked away.
She returned carrying a parcel.
It was a big bottle of Arpège de Lanvin, a present for me.
The bus 
came, and I went to look for Max.
As we were being taken to the air-
plane, I started to sing "Ou mi anaknou".
And every person on that 
bus answered: "Israel!"
And I was waving my bottle of Arpège de Lan-
vin.
We got to the plane, and I said to myself, "Before I start to 
remember more miseries of my life, I will write a letter to 
Mrs.
Gertzman."
Mrs.
Gertzman was the president of the Women's Feder-
ation.
I wrote to her: 
I was informed by Phyllis Waxman that the Women's 
Federation wants to know everything about my friend Nicola.
I answered all the questions that the social 
worker from Baron de Hirsch asked me.
But if the Women's Federation wants to ask me more questions, they 
should invite me to one of their meetings.
I will answer all their questions to the best of my ability.
I 
give you the name of Nicola's friend George in Greece.
Nicola has another friend in Montreal.
They went to 
school together in Switzerland.
Maybe he can give you 
more information.
About my stay in prison: it is only 
a topic of conversation among the ladies of the Wo-
men's Federation.
My only crime was that I was a Jew.
I enclosed the addresses and reread the letter.
We arrived in New York.
I was still fixing up my letter, because 
my English is not so good.
In New York, the first thing I did was to 
give my heartfelt thanks to Mrs.
Esther Glazer of Boston for the courage 
that she gave me during the trip.
Mrs.
Glazer is Mr.
Shofild's sister-
in-law.
Without her, I don't know if I could have gone on.
We came to Montreal on an Air Canada flight.
I felt like I was 
pushing the airplane to make it get home quicker.
On the airplane, 
a fellow traveller near me said "Mrs.
Garfinkle, do you agree with 
the way I live?"
I was very surprised.
"You are grown up enough to distinguish between good and bad," 
I said.
"It is not in my character to judge anybody."
He looked at me and he said "But you yourself introduced me to 
my girlfriend."
I said "I introduced a housekeeper to your family.
Your father 
requested me to do it.
But I never knew that she became your girlfriend.
There is a Ladino proverb, and you deserve to be told it: 
'non agas al malo ni mersed ni grado'.
It means 'never do favours 
for bad people.'
And there is another proverb: 'kien non tiene 
akien yorar yora al rey que non tiene padre'.
It means: 'if you 
have no-one to cry for, cry for the king, who is an orphan.'
I don't 
want to cry for you, because you haven't got any brains."
I turned 
away so that I wouldn't have to talk to him any more.
I started to 
think of the time that I introduced the housekeeper to this man's 
father.
I didn't even know his wife, who had just arrived from New 
York.
I met her three weeks later, at a children's purim party.
And 
years later, I went with Ely to the son's business, to buy something, 
and I met him for the first time.
Thank God, we finally arrived in Montreal.
I was disgusted and 
exhausted.
I was relieved that the trip was over.
I felt like the 
time I arrived in Turkey with Miki.
Oh, thank God, we were home!
I opened the door to my house.
There was a pile of mail on 
the floor.
I took all the letters and put them on the table unopened.
Max and I had coffee, and went to visit Nicola.
It was suppertime.
All the pensioners had just a very little bit of salad.
Nicola had 
just two tomatoes.
Uncooked vegetables or fruit were like poison