% pages 71 to 82
We went to see a jewellery factory staffed by war amputees. The Israeli govern-ment 
opened this factory as part of tkmi its programme for amputees. The people 
were working in their wheelchairs, making gorgeOns jewellery. I wanted very badly to 
buy something, but I needed the money to pay for the gravestone. 
According to Sepharady tradition, the stone was to be placed thirty days after the 
burial, and it had been placed, but now the widow was in debt. The boy had been sick 
for some years, and every once in a uhi3e I would send him a few dollars. In 1968, 
a group similar to ours had visited Israel. Phyllis Waxman was one of the people who 
had gone. I begged her to take 35 dollars to my sister. I didn't tell her that the 
Painu-p 
money was for this boy. Then she came back, she gave me a big argument. "Yam. sister 
is very ridh," she said, "and when she came to get the money, she was dressed very 
elegantly. Instead of giving it to a wealthy woman, you•eould give the money to the 
community. You are not a very big donor. 'You have to learn that the more money you 
give, the more you will be honoured by the community. And don't forget, you speak Rng-
lish very badly. With money, this could be forgotten." 
I answered "If this is my biggest defect, I am very happy. But like everyone else, 
I have defects too. As for my sister, who came to see you so well dressed, she took mmdtd 
nothing from the commnnity. And I don't want to buy honour with money; I want to deserve 
it by my actions." 
I was thinking of all this in the factory, when Hr. Gelber approached me. He asked 
me what I was going to do with the liras that I had left. I didn't want to tell him 
that I was going to pay for the gravestone. This was personal. I answered "I'm going to 
give it to my sister." When Phyllis Waxman heard, the gossip started. Everyone knew 
that my sister was very rich, and that I was going to give my left over liras to a 
wealthy woman. 
Instead of Gaza, they took us to a'kibbutz. They showed us a discotheque for teen-
agers. It had been a bomb shelter before, but now they had made better shelters, and 
they had converged this one to a discotheque. It was very nice. But when they showed 
us mt: where the babies slept, my heart was sore. In an underground shelter were cribs. 
They didn't take chances by putting the babies in a nursery. 
103012ganc 
give you 5000 dollars. But I won't give one penny to the Women's Federation." Be 
asked me why. "If the Women's Federation can have Phyllis Waxman for chairman, they_ 
don't need my money. I can give to the men's division, or I can send the money dire( 
ly to Israel." 
In the bus, the El-Al agent gave us a bill for the food for the fifteen days, fir 
also a statement of how much tip we had to give to the guides. My husband paid right 
away, but I was burning up inside. After the treatment we got, we were expected to 
give a tip! But I didn't say anything. I think it is time that the Eged put an end 
to these tips. They are very well paid, the chaverim of Eged. And look at their be-
haviour! I find this disgusting. As well as being very well paid, they have a store 
where they get everything at hPlf price. After 20 years of service, they get a big, 
big, big sum of money that I don't even want to mention. Find out for yourself. Plu, 
when they retire, they continue to get their monthly salary for life. If they put ma 
ney in the bank, they get 20% interest. This, too, I want you to Amp= check for yoi 
self. The big sharehAlder in the Eged is the Istaderouth, and if the Istaderouth is 
so generous, *low it should no have a fundraising campaign. They have to understand 
that the Sews of the diaspora want to help Israel, but not to make individuals into _ 
millionaires. 
So we arrived in Tel Aviv. We went to a very nice hotel, but I was reanhing the` 
and of my strength. I was suffering because I was reliving the miseries of the past. 
But I never stopped singing and dancing; I didn't want to show others how much I was 
suffering. 
We went to see Jaffa. The Israeli government had conserved ail the Arab monument! 
very well. In the middle of the town square was an old well. We thought that it was - 
ti 
very considerate of the Israeli government to preserve the history of the Arabs and tb 
Jews of Jaffa. 
-11 
I started to remember Evia. I used to go to a similar place to wash cloth 
for George, Nicola and Miki. When I finished washing, I couldn't wring the clothes 
out. I hung the clothes up and the water ran off them. But the sun of Greece is very-
strong. 
laundry was clean enough for me then, and it had a very nice smell. 
r41- 
When I came back to the hotel, I found out that there a. was to be a reunion for 
all the groups that night, but it had been kept a secret fronime. The next day, the la. 
day of our trip, I had an appointment at four o'clock in 3.11 the morning. Some of the 
children were going to cone to the hotel. At nine o'clock in the morning, I was to 
meet my niece, whA wanted very badly to buy me a present. 
At 4 A.M. I went to the hotel lobby. Thre was a soldier standing there. I asked 
the clerk at the desk if someone had come for me. He said "This soldier." When 
turned my head, the soldier started to kiss me. When he was smn11, he couldn't pro-
nounce the letter 'r'. He used to call me Malitsa. 
I realized it was miki right away. 
You can imagine what kind of reunion we had. The taxi that Mill had come With was at 
141ofift-the 
door.AI couldAtt talk, I was crying too hard. I started to remember the day that 
I left him in Chesme to go with the lady to Palestine, to the Beth Haolim. 
g. 
Two or three days later, I asked for a telephone book. I was looking for the name 
'Nakmonly4. I found 'it. It was my brother-in-law, who had a store in Makalat Ben-
jamin Street. I called him. He was very happy to hear my voice. The next day, I met 
my other brother-in-law, Alfandary, for the first time. He came to pick me up. He 
gave me news of everyone, and he said "Tonight we are gitk going to have a big party. 
Daisy Nakmouly, my niece, is engaged, and we will have the engaement party." I was 
very happy for her, but I said "I can stay here for A another day or two. 361 I'm 
just not in the mood to go to a party x tonight." I gave him my corset with the jewell-
ery inside to keep safe for me. 
In the Beth Haolim, there were other Jewish people from Greece. Everyone encour-
aged me to go with my brother-in-law, to start my new life with parties. But there was 
nothing there for me, and I didn't want to go. After my brother-in-law left, rumours 
started in the Beth Haolim. The camp of Alepo was to came to Gaza. And I started to 
make plans to take the children from Gaza. They wouldn't get certificates. I was a-
fraid to talk to anyone about this. The priest had registered the children as non-Jews, 
with names that he had invented. For sure they would have to wait out the end of the 
war in Gaza. Without parents, it was not the place for children to grow up. 
The next day was Sunday, ant I didn't have too much time to think of A 711an 11:GVE11. 
could I take the children, and where would I put them? I remembered Rachel Yanait bei 
Zvi. I would put them in Hayad Halimaud, her school. But how could I get the 0'111- 
dren from the train, before they went to Gaza. I took the bus and went to Tel Aviv, 
to my sister. A bath and clean clothesx u were waiting for me. But my thoughts wer( 
only of the ohildren. 
Many people came to see me. I felt like the Fat Lady in the circus. All of a 
sudden, a young soldier, a Greek Jew, came. He came to ask me if I knew his family. 
answered no. He invited me to go downstairs with him to have an ice cream in the cafe-I 
I accepted right away. The only thing I heard around me was what dresses the women 
would wear to the wetitiing, and who would be invited, when the dressmalrer would come 
to make the achougar (all the clothing that the mother would give to the bride). It LT 
was a machia for me to go with this boy, to get away from that atmosphere. 
We sat down in the cafe, and he told me he was stationed in Gaza. I jumped. I 
the 
told him about War children. He told me that he was on leave until BE Friday, and we 
made an appointment. On Tuesday, we would go together to Gaza. He asked me "What are 
you going to tell your sister? Your brother-in-law won't let you come with me!" Ify 
brother-in-law was a very nice person, very well educated. He was one of the most 
elegant men in Tel Aviv, and he was very humanitarian. But he had a head of 2000 year 
ago. Girls, after they reached the age of 12, couldn't talk to boys. He believed tbn 
a girl must be married before she became 20 years old, From the first night that I 1-1 
arrived in Tel Aviv, my suitors were ready. When they introduced me to one of them, 
I saw that he was interested in me, I would say "I want you to know everything about n 
I can't see at night, I'm a little WinA. But only a night." And he wouldn't come 
again. Everyone knew that I couldn't see that night. And like this, I was left in 
peace for a few days. Until another suitor came. 
As we were sitting in the cafes, a young lady came up to me. It was my cousin 
Suzane. She bad escaped from Gaza. She had been in Tel Aviv for only one day, and ski 
was staying at the Beth Haolim of Tel Aviv. Suzane and her husband came with me to 
my sister's. We didn't appraise the bracelet I had for her, because at the time, 600C 
drakmes werenEt even worth one penny. Suzane and her husband couldn't stop blessing r 
and Eliaou. 
- 4 
10 
The next day I told my sister that Rachel ben Zvi had invited me to go to atm Jerp-
salem to spend a few days with her. The ben Zvis don't need an introduction in the Jew-
ish world. They are very sweet people. Mr. ben Zvi was later the president of Israel. 
My sister believed me, and I went to meet the soldier. 
We went to Gaza in a track full of Greek soldiers. The first thing that the soldier 
did was to find accomodations for me. Be took me to an Arab home where he had lived a 
few months earlier. They received me there very nicely. The soldier didn't tell them 
that I was Jewish. I was Greek, and my name was Maritsa again. The next day, at eight in 
the morning, he came to pick me up, and he told me that many of his friends wanted to help 
us. A few trucks were going to Jerusalem that day. If we could intercept the trucks, we 
would have mazal. 
From 8:30 A.M. until late in the afternoon, we waited for the train to COMB. When it 
arrived, I looked in all the windows. Suddenly, as the train stopped, I heard a voice: 
Malitsa!" I never knew how the children got off the train, because as soon as he saw the 
group, the soldier told me to get into one of the closed trucks. The truck left at full 
speed, heading for Jerusalem. 
After a while, the truck stopped, and I was told to get off. The children were already 
waiting for me! In Each of them was holding,* chocolate, bread and corned beef in his hands. 
We sat down under a tree and ate. The sun was very strong. We rested and slept. When we 
woke up it was very dark. We spent the night under the tree. Very early in the morning, 
we were ready to walk to Jerusalem. 
We out a branch off the tree, and we made a stick. We used the tins of corned beef to 
make music. The orchestra was ready. I told one of them "You will be the captain." Be 
liked that very much. Be said "Attention!" He told us what song to sing. He raised the lox 
branch very high, and he said "March!" The song he told us to sing was a Greek lang mili-
tary march, 'Zoom! : 
Zoom krianda hena 
Zoom trianda dio 
Zoom trianda tria 
lie misso 
We marched for I don't know how many hours, until we couldn't stand the sun any more. 
We made a tent from the branch of the tree, and a blnnket the children bad. But it kept_ 
4.0114. 
t_ ft* 
. 
It was impossible to walk. 
The ehildren were between the ages of six and twelve. But there was one who was 
ninetten years old. The priest had declared him as sixteen, to avoid the military ser= 
vice, and he had dada declared him as his brother with the same last name. The child-
ren were sitting in a circle all around me, in the middle of nowhere. The nineteen year 
old bdy said to me "Maritsa, will you marry me? Me mill have a home, with all the chit- 
dren." I answered "A boy who marries someone older than himnPlf is looking for a mo— 
ther and not for ax wife. The pride of a man is to have a young wife, and one who is 
pretty." The (ihildren said in unison "you are pretty!" If a man marries an older wo— 
man, his wife's age will show in a few years. He will be ashamed to go out with her. 
love you all, and I want to keep it that way." Everyone promised to marry a younger 
girl, or one of the same age as himself. "You know what?" I asked. "My grandmother' 
knew a very nice song for men who married older women. Itll 
of the song is El Amor: 
El amor de las mossas 
Es como la manssana de Escopia 
Comb ouna i quierb otra 
Tagnb qub non sos tau 
El amor de la rezin cazados 
Es como la carnb assada 
Comb i comb i non sb afarta 
Tagnb a sb bivas tou 
1011b qub non sos tou 
El amor de las viejas 
Es como la samara vieja 
Tiene pelo i non caenta 
Tagnb a sb vivas tau 
dilb qub non sos 
This song means: 
The love of youngsters is like the apple of Scopia 
You eat one, you want another one 
The love of the newly married is like a steak 
You eat and eat, and are not satisfied 
The love of old people is like old fur 
It has hair but it doesn't keep you warm. 
sing it for you. The name 
After I had sung this song, the nhildres -wanted to sing the song of Palestine: 
Palestina tau lous ermoza i santa 
Couanto ton sos dezeada 
Alevanta i tou sola canta 
Pass i amor 
adientro el 
corasson 
t--
-73— 
Macabi tou ovra ese querida 
Fina el MP de la rekmission 
Mouestra tierra tierra conocida 
La tierra qua alegra el corasson. 
Suddenly, one of them said to me, "Why don't you marry the priest? You let him go 
to Gaza, you didn't bring him with us. in& He's older than you, nice looking, and a 
good man." I said "Look. I suffered much for being a Jew. And I'm going to die a Jew. 
The priest is Greek Orthodox, and I would neveriM marry him. And I advise you never to 
marry a woman who is not Jewish. I love you all, but it's not enough to marry you. In 
Greek there is a proverb: 'Papoutsi apo tom tog= topo sou asse inez balomeno.' It means 
'Shoes should come from your own country, even if they are torn.'" 
Very early in the morning, as soon as it was light, we prepared to walk again. I 
chose a captain to hold the broken branch. And we had our music. Everything was ready, 
the boy said "Attention!" when I noticed that Rua was still on the ground! "Miki, get 
up, we're going!" "Oh, no," he said. I said "I'm going to leave you here. We're go— 
ing!" "Carry me," he answered, 
"Are you a baby, that I should carry you?" 
"You treat me like one." 
"I made you captain for the meal last right," I said. 
44S44,VSAVEVCCIVCSIG4AVRIA4A1A4A-A44014.44V76PAW4•:•SAIG4Cre4444C1.11411A7.T44.1,A40471,4RA14A7flOWILJE 
"You just gave me to understand that you won't marry me because I'm too small. 
That's number one." 
a 
"Look, Miki, I will answer number one. You can tell me number two after. When 
you will be 20 years old, and if you still want to marry me, I will marry you. Now," 
'I said, "let's hear number two." 
"You've been in the Middle East so long," he said, "and you still haven't found. 
my father!" 
"Look, Miki," I answered, "a man without a name is impossible to find. Whom should 
I ask for?" 
"I told you, my father is decanea. He has an olive on his back. How many times 
Tritto+ T Ittnt45,A419 ---t11 vt11 .02-2 
"Maybe you can explain to me how," I said. 
"When we walk, many trucks filled with soldiers go by us. You always take us to 
the side of the road. Me, I'm going to stay in the middle. They're not Germans. They 
not going to hurt us. They're our soldiers. We'll stop the trucks, and we'll ask abot„ 
my father." 
I asked for everyone's attention and I told them that Miki would be the captain 
that morning. I gave him the broken branch. Roght away, he shouted 'March!" 
We were walking and dancing, and everyone was singing 'Zoom'. After half an hour, 
we heard some trucks coming, sounding their horns. Miki stayed in the middle of thw 
road, and we did too, because we had to obey our captain. The trucks stopped. An 
officer got out of the first truck. He was ready to kill us, He was saying things in 
Greek that you don't hear every day. I went over to him. He didn't even want to talk 
to me. I told him very loudly, over his yelling, "We are Jews from Salonica!" 
a Jew from Salonica, too," said the officer. He told us his name, but no-one knew him. 
I said "Look, we stopped you because this little boy is looking for his father." 
"What is his name?" 
"I don't know. He is decanea, and he has an olive on his back.". And I showed hin 
Mike's olive. The officer was kaki's father! I will never forget that scene. The 
officer cried, and Miki told him. "A soldier doesn't cry." 
In the trucks, we fonntl fathers, brothers, cousins, for the rest of the children, 
all but one child. The soldiers asked us where we were going. We answered "To Jeru-
salem." They answered "But this is not the road to Jerusalem!" 
Everybody was given an address to go to.ik The soldiers gave us money for the bus 
to Zerawskon. They showed us where to take the bus to go to the sister of Miki's fa-
ther. Her husband was a taxi driver. When we got there, he took each nhild to the 
address that he had been given. There was just one boy left, and I didn't know what to 
do with him. A neighbour came to visit. As soon as she saw the boy, she said "He look 
like my family! And he has my name!" The neighbour asked the boy where he had been 1 
born in Salonica, at what address. "I 1M3Mt was born near the synagogue of the Monaster 
-J75- 
lis." The lady asked "Are you a Monasterli?" "No, but I WI* my father is, because 
he used to go to this synagogue." 
The lady took the boy to her home. She showed him some pictures of her family. In 
the first picture, the boy recognized his Nona, his father, and his mother. It was too 
late at night now fmxx for me to go to HavadRAlimoud. I could now return to Tel Aviv, 
to my sister. 
Suddenly I heard Miki's voice. We had arrived at the cemetery. EVeryone was wait-
ing for my arrival. They all had their young wives with them. I was introduced first 
to the widow, and then to Mils wife and all the other women. The widow took a letter 
out of her purse, written by her husband. He said that he wanted the stone to be placed 
on his grave within 30 days of his death, according to Sepharady xmligim tradition. But 
he didn't want an unveiling, and he didn't want anyone to visit his grave, not even his 
wife. "When Maritsa comes, that should be the first time that anyone visits my grave. 
After that, anyone can come. And you, also, " he had written to his wife, "as long as 
Maritsa doesn't come, you cannot come to visit my grave, either." Be gave directions 
for the unveiling. "They should all sing songs. If it is possible, a reunion should 
be held for all the people who know Maritsa, including the baby." 
The widow told us that she had said to her husband "Maritsa is in Canada, and it 
will not be very easy for her to come here for your unveiling." He had said "Maritsa 
went from Syria to Greece at the time.of the German occupation. Coming to Israel is 
much, much easier." 
We started with avdela, a traditional prayer at the end of shabbat: 
Adio alto con sou gracia 
Mandamos moutha ganancia 
Non -mamas mal ni ancia 
A nos i a todos Israel 
Venimos mos adjountemos 
I sou nombre bendiziremos 
I a el demondavemos 
La rekmission de Israel 
Tou club sos padre Rakman 
Mandamos a mossad Leman 
Que mos sea bien siman 
Bouenas semanas vemos 
Venir 
Las salimos arecevir 
club mos dechb el dio bevir 
A nos i todos Israel 
Every shabbat, while Nikits father was in Palestine, he would entertain all the chil-
dren, Before he sent them home, he would make avdels. This prayer at the graveside 
was in memory of Nikits father, who had died in the revolution in Greece, after the 
Germans had gone. 
Afterwards, we sang the 'Potirakit. The priest used to sing this song to the rbil-
dren when they were waiting for manhin to come to pick them up from Evia: 
Mia comssi mondaine san Kie Mena 
Den bori na taki kamena 
Den bori na treki stin taverna 
Kie na pini crassi 
Kiomos mena dio potirakia 
Phernoun tis candies to merakia 
Kie phonazo messa stin cardiamaa 
Agapi mon krissi 
Acoma ena potiraki 
Acoma ena tragoudaki 
Sto cosmo you vrefica 
Ta panda varefica 
Agapes pikres Kie farmaki 
Acoma ligo coquineli 
Acoma ligaqui ti sas meli 
Mon ferni mia zali 
Messas sto mialo 
Kie tots xero Kie kamoyelo 
This was sung in memory of the priest, who had died in the revolution in Greece. 
We sang another song, tLa Rosa Enfloresset (tThe Rose is Flourishing'), in memory 
of Nikits uncle: 
La rosa enfloresse 
0 lib el MoS de Mars 
NI alma sb escouressb 
Adientro del lounar 
Los derbelicos cantan 
Con souspiros de amor 
De la passion los mata 
Oulvida ni golor 
r 
We sang Ava Nagila Ava. We made acheava. It was read by the baby. The widow told 
us to get ready for the Greek national dance, tBiva la Democraciat ('Long Live the De- 14 
mocracyt). We took each other's hands, and we started to sing a slightly altered ver-
sion of this song: 
-77- 
Oli oli na piastaumb 
Kie na arkisomb Koro 
Na efkaribtifoumb oli 
Pou irfamb y4a na se edo 
Zito zito i patrida 
Kie I Maritsoula Mazi 
Zito kie i iimocratia 
Kie i Bouena Tova Mazi 
The translation is: 
Everyone take each other's band 
And start to dance 
We should all be thankful 
To be able to come to see you 
Long live our homeland 
Together with our small Maritsa 
Long live the democracy 
Together with Buena Tova 
The widow gave me a present in tkixam everyone's name. She thanked me warmly for 
the chicoun (condominium) the baby had bought for her. We left the cemetery singing 
'Zoom'. Opposite the entrance of the cemetery, I saw many, many children. They were 
the second generation. All the children said in unison "Shalom, Nona." The smallest 
one came forward to give me a present in the name of all the children. He was Miki's 
son. I said goodbye to everyone, and I went to the taxi. I was crying with all my 
heart, and Miki was, too. I was very happy. I thought inxmipmakt that I had 1 done a 
very good deed. I thought to myself that there were few people as rich as I. 
Miki started to talk to me in the taxi. "Do you remember when I told my father 
that a soldier doesn't cry? Now I am a soldier. A soldier also cries. Do you remem-
ber when I told you 'You see, Nalitsal how I found my father!' Do you remember when I 
said 'You see, MAlitsa, even the best Malitsa makes mistakes.' Now I understand that it 
wasn't Malitsa's mistake, it was just mazal that we had." 
We came to the hotel, and of course Miki had to go. I sat down in the lobby in a 
very comfortable chair. Two minutes later, my niece came. I didn't want to go shopping, 
I didn't want any presents, but I had to go. She bought a present for my husbamd. 
When I returned to the hotel, Phyllis Waxman and Mr. Gelber were at a table in the 
lobby. Mr. Gelber invited me to sit down with them. In my thoughts I was praying "Oh, 
God, I don't want to fight with Phyllis." I sat down. Mr. Gelber asked me what dress 
e party that 
evening. Right away, Phyllis gave him a sign with her 
-78- 
eyes, not to talk to me about the party. I was not supposed to know. Mr. Gelber ex-
cused himself and left. AndLI stayed alone with Phyllis 
She understood that I knew there was a party. She said "Wear the white dress." 
answered "No". She said "No? I I thick that dress is very nice, and you don't?" I 
looked at her without saying anything, and I started to remember. 
Three years ago Phyllis used to wear an old camelhair coat that was two sizes too 
big for her. Her shoes were always one or two sizes too big. She had a kamimxt: hand-
kerchief in her hand all the time. I remembered a meeting at Adath Israel. She was 
wearing a suit one size too smAll, and a hat that looked like a nhamber pot. She was 
supposed to sit at the head table. The hat was so ridiculous that one of the ladies 
took her awn hat off and gave it to Phyllis to war* wear. And now, because she had 
bought a dozen new dresses, and Sadie Neamtan had taught her how to wear them, she want_ 
ed to tell me how to dress. 
