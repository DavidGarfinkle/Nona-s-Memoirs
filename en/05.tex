% mempages 95 to 110
% pdfpages 100 to 115
I.
-/7- 
.I said "Your husband isn't going to give to the Hebrew Academy?
You have a child 
in the school."
She answered "Leave my husband out of this.
He is not going to listen 
to me."
I said "If even your husband isn't going to listen to you, doh you expect me to' 
listen to you?
Phyllis, it's my pocket, and not yours.
And I'm going to give what 
pleases me.
I believe in Hebrew education, and I'm going to support the Hebrew Academy.
"This stinker called you to give to the Hebrew Academy?"
she asked.
I answered "you and, 
your rabbi are stinkers.
You discriminate against children learning Hebrew because thei 
Jewish 
parents are members of Young Israel Synagogue.
This is a disgrace to the imirm Communli 
I would like to know who this hipocryte rabbi is.
The Sisterhood has already been ruineu 
since we elected you as President last year because no-one else wanted the post.
But wi 
this question of the schools, the Sisterhood will be ruined even further.
Don't think 
f 
you're an expert in deciding who * should learn Hebrew and who should not.
And your rab 
would be better occupied with religion than with politics."
The next day I sent a letter of resignation to the Sisterhood.
Mr.
Sternthal asked 
me many times whu I had resigned.
I never answered him, because I didn't want to take 
part in these intrigues.
The President, Mrs.
Bessie Gorolnik, wrote to me that she didn'.
want to accept my resignation.
It was suppertime at the hotel in Tel Aviv.
I got up and got dressed.
My husband f 
and I went to the dining room.
In the lobby, I saw Mr.
and Mrs.
Syd Gotfried, very nice, 
people.
Mr.
Gotfried was an officer of the Combined Jewish Appeal.
It was at his office' 
in Mohtreal that I paid for our tickets for this trip.
Every day, for the last fifteen 
days, he would look for us to say hello, and to ask how we were.
You could see in his 
eyes that he wanted to say "I'm very sorru, Mrs.
Garfinkle, for what is going on here."
I shook hands with both of them, and I said "Thank you very much for everything."
The next day was our departure date for Montreal.
We arrived at the airport.
Half 
the group went to Spain, and the other half, us included, went to New York, and from New '• 
York to Montreal.
I sat down in the airplane and I started to remember the Jewish life i 
Salonica before the war.
When someone invited people who were not in their milieu, it was traditional to do 
everything in their power to make these people feel welcome.
After 
hallang grown up in thr 
atmosphere, how could I accept the treatment I had received from some oreem of the people oni 
-96— 
this trip?
I couldn't believe that I hadn't opened my mouth and told them "Look here, 
chaverim, if we are so unwelcome here, my husband and I, Madame Chairman should go up to 
Mr.
Garfinkle and say 'Get out of here!'.
This would be better than to be humiliated every!
hour of the day."
One hour into the flight, the stewardess came over.
She was carrying some small 
pillows.
She gave one to my husband.
I told her "If you have something else to do, give 
MB the cushions and I will give them out to everyone."
Mr.
Waxman was sitting near me, 
and I went to give him a pillow.
Suddenly I heard Phyllis Waxman talking to the wife of 
the doctor in our group.
The conversation consisted of gossip about Phyllis Wexler.
I 
didn't want to listen any more.
But Phyllis Waxman told the doctor's wife "Now, ask her 
who Phyllis Wexler is, if you don't believe me."
The doctor's wife didn't want to ask me, 
but Phyllis did.
"Na, Tova, tell her who Phyllis Wexler is, your associate chairman 
in 1967."
"Phyllis Wexler is the most charming lady that I know," I answered.
Phyllis Waxman 
was mad, but I went on.
"You've finished godsiping about me.
Are you now going to 
gossip about Phyllis Wexler?
You do this to everyone who has a bit more education than 
ypu, or is more capable or more knowledgeable than you.
You are scared because we know 
who you are.
These wonderful people believe you because they don't know you.
Madame 
Chairman, the leadership has gone to your head.. I told you to get off my back.
Now I'm' 
telling you to get off Phyllis Wexler's back.
She is a person whom you don't even deserve 
to know."
I gave Mr.
Waxman a cushion, and Phyllis screamed "Don't do what I don't do for 
my husband!"
I left them.
I said to myself "If Thomas Hecht is not sleeping, I will go to talk to 3 
him.
now.
I'll tell him what has been going on on this trip.
He has a good idea, I'm sure, 
but I want to tell him myself."
I went to see him in the First Class section.
I opened 
the curtain, but I said to myself right away "I'm not going to talk to him..E.Iikkaama I'll 
talk to him when the Combined Jewish Appeal asks me for a donation."
Thomas asked me 
if I needed anything, I said "No, thank you," and I went to sing with some wonderful 
young people who were in our group.
I started to think about what I did after I sxttimixtkx got the children settled.
I came to my sister's that night, very tired.
I was happy because the children ha-
homers.
But I felt myself very alone.
When I came in, my sister said "You have a lette 
from Motsa."
I replied "Motsa?
I don't know what that is."
They explained it to me.
"It is a rest home for people who have been gravely ill for a long time.
They go there 
to rest.
But in this place," my sister said, 2beds are hard to find.
But Motsa is in-
viting you to go thdre to rest.
An anonymous person has paid for your stay there.
We 
called you on the phone at Havad Halimoud, but you weren't there.
The invitation is to 
start tomorrow."
I thought to myself that this is what I needed: rest.
The next day at eight in th 
morning I was i on the bus to Motsa.
I went to Jerusalem, changed buses, and arrived at 
Motsa.
Naturally, even the bus tickets had been provided by Motsa.
I came to Motsa, 
and I thought I had arrived in ganeden (paradise).
There were beautiful gardens; the 
grass was like a green velvet carpet.
A nurse came to receive me.
She took me to the 
physician, who examined me and told me I needed complete rest and very good food.
The niurse showed me where to go to eat three times a day.
Twice a day I would 
get a snack.
She showed me where to get milk, as much as I wanted during the day.
"Oh, no," I said, "I don't drink Milk."
She took .me to my room.
It was during the coax{ 
pulsory rest period.
A woman of about 22x or 23 years was getting ready to go to bed -1 
and rest.
The nurse introduced us.
"This is Leah".
Then she asked me "Bouena, R what 
is your name in Hebrew?"
I replied "Tova".
The nurse turned to Leah and she said 
"This is Tova.
She will be your roommate."
Leah spoke Hungarian and a little bit of Hebrew.
From the moment we met, we liked' 
each other.
After we rested, we went to the game room.
The game room had many things 7. 
it to amuse people without tiring them.
Leah introduced me to her friends.
It was vi-
siting hours, but neither Leah nor I had any visitors.
One day I saw a very handsome officer, who had eyes only for us.
I said to Leah, 
"Hey, Leah, this officer og the Jewish Brigade is looking at you a lot."
Leah turned 
around, and recognized her husbands She had not seen him for a very long time.
Leah 
introduced me to him, and he talked& to me a bit in French, a bit in Hebrew.
I told ti 
-98- 
that they shouldn't bother themselves about me, they should go to our room.
Leah's hus-
band wanted to spend the night.in Motsa.
At four in the morning, he had to be in camp, 
and his camp was near Motsa.
He came to say good-bye to his wife because his regiment 
was being moved from Palestine to Europe.. 
I said "You go to our room.
When I come in, the nurse won't notice that there are 
three of us instead ms of two.
I know what it is to have your loved ones go to the bat-
tle."
Leah went to oyr room first.
Her husband followed, and I was the third to go.
When I. came into the room, I saw Leah's husband's uniform on the chair.
I was afraid the 
nurse would see it, sa I took the bedspread from Leah's rommx bed and covered the uniform 
with it.
I told them "The nurse is going to come to check if we are sleeping.
If she 
76) 
sees this uniform, we're going to be in trouble!"
Leah's husband kris spent the night in our room.
The next morning, I was still asleep 
when he was ready to leave.
He woke me up and took me in his arms, and with tears in his 
eyes he thanked me.
I remembered when Chaim went to the battle.
At the time there was 
a song that the people would sing to the soldiers at the train station, to say good-bye.
It was called 'Ayde sto calo'.
This means 'Go with good health'.
I sang the song for 
Leah's husband.
The next day I had visitors, George and Nicola.
They had left the camp at Gaza.
They 
cape with Irene, Nicola's sister, and her husband Henri.
We took pictures together.
We 
spent a very pleasant two hours together.
In Motsa there was usually entertainment for everyone; Newspapers, a library, games, 
radios, singing, music.
There was also very good food, and fresh air.
We were like a big 
family in Motsa.
No-one was alone.
There were often conferences held on very interest-
ing subjects, with very good speakers.
There were beautiful pine trees on the grounds.
From the pine cones we would make birds, each of us ustIng our imagination, and we would 
teach each other.
There was the greatest spirit of camaraderie that could exist in life.
It was exactly what I needed after what I had gone through.
But I never knew (and I still 
don't know, to this day) who had arranged all this for me.
initim left Motsa with many regrets, and Leah was very sorry to see me go.
But she 
was also supposed to qo home 
-99- 
I came back to Tel Aviv very well rested, and ready to find a job.
The next day I 
went to WIZO.
They received me very well.
They asked me if I had any references.
I 
said no, but I sent a telegram to Mr.
Burcard of the International Red Cross in Geneva.
I would get an answer for sure, and the International Red Cross would testify An to what 
I had dote for them.
WIZO gave me the address of a nursery for newborn babies.
They told me "If you don' 
have any references, you will be paid less," They explained the kind of work I would be 
doing.
Working mothersx would bring their babies in the morning and pick them up in the 
evening.
I said "Working with children!
I speak very little Hebrew.
How will I speak - 
with them?"
The lady from WIZO answered "They are babies.
You only have to prepare the 
milk, and feed them, and change them.
You don't have to talk to them."
She advised me 
to present myself at the nursery the next day.
1 
I went home and my sister gave me a telegram from the International Red Cross, signt 
by Mr.
Burcard.
He testified to the work that I had done for the International Red Cros 
in Salonica, and to my background.
The next day I went to the it) nursery.
The director 
received me.
Since I spoke very bad Hebrew, she asked me if I knew Yiddish.
I said no2 
and t gave her the telegram frpm the International Red Cross.
She didn't even look at i 
She just asked me "Are you Sepharady?"
I answered yes.
She replied "I'm very sorry.
We don't hire Sepharady here."
I wanted to spit in her face.
But all I said to her wask 
"The German's don't want the Jews.
The Palestinians don't want the Sepharady."
I thanki 
her for the time she had wasted on me, and I left.
Outside there were a couple of steps.
I sat down on them.
I wanted to cry.
But I 
remembered that I didn't have to cry.
Chaim had recommended singing instead.
I started,' 
to remember that the Jews in the ghetto would ask each other what was new.
Somedne woulk_ 
answer by singing the 'Phtlosophy of Koelb': 
Nada de nada 
Duho Koelb 
Nada de nada 
Todo nada 
Que avantage 
A el hombre 
Con toda son lazeria 
Que bive 
Debacho de el sol 



-79- 
While I was thinking about this, she was talking to me.
She said "I gave $500 for 
the campaign.
But you gave to your sister.
I saw thw way she came here last night, so 
elegant."
I answered "My sister didn't take anything from the commmaity, even if she 
is so elegant.
And I want you to know one thing, Phyllis.
As long as you are in the 
Women's Federation, I'm not going to give one penny.
Every Jew mnst help Israel.
But that doesn't mean that I have to give to the Women's Federation.
I can send money 
directly to Israel.
Or I can give to the Men's Division.
"I refused to work with you in the 1969 campaign.
In 1966, it wasn't so bad to 
work with you, because you were district chairman and you wanted to collect every penny 6 
to show that you were somebody.
But in 1967, because you refused to work miik under the 
direction of Hrs.
Fisher, you did everything to destroy my district.
Friday, before 
the Sunday we were to mike the collection, I went to prepare the allocations.
I was al-
most finished.
I left for an hour to give my husband his supper, and you destroyed the 
FOlocations.
I really wanted to cry.
When I asked Phyllis Wexler who had done this, 
she said 'I don't want to talk.
Ask Phyllis Waxman.'
So I asked you.
The only thing 
you had to say was 'So what?
2/ Start your campaign next week!'
In 1967, Phyllis, there 
was a war in Israel, and soldiers were dying.
"A lady came to the office just then, Phyllis Jackson.
She told MB not to worry.
Even ie she and her sister, Sheila Lipson, had to work all night, the cards would be 
finished.
They would make the allocations.
You wanted to kill her when she offered to 
do this.
These two ladies worked all night fisting the cards and putting them in order.
But people like Phyllis Jackson are of no use to you.
"It was only on Sunday that I found out why you had done this.
You called all my 
workers; it had taken me weeks to convince them to become canvassers.
You told them 
to go to Adath Israel for the distribution of the cards.
You wanted to form a caval-
cade of canvassers, but Phyllis, if you want to do this, you have to find your own wor-
kers and not take mine.
I complained to you.
Your answer was 'If I didn't do this, I 
would be washed out of the Women's Federation.'
I replied 'You do things at the ex-
pense of otherset Even after all this sabotage, I got the campaign into very good shape.
In 196 you told me that 1 had done enough, that I shouldn't run around any more.
I should leave it for the next year.
I answered, I don't work for Mrs.
*lkin, I 
work for the organization.'
You weren't satisfied.
You told the associate °halms 
take my cards before the campaign was over.
I called Mrs.
*akin and I told her what 
was going on.
Mrs.
*Thin told me that she would speak with the associate chairman.
When I put the phone down, my doorbell was ringing.
It was the associate chairman, 
coming to take my cards.
I'm sure she was coming from your house.
I asked her not to 
take them, because traditionally the cards are taken after the closing luncheon.
She 
answered it 'If I take the cards now, Phyllis will give me a job of honour in 1969.'
I don't work for honnour.
I repeat it again for you.
I work for the organization.
My honour is in my conscience, for the acts I am doing.
"You ruined my son's life in school with your gossip.
You ruined my trip with 
gossip.
These wonderful people don't know who you are.
You would burn the wholv 
world down to get IT what you want.
When you want to ask where my husband is, you 
make your hsnas shake.
But I have never answered you.
I want you to know that the 
tremor my husband has is a disease, like an ulcer, except no—one can see an ulcer.
Et 
in my house we hays love between us, and you don't have this, Madame Chairman.
In my 
house, we make sacrifices for each other, and this you haven't done and never will do, 
Madame Chairman.
Only when I decide to write my memoirs will these wonderful people 
know who you really are, Madame Chairman.
"When we went to Yad ve chen, I wanted to vomit seeing you represent the Jewish 
women of Canada.
This was a disgrace for all those people who died in the crematoriums 
or by German bullets.
There were names of people there who had had salt poured into 
the open wounds on their backs so that they would reveal where the partisans and Jewish' 
families were hiding.
And you represented American Jewish women at Yad ya chen.
You 
love it when someone else cries.
I remember, and you remember, when a little girl ask' 
you howx you could laugh at a time like this.
You see, Phyllis, it is not the $500 donation, and not the dozen new dresses, ths: 
change you inside.
The only thing that has changed is that you no longer wear your car 
camelhair coat.
My father was a prophet.
He was rikht.
But I don't know if I shauldm 
spit in your face or feel sorry for you.
But I am sorry for the ladies in the Women' 
-81- 
Federation who voted for you.
"People like you shouldn't collect money for sick people.
How will you have the 
face to ask the public for money for Maimonades and Convalescent Hospitals?
Those 
two hospitals are the centers of research for Parkinson's disease.
If you want to 
bury the sick people and their families alive, you have to find an organization that 
collects money for cemeteries, and then bury the sick people and their families."
She replied "Me?
Me?"
But I didn't let her continue.
"Yes, you, you!
When we went to the hxipmx houlpan, you told me to shut up.
But 
now, it is my turn to tell you to shut up.
In the haggada of Pesach, we say 'be 
°haw aba be yerounhnlaim.'
But you forgot.
In the middle of the haggada, we say 
Idaenoul.
But with the insane superiority complex that you have developed since Sadie 
taught you how to dress, the daenou is scratched out of your haggada.
"You told people that my husband and I didn't want other people to sit mi at our 
table, and no-one did.
It was a miracle if someone sat down with us.
I showed you 
how myhusband and I eat, not like you.
You ate the fish with your hands, Madame 
Chairman.
In my city, they would throw tomatoes at a person like you.
"You go to the meeting and you laugh at people.
You say they are wearing slippers.
You make fun of them.
They aren't slippers.
They are just shoes with low heels.
People who were in the interrogation room with the Germans can't wear high heeled shoes, 
Madame Chairman.
You told everyone in the Women's Federation, 'Hah, Mrs.
Garfinkle 
was in jail!'
My only crime was being a Jew, Madame Chairman.
"In 1968, you promised my associate chairman a reward if she took my cards before 
they were due.
I refuse to work with people like you.
But Mr.
King is going to make 
you a star.
He is the best advertising person I know.
It's a pity that a talented 
man like he doesn't go to Hollywood.
"Get off my back, Madame Chairman.
Many people took advantage of us since my 
husbands hands started to shake.
We used to have blind faith, but they swindled us.
Friends of my husband!
Bat it was only money.
But you, in these fifteen days, you 
took our blood, Madame Chairman.And one day you will see, Madame Chairman, in Ladino we x 
say 'Las del avoua al avoua, las del vino Al yinn_ r 
God.
What belongs in the water will go to the water, what belongs in the wine will go 
to the wine.
You practice in the mirror, immitating the way Mrs.
Gerstman talks.
I 
hope you will be able to look at yourself in the mirror after what you did here these 
past fifteen days."
I got up and left her.
On the stairs, I met Mr.
Gelber, a very nice person.
told him "I want everyone in this group to know one thing.
I'm not the poor cousin 
who came to visit the rich one.If you knew where I came from, the religious and hones' 
man that you are, you would understand.
If there's one person here who deserves to bA 
invited by the community, it's me.
And don't think that I came for free.
I paid like 
everyone else.
But this Madame Chairman made my life miserable for these fifteen day 
I have remembered all the miseries of my life.
I spent the fifteen days remembering 
the miseries of the 1940's, and it is now 1969.
Niseries that I never wanted to reme 
ber, and never wanted to talk about then to anyone.
Bat $500 light donation is strong.
er than being a hrmnnitarian.
I don't ask that the community give me a gold medo, for 
what I did.
But I will not let myself be insulted any more, especially when it is 
owing from some ignorant person because she gave SAR a $500 donation.
She thinks 
that because of this, she is permitted to insult someone like me!"
I left Mr.
Gelber and I went to my room.
I closed the door.
My husband asked 
me what I had bought.
This made me feel a little better, and I started to remember 
again.
I started to remember the things I didn't tell Phyllis.
When we were in Quinerette, 
Max wanted to visit Ein Hachofet on our day of leisure.
Max had spent thirteen years 
on this kibbutz, and he had laid the first stone.
We called a taxi, and we were ready to 
go.
The taxi driver asked if there were only the two of us going.
We said yes.
He askec 
us if he could take his wife along.
We went to Sfat, and picked up his wife.
On the road, I started to talk to his wife.
I asked her where she had been born.. 
"I am from YfIninP, in Greece."
Oh, we had so much in common.
We spoke Greek, and we 
spoke Lading, and we soon arrived at Ein Hachofet.
The cbaverim were very happy to see us.
They showed us the beautiful dining room, 
the two swimming tin: pools, one for little children, and one for adults.
They showed 
us their industry; they manufactured screws.
We had one day to feel like human beings.
We returned later in the day to Quinerette.
Everyone was ready to visit the kibbutz at X Quinerette.
We also went to see it.
Mrs.
X was near us.
When we started to vac walk, we saw a beautiful tall China vase in 
one of the doorways.
1 
"Oh," I sad to Mrs.
X, "this is early Chinn."
She jumped.
"How do you know?!"
I answered "Is it because I speak English with an accent that you ask me this?
Or 
is it the portrait that the chairman has painted of me?"
They shoimd us where the children played, and the barn.
Suddenly Mrs.
X said to me 
"you know, Mrs.
Garfinkle, my mother wouldn't do what you are doing!"
I replied, "of 
course not!
I'm special" 
We went back to the dining room of the kibbutz.
She said to me "You have money 
for taxis.
But you don't have any for the Women's Federation."
I answered "Yougre 
mistaken.
I have plenty of money.
But I won't give one penny to the Women's Federation 
as long as you are there.
My grandmother used; to know a proverb: Quien al nouevo rico 
servio sou aiempo piedrio.
A la fin del agno lo yanaron ladron o poutagnbro.
It means 
'People who serve the nouveau riche are only wasting their time.
At the end of the 
r 
year they will call them crooks and whores.'"
-1114-remitidu1 mu 
Phyllis Waxman asked me if I wanted to speak to n 
group of French-speaking ladies.
I accepted right away.
Phyllis came to pick me up.
The subject that I talked about was the situation of the Jews in Arab countries.
I 
said that we needed a strong Israel.
The next day, Phyllis called me.
Instead of tell.'
ing me if I spoke nicely or not, she said "I'll kill you if you tell Rossetta that you 
spoke last night."
I asked her why.
She answered "Because I told Rosetta that I spoke 
last night."
I never told Rossetta that I had been the speaker.
I remedbered another episode in 1965.
I was district chairman, and I had to call 
a meeting.
I told the Women's Federation the date on which I planned to hold ig the 
meeting, so that they would arrange for some speakers.
The speaker was Mrs.
Fisher.
Mrs.
Fisher always speaks very well.
I called all the workers and area chairmen.
The 
day of the meeting, most people showed up at my house.
Whx: As soon as we were ready to\r 
start the meeting, Phyllis Waxman arrived.
I was surprised, because she hadn't been 
invited.
As soon as she arrived, she said to Mrs.
Fisher "Let's gol2 Let's got" I sail' 
"Phyllis, she's the speaker!"
"I want to go to the Y. There's a very important meet-
ing.
All the ladies are going to be there."
I said "Phyllis, you weren't invited here 
You can go to the Y by yourself.
I'm going to wait another five minutes for Ella Cohen,,.
the associate c►.hAirma_n, and then I'm going to start the meeting."
Phyllis answered 
"Ella Cohen is not coming.
She's sick."
I said that since Ella was sick, I would 
start the meeting right away.
As soon as I had said this, the doorbell rang.
I opened the door.
It was a taxi 
driver with a box in his hand.
He said "I took a lady from Cote St.
Luc to the I, and 
she gave me this box full of papers to give to you."
I took the papers.
Fla Cohen 
had sent them for the meeting.
The taxi driver asked me for # $3.001 and I paid him.
I said tp Phyllis "You told me Ella Cohen was sick, but she went to the I, like 
you want to do.
I just paid 0.00 so that she could go to the Y by taxi."
Mrs.
Fisher started the meeting.
In her speech she asked who were the spokesmen 
for the Jewish community of Montreal.
One of the ladies answered "The Congress."
Phyllis Waxman said "Let's go, let's go!"
to Mrs.
Fisher.
At the door.
I said to PhArlie 
"I paid $3.00 and the meeting is ruined.
You people take others for suckers."
They 
left.
-85- 
The ladies at my house were very insulted and angry.
They asked me "Is it only 
the ladies at the Y who are important?
If we're not important, why did you call us 
to a meeting?"
I said "Let's go, I'll serve some coffee."
But everyone got up and 
left without having any coffee.
I felt terrible.
I put my coat on, took the car, and went to the Y myself, just to see what kind 
of meeting they were having.
As soon as I entered; I'saw Ella Cohen in the lobby.
I told her "Between you and Phyllis, you ruined my meeting."
She answered "You don't 
understand.
Phyllis and I want to climb to the top in km life.
And to go up, to 
have honour, you must be seen talking to Mrs.
Bronfman.
And that's what we did today."
21 I said "But don't you care about the campaign?
Even with all the boycotting that 
you and Phyllis are doing, I'm going to get the campaign in good shape."
She replied 
"Of course, because you are friends with Rossetta."
"I don't work for Rossetta.
I 
work for the campaign.
And Mrs.
Cohen, please pay me the $3.00 for your taxi to the 
1.." But she never paid me.
The second person I saw was Phyllis Waxman.
"Oh, Tova, are you invited here?"
I said "No, I came to see what meeting you were having here, important enough to ruin 
mine."
She said "You are a coo-coo.
Do you think I would sit down with a bunch of 
women from Outremont if I have the chance to be with Mrs, Bronfman?"
"But you weren't 
invited to my.
meeting," I said.
"I went to rescue Mrs.
Fisher."
"The best thing for 
MB to do now is leave," I said, "because I feel like pulling your hair out in publim," 
You can imagine how hard it was to find workers after this.
I did most of the work 
myself.
A few weeks later, it was decided to have another meeting in my district.
Phyllis 
Waxman informed me that the meeting would not be held at my house.
I was very glad 
about this, but just the same, I asked why.
I also wanted to know where it would be 
held instead.
She said "Your house is not up to the standards of the community."
i said "I'm very proud of my house.
I have a beautiful home."
They looked for a 
house, and they decided to have the meeting at the home of Mrs.
Rosenfeld.
Mrs.
Ros-
enfeld is always very generous towards all the organizations in Montreal.
She lives 
in a beautiful cottage.
Not more than five ladies showed up at the meetit Thank 
God that the board of the Women's Peapratinn wpm +.11=ePA fin mmkp fihP mpAting, a -1-0Athrz 
little warmer.
When the meeting ended, I thanked Mrs.
Rosenfeld.
"It was very nice of you to offer 
your house, and I'm so.t-Ly that they disturbed you.
You have a beautiful home, but mine 
is nothing to be ashamed of."
Lying in bed in the hotel in Tel Aviv, I wanted to forget all of this unpleasantness, 
so I started to think about all the children and about the widow.
She was so happy to 
have the chicoun, to get out of the slum she was living in.
And I started to remember 
happy things.
I had been liberated for three months in Israel, and I had heard nothing about the 
children or the baby.
One day I received a letter from the baby's uncle.
He told me he 
had just started to walk without crutches.
On the third day of Pesach he would be in Tel 1 
Aviv.
He wanted to make a third seder with me.
At six P.M.
someone would come to pick me' 
up and take me to the hotel where he would be staying.
And I would see the baby, too.
At six P.M.
on the third night of Pesach, the doorbell rang.
From the balcony where 
I was standing, I could see a taxi.
The taxi driver was all dressed up in blue, and his 
shoes were newly shined.
It was Miki's uncle.
He took me to the hdttel where the family wl 
staying, with the babies.
I took the baby in my arms.
My cheek against the baby's cheek felt like velvet.
My 
I) 
tears were falling like rain, and I said to the baby's uncle "Don't think I gave you this ri 
baby because I wanted to be rid of him.
I'm too attached to him.
I suffered so much to 
save him from the Germans.
I gave you the baby because I have nothing to offer him."
Suddenly we heard people singing a Greek song.
I asked if we were in a Greek hotel.
;1 
"Oh," said the uncle, "let's go to the dining room to hear the singing!"
"No," I said, 
"I want to be with the baby."
g "The baby has to go to sleep."
He called the nurse and 
she took the baby away.
We went to the dining room, and the first person I saw was the priest.
Of course, he 
was now a doctor, with a very fancy army uniform.
The baby's uncle had arranged a reunion 
for me.
Miki's uncle was there, too.
The priest introduced me to two Jewish officers 
from the Brittish Army who had come with him.
Suddenly, the fisherman with a new suit 
and shined shoes appeared.
Suddenly, all the children came in as well.
There was a lovely buffet, with all the best things to eat.
The baby's uncle said 
"attention!"
And he started to tell us the story of Pesach, how the Jewish people had 
left Egypt.
"Pharaoh had the ten dinims.
I'm telling you this so that you will know that ic his-
tory repeats itself.
The Germans, on the Russian front, have exactly the ten dinims.
Moses helped the Jews get out of Egypt.
You see how history repeats itself; Maritsa helped 
us get out of the claws of the Germans.
The Jewish people wandered in the desrt for 40 
years, and they received manna so that they could survive.
You people, you received manna 
also.
The priest taught you how to kill birds for food.
If the Turkish government hadn't 
given us trucks and trains to go to Syria, we would have spent 40 years in the desert, too.
If the Brittish government hadn't taken us by train to the Promised Land, we would have 
spent 40 years in the desert, too.
And now, we are going to start the programme."
The baby's uncle said to me "The children are well prepared."
They took a big table, 
and two children tied up all the others around the table.
It was exactly the same scene 
as in Evia, when the bandits tied us up.
They called me to sing the 'Elia elia mo yalessa."
When this was finished, we sang the traditional Pesach songs.
We ate from the buffet, 
and we sang the traditional song, 'Eloenou the ba chamaym.
Eloenou chb ba chamaym 
Eloenou chi ba chAwAym 
El dio club mos mand 
A yerouchalaim 
Con la caratana grand 
Coualo son los dos 
Dos maestros padrbs son 
Primero eze el creador 
Barekou barouk chemon 
Eloenou ebb ba chwinnym 
El dio qub mos mand 
A yerordir 1 aim 
Con la caravana grand 
Coualo son los trbs 
Trbs mouestras padrbs son 
Dos Moehb I Aaron 
Primero eze el creador 
Barekou be.rouk chemon 
Eloenou chi: ba cihannym 
El dio qub nos mand 
A yerouchasim 
Con la caravane grand 
Coualo son los quatro 
Quatro madres de Israel 
Tres mouestros padrbs son 
Dos Mochb i Aaron 
Primero ese el creador 
Barekou barouk ahem= 
Eloenou chi ba piipTnnym 
dio qua mos mand 
A yerougthn 1 n im, 
Con la caravane grand 
Coualo son los cinco 
Cinco livros de la ley 
Quatro madres de Israel 
Tres mouestros padrbs son 
Dos Nbchb i Aaron 
Primero ese el creador 
Barekaa barouk chemon 
Eloenou chb ba ehAmaym 
El dio qub mos mand 
A yerouelin 1 slim 
Con la caravana grand 
Coualo son los seih 
Seih dias de la semana 
Sinco livros de la ley 
Quatro madras de Israel 
Tres mouestros padrbs son 
Dos Moab i Aaron 
Primero ese el creador 
Barekou barouk chemon 
Eloenou chl ba elapripym 
El dio qub mos maned 
A yerounhAlAtim 
Con la caravana grandb 
Coualo son los sietb 
Sietb dias con chabat 
Sech dias de la semana 
Sinco livros de la ley 
Quatro nadres de Israel 
Tres mouestros padrbs son 
Dos Mochb i Aaron 
Primero ese el creador 
Barekou barouk chemon 
Eloenou chb ba chamaym.
El dio qub mos mande 
A yerouchnl aim 
Con la carotene.
grandb 
Coualo son los otho 
Otho dias de 4textfala ke9pck 
Sietb dias con ehabat 
Seth dias de la semana 
Sinco livros de la ley 
Quatro =Ares de Israel 
Tree rnouestros padrbs son 
Dos Mochb i Aaron 
Primero ese el criador 
Barekou barouk chemon 
Eloenou dab ba elinTnn3rm 
El dio qub mos mandb 
A yerouvhalaim 
Con la caravan grandb 
Coualo son los noueve 
Noueve mezes de la Pregnada 
Otho dies de las t oopck.
Siete dias con chabat 
Sech dias de la semAns 
Sinco livros de la ley 
Quatro madras de Israel 
Tres mouestros padrbs son 
Dos Mochb i Aaron 
Primero ese el criador 
Barekou barouk chemon 
Eloenou chb ba chamaym 
El dio qub mos mama 
A yerouelaninim 
Con la caravana grandb 
Coualo son los dies 
Dies comandamientos de la ley 
Noueve mezes de la pregnada 
Otho dias de twine.
1-0 %.)
r-
Sietb dias con chabat 
Seth.
diets de la semana 
Sinco livros de la ley 
Quatro madras de Israel 
Tres mouestros padrbs son 
Dos Mochb i Aaron 
Primero ese el criador 
Barekou barouk chemon 
Eloenou chb ba clanTvriym 
El dio qub nos mandb 
A yerouelininim 
Con la caravan grand 
Coualo son los pnzb 
Onzb yos sin Joseph 
Dies comandamientos de la ley 
Noueve mezes de la pregnada 
Otho dies de # UsuNka INovrok.
Sietb dies con ehabat 
Sech dies de la semana 
Sinco livros de la ley 
Quatro madres de Israel 
Tres mouestros padrbs son 
Dos Mochb i Aaron 
Primero ese el criador 
Barekou barouk chemon 
Eloenou chi ba ellgymAym 
El dio qub mos mand 
A yeraunbAlAim 
Con la caravana grand 
Coualo son los dose 
Doje ijos con Joseph 
Onzb izos sin Joseph 
Dies comandpmientos de la ley 
Noueve mezes de la Rregnada 
Otho dies de kedispa 
Siete dies con chabat 
Sech dies de la semmin 
Sinco livros de la ley ' 
Quatro madres de Israel 
Tres mouestros padrbs son 
Dos Mochki Aaron 
Primer() ese el eriador 
Barekou barouk chemon 
Eloenou chb ba chmmaym 
El dio qub mos mand 
A yerounlisa 1 n 
Con la caravana grand 
Coualo son lo treje 
Treje ijos con Dina 
Doje ijos con Joseph 
Once izos sin Joseph 
Dies comandamientod de la ley 
Noueve mezes de la pregnada 
Otho digs de koalas.
I..ovpu\ 
Sietb ding% con chabat 
Sech dies de la seminn 
Sinco Iivros de la ley 
"Quatro madres de Israel 
Tres mouestros padrbs son 
Dos Hoch i Aaron 
Primero ese el criador 
Barekou barouk chemon 
This song means, roughly: 
God send us to Jerusalem 
With a big caravan 
First is the Creator 
Second, we have two fathers, 
One is Moshe, and the other, Aaron 
With the Creator there are three 
Four, we have four mothers of Israel 
Five, the five books of the law 
Six, the xi six days of the week 
Seven, seven days with shabbat 
Eight, the eight days of houpa 
Nine, the nine months of pregnancy, 
Ten, the ten commandments 
Eleven, the eleven sons without Joseph, 
Twelve, twelve sons with Joseph 
Thirteen, thirteen children with Ding 
But first is the Creator 
Barekou barouk chemon 
Again, I started to mRke a resume in my mind of this trip I was on to Israel.
It is a shame that this chairman and others like her think that with a $500 donation, 
they are militant zionists and they have bought Israel.
I started to remember the 
boat Sarah Primo, that came to Salonica.
It was a school on a boat, belonging to the sukonouth.
The sukonouth, for the 
Jews of Salonica, was synonymous with the salvation of Israel.
We had a reception for 
the crew of the Sarah Primo in one of the clubs in Salonica, called the Zentin.
All 
the zionist organizations were gathered there.
Eliaou introduced the captain.
Eliaou 
represented the Keren Hayssod.
Mr.
Bessanthi introduced the captain's speewch.
Mr.
Bessanthi was one of the foremost intellectuals in Salonica.
He knew ten languages, 
perfectly.
He was a journalist.
He had a wife and two daughters, as intellectual as 
he was.
Mr.
Bessanthi was later taken by the Germans, and we never found out what 
happened to him.
This meeting with the crew of the Sarah Primo was like a holiday.
The Jews were 
very happy to see the first boat—school from the sukonouth.
It was in the gulf of 
Salonica, at Niki Street.
The atmosphere was like that on March 25, the Greek Nat— 
ional Holiday.
There were so many people going to see the boat.
All the flags on 
the boat has Marron T vAiA nn -Fham Theme wrefirnilma of -Eho nrpm were white with Maven 
1 
David.
All the crew was Jewish.
Officers, Jewish.
We thought at that time that Pales-
tine was ours.
We saw the Jewish flags with Magen David on the boat with our own eyes.
The name of Sarah Primo was written on the boat with Hebrew letters.
You had to 
see it all to understand the reaction.
The captain gave a present to Eliaou, a. miniature 
of the Sarah Primo.
He gave Jewish flags with 'Strap Primo' marked on them to Regina as 
to me.
Eliaou ordered a vitrine to be made, in which to put the boat and the flags, and 
this was put in the best place in our library.
Everybody would tell Regine and me "How 
lucky- you are to have a brother like Eliaou!
Because of him, you have the Sarah Primo, 
too."
And just look at the way I am considered here, on this trip.
All I could think of 
was that people like this made something of themselves while Hitler made us, the ones 
Europe, lose.
How many Jews died with the hope that the American Jews would save us?
I couldn't swallow.
She told me she gave $500.
Jews of America, it is time to 
show people like this that it is not the $500 that we need.
What we need are clean com-
munities, and hrimnnitarianism.
in 
7 
1 
- 7.1* 
I started to remember 1968 again.
I hated that year, but I couldn't help myself.
My associate chairman would come every morning to pick up the money I had collected the 
day before.
Everything was going very smoothly because I hate to have money that isn't 
mine in the house.
One day Phyllis Waxman called me.
"Your associate chairman is sick 
and can't come for the money today.0 "Oh, God, I have $6000 that I collected yesterday 
in the house."
She said that she would come to collect the money if I wished.
I was very 
glad that someone would take the money.
Phyllis said that she would be at my hpuse in an 
hour.
I said "I won't be home in an hour.
But I can take the money to your house right 
away."
She agreed.
Five xxst minutes later I was ringing her doorbell.
Her two children opened the door.
The two together =INN called their mother: "Hey, Mh, Mrs.
Contaminated is here!"
I 
said to the children in Ladino :"De la cavessa fieda el peche."
This is an old proverb 
whicif.
means "The fish starts to smell from the head."
I had just finished saying this 
when I heard Phyllis' voice.
"Tova, I'm in the bathroom, wait a little."
She knew to 
whom her children referred by that name.
This was the niCknAme she had given us because 
my husband's hands were shaking.
But she never dared to call me tic this to my face.
,I started to remember the question of the amalgamation of Adath Israel School with 
Young Israel School.
It was no secret that all the Jewish schools were having money prob-
lems, and Adath Israel was no exception.
Mr.
Sternthal, a member of the Board of Education 
and one of the founders of the school, called the parents together to try to find a solu4 
tion.
Ten or fifteen parents attended the emergency meeting.
Mr.
Sternthal explained the 
school's situation to us.
He told us that the teachers there were very good, and to have 
this high quality of teaching, one had to pay for it.
Many pupils who came to the school 
in the afternoons didn't pay.
There were others in regular school all day whose parents 
couldn't pay.
A solution had to be found, or the school would have to close.
I spoke and I said "The district has changed.
It has become smaller.
People are 
moving away.
There is no place for two Hebrew schools.
If we join forces with Young Is-
rael, we can keep the school."
There were many members of Young Israel at the meeting,' 
parents of the pupils in the High School.
They found the idea very good.
We decided that 
Mr.
Sternthal would speak with the Board of Education, of Adath Israel, and they would 
speak with the Board of Education of Young Israel.
A few weeks later we found out that Mr.
Loewy, member of the Board of Education of YovnE 
Israel, and Mr.
Sternthal had taken the proposal very seriously.
After many discission 
the amalgamation took place.
There was a meeting of the Adath Israel Sisterhood, and 
the question of the schools came up.
The Sisterhood was split.
Some were for the amal 
gamation, and some were against it.
Some of the members went so far as to say that if 
the amalgamation took place, the Sisterhood would no longer give assistance to the schoo: 
I explained to the women that we had committments: "We took children without pay.
The: 
are many Moroccan children in the school whose parents cannot pay.
What is the different 
if the school is Adath Israel or Young Israel?
All the children are Jewish pupils lean__ 
ing Hebrew.
The pupils of Young Israel don't need us; the families there can pay very 
well.
And also, new blood on the Board of Education is very good for the school."
All 
of this was marked down in the minutes of the meeting.
The meeting was then adjourned.
At the next meeting, Phyllis sat down near me.
Another friend came to sit with U.S.
They both started to talk to me.
"We are going to remove what you said at the last meet., 
ing about the schools from the minutes."
I looked at both of them.
I couldn't believe 
my ears.
I asked them why.
They both replied "Yes, people are against this amalgamatiot 
I said "Everyone wan have her own opinion.
If you, or you, want to contradict me, you 
can speak today, and it will be marked down in the minutes."
"Oh, no; This has to be 
taken out of the minutes."
They looked for the Secretary, Mrs.
Glazer, and they told 
S. 
her "Remove what Mrs.
Garfinkle said at the last meeting from the minutes."
At first I wanted to walk out of the meeting, but I decided to stay to see how far x: 
would go.
A recommendation was made, not to give donations to the Hebi,ew Academy, the n/ 
of the school after eiTolgamation.
When the meeting ended, Phyllis gave me a ride home.
I said to her "I gave a donation to the Hebrew Academy yesterday.
Barouch Pollak called 
me, and I'm very happy to be able to give."
She answered "You are a fool!
The rabbi is 
against this amalgamation.
If you want to be somebody in life, you have to agree with th) 
1 
rabbi."
I said "Which rabbi?
Tell me the name."
She didn't want to tell me, and I nevc 
did find out who this famous rabbi was.
I told her "you can tell your rabbi that he is 
not the boss of my pocket."
She replied "Sternthal formed the school for his own child-
ren.
Only Sternthal should give money to the school, and not you.
The Sternthal girls 
should come and work for the Sisterhood.
But they don't want to come.
They are bigshoter 
I.
-/7- 
.I said "Your husband isn't going to give to the Hebrew Academy?
You have a child 
in the school."
She answered "Leave my husband out of this.
He is not going to listen 
to me."
I said "If even your husband isn't going to listen to you, doh you expect me to' 
listen to you?
Phyllis, it's my pocket, and not yours.
And I'm going to give what 
pleases me.
I believe in Hebrew education, and I'm going to support the Hebrew Academy.
"This stinker called you to give to the Hebrew Academy?"
she asked.
I answered "you and, 
your rabbi are stinkers.
You discriminate against children learning Hebrew because thei 
Jewish 
parents are members of Young Israel Synagogue.
This is a disgrace to the imirm Communli 
I would like to know who this hipocryte rabbi is.
The Sisterhood has already been ruineu 
since we elected you as President last year because no-one else wanted the post.
But wi 
this question of the schools, the Sisterhood will be ruined even further.
Don't think 
f 
you're an expert in deciding who * should learn Hebrew and who should not.
And your rab 
would be better occupied with religion than with politics."
The next day I sent a letter of resignation to the Sisterhood.
Mr.
Sternthal asked 
me many times whu I had resigned.
I never answered him, because I didn't want to take 
part in these intrigues.
The President, Mrs.
Bessie Gorolnik, wrote to me that she didn'.
want to accept my resignation.
It was suppertime at the hotel in Tel Aviv.
I got up and got dressed.
My husband f 
and I went to the dining room.
In the lobby, I saw Mr.
and Mrs.
Syd Gotfried, very nice, 
people.
Mr.
Gotfried was an officer of the Combined Jewish Appeal.
It was at his office' 
in Mohtreal that I paid for our tickets for this trip.
Every day, for the last fifteen 
days, he would look for us to say hello, and to ask how we were.
You could see in his 
eyes that he wanted to say "I'm very sorru, Mrs.
Garfinkle, for what is going on here."
I shook hands with both of them, and I said "Thank you very much for everything."
The next day was our departure date for Montreal.
We arrived at the airport.
Half 
the group went to Spain, and the other half, us included, went to New York, and from New '• 
York to Montreal.
I sat down in the airplane and I started to remember the Jewish life i 
Salonica before the war.
When someone invited people who were not in their milieu, it was traditional to do 
everything in their power to make these people feel welcome.
After 
hallang grown up in thr 
atmosphere, how could I accept the treatment I had received from some oreem of the people oni 
-96— 
this trip?
I couldn't believe that I hadn't opened my mouth and told them "Look here, 
chaverim, if we are so unwelcome here, my husband and I, Madame Chairman should go up to 
Mr.
Garfinkle and say 'Get out of here!'.
This would be better than to be humiliated every!
hour of the day."
One hour into the flight, the stewardess came over.
She was carrying some small 
pillows.
She gave one to my husband.
I told her "If you have something else to do, give 
MB the cushions and I will give them out to everyone."
Mr.
Waxman was sitting near me, 
and I went to give him a pillow.
Suddenly I heard Phyllis Waxman talking to the wife of 
the doctor in our group.
The conversation consisted of gossip about Phyllis Wexler.
I 
didn't want to listen any more.
But Phyllis Waxman told the doctor's wife "Now, ask her 
who Phyllis Wexler is, if you don't believe me."
The doctor's wife didn't want to ask me, 
but Phyllis did.
"Na, Tova, tell her who Phyllis Wexler is, your associate chairman 
in 1967."
"Phyllis Wexler is the most charming lady that I know," I answered.
Phyllis Waxman 
was mad, but I went on.
"You've finished godsiping about me.
Are you now going to 
gossip about Phyllis Wexler?
You do this to everyone who has a bit more education than 
ypu, or is more capable or more knowledgeable than you.
You are scared because we know 
who you are.
These wonderful people believe you because they don't know you.
Madame 
Chairman, the leadership has gone to your head.. I told you to get off my back.
Now I'm' 
telling you to get off Phyllis Wexler's back.
She is a person whom you don't even deserve 
to know."
I gave Mr.
Waxman a cushion, and Phyllis screamed "Don't do what I don't do for 
my husband!"
I left them.
I said to myself "If Thomas Hecht is not sleeping, I will go to talk to 3 
him.
now.
I'll tell him what has been going on on this trip.
He has a good idea, I'm sure, 
but I want to tell him myself."
I went to see him in the First Class section.
I opened 
the curtain, but I said to myself right away "I'm not going to talk to him..E.Iikkaama I'll 
talk to him when the Combined Jewish Appeal asks me for a donation."
Thomas asked me 
if I needed anything, I said "No, thank you," and I went to sing with some wonderful 
young people who were in our group.
I started to think about what I did after I sxttimixtkx got the children settled.
I came to my sister's that night, very tired.
I was happy because the children ha-
homers.
But I felt myself very alone.
When I came in, my sister said "You have a lette 
from Motsa."
I replied "Motsa?
I don't know what that is."
They explained it to me.
"It is a rest home for people who have been gravely ill for a long time.
They go there 
to rest.
But in this place," my sister said, 2beds are hard to find.
But Motsa is in-
viting you to go thdre to rest.
An anonymous person has paid for your stay there.
We 
called you on the phone at Havad Halimoud, but you weren't there.
The invitation is to 
start tomorrow."
I thought to myself that this is what I needed: rest.
The next day at eight in th 
morning I was i on the bus to Motsa.
I went to Jerusalem, changed buses, and arrived at 
Motsa.
Naturally, even the bus tickets had been provided by Motsa.
I came to Motsa, 
and I thought I had arrived in ganeden (paradise).
There were beautiful gardens; the 
grass was like a green velvet carpet.
A nurse came to receive me.
She took me to the 
physician, who examined me and told me I needed complete rest and very good food.
The niurse showed me where to go to eat three times a day.
Twice a day I would 
get a snack.
She showed me where to get milk, as much as I wanted during the day.
"Oh, no," I said, "I don't drink Milk."
She took .me to my room.
It was during the coax{ 
pulsory rest period.
A woman of about 22x or 23 years was getting ready to go to bed -1 
and rest.
The nurse introduced us.
"This is Leah".
Then she asked me "Bouena, R what 
is your name in Hebrew?"
I replied "Tova".
The nurse turned to Leah and she said 
"This is Tova.
She will be your roommate."
Leah spoke Hungarian and a little bit of Hebrew.
From the moment we met, we liked' 
each other.
After we rested, we went to the game room.
The game room had many things 7. 
it to amuse people without tiring them.
Leah introduced me to her friends.
It was vi-
siting hours, but neither Leah nor I had any visitors.
One day I saw a very handsome officer, who had eyes only for us.
I said to Leah, 
"Hey, Leah, this officer og the Jewish Brigade is looking at you a lot."
Leah turned 
around, and recognized her husbands She had not seen him for a very long time.
Leah 
introduced me to him, and he talked& to me a bit in French, a bit in Hebrew.
I told ti 
-98- 
that they shouldn't bother themselves about me, they should go to our room.
Leah's hus-
band wanted to spend the night.in Motsa.
At four in the morning, he had to be in camp, 
and his camp was near Motsa.
He came to say good-bye to his wife because his regiment 
was being moved from Palestine to Europe.. 
I said "You go to our room.
When I come in, the nurse won't notice that there are 
three of us instead ms of two.
I know what it is to have your loved ones go to the bat-
tle."
Leah went to oyr room first.
Her husband followed, and I was the third to go.
When I. came into the room, I saw Leah's husband's uniform on the chair.
I was afraid the 
nurse would see it, sa I took the bedspread from Leah's rommx bed and covered the uniform 
with it.
I told them "The nurse is going to come to check if we are sleeping.
If she 
76) 
sees this uniform, we're going to be in trouble!"
Leah's husband kris spent the night in our room.
The next morning, I was still asleep 
when he was ready to leave.
He woke me up and took me in his arms, and with tears in his 
eyes he thanked me.
I remembered when Chaim went to the battle.
At the time there was 
a song that the people would sing to the soldiers at the train station, to say good-bye.
It was called 'Ayde sto calo'.
This means 'Go with good health'.
I sang the song for 
Leah's husband.
The next day I had visitors, George and Nicola.
They had left the camp at Gaza.
They 
cape with Irene, Nicola's sister, and her husband Henri.
We took pictures together.
We 
spent a very pleasant two hours together.
In Motsa there was usually entertainment for everyone; Newspapers, a library, games, 
radios, singing, music.
There was also very good food, and fresh air.
We were like a big 
family in Motsa.
No-one was alone.
There were often conferences held on very interest-
ing subjects, with very good speakers.
There were beautiful pine trees on the grounds.
From the pine cones we would make birds, each of us ustIng our imagination, and we would 
teach each other.
There was the greatest spirit of camaraderie that could exist in life.
It was exactly what I needed after what I had gone through.
But I never knew (and I still 
don't know, to this day) who had arranged all this for me.
initim left Motsa with many regrets, and Leah was very sorry to see me go.
But she 
was also supposed to qo home 
-99- 
I came back to Tel Aviv very well rested, and ready to find a job.
The next day I 
went to WIZO.
They received me very well.
They asked me if I had any references.
I 
said no, but I sent a telegram to Mr.
Burcard of the International Red Cross in Geneva.
I would get an answer for sure, and the International Red Cross would testify An to what 
I had dote for them.
WIZO gave me the address of a nursery for newborn babies.
They told me "If you don' 
have any references, you will be paid less," They explained the kind of work I would be 
doing.
Working mothersx would bring their babies in the morning and pick them up in the 
evening.
I said "Working with children!
I speak very little Hebrew.
How will I speak - 
with them?"
The lady from WIZO answered "They are babies.
You only have to prepare the 
milk, and feed them, and change them.
You don't have to talk to them."
She advised me 
to present myself at the nursery the next day.
1 
I went home and my sister gave me a telegram from the International Red Cross, signt 
by Mr.
Burcard.
He testified to the work that I had done for the International Red Cros 
in Salonica, and to my background.
The next day I went to the it) nursery.
The director 
received me.
Since I spoke very bad Hebrew, she asked me if I knew Yiddish.
I said no2 
and t gave her the telegram frpm the International Red Cross.
She didn't even look at i 
She just asked me "Are you Sepharady?"
I answered yes.
She replied "I'm very sorry.
We don't hire Sepharady here."
I wanted to spit in her face.
But all I said to her wask 
"The German's don't want the Jews.
The Palestinians don't want the Sepharady."
I thanki 
her for the time she had wasted on me, and I left.
Outside there were a couple of steps.
I sat down on them.
I wanted to cry.
But I 
remembered that I didn't have to cry.
Chaim had recommended singing instead.
I started,' 
to remember that the Jews in the ghetto would ask each other what was new.
Somedne woulk_ 
answer by singing the 'Phtlosophy of Koelb': 
Nada de nada 
Duho Koelb 
Nada de nada 
Todo nada 
Que avantage 
A el hombre 
Con toda son lazeria 
Que bive 
Debacho de el sol 
-100- 
This meanss 
Nothing of Nothing 
Says Koele 
Nothing of nothing 
Everything is nothing 
What advantage to the man 
In all his labcqdr 
Who lives under the sun 
I remembered another aspect of life in the ghetto.
Even with all the miseries, the people 
would sing.
They would curse Hitler with sings.
Now it was my turn to curse Hilter.
Hah, Hitler, Hitler 
Club tb vea oun estremesser 
Qub sea neigro i crouelle 
Si el dio etha oun trouello 
Creo qub tb pella entero 
Non tb decha ni oun pello 
Mich 
This means: 
Oh, Hitler, Hitler 
I. should see you terrified 
In a bad and cruel state 
If God sent.a thunderclap 
I believe you would be without hair 
He won't leave you one hair 
Mich 
Sitting fon the airplane, on the way back to New York, I suddenly remembered that they 
had taken us to see a bomb shelter, miklat.
I knew what this was because we used to have 
them in Salonica during the war.
Our group came in two buses.
The mayor of the city re-in 
ceived us with a delegation.
There was a nurse mikk a very white dress with a lantern in 
her hand, so that when we went down into the shelter, we would be able to see.
There was 
no greenery in this village, unlike every other place we had been to.
There weren't even a 
any sidewalks.
You could feel the misery of the inhabitants, even in the streets.
We finally came to the shelter.
In front of the entrance there was a big hole.
We 
had to jump over it to get to the door of the shelter.
My husband jumped and went in, and 
others did also.
Just as I was ready to jump, I heard someone a speaking kadino.
Two 
ladies from the village were saying, one to the other, "Oh, God, tell them not to go in:" 
I turned around and asked them why we shouldn't enter.
They answered "We are inhabitants 
3.
I 
-101- 
there to make."
Right away my husband and the others came out and told the rest of us not 
to enter.
What the two ladies had told me was confirmed by my husband and the others.
I 
asked the ladies "How can you let the children do that?"
They answered "We came g here 
twenty years ago.
They promised us decent housing, but they did nothing for us.
You can 
imagine how we live.
Come with us, we will show you."
I wanted to go and see.
But a gentleman from our group standing near me told me not 
to go.
I said to the ladies "If the bombs come, you'll have to stay in the dirt.
Won't, 
that be worse?"
One ad of them answered "When we came here twenty years ago, we came wit1 
small children.
These children are now grown.
But more children were born in this vil-
lage.
They are grown now.
When the villages started to build this shelter, these child, 
mei ren asked them to build houses with shelters, instead of just the shelter in the midq 
of the village.
They spent a lot of money on this shelter.
If they had spent just a 
little more, we could have had shelters and houses.
But they wouldn't is listen to us.
We are Sepharadis.
The big children, as ass an act of protest and sabotage, sent the - 
k little children to use the shelter as a toilet.
Look, lady;" she said to me, "look at 
the hole at the entrance.
The mayor has no imagination.
How does he think we are going 
to go in?
Not everyone can jump over the hole.
When they finished the shelter, no-one 
came to inspect it.
They spent so much money.
W They could close this hole with two 
bags of cement."
etd.amect.
The mayor, the nurse, and other members of the delegation I'm sure were a4 aid.
Or 
member admitted that the government had neglected this part of Israel.
I asked him "Is '1 
it because they are Sepharady?"
He didn't answer.
But someone else on the delegation 
answered me: "If you know how to do things better, come to Israel and do them."
I thou9 1 
that this was a very stupid comment.
There followed a discussion about this in our group.
One lady said "These people 
came from Arab countries, and they don't know any better."
I answered "In twenty years 
these people should have received betteraccomodations.
If the 4 government had money f 
the newcomers, it should have first taken care of the old ones.
A little bit less luxtul 
in the houlian, and they would be able to do a little more for the people here."
But t 
had taken us to see, and they didn't want to receive criticism.
It is not good tO 
say, 
-102- 
I started to remember the life of the Sepharady in Salonica: Jewisf life, and Zion-
ist life.
It was very colourful.
I remembered how engagements were celebrated.
First, 
the two families had to agree on what the bride would bring as biankueria (trousseau in 
linen).
Everything would come in sixes, or in twelves, or in twenty-fours if the people 
were very wealthy.
Achougar was the trousseau in clothing; would the bride bring clothes 
only fop winter, or only for summer, or for all four seasons?
The bride and groom would 
hold an open house at the bride's house on the day of the engagement.
The house would be 
full of flowers that friends and neighbours and relatives had sent.
The groom would send 
flowers also, called jerbe.
The flowers went from the floor to the ceilinge.
It was 
very difficult to bring the flowers into the house.
The groom would give the bride a 
ring.
Both sets of parents would give tk presents to the bride.
Fifteen days before the wedding, the dowry would be given.
The groom gave the bride 
a box of jewellery, the quantity and quality of the jewels depending on his financial 
situation.
This gift was called the-coffre.
After the dowry was given, everything that 
the bride would bring was displayed on tables.
A rabbi would come, whose only business 
was to price the goods.
The sum of mp money that the bride brought, and the value of her 
trosseaus, would be marked down in the kitouba (marriage contract).
Relatives and friends were invited.
The tables were removed to make room for music 
and dancing.
On the shabbat before the wedding, after the avdela, there was a big party, 
(almoussama)1 that would go on all night.
It was the biggest party the bride and groom 
could afford.
On the day of the engagement, the bride's mother would send candy favours to all the 
friends and relatives.
All the young relatives would come and help.
X They would all 
participate in the engagement of their cousins.
Before the wedding, the groom's mother 
would make korban (a sacrifice).
She would bake pitas and give them to everyone, neigh-
bours, friends and relatives.
The wealthy people would barbecue a lamb outside, in addi-
tion to the pitas.
They would cook accompanied by traditional songs to the bride and 
j groom at this barbecue.
They would dance and eat, They never forgot to give donations 
For example, they would give to the dispensary Hassid, or to the other one, Pinkas, 
which belonged to the Bikour Kolim, a charitable organization for the sick people of the 
community.
It was considered a great deed to give to sick people, rspecially in times 
of happiness.
The youth of Salonica would dress like other Europeans of the time.
The old people' 
like my Nona, would wear traditional Sepharady clothing.
The women would wear a dress 
like a housecoat, with a very tight bodice and a very full skirt, made of print material 
This dress was called antari.
The antari had a very low decolletage, and underneath the 1 
dress, exposed by the decolletage, a very beautiful white blouse was worn.
This white 
blouse was called Koyar.
On top of the antari there was another dress of heavy silk, a 
material called stofa.
From the shoulder to the bust there were two ribbons of embroider 
done with gold thread, called sayo.
The women would wear a necklace of many strands of 
pearls, calleds yadran.
They would wear a soft bottle green hat called tocado, with a 
train that would come down to their waists.
At the bottom of the train there was a squa 
panel, covered entirely with pearls.
The front of the hat, on the forehead, was embroid, 
ered with pearls.
This part of the hat was called tserekve.
Everyone had the same out-
fit, but the wealthy people had precious pearls, while the others had false ones.
The women would wear bracelets on both wrists, and many many rings.
It was beauti 
ful to see.
If it was during the winter, they would wear a coat lined with fur2 called 
kirim.
The rich people were members of organizations like Matancith Laevionim, or Orphell 
nat Aboav, or Hospital of Hirsch, Loge Masonique, or Orphelinat a Latini.
In every dis-_ 
trict, rich or poor, there was a Zionist organization.
The Keren Hayessod was for the 
wealthy.
All the youth would take part in sports, and they would all go to collect money; 
for the K.K.L.
(Jewish National Fund).
At my house, the biggest holiday was the day of Rabbi Simon Barioka.
There would be' 
a very km big party, like a ball.
There would be music all night.
Relatives and friend, 
-$ 
would all come to be happy with us.
There was confetti, and streamers, and party hats.
All the guests would give donations to their favorite organizations.
XXXXXXX 
I remembered how we used to celebrate purim.
In every district there was a mas-
querade ball.
A month before purim the business would decorate their windows in a purim 
atmosphere.
Every little thing in the windows was made of sugar.
All the families bougi
-104- 
- gifts made of sugar to send as jokes to their relatives and friends.
They sang the tra-
ditional songs of Purim: 
En Pourim Non Bevras Vino 
En pourim non bevras vino 
Ni aborrescas el tino 
Todo el mal qub lb vino 
En los dias de Pourim 
Ven aqui tou zera de loca 
Ati avlar non tb toca 
Y'o por ti ize la orca 
Qub la estrenes 
In Pourim 
Se Vistio la Reina Esther 
Se vistio la reina Esther 
En el primer dia 
Vestido de sou collor 
Qub al oro paressia 
Con grandb manzia 
Sb foub andb el rey 
El rey Kb %a vido venir 
Piedrio sou tino 
Acolor del vino 
Kes to tou venida Esther 
Kes la tou demanda 
Oun coumbit quiero azer 
En la touza caza 
Con toda mi compagna 
I Aman tambien 
There was an operetta composed especially for purim.
It was the story of Aman and 
Esther, and it went like this: 
Esther mirada mi desgracia 
0 vos rogo mi vida salvar 
Reyna djousta yena de gracia 
Delantrb,de vos mb vo a encorvar 
Basta mizeravie tou desgracia.
Por tou vida i tou croueldad 
En la vida en la balanssa 
Yo non pouedo tener,piadad 
Piadad Esther piadad 
Piadad por tou majesta 
Enfaxmax 
Enforcar Aman enforcar 
I oun negro descarancar 
Esther basta, Esther basta 
-105- 
Al pounto qub vengan 
Qub sb lo yeven 
Deploravle, dio de los sielo 
O sient nouestra orassion 
Oh qub maravias qub miraclo 
grub vino azermos el bouen dio 
El y'a mos escapo 
De hombrb malo de oun 
Enemigo de djidios 
Cantar cantemos i cantar 
Baylemos sin quedar 
Aman enforco 
Vino la claridad 
De nouevo reynara Sion 
Con toda sou nacion 
Souvio la alvorada 
.De nouestro corason 
Souvio la alvorada 
De nouestra rekmission 
Cantar cantemos 
Baylar baylemos 
Cantar Cantemos 
Baylar baylemos 
The translation is: 
Aman: Esther, look at my disgrace.
I beg you to save my life.
Quuen 
of justice, full of grace, I bow in front of you.
Esther: Enough: Miserable Your disgrace is because of your life and 
your cruelty.
For this life, as I judge you, I will not have 
pity on you.
Aman: Pity: Esther, pity: Pity from your majesty: 
Esther: Hang, Aman: Hang: We will be rid of a terrible person.
Aman: Esther, it's enough; Esther, it's enough: 
Esther: Come right now, and take him away: 
Aman: Deplorable, God in Heaven: Listen to my prayers: 
(he is taken away) 
Chorus: How marvelous, how miraculous, the Good God has helped us.
He 
helped us be rid of a bad man, an enemy of the Jews.
Sing, everyone, 
We should dance 
Aman is hanged.
All the nations 
There will be a 
sing and sing: 
without stopping.
Night has become 
of Zion will be a 
beautiful morning 
day.
kingdom again.
in our hearts, 
Beautiful in morning with blessings.
Sing, everyone, sing and sing: 
Dance, everyone, dance and dance: 
-1.0671- 
The people whose name was Saragoussi would make purim by themselves.
The Sara-
goussi became a legend for this purim.
There was a synagogue near the King's palace.
All the members of the congregatiom were Saragoussi.
The King gave an order.
Every 
day the shaffer thora must be put in boxes, and these boxes must be paraded around the 
palace every morning.
The Jewish people did this, but the baxe they didn't fill the 
boxes.
They paraded with empty boxes for months and months.
One night, the chamach dreamt 
that he was told to put the shaffer thora in the boxes, because the boxes must be full 
the next morning when they were taken to the palace.
The chamach got up in the middle of 
the night, and he put the shaffer thoras in the boxes.
The next morning at the palace, the Royalm Guard came to open all the boxes.
They 
said that if the boxes were empty, they would kill all the Saragoussi.
The people who 
held the boxes were afraid, because they believed the voxes were empty.
As each box 
was opened, the bearer was very surprised.
Thank God, all the boxes were filled with the 
shaffer thora.
On chabat hagadol (the big shabbat) everyone dressed in new clothing from head to 
toe.
The wealthy people could afford to do this, but to the others, the community gave 
malbich haroumin (dressing the naked).
Everyone would give voluntary donations to this 
cause.
The community had a matzah factory, and the matzah was tasty and well done.
On 
pesach, the wealthy people paid a very high price for the matzah.
The middle class paid 
the cost of the matzah.
The poo r people got the matzah for free.
The sugar for pesach' 
was sold by the community.
It was shaped like small pyramids.
The pesacb seder was 
held with great happiness.
Each person said the passaskiaximmitam passock in Hachon (heb-
rew) and in Ladino.
On chevouotte, one would hear people going for excursions with baskets full of food 
in the middle of the night.
They would call out "We're going to recite the torah."
They' 
would sing the song of Ruth: 
I fOub el dia 
De djousgar a los djoueles 
I foub ambrb en la tierra 
I andouvo varron 
Del bet lekem yeouda 
Por moray en campos de Moa 
-10Y-
El i sou moujer 
I dos sous yos 
I nombrb de el varron Elie Melek 
I nombre de sou moujer Naoumi 
I nombre de dos sous yos 
Elion tefilot Efratin 
I moro ayi comi dies agnos 
Everyone was dressed in white organdy, beautifully embroidered by hand.
In the evening 
there was a ball held outdoors.
All the Zionis* p organizations would hold balls outdoo: 
They called these balls 'La Fete des Fleurs'.
The last ball I went to was given by the 
Tel Aviv Organization.
It was the annual ball, 'La Fete des Fleurs'.
I was wearing 
white organdy with hand embroidery.
(This dress was like a photograph in my memory, and 
I finally made the same dress to wara wear to Ely's Bar Mitvah.)
On Rosh Hashonnah, there was to be a pomegranate on every table, for blessing the 
first fruit of the season.
On soucot, on every balcony, rkk rich or poor, there were 
beautiful soucas.
The smell of the flowers in these soucas filled the city.
On Yom 
Kippour and Rosh Hashonnah, the people didn't buy seats in the synagogue.
Instead, every 
mitsvotte was put on auction, and the money from these auctions on these holy days was 
enough to keep the synagogues going all year.
On Hannoukah, the hanoukia was lit in every house.
Afterwards, we would visit a 
different relative each night, to celebrate Hannoukah.
Everyone gave money to the child 
ren, but it wasn't theirs to keep; it was for pg philanthropic institutions.
The assoc-
iation of Jewish youth, called A.J.J.
(Association des Jeunes Juifs) organized a very bid 
bazaar.
The profits were given to the K.K.L.
(Jewish National Fund) as a Hannoukah pre-.
sent.
Each Jewish woman donated hand-made artifacts to the bazaar.
This baliar consti-
tuted the biggest income of the K.K.L.
On the last day of Hannoukah, the A.J.J.
gave the biggest ball in Salonica.
The 
ball took place in the hall of Matenoth Laevionim.
Before the ball started, the big ha-
noukia was still unlit.
The privilege of lighting the candles was put on auction.
The 
K.K.L.
would receive a large income from the auction of the lighting of the hanoukia.
1 
big companies and the merchants were members of Keren Hayissod.
The committee of Keren 
Hayissod decided how much each of its members would give as a donation.
This money was 
used to help build Israel.
-log-
On the anniversary of the day of the Balfour Declaration, every Jew was in the 
streets, waving Israeli flags.
The band of the Macaby would play.
Jaques Sarfatty was 
the leader of the band.
When he led the band, he would dance along to the music.
The 
band was in the front of tbe parade.
In the back were the great numbers of the scouts 
of the Macaby.
In the evening, there was a ball in each district, with its own people, 
because there was always a division of class in Salonica.
After having grown up in this atmosphere, it was not easy for me to swallow the IMAM 
treatment that I had received from some of my companions on this trip to Israel.
I start-
ed to think of how I finally found a job in Tel Aviv.
A few days after the incident at the nursery school, I presented myself at a haute 
couture clothing store.
Because I worked very well with my hands, I was sure that I 
could work there.
But these people wanted only .a designer.
Coming from the mountains, 
I didn't even know what the current styles were.
One day, I arrived at my sister's house, desperate because I had found no jobs.
My sir 
sister said to me, "You have a letter from the Sokonouth of Jerusalem."
They called me 
for an interview.
The next day I went to Jerusalem, to the Sokonouth and introduced 
myself.
A gentleman invited me into his office.
He started to tale.
As I listened, I 
thought that this man was machia for me.
He said "As soon as the war is over, we are 
going to send a group to Greece, to help people.
The name of the group is P.J.C.R.A.
In Hebrew this is plougot assad (rescue organization).
It is attached to U.N.R.R.A.
Of course, first we must prepare you.
You must procure a passport for yourself.
Then, 
you will go to courses on Greek Judaism.
Lastly, you will learn to be a dietitian for 
a refugee camp.
These courses will take place in Pitak Tikva."
My first task was to look immediately for a passport.
I went to the Greek consulate 
with the refugee book that I had received in Syria.
There was a fiery nice gentleman there.
He said he would let me speak to the Consul.
The Consul said to me, "You are Bouena Sarfatty.
Yet you speak beautiful French."
I wondered how he knew I spoke French.
I sai es, spec r 
Half an hour later, my passport was ready.
I we left the consulate and went to Rachel 
Yanait ben Zvi.
I was always welcome there.
She and her husband were people made of 
gold, 
When I came in, no-one was home except for a lady working in the garden.
She said 
"Shalom, Tova."
I said "Wha are you?"
She answered "I am Rachel Nakmouly, your brother 
in-law's niece."
She made me some tea.
She asked me if I knew Jerusalem.
I said no.
She said "My brother-in-law is going to come to take me on a tour of Jerusalem."
In an 
hour, her brother-in-law arrived.
He was a very nice man in his forties.
He took me tc 
Cotel Maravi, the Wailing Wall.
It was exactly three months since my dream about my 
Nona.
I told him about my dream.
He said "I left 4 job today, just to bring you to t' 
Cotel Maravi.
Since the day we heard that you had escaped, I dreamt that an old lady ca 
dressed in clothes I had never seem before, to beg and push and order me to take you to 
the Cotel Maravi, to put oil in the wall."
A few weeks later, I went to Pitak Tikva for the courses.
At the courses in Pitak 
Tikva, I met my husband.
He was very sweet to me.
He would explain in French whatever 
I didn't understand in Hebrew.
The day to go to Cairo soon arrived.
The headquarters 
of the U.N.R.R.A.
was in Cairo.
We arrived in Cairo, and they took us to Meady, the best district in the city.
The 
gave us beds, and tents, and linen.
The only building existing so far was the officers 
mess and the dining room.
The men were given driving lessons.
They supplied us with 
trucks and uniforms.
One day, I decided to go to SE Alexandria, to look for the brother-in-law of Dr.
A 
rio, my family's pbysician in Salonica.
At the time of the war, Dr.
Amario couldn't' 
support his family by purchases on the black market.
We supported them.
He gave us a 
cheque, to be honoured by his brother-in-law in Alexandria.
Dr.
Amario's wife had a 
large inheritence in SE Alexandria.
The family ol§ Dr.
Amario had been deported, and we 
never heard of them again.
The brother-in-law was not hard to find, because the fami 
was very wealthy, and well-known.
They were wholesale druggists.
I went there and intl.
duced myself.
I presented the cheque, and the man refused to pay me.
I came back to Cairo.
During the day, we would go to visit the museums, very intel 
estina places.
In the evenina.
sometimes.
by moonlight.
we would go to see the pyramid 
117 
A few days after I returned, I started to look for Mr.
Soule, the Lebanese Consul from 
Salonica.
I wanted to meet him and to give him news of his wifee and children.
I went 
to many embassies, but no-one knew where he was.
The evening before I was to leave Cairo 
for Greece, someone in the camp asked me if I had found Mr.
Soule.
He told me that he 
had his address, and he would take me there right away.
In ten minutes, I was in the 
truck.
X We wit went to where the trucks usually stopped in Cairo, a kind of meeting 
place.