% pages 1 to 34

It took me six years to give my impressions of a trip I made  to Israel in the year 1969, with members of the Jewish community of Montreal.
I have tried very hard to forget these impressions. 
There are parts of my life that I always try very hard to forget. 
Disasters of the second World War, nazis, cruelty, starvation, persecution, I try my best always to forget. But in times of difficulty, you remember your life in the fullest detail.
At these tax moments, you feel that your world has ended; only vain people would force you to remember the past miseries. 
After the combined Jewish Appeal of Montreal honoured me, in 1968, with the gift of a silver plate from the Women's Federation, I received a letter. I was invited to visit Israel with a group consisting of the leaders of the community, to see with our own eyes the needs of our brothers in Israel.
I showed the letter to my husband. He was very enthusiastic about accepting the invitation.
But I was not very enthusiastic. All of a sudden I had a premonition telling me not to go, as if I knew that this trip would bring to life my memories, memories that I was never willing to think about.
On the other hand, I said to myself, "I'm crazy! Why not go? Why have these ideas?"
My grandmother used to say \ladino{'kien no tiene ventoura no devia. de nacer'} - if someone has no luck, he shouldn't have been born. 
The night before our departure, my so-called friend, Phyllis Waxman, who had just been named Chairman of the Year campaign of the Combined Jewish Appeal of 1969, telephoned me. 

"I just learned that you are going to Israel with a Mission." 

"You just found out today?" I asked her. 

“Oh, only today,” she answered.

"You know, you are a liar," I said.
"First, I went to Ottawa to get a passport with your son.
Second, I gave my passport photos to you, to be signed by your husband, who is a Justice of the Peace.
Third, you did all that was in your power to prevent us from going on this trip.
You terrified everyone in the Combined 
Jewish Appeal by saying that my husband is a very sick person who cannot walk.
And you terrorized everyone who is in the Mission.
Fourth, when I paid Mr Syd Gotfried, officer of the Combined Jewish Appeal, for my trip, you were in the office, and you called me afterwards on the telephone.
You screamed at me that one had to  walk on this trip to Israel, and my husband cannot walk.
I told you then that my husband walks better than you, he is more intelligent and more educated than you, and you know your education leaves something to be desired.
My husband's hands tremble, only the hands." 
After I had said this., Phyllis Waxman tried to excuse herself, and to say that she didn't know what I was talking about at the same time.
"Don't worry about us," I said. "The neurologist knows better than you." 
We ended our conversation on apparently good terms. 
The day of our departure, my husband and I arrived at Dorval airport and went into the waiting room.
As soon as I entered, I saw Phyllis, and I learned that she herself was supposed to come with us on this trip.
After our unpleasant conversation of the night before, I thought to myself that we were going to spend fifteen days together and I had better say hello.
If I don't speak to her now, I won't be able to speak to her during the entire trip.
My husband and I approached her, and instead of answering our greeting, she said "Take him away from here!" in a very nasty way. 
"Drag me!“ I answered.
They were taking photographs, and I didn't move from there. 
They took a picture of Phyllis and her daughter, my husband and me. 
"This is your friend?" my husband asked me later. 
We took an airplane to New York. In the plane, I sat as if I was nailed to the seat, and I began to remember my father. 
I don't remember my father well, since he died when I was very young. But my brother Eliaou, who had raised us like a father, used to talk a lot about him.
One day, Eliaou told us something that my father had said once, that he wanted to pass on to the family. 
"Never be scared of the truth in life," he had said "Hardships, everything, will pass.
In times of difficulty, you think the world is ending.
But you don't have to be scared.
The only  time you have reason to fear is if someone close to you all of a sudden, night to day, becomes rich and buys a dozen new dresses to  impress others.
You must fear a person like this.
Changing from one day to the next, they develop an insane superiority complex.
And these people don't know how to stop hurting others.
The more they have, the more they want." 
My father came from a very ancient Sephardic family.
My father used to wear a ring with a dark green stone.
There were Roman, Latin and Hebrew letters, and a small picture of a king carved into the stone.
This ring came from our ancestors, and it was passed down for many generations from father to son. Our family name is Sarfatty.
In Hebrew, Sarfatty means "French". It means that the family came from the Pyrenees, from the French and Spanish side.
They came to Salonica, like all the Sephardim, in the time of the Spanish Inquisition.
They lived under the Turkish regime for many generations.
When the Greeks came to Salonica, a large majority of the Jews became Greek citizens, and a small minority remained Spanish.
The Spanish people used to have certain privileges during Hitler's regime.
The last Jews to leave Salonica during the war were sent to Germany by the nazis, and after a while were sent to Spain, and from there to Palestine to a Greek camp in Gaza.
Salonica was a Jewish city with all the Jewish traditions. 
All services were provided by the community. The community was our government; it presided over marriages, deaths, circumcisions, and even legal judgements.
The name for the legal court was Bet Din.
The Bet Din made the final decisions in disputes between Jewish people. 
If two people wished to get married, the community would send the Rabbi of their Khila.
The city was divided into sections, each section with its own Khila.
People would pay an amount for their weddings according to their positions in life.
We had synagogues which were supported by donations.
People would give donations for the honour of reading the Torah.
The name for these donations was Mitzvoth.
People would bid for the honour, and the highest bidder would get to read the Torah.
During Yom Kippur and Rosh Hashonah, the synagogue would collect enough money to keep going for the rest of the year.
My father was one of the leaders of the synagogue, and later, this honour was passed on to my brother, who knew Hebrew very well and had a beautiful singing voice.
He used to help the Hazan. 
This was a great honour for our family. 
We arrived in New York and boarded the El Al airplane to go to Israel.
We were soon made to get off the airplane, since there  was something wrong with the engine.
We sat in the lounge and waited to find out if we were going to get another airplane or if we were going to spend the night in a Hotel.
My husband wanted to go to the rest room, like many others who were in the lounge.
A few minutes later, a bus came to take us to a hotel for the night.
The people who had gone to the rest room before my husband came back, but he had not returned yet.
The bus waited for about two or three minutes.
All of a sudden, Phyllis Waxman said to me: "I told you, this man can't walk".
I didn't answer her.
I didn't want to start the fight again.
The only thing that came to my mind was: "My father was right when he said to fear people who bought a dozen new dresses.
They don't know how to stop hurting others.
And they are different because they have an insane superiority complex".
While I was thinking this, Phyllis said to me: "Your husband is an invalid". 
"You are a \ladino{'bevedera de sangre}{(a vampire)}", I replied in Ladino.

"My husband is no invalid.
My husband has a tremor in his hands only.
The neurologist told us before we accepted this trip, "If you don't go, don't come to me any more.
Your husband is a very intelligent man.
He walks like a soldier.
I don't see any reason why you should not go.
If this woman doesn't want you to go, it is not because of your husband.
She wants to be rid of you, because you are more capable than she is.
She can't be rid of you, so she talks about your husband." 

We spent the night in the hotel in New York, and the next day we flew to Tel Aviv.
When we arrived in Tel Aviv, there were two buses waiting for us, one red and one blue.
Thank God I was in a the blue bus and not in the red one with Phyllis Waxman. 
We were introduced to our guide and our chauffeur.
We were  going to spend fifteen days together and see all of Israel.
The bus went along the road to Jerusalem, and we soon came to our hotel, which was only half completed.
The workers had left the elevator on the fourth or fifth floor.
My husband went up and down like anything, but I was dead tired by the time I walked upstairs! 
After noon, we went to the Old City of Jerusalem, and to the Wailing Wall, Cotel Maraul?? in Hebrew.
Coming back, we saw some murals.
I asked the guide if there was a school in the area named Havad Hallimoud. 

"Don't dare ask me another question.
I have orders not to answer you," he replied in a nasty way.

I looked in his eyes, but I didn't answer. 
We went to see a miniature mosque that the Israelis had made, in Old Jerusalem.
A great artist must have made it, it was so perfect in all its detail, and the gold colours were so beautiful.
I was filled with admiration, and I enjoyed myself very much.
The guide, who was from the other bus, was explaining all the details to us.
He asked if we had any questions.
I asked a question, and his  answer was very sour.
"Don't have the chutpah to ask me any questions.
I have orders not to answer you," he said.
And this one, too, I looked, in the eye.

"As soon as we arrived in Jerusalem," I thought to myself, "they gave orders to the guides to annoy me.
They are hoping that we will leave the group and go off on our own, because Phyllis wants to get rid of me.
I think they know me very little!
It was  in very poor taste for the guides to talk to me in this way."

We went to our bus, and the guide gave us orders.
We were to sit one seat back the next day, and the people in the back were to move to the front, so that everyone would have a chance to sit in good seats.
I sat down in the bus, and I felt like someone had given me a good beating. And like this I started to remember my life. 

% War

It was Pesach, 1941.
I was 23.
Tia Donna ('Tia' means "aunt'  in Ladino), my father's sister, and I were alone in the house.
Regina and Eliaou, my sister and brother, had gone downstairs to visit neighbours, friends of the family for many years.
We had not been living in this house for very long.
We came to this house when the war with the Italians started, because the family home did not have a bomb shelter, and this house had a very good one
We were standing at the door of the balcony.
The were closed.
We could see outside, but people couldn't see in.
We were watching the Germans coming with tanks, and the Greek soldiers retreating.
They were taking the soldiers to the schoolyard of Alliance Israelite Universell by the hundreds. 
Tia Donna was widowed when her son Samuel was five years old. 
When the boy turned nineteen, he went to do his compulsory military service.
Tia Donna had not received one letter since the day Samuel left.
A few months after Samuel left, Chaim, my fiancee, was drafted.
I hadn't received any letters from him either.
We were out the window to see if we knew any of the retreating soldiers.
Tia Donna asked me if we were going to go to the Soupe Populaire. 
I answered that now they needed us more than ever.
The Soupe Populaire was organized by the Matanoth Laevionim.
They gave food to  poor children and to the children of soldiers.
The Matanoth Laevionim was founded by a large, ancient Sephardic family, the Revak. 
The Revaks were wealthy, and they were very dedicated, philanthropic people.
The life of the Revaks was Matanoth Laevionim, and looking after poor people.
They had a very large building with a hall, and 
in this hall were held the weddings of the very wealthy, who could pay well.
There were balls given on occasion by rich organizations that could afford to rent the hall.
On Yom Kippur and Rosh Hashonnah services were held in the hall.
All the income from these activities was used for Matanoth Laevionim, and donations came without 
any difficulty.
Regina, my sister, Tia Donna, Sara Trabou, a very good 
friend of mine, and I used to volunteer to supervise the meals.
At seven o'clock in the morning we would go to make sure that the food was of the best quality, and the bread was of the measure of Matanoth Laevionim. 
Tia Donna and I were watching the retreating soldiers, and it was the most depressing sight of my life. Some of the soldiers in the ranks were crying.
Others couldn't walk any more. Others were wounded and in pain.
It was a very dark tableau; there was silence in the house. Tia Donna broke the silence. 
"With this good deed (zakoud) that we will do tomorrow, God is going to help Samuel and Chaim," she said.
She had not finished saying this when a soldier escaped from the ranks.
He was heading for our door.
I went down the stairs and spoke to him from afar. 
"Come in, come in to my apartment." 
"It's me, Chaim, it's me!” I  heard a very familiar voice say.
He came upstairs.
Tia Donna, a woman who never lost her courage, had the 
bath water warming before we got upstairs.
She took Chaim's uniform and he put on pajamas until the bath was ready.
She put the khaki clothes in a laundry sack, made a parcel and tied it well, and threw it into the yard as far away as she could.
Chaim got clean, very clean. 
Tia Donna rubbed his back very well. 
Two hours later, we could see only German tanks in the street.
But the brave Tia Donna went to Chaim's mother to tell her the news.
There were no streetcars, no taxis.
She walked. 
When we got up the next morning, the yard was full of parcels with uniforms inside.
Everyone had copied Tia Donna's idea; every apartment 
had soldiers hiding inside. 
In two days, the situation had returned to normal, but the Jewish people were very scared.
On the third day it was decreed that all Jews were to bring their radios and their dogs to Gestapo headquarters.
The Jews who worked in the port of Salonica were sent away the day after the Germans came.
Jews who worked for the railways were immediately 
fired, as were those working in the tobacco industry, The working people 
were the first to realize that they were Jews. 
The International Red Cross started to organize the distribution of 
milk to children in the working districts. Many people had no food, and 
they were bloated from starvation. My Soupe Populaire was in Regie Vardar, a working class Jewish district. The people there had lost their 
jobs, Many were dying, and we would see wheelbarrows with bodies piled 
onto them all the time. The community had no more horse-drawn 
carriages for funeral services. All of this was misery to see. 
Sarah Trabou, Regina and I used to prepare milk from the condensed 
milk provided by the Red Cross. Each mother would come with a bottle, 
and we would give different amounts, according to what was marked on 
each mother's card. 
The Germans next decreed that Jewish people were forbidden to live 
in certain districts. Mitropoleos Street, where we lived, was 
in the prohibited zone. We found a house in the ghetto zone, on Singrou 
Street, near the synagogue Monasterlis, and not too far from the Regie 
Verdar. We moved on Friday. On the next day, on the Sabbath, every Jew 
between the ages of fifteen and sixty-five was ordered to go to a large 
square near the port of Salonica. This was half a day of mourning for us. 

When the Jews were allowed to leave the square, panic started. Everybody started to run, to be the first out of the square. But this panic 
lasted only one day, and everything soon came to order. Chaim and Eliaou 
were looking for ways to escape, but it was difficult, even if one had 
a lot of money. One had to know exactly the right person to help one 
escape, and we were afraid to talk to strangers. 

One day, as we were going to the Soupe Populaire, we saw some
Polish soldiers, prisoners of war. They were repairing the road. Regina 
saw that one of them was wearing a Mezuzah. Tia Donna said that the 
young man was very sick, that he had a fever. But we could do nothing 
about it. In the evening the whole family discussed what we could possibly 
do for this boy. We concluded that we could do nothing with the Germans standing guard. Next day, when we were going to the Soupe Populaire,
there was a fire near where the soldiers were working. The fire engine 
came. In the confusion, Regina took the soldier to our home, without 
anyone noticing, not even us. When we arrived at the Soupe Populaire, we 
realized that Regina was not with us. Tia Donna thought that she had 
stayed to watch the fire. 
When we came home later, we realized what Regina had done. The 
great Tia Donna warmed the bath, and washed the soldier very well. She 
fed him good soup, and put cold compresses on his head, since he was 
suffering from Malaria. But we didn't know what to do with his uniform. 
The bell rang. It was Sarah Trabou wanting to find out what had happened 
to Regina. We told her, and we started to discuss what we were going to 
do with the clothing. We decided that Sara would take half the clothes 
in her shopping bag. Her father would know what to do with them, she 
had confidence in him. After a half hour, Sarah's father came and 
the rest of the clothes.

When Eliaou came home, we discussed with him what we were to do with 
the soldier. Eliaou knew of an organization in Salonica that looked 
after soldiers who escaped from prisoner of war camps. But we were 
scared of telling anyone that we had an English soldier in the house. 
Tia Donna looked after him as if he were her own son. She used to sleep 
in the same room with him. Every three hours she gave him quinine, which' 
was very difficult to obtain. Eliaou would buy the quinine on the black 
market with gold. We got milk for the soldier from the Red Cross. 
We used to hear Tia Donna in the middle of the night, breaking ice from 
the ice box to put on the lips of the sick soldier. 

The real problems started when the soldier died. 
"We thought that getting rid of the uniform was a big tragedy," 
Eliaou said. "Now what are we going to do with the body?" We decided 
to go to the Jewish Community and to talk with Alphonse Levy. He was 
one of the bigger leaders of the Jewish Community. This was to be my job.
As soon as I arrived at Alphonse Levy's, I saw Mr. Albala, with 
chevrons on. He was a member of the Jewish Police. The Jewish Police 
were worse than the Germans. I asked to speak to Alphonse, who was a 
very good person, but then I saw him from afar, and I ran over to talk 
to him. 
"We found a man on our doorstep," I told him. "We thought he had 
just fainted, but he died soon after. We have to bury him, but he has 
no identification papers!" 
"How do you know he is Jewish?" was the only thing Alphonse asked. 
"Tia Donna saw!" I replied. 
Alphonse provided me with the necessary papers, and he sent two 
men to help, 'embagnadores', people who washed the dead. He also gave 
us a coffin, and we took a taxi home. We gave the embagnadores some 
bread, and they helped us take the coffin downstairs. We needed a carriage, but the community had no more carriages. Sarah Trabou lived in 
my ghetto on a street where there were many stores selling Lemons and 
oranges. Sarah's father was a wealthy Zionist, and he was always 
willing to help. He went to one of the stores and bought a carriage. 
Nobody wanted to sell a carriage for drakmes; everybody wanted gold. 
Here in Montreal, if people see people like us, Trabou and Sarfatty, 
they would ask right away why we did something like this. They would 
start an investigation, and soon, the whole city would know. But for 
us, it was a very natural thing to do. 
Someone had to push the carriage to the cemetery. Regina, Sara 
Trabou, Tia Donna and I pushed this carriage. There were usually some 
poor people in the cemetery. We took bread with us, to 
we would find someone to help us bury the body. After the burial, 
we left the carriage for those who helped us, as payment. 
(I saw Alphonse Levy again after the war, and before I could 
start to talk about this particular incident, he said "don't say 
a thing. I know it was an English boy." But people like Alphonse Levy 
you don't find every day.) 
When Eliaou and Chaim came home, we told them what had happened. 
Reyna, our cleaning woman, couldn't stop crying. She was sorry that 
Tia Donna had looked after the soldier, and not her; her conscience was 
bothering her. Eliaou and Chaim and Mr. Trabou went to the synagogue 
to say kaddish for eight days. After the eighth day, we went again to 
distribute the milk. Everybody came as usual except the mothers of 
the Baron de Hirsch district, a Jewish working class district. We were 
supposed to prepare the tables in the dining room at eleven o'clock, 
and the children were to come to eat at noon. But since the Baron de 
Hirsch people didn't come, the containers were full of milk. We decided 
to call the Red Cross to find out what to do with the milk. I spoke 
Mr. Bekard, the Red Cross delegate in Salonica, who insisted that we 
not leave the children of Baron de Hirsch without milk. He sent a car, 
and I went with the driver to distribute the milk. 

When we were still quite far away, the driver cried, 
"Look, look!". There were high walls made of wood all around the 
trict. In the center of the district was a tower; at the top of the 
tower were German soldiers with a machine gun. We entered the district through large doors in the wall. There were Jewish Police all around. It was the first day that the Jewish Police, supervised by the Germans, refused to let anyone leave the Baron de Hirsch district. 
Hasson, the chief of Jewish Police, came over to talk to me. I thought 
to myself, I came in, but they will never let me leave again. I had the 
flag of the International Red Cross on my car, and this gave me some consolation. 

They took me to the Soupe Populaire in the school. We brought in 
all the containers of milk. With much anger in his voice, Hasson 
said to me "Drink!" He gave me a mug full of milk, and I drank. As 
soon as I finished, he poured some more. I couldn't drink any more. I 
was vomiting and drinking, and my dress was soaked with milk. I was 
thinking of all the babies who would be without milk on account of this 
brute. I couldn't drink any more, and I was desperate. When Hasson saw 
this, he took the mug from me and forced me to drink until my gums were 
bleeding. He kept pushing the mug into my mouth. He pushed some more, 
and my teeth began to shake. My white uniform was covered with milk and 
blood. Hassan began to beat me on the face. When I was exhausted, he 
decided that I should distribute the left over milk. 
Hasson was from a very modest family. His sister worked as a secretary in the Jewish Community, and she was always in contact with the 
bigger Jewish personalities of the city. She wanted her brother to be 
as rich as the people for whom she was working. She placed Hasson in 
the Italian school, an outstanding school in an outstanding milieu. She 
worked all her life to give him an excellent education and environment. 
But Hasson was not socially accepted by his schoolmates. 
When Hasson was in secondary school, he became a member of the Fifth 
Column. His job dealt with the Jewish question. Since his sister 
worked for the Jewish Community, Hasson knew exactly what was going on 
there. He had a list with an evaluation of every person in the community. He planned how he was going to become wealthy, and how he was going to acquire the fortune of the richest family in Salonica. He also had a List of all the girls who had refused to date him. I was one of 
the girls on that list. 
The Jewish Police was organized by the Grand Rabbi, Koretz, who was 
born somewhere near Russia. He was the first Grand Rabbi in the history 
of Salonica who was not Sephardic. When the Germans came to Salonica, 
Koretz was sent to Vienna to receive training on how to treat the Jews. 
Albala was a nobody who came from another country. Nobody knew him, 
and I don't know his history at all. 
Thanks to the driver from the International Red Cross, they let 
me leave Baron de Hirsch. You can imagine that by the time I came  
home Tia Donna and Regina were very worried. In the evening there was 
a family conference, and they all came to my room, since I was in bed. 
It was the same problem again -- how to escape. But we couldn't see 
how. Chaim said, "the first thing tomorrow, we will be married. As a 
matter of fact, I made the arrangements today." We had been on the list 
of the community a long time before. By the time we finished talking 
about how to meet, it sounded as if we were going to elope. 
Suddenly, the doorbell rang. We were very scared. A man whom we 
had never seen before, a non-Jew, came in, and he said, "Bouena (That's 
me), do not sleep in this house tonight." He told me to go outside the 
ghetto. I went to another ghetto, in another district, where friends 
of ours, the Arditis, were living. They were very good friends of the 
family. They were surprised and scared to see me arrive at that time of 
the night. They didn't ask me too many questions. The only thing 
I said was "It's too late to go home. I want to sleep here 
with you." 
The house of the Arditis was not very far from the Midrach. The 
Midrach was a small synagogue where rabbinical students came to pray. 
At the time of the rendez-vous I had with Chaim, the rabbi was supposed 
to be there, to give us the marriage blessing. I went there from the 
Arditis. When I arrived, poor Chaim was there, dying from a German bullet. 
The only thing I can remember him telling me was "Don't cry. Just 
sing and sing, only that can console you." Two Germans came and took me 
away in a closed truck. I only found out where I was the next day--in 
a prison for criminals, Pavlo Mela. I had a guard staying with me at 
the door. The guard would take me to the interrogation room. She pushed 
me, but only on the first day. After that, she wasn't cruel to me. She 
Was an interpreter, and spoke very good Greek. We started to be a little 
friendly. She used to talk, and then suddenly she would retreat and say
can't talk to you. I have orders not to talk to you." Just like the guides here 
in Israel, who have orders not to me. 
One day I wanted to be friendly with the guard, and I told her that 
her skirt was very long. I told her to bring me needle and thread, and 
I would fix it for her. I said, "Look, I can make a blouse for you, 
You go to this place, they have the best material in Salonica." The owner 
of the place was a friend of my family. I sent the guard there so 
that these people (who were non-Jewish Greeks) would know that I am alive, 
and where I am, and maybe would be able to save me from this hell. 
I made the blouse and gave it to her, some days passed, and I was 
still hoping that these people would be able to help me. Interrogations 
were held in the prison at any time of day or night. I was awoken 
suddenly one night, and my guard told me that a superior officer wanted 
to question me. It was the same song all the time, and very unpleasant. 
They always wanted me to give them the names of the saboteurs and the 
members of the underground. They would make me stand straight against 
the wall for half an hour without talking. After half an hour they 
would tell me to raise my foot, and they would whip me with a leather 
whip on the sole of my foot if I refused to answer questions. 
The guard took me. As soon as she saw the officer, she shoved me, 
to show him that she didn't treat me well. The officer complimented her. 
We came to the interrogation room, and the officer told my guard that 
she would translate his questions and my answers. The held the 
leather whip in his hand; I was very familiar with it. The officer asked 
me who the chief of the underground was. Just as he finished asking, 
he took a handkerchief from his pocket. He put the handkerchief 
over the mouth of the guard. I don't know what was in it, but the 
guard fell asleep. He took the clothes off the girl and told me I'd 
better hurry up, because I was standing there like a statue. He tied
her up, closed the door, and he and I, wearing the guard's uniform, 
passed the guards outside like two. lovers. The officer was one of the 
organizers of the partisans. The sentinel saluted us. The sight of a 
German uniform was enough for him. Even after the guard and I had 
become friends, she pushed me when she saw a German uniform, too, 
without knowing who was in it. 
You guides, when you hear the name of Chairman, you think that it 
is a very important personality. But my grandmother used to recite a 
proverb in Ladino: "de casta que vengan los reyes para que sean imperadores." 
- to be an emperor, you have to have royal blood. 
We started to walk. The officer took me to a house, I don't know 
where. A nice suit was ready for me for the occasion, and an iden-
tity card and papers to get me to Athens. The papers were in the name 
of the daughter of the chief of the Fifth Column. I also had one small 
valise. The officer took me to the station. When we arrived, we saw 
two Jewish Police going on the train and checking to see if there were 
any Jewish people using false names to go to Athens. Right away, the 
officer asked me, "Do they know you?" "Of course, I know them. They 
are the brothers Amareo." The only answer the officer gave to me was 
"I regret very much, but you can't go to Athens tonight." He took me 
back to the house. 
The officer told me what had happened to Sara Trabou. She went to 
work in my place to Baron de Hirsch, with the assurance of the Red Cross 
that she would not get the same treatment as I did from Hasson. But Sara 
Trabou was on the same list that I was on. Hasson was very good with 
Sarah, despite this. 
Sarah lived in a separate bungalow, with her mother and father and 
younger brother. One day Hasson came to Sarah's father. Tradition had 
it that the groom had to ask the father of the bride for her hand in 
marriage. But in this case, Hasson went to ask Mr. Trabou in the name 
of his brother. Sarah's father was very scared. He told Hasson he
would have to ask Sarah. In any other circumstance, the father would 
have said no right away. Hasson said,"You're going to say yes. If not, 
all your family is going to be killed today." 
When Sarah finished the distribution of milk, she saw Hasson's car 
all decorated with flowers. This was seen at every wedding in Salonica. 
When she arrived home, she saw a wedding gown on her bed. At this time, 
there were not too many Jews left in Baron de Hirsch, except for the 
Jewish Police. But there was one Rabbi still there, who would 
give them the kidouchim. Sarah said she did not want to marry Hasson's 
brother. Hasson answered "The Germans are going to start with your 
little brother, and then your father and your mother. First you will 
see them die. After that, it will be your turn." The wedding took place. 
The Hassons' car went around Baron de Hirsch several times. Sarah 
was wearing her bridal gown. They told her to wave. Sarah did everything that Hasson asked. The next day, Hasson wanted to send his brother and Sarah to a very good camp for privileged people in Germany. 
Sarah refused: "I prefer to go to Poland, to Auschwitz, with my family." 
We never found out what became of this family. We only knew that the 
girls on Hasson's list had the worst fate and the worst disgrace in the 
world. 
The day after the officer told me this story, I went to my good 
friend Georgette Modiano. Georgette was doing volunteer work in the 
office of the International Red Cross. Her husband Daniel was out of 
town. I spent two nights with her. Her house was located in the district 
where I was born and grew up, and I never even stuck my nose out 
the window. I was scared even to see people who knew me. The only 
Jewish people in the street were the Jewish Police. 
Daniel came home, and they sent me to Dr. Scouros in the same district. 
Dr. Scouros had just come out of jail. He had had an Australian soldier hiding in his house. The Germans found out, and they took the soldier, and Dr. Scouros. When he came out of jail, he was deaf from the bad treatment the German had given him. 
I stayed there a few days, and then they sent me to Mrs. Soula, in 
the city of Verea. Daniel took me to the bus. I arrived in Verea at 
the address they had given me, Platia Elies. The owner of the house 
was Dr. Moratoglou. The house was a duplex. Dr. Moratoglou lived upstairs with his family. Mrs. Soula lived downstairs, with a boy 6 or 7 years old, an old grandmother, and the housekeeper. Mrs. Soula's little daughter was in Salonica with Mrs. Soula's mother. Mrs. Soula's husband 
was in Cairo. He was the Lebanese consul in Salonica, and he had escaped to Cairo. Dr. Noratoglou was the chief of the Fifth Column. But no-one ever found out that I was a Jew. 

The bus stopped and we were at our hotel in Jerusalem. We had 
dinner, and they called us to go to the reception room. We went, and 
they gave everyone but me sun hats. They gave us whiskey, which was a 
present of Seagram's of Montreal. I took a whiskey. I had my whiskey in 
my hand, and before I could take a drink, the chairman called me shicor. 
(drunk). For the rest of the trip she called me shicor. Not once in 
the 15 days did she call me by my name. I never answered her. I remembered my father and what he said before he died. "Stay away from people 
like this." I remembered what Chaim had said, and I sang and I danced, 
but believe me, I wanted to scream. My heart was full of tears. 
The next day, we were to go to the Dead Sea. We sat down in the 
bus, one seat behind where we had sat the day before. Beside me sat a 4 
young couple, Mr. and Mrs. X. All of a sudden, the man jumped up. 
"Mrs. Garfinkle, go to the seat in the back." We went to sit in the 
back, and I started to remember. 
Mrs. Soula was a very modern person, very well dressed, very humanitarian, and very well known in Verea. The old lady that lived with her asked me very nicely why I had come to Verea. I answered that I had come to find a job. The old lady discouraged me about finding a 
job in Verea. I was very welcome there, and had as much food as I wanted. The little boy took me to the movies. One day Mrs. Soula asked me if I wanted to go with her to the Soupe Populaire, similar to the one we had in Salonica. But in Salonica we used to say amotsi lekem min haharetz. Here, everyone made the sign of the cross, and I did too. 
But I made my cross upside down. At the end of the meal we used to say 
et a mazon, the traditional Jewish blessing after a meal. One 
lady saw me making the cross upside down. At the end of the meal, she 
stayed very close in front of me so that no-one else would see me do it 
wrong. As soon as she finished making her cross, she called me into the 
office. She showed me how to do it right, and she said "don't worry 
about me”.
We went out to the dining room. The children were singing the song of the farmers ( 0 Jeorgos): 
Acoma glico kalama 
Me to ayeleno 
Olos photos you Iambi eki 
Psila ston ourano 
Ksinay alpida me kara 
Na pane stin doulia 
Psilo pono ergatico 
0 Jeorgos La la la la la la la la 
They sounded like a professional chorus. 
This young man, Mr. X, who told me to sit at the back of the bus, 
he thinks I'm the poor cousin coming to visit the rich one. Look at 
this lady who protected me and took me into the office to show me the 
right way to make the sign of the cross. And you, young man, you hear 
all the gossip and the orders that the chairman gives. Look, young man. 
Hitler was a nobody who became very big only through friends like himself, and through extortion. Look, Mr. X, if you knew my life, I am 
richer than all the people put together in these two buses. 

I started to remember again. Mrs. Soula could no longer keep me. 
My life was like a parcel being moved from one place to another. I took 
the bus back to Salonica. Where else could I go? I went back to Georgette and Daniel. Daniel’s was one of the most respected Jewish families in Salonica. His father had a title from another country. He was called Signor Jacov. 
When I got off the bus from Verea, all I saw were some members of 
the Jewish Police. I arrived at Daniel's. Georgette was ready to leave 
for Belgium, where she was from. I spoke go Daniel, and I said, "My 
only solution is to join the underground. I am ready. I had good treatment
at Mrs. Soula's, I rested, and now I am ready to fight the Germans. 
Each new partisan in the mountains is one less soldier at the Russian 
front." Daniel refused. He was preparing to go to Italy. He said "I 
want you to know one thing. I'm not leaving for Italy before you are 
settled somewhere." 
I spent the night again at Dr. Scouros". Of course, his family was 
scared to death because I was there. His daughter Afro was one of the 
organizers of the underground. Daniel took me from this house and sent 
me to the house of another doctor, not far from where I was born. I 
went there and I found the mother and daughter. The father was working 
in a small town and came home only on the weekends. The fees that I had 
to pay at this house were so high that sometimes I didn't know how I was going to pay. But Daniel knew what he was doing. He obtained an Italian passport for me, with the help of the Italian consul. My name was Flora Tivoly, born in Livorno in Italy. He gave me a corset stuffed full with jewellery, and he told me that my family had sent them. He said, 
"You will sell these one by one, and you will be able to survive in 
Athens." 
One morning at 6 A.M. Daniel came. We walked to the railway sta-
tion. We went by back streets. I wore a black veil to cover my face. 
We passed by Venizelos Street. All the Jewish stores were closed. We 
passed by Olympio Diamandi. This street was where my brother's mill was 
located. The mill was open, and the Germans were removing all the machinery. We came to the railway station. The consult himself was checking all the papers, and of course, Hasson too. Hasson recognized me despite my veil.. The consul looked him in the eye, and Hasson remained silent. There were many people around who were ready to kill him if he 
did. 
I was the first one to take a seat in the train. There appeared 
another lady with her husband. The lady was Ida Simantove. She sat 
down near me. She was the daughter of a friend of my mother's. The mother was Italian born, and Ida was, too. But Ida married a Greek citizen and became Greek, and so was deported. But the Italian consul had given her and her husband Italian passports. 
All the passengers had embarked, and the train started to leave. 
After a few hours of travelling, they abandoned us in a field and left. 
The only thing we had in the field was water. But everybody had food. 
I asked Ida where her son was. She answered that he was in Athens with 
her mother and her brother. She said "My son and my brother received 
military uniforms from the Italian consul and they were put on a train 
full of Italian soldiers. Nobody discovered them, because they spoke 
Italian so well." 
Most of the people on the train had places to go to in Athens: parents, relatives. In the train, each one was given a card which told them where to go to eat. I was not given a card. The Italian government had put one school at the disposal of these people, to sleep in. I couldn't go there either, and I was worried about where I would sleep when I arrived in Athens. After two days of not knowing where we were, a locomotive with Italians on board arrived. They took us from the German zone to the Italian zone. Once in the Italian zone, we were taken 
by truck to Athens. 

We arrived in Athens, and I had no place to go. I had met a young gentleman on the train, travelling alone, whose name was Salamon. He now said to me "Come with me, you will find a home in my house." I didn't think twice; I accepted his invitation. I went to his house and 
met his mother and two sisters, and a friend of the family. About two 
hours later, two boys, George and Nicola, came to the house. They told 
me not to worry, they would find a room for me. The next day they found 
one. I took a bracelet from my corset and I sold it so that I could pay 
the rent on my room. Daniel had given me cheese and oil and a few pounds
of wheat, enough to last me a few weeks. 

I had with me in Athens the address of a friend of my father's. 
My mother had often talked about this friend, and I had gone to school
with his daughter. One day Nicolas and George introduced me to a friend 
of theirs, a very distinguished Greek. I showed him the name and the 
address that I had. I asked him if he knew them, and he said to me 
"This man can do a lot for you, and he can save you. The political situation of the Germans and the Italians is not very good. If Italy falls, you will be as lost here as you were in Salonica." He asked me what I had done with my Italian passport. I told him I had burnt it on Daniel's 
request, and that I had a new identity card that Daniel had obtained for 
me. My name was Maritsa Serafimidou. I was born in Comotini. The I friend of Nicolas and George advised me that I should go to see my father's 
friend. 
The next day he came to take me to the address that I had. We went 
there and it turned out to be a beautiful villa in Kefissia. The guard 
at the villa asked me what I wanted. He went in and told his employer 
that Maritsa Serafimidau wanted to see him. Of course, he didn't know 
me by that name. The guard came back and said "I'm sorry, but my boss 
doesn't know you." He wouldn't let me enter, and I couldn't tell him 
that I was really Bouena Sarfatty. I waited outside. Maybe someone would 
come and recognize me. After two hours, a car arrived. I saw a face 
looking out the window. It was the daughter of my father's friend. She 
took me in. I saw lights, maids, black and white, everybody getting 
dressed up. It was the girl's birthday. Cocktail dresses, white tuxedos. 
The guests started to arrive. 
They put me in a room, and the daughter brought me food. When all 
the guests had gone, the father came to my room. He first asked me if 
I had money. If not, he would give me some. I told him that I had some. 
I explained my situation, and that I was afraid that Italy would soon 
fall. He said that it was just a question of days. He called someone 
on the telephone, and then he came back. "Tomorrow morning at 8 A,M. 
you will take the bus to this address. Make sure that the name of the 
driver is Photo. You will tell him that you are Maritsa Serafimidou." 
I told him that I wasn't alone, that I was with the two friends who helped 
me when I arrived, George and Nicola He said the three of us could go. 
In the house where I was staying, very good people lived downstairs. 
They had a daughter of sixteen, and a boy of five. The boy was a Jewish 
boy whom they were hiding. When I came home, Nicolas and George were 
waiting for me downstairs. I told them what I was going to do, and told 
them the news that in a matter of days the Germans would come. It would 
be our turn to go to Kaidary, a criminal prison. No, I wasn't going to 
go places like that any more. "You don't understand what Kaidary is! 
The Jewish people are either killed or terrorized. If they are lucky, 
they are sent to Poland, directly to the crematorium. You can stay. I 
am going!" 
The next day, in the morning, when I was ready to leave, Nicolas and 
George were there also, ready to come with me. Before I left, I went 
to see the two wonderful people who had rented me my room. They gave 
me their blessings and told me "Anytime You can come back here and stay without paying rent. You will eat the way we are eating. And don't worry, you can come back any time." I cried, the two old people cried, and we left. 
As soon as we left the house, I saw a good friend of mine from Salonica with another gentleman. I was very frightened. How did they know I was here? My friend said he came to do me a favour. "You have too much money. Give it to my friend to keep it for you." I was so 
scared, I gave him half the money I had, and we left. 

We came to the bus station. I asked the driver if he was Photo, 
and he said "You are Maritsa". The three of us sat down in the bus. 
Photo said to me "You there! In the back!" in exactly the same tone as 
Mr. X. The only difference was that Photo was doing it for my own good, 
so that no-one would recognize me. 

The bus started to go and go, and we had no idea where we were going. We came to a small town near the sea. Far away, we could see an island. We didn't dare ask where we were, or what island this was. The driver took us to a beautiful farm. The wife and the daughter couldn't 
do enough for us. They made roast lamb for us outside, and we drank 
ouzo. In the evening, the owner of the farm said to the driver "I took 
rooms for them in the hotel. My house is very big, but I have no place 
for them." The next day, he came to the hotel and took us home to have 
breakfast. He gave us tickets for the boat that would take us to the 
island. The name of the island was Evia. We were told that as soon as 
we arrived on Evia, we should sit down at the cafe, and somebody would 
come to take care of us. 
We came to Evia, and we sat down at the cafe. We ordered a coffee, 
and a lady came up to us. She introduced herself as Mrs. Tsacouste. She
took us to a room. She fed us, and she invited us to listen to the radio. 
This was a great luxury for us, because in Greece it was prohibited to 
listen to the radio. We heard a London broadcast in Greek. The next 
day, Mrs. Tsacouste took us to Ano Mamoula to a barn with three horses 

-25- 
two donkeys, and it was full of flies. Big flies! In the dark, the 
flies had a green and silver color. I never saw flies like this before 
in my life. But we had a radio. I was afraid to say even one word, be-
cause Nicola and George would say "Where did you bring us?". The owner 
of the barn was the best man I ever met. If he slaughtered a sheep, he 
 would give us a piece. He gave us oil, but even with what he gave us, 
we started to feel starvation. The people of the village didn’t want 
my jewelry, only gold coins. Most of my gold coins I had given away 
in Athens. 
One day, Nicola and George, feeling desperate because of the life 
we were leading, decided to go to Steny. It was the capital of the 
underground. I told them they should take me with them. They refused. 
If they found a room, they would come back for me. The owner of the barn 
told them not to go. But the next day, at four in the morning, they were 
walking toward Steny. I was alone with the horses, the donkeys, and the 
flies. The flies were killing me. The wind from the mountains sounded 
like a regiment coming to take me away. I wanted to cry. But I remem-
bered Chaim. He told me to sing and never to cry. And I started to 
sing the song of solitude: 
Monaxia 
Kafe tosso kic mirca 
Monaxia 
Ice pio sklire parea 
Opios pi i zoi pos iney orea 
Den faki tiki zi 
Me senane mazi 
This means that if somebody says that life is very good, it 
is because he has never had the company of solitude. As soon as I sang 
the last word's of my song, the door of the barn was opened and the 
owner of the barn was at the door. 
He told me "Maritsa, do you see that boat far away that is 
approaching? It is full of Germans. Full of Jewish people. Take your 
belongings and those of the boys. The Germans can come to check here." 
I asked him how he knew. "I had a telephone call from my people," he 
said, "I told the boys not to go to Steny! But they wouldn't listen 
to me. The whole island is terrified. Nobody will take strangers without a recommendation. This boatful of Germans will take everyone prisoner and send all of them to Haidary. What hurts me is that there are many children in this boat. Hurry up, hurry up, I'll explain to 
you as we go. Let's clean up the barn. If they come to inspect, they 
shouldn’t see that there were people here." 
He took all the animals and changed their places. I took all our 
belongings and we left. Suddenly we remembered that we had left the 
radio. We went back. We got the radio and started to run. We could see 
the Germans with their machine guns. If somebody was going to try to 
escape, they were ready. The partisans were there, but only if they 
were wearing uniforms would we know them, and they were without their 
uniforms. They were there to rescue Jewish children as if they were 
their own. 
We were walking, running, to reach the farmhouse of the owner. 
Suddenly we heard "Voifia, Voifia" - "Help! Help!" It was a little 
boy about four years old and wounded. The owner of the barn picked 
up the boy. I had one of Nicola's shirts in my hand. I tied his foot 
so that the blood wouldn't give our trail away. We reached the farm-
house. I asked the owner where his wife was. He said to me, "I went 
to you, but my wife went to the boat. There are many pregnant Jewish 
women on this boat and she went to help them. You will ask yourself 
why I'm doing this. But you have to understand that every day I 
remember that the German flag is waving in the Acropolis." 
We came to the farm and we asked the boy his name. It was Miki. 
"And you?" he asked me. "And you?" he asked the man. "Afendico is 
my name (this means landlord in Greek). Alright, Miki," said the landlord, 
"I'll take you to the doctor. We'll say to him that you are my son." 
Miki understood very well. He had been brought up in this kind of 
atmosphere. Before they went away the owner said to me, "If you see 
people running, Jewish or not, take them in." I begged him to wait 
until everything was quiet and his wife returned, and then go to the 
doctor. "One bullet can come and you are finished, the way it is now 
outside." "Don't worry, " he said, "we know our mountains very well. 
Nothing will happen to me." He took Miki and went to the doctor to 
have the bullet from Miki's leg removed. Two hours later, he came 
back alone. I asked him where the boy was. 
"The doctor will keep him at his house until tomorrow, and then 
I will go to pick him up." The next day I found Miki; the Owner of 
the barn had brought him home. I was starting to believe that everything I had was wrapped up in this little boy, so sweet and nice-looking. I asked him who his father was. "0, my father is in the Middle East," he said, very proud. "He went with the British when Greece 
lost its battle. But we won't lose. My father is still fighting against 
the Germans to save Greece. My father is Decanea (which in Greek means 
19). He is the commander of nineteen soldiers. And you know, Maritsa, 
my father is very easy to recognize. He has an olive birthmark here, 
(he showed me the back of his neck.) Do you want to see the olive of 
my father? Come here and see mine. My mother and I went to a boat that 
was going to the Middle East to find my father. While we were waiting 
for the boat to cross to the island, we sat down in the cafe. I went 
to play with one of the captains. My mother was still drinking her 
coffee when I saw the Germans coming to take her. For sure they took 
her to Haidary. The captain treated me like I was his own son. But as 
soon as I came to the island I started to run. All the people who 
ran were shot at by the Germans. But don’t worry, Maritsa, my father 
is going to shoot them, because my father is a Decanea." But for me, 
it wasn't so easy. There was one more mouth to feed. But Miki never 
complained of hunger. When I could not give him enough to eat, he 
would look at me with wide eyes, but never complained. 
We were on our way to the Dead Sea. On the way, the guide showed 
us the beautiful and unbelievable things that the government of Israel is doing. They were washing the land, because all the soil was filled 
with salt. They were washing it to make it good for agriculture. We 
saw a kibbutz where they were doing experiments with plants. There 
were soldiers working there who were North African Jews. I was very 
excited. I spoke Ladino with them. The Chairman came over in the middle 
of the conversation. She said to me, "Hey you! You can't talk another 
language. I have to understand what you say." I answered, "I am to 
be censured here?" I turned my back to her and continued my conversation 
with the boys in Ladino. I called the photographer over to take a 
picture of me with the soldiers. The photographer said "I 
can't take your picture. I have orders not to take your picture." 
I jumped. The Chairman saw that there would be a big fight so she told 
the photographer to go ahead and take the picture. My husband gave the 
boys a few dollars and we left. I was ready to go and complain to the 
head of the expedition, but I didn't because he was always busy and 
I could not find a moment alone with him. 
In the evening, we had an invitation to dinner from the Israeli 
Government. I dressed very nicely as if nothing was upsetting me. I 
wore a beautiful Ruban Lam dress that I had designed and sewed myself. 
I had taken my best wardrobe on this trip, everything designed 
and made by myself. The dinner was held at a fine hotel in Jerusalem. 
As soon.as we came to the door, we were all told to go to the bar and 
get a drink. My husband went to sit down. I took two glasses to get 
us each a drink. A man came up to me. He was a government employee. 
He said to me "I have instructions to tell you to go and sit down. Your 
husband should come and get the drinks." I answered "We have been to-
gether for two days now, and you must know very well that my husband 
cannot carry two glasses at the same time." "I'm an employee," he answered me, "these are my orders." I took the two drinks and told him 
"I"m not taking any orders." 
As I sat down, I started to remember how this chairman (or friend, I don't know what to call her) destroyed the life of my son in school. 
She terrified all the children by telling them how my husband was shaking. She even gave us a nickname. My son used to come home and cry. 
I wanted to say to her "You destroyed the childhood of my son, and now 
you do everything in your power to destroy my trip. You are a chairman, in charge of raising money for sick people, for widows, and for the poor!" 
When I found that I was about to cry, I started to sing "Hava Nagila". 
We returned to the hotel. I couldn't sleep all night. I was thinking of how my husband first began to shake. I had a little girl after my son Ely, whose name was Regina. I was very happy. I had Eliaou and Regina with me. Regina died when she was just three months old. 
When we returned from the cemetery, we were very bitter, my husband and 
I.The doctor came and gave us some sleeping pills. We went to sleep, 
but were suddenly awakened by the telephone. My husband answered. It 
was the police. They told us that our factory, children's wear, was in 
flames. Max started to shake while he was on the phone. 
He never spent a day sick in bed; he just shook. We went to see 
different neurologists. One day a new series started on television, 
"Ben Casey", and they showed a brain operation that was done to control 
tremors. There were advertisements in newspapers for this type 
of operation. My sister-in-law was very excited about Dr. Casey on 
the television and in the newspapers. Before I made a decision, I went 
to see patients before and after this operation. I saw many patients 
who were sick before the operation, but none was sick like my husband. 
I started to go from one neurologist to another to get an idea of what 
to do. The majority of the doctors told us not to go ahead with 
the operation. They said that in a few years there would be pills to 
control the tremors. The pills were being researched, but they were to 
be available soon on a first come first served basis. Our name was on 
the list at the Jewish Convalescent Hospital, and I was sure that when 
we returned from our trip to Israel, our name would be called. You'll 
see, Madam Chairman, that my husband will not shake any more. 
The next day we were to go to Jericho. In the morning I went to
the bus. My guide stood at the door of the bus. I told him that Mr. 
Garfinkle was coming. 
"I don't wait! I have orders," he answered me very rudely. 
"Then let's go!” I said. 
"But the others have not come yet," he answered. 
"Oh," I said, "you mean you will wait for the others but not for 
us?" 
"Not for you, and not for your husband." 
"Oh, Mr. Guide, you can tell the people who gave you the orders that 
we are here and we are going to stay." 
My husband arrived, and we went to sit down. Mr. X., the ca-
valier servant of the chairman, appeared. "Mrs. Garfinkle, in the back.” 
he said. I remembered the prayer that one says at Yom Kippur. Eliaou 
used to help the Hassan in this prayer at "0 Dio piadozo" - but I said 
this prayer in my own words - let me not fight with these people. I said 
nothing. 
I went to sit down in the seat that he had indicated. I was sitting under the air conditioning, and water was dripping from it. I was starting to get wet. The water was like chemical water; it irritated 
the skin. My backside started to itch, and I was very uncomfortable 
and miserable. And I started to remember once more. 

A few days after Miki came back from the doctor, I took him for a 
walk, to get some fresh air. We were now staying at another barn, with 
another radio. We saw some people approaching. There were Jews and 
Christians and partisans. Hitler was supposed to give a talk that night 
and everyone wanted to hear what he had to say about the Jewish question. 
There were rumours at the time that the Germans had asked the Americans 
for money, tanks, airplanes and other supplies. If the Americans agreed, 
the Jewish people would be allowed to go free. But Hitler didn't say 
that he had asked the Americans for help. Instead he attacked the Americans and said that they were led by Jews and that money was their 
whole life. He was going to spare the Jews, but they were going to be very miserable. They were going to regret that they didn't die. 
I had developed a stomach ache, so I went outside with Miki. Suddenly we saw Germans approaching with torches. We hid in a ditch and watched as the Germans set the barn on fire, with everyone who had been listening to the radio still inside. They died singing "ou mi anaknou Israel". 
Miki and I spent the night in the ditch, and I slept a bit. I 
dreamed that I was at home. My Nona (grandmother) came to visit us. 
Whenever she came to see us she would bring us peppermint candies. She 
said to us "All my life you used to ask me, “You're so religious, where 
will you go in the next world?” Now is the time to tell you. My job 
consists of putting oil for Buena al cote maravi (at the Wailing Wall). 
And I swear to you, in three months, you yourself are going to go to 
the Wailing Wall. And you yourself will put oil in the containers at the Wailing Wall." I woke up, and naturally I found myself with Miki in the 
ditch. I woke him up. "Let's get out of here!" 
We started to walk. Because of the echo from the mountains, it was very easy to hear voices. We heard a baby crying. We came across a 
mother and her child. The mother was covered with blood; she was dying. 
The baby was three or four months old. The only thing she said to me in 
Ladino was "Salvalo" - save him. And she told me not to lose the booties. 
It didn't make much sense to me. 
In the meantime, Nicola and George had come back from Steny. They 
had not found accommodations. Nicola asked me if I was planning to start 
an orphanage. "What do you want?" I answered. "Should I let them die?" 
 I decided at this moment that I had to find a way to get to the Middle East. 
There were no diapers with which to change the baby. The wife of 
the owner of the barn gave me a few rags to use. The baby's 
backside became redder and redder every day. It was a pity to see. 
George, Nicola and I had a conference. We did not have enough money to go to the Middle East. I told the boys that I had a corset full of jewellery. We would go to the fishermen and find a boat and get out of there. We found two families that had gold coins The fishermen agreed to take us  if they were paid half in jewellery and half in gold coins. 
George and Nicola told me to go to Athens and bring back Tamo, who 
was a cousin of theirs. Of course I refused. After six hours of discussion I still did not want to go into the fire of Athens. They made me go. 
I arrived in Athens and I phoned the proprietor of a hotel who was 
supposed to tell me where to find Tamo. He told me to call back the 
next day. I said "Look, every moment counts. Every moment I stay here 
I risk going to Haidary." I called my friend to try to get my money back.
I didn't get the money; instead he begged me to take him with me. 
I had to spend the night somewhere, so I went back to the two good 
people who had rented me a room when I first came to Athens. If these 
people were discovered, they would be killed. They welcomed me with 
food. I told the neighbor downstairs who was hiding the little Jewish 
boy of my plan to go to the Middle East. She begged me to take the 
boy's father with me. 
I called the hotel manager again and he gave me Tamo's address. 
It felt as if the bus took a year to get to her place. I got off and 
walked a bit, and came to a gorgeous villa. The housekeeper came to 
receive me and he asked me to take a seat in a beautiful dining room. 
I was surprised to see that people still had tables. 
Tamo used to be the most elegant lady in Salonica. She had beautiful dresses, shoes, hair. I thought of how, when Tamo walked down the street, people would look at her to see what the latest styles were. 
She now appeared in the dining room. She was wearing a beautiful house  
dress. Her hands were manicured, and her hair was dyed in the latest 
fashion, a touch of blue in places. I told her our plan. 
"I came to take you and Victor, on the request of George and Nicola: There is no dining room, no blue hair dye, no manicure in the place where I have come from. Cut your hair very short, try to take the blue dye out of it, and tie a kerchief around your head."  
Tamo's perfume was opening up my lungs. When she heard all the 
things I had said, Tamo assured me that she was fine where she was, she 
was hiding very well. I asked about Victor, her husband. She assured 
me that he, too, was well. I left before Tamo could change her mind 
about coming with me. 
In Athens the streets were filled with terror. Even the Christians 
were afraid to walk in the streets. You can imagine me. I was wearing 
a veil to cover my face. I got to my room. I took my friend and the 
little boy's father. My landlord and the people downstairs both gave 
us parcels of food. We took the bus to the same small town to take the 
boat to Evia. We sat down in the cafe, outside, to wait for the 
boat. My friend and the boy's father both got up to go to the bathroom. 
This had become a tradition - all the Jewish people who had to wait for 
something were so scared they had to keep running to the bathroom.  
Thank God that they went away! 
A German truck arrived, and was taking all the men in the streets 
away. I saw many women whom I knew who had escaped to Athens from Salonica, sitting in this cafe. The Germans took them too - they were 
recognized by the way they were dressed. But I was dressed like a peasant, and I had my shoes under my arm, the way peasants did when they 
sat in a cafe. Only the shoes saved me. When the truck left, the own-
er of the cafe brought me an ouzo. I felt as if I was paralyzed. The 
owner urged me to drink. I told him that I hadn't ordered anything, 
that I had no money. He said "Drink! It's an order! It is on the house, 
you need it." 
 
The two men came out of the bathroom. The boat was ready for boarding, and we took it to Evia. In the boat I did not speak to the two men I was with. I asked myself if I was an angel, or just stupid. If George and Nicolas wanted to save their cousin Tamo, why hadn't 
they themselves gone to Athens? I called this doing good with someone 
else's back. But I said nothing. 
When we arrived at Evia, everyone was ready to depart for the Middle 
East. They would not have waited even five minutes for me. I took Miki 
and the baby, and the whole expedition headed for the rendez-vous that 
we had with the fishermen. 
